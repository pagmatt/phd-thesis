\documentclass{mimosis}
\usepackage[a4paper]{geometry}
\geometry{left=1.55in}
\geometry{right=1.55in}
% PDF-A output
%\usepackage{metalogo}
%\usepackage[a-3u]{pdfx}

% Compile just one chapter
%\includeonly{Sources/chapter_to_include.tex}

%%%%%%%%%%%%%%%%%%%%%%%%%%%%%%%%%%%%%%%%%%%%%%%%%%%%%%%%%%%%%%%%%%%%%%%%
% Some of my favorite personal adjustments
%%%%%%%%%%%%%%%%%%%%%%%%%%%%%%%%%%%%%%%%%%%%%%%%%%%%%%%%%%%%%%%%%%%%%%%%
%
% These are the adjustments that I consider necessary for typesetting
% a nice thesis. However, they are *not* included in the template,   as
% I do not want to force you to use them.

% This ensures that I am able to typeset bold font in table while still aligning the numbers
% correctly.
\usepackage{etoolbox}

\usepackage{siunitx}
\DeclareSIUnit\px{px}
\DeclareSIUnit\dBm{dBm}

\sisetup{%
  detect-all           = true,
  detect-family        = true,
  detect-mode          = true,
  detect-shape         = true,
  detect-weight        = true
}

%%%%%%%%%%%%%%%%%%%%%%%%%%%%%%%%%%%%%%%%%%%%%%%%%%%%%%%%%%%%%%%%%%%%%%%%
% Customs 
%%%%%%%%%%%%%%%%%%%%%%%%%%%%%%%%%%%%%%%%%%%%%%%%%%%%%%%%%%%%%%%%%%%%%%%%


% Openright with oneside class
% Taken from http://mirrors.ctan.org/macros/latex/unpacked/latex.ltx
\makeatletter
\def\cleardoublepage{\clearpage\ifodd\c@page\else
    \hbox{}\newpage\if@twocolumn\hbox{}\newpage\fi\fi}
\makeatother

% No footnote ends in different pages
\interfootnotelinepenalty=10000

\usepackage{amsthm}
% Argmax
\DeclareMathOperator*{\argmax}{arg\,max}

% Lemma environment
%\newtheorem{thm}{Theorem}[section]
\newtheorem{lem}{Lemma}
% Better QED
\renewcommand\qedsymbol{$\blacksquare$}

% Algorithms
\usepackage{algpseudocode,algorithm,algorithmicx}
% Asymptotic computational complexity
\newcommand\bigO[1]{$\mathcal{O}(#1)$}

\definecolor{UNIPDRED}{RGB}{155,0,20}

\setcounter{tocdepth}{3}
\setcounter{secnumdepth}{3}

% Coloured chapter numbers
\renewcommand*{\chapterformat}{%
 \selectfont\textcolor{UNIPDRED}{\huge\thechapter}\hspace{0.2in}%
}

\usepackage[inline]{enumitem}


% Better tables
\usepackage{colortbl}
\usepackage{multirow}
\usepackage{hhline}

\renewcommand\chapterlineswithprefixformat[3]{%
  \MakeUppercase{#2#3}
}

% Placeholder text
\usepackage{lipsum}

% TikZ & Co.
\usepackage{tikz}
\usepackage{pgfplots}
\pgfplotsset{compat=1.16}
\usepgfplotslibrary{polar}
\usepackage{pgfplotstable}
\pgfplotsset{plot coordinates/math parser=false} 
\usetikzlibrary {patterns.meta}
\usetikzlibrary{plotmarks,patterns,patterns.meta,decorations.pathreplacing,backgrounds,calc,arrows,arrows.meta,spy,matrix}
\newlength\fheight
\newlength\fwidth
\usepackage{booktabs}
\usepackage{multirow}
\usetikzlibrary{patterns,decorations.pathreplacing,backgrounds,calc}
\usepgfplotslibrary{patchplots,groupplots,fillbetween}

\usepackage{afterpage}
\usepackage{tabulary}

%%%%%%%%%%%%%%%%%%%%%%%%%%%%%%%%%%%%%%%%%%%%%%%%%%%%%%%%%%%%%%%%%%%%%%%%
% Hyperlinks & bookmarks
%%%%%%%%%%%%%%%%%%%%%%%%%%%%%%%%%%%%%%%%%%%%%%%%%%%%%%%%%%%%%%%%%%%%%%%%

\usepackage{bookmark}
\bookmarksetup{color=black}

%%%%%%%%%%%%%%%%%%%%%%%%%%%%%%%%%%%%%%%%%%%%%%%%%%%%%%%%%%%%%%%%%%%%%%%%
% Bibliography
%%%%%%%%%%%%%%%%%%%%%%%%%%%%%%%%%%%%%%%%%%%%%%%%%%%%%%%%%%%%%%%%%%%%%%%%
%
% I like the bibliography to be extremely plain, showing only a numeric
% identifier and citing everything in simple brackets. The first names,
% if present, will be initialized. DOIs and URLs will be preserved.

\usepackage[%
  autocite     = plain,
  backend      = biber,
  doi          = true,
  url          = true,
  giveninits   = true,
  hyperref     = true,
  maxbibnames  = 99,
  maxcitenames = 99,
  sorting		= none,
  style        = numeric-comp,
  ]{biblatex}
\hypersetup{
  % Document-wide hyperlinks color
  colorlinks=true,
  citecolor=UNIPDRED,
  %allbordercolors=UNIPDRED
}
  
 \usepackage{xurl}

\addbibresource{Sources/bibl.bib}

%%%%%%%%%%%%%%%%%%%%%%%%%%%%%%%%%%%%%%%%%%%%%%%%%%%%%%%%%%%%%%%%%%%%%%%%
% Fonts
%%%%%%%%%%%%%%%%%%%%%%%%%%%%%%%%%%%%%%%%%%%%%%%%%%%%%%%%%%%%%%%%%%%%%%%%

%\usepackage[lf]{ebgaramond}
%\usepackage[scale=0.8]{sourcecodepro}
%\usepackage{newpxtext,newpxmath} % Palatino like text and math
%\sffamily%

\IfFileExists{mathpazo.sty}{\RequirePackage[osf,sc]{mathpazo}}{}
\IfFileExists{helvet.sty}{\RequirePackage[scaled=0.90]{helvet}}{}
\IfFileExists{beramono.sty}{\RequirePackage[scaled=0.85]{beramono}}{}
\onehalfspacing

\renewcommand{\th}{\textsuperscript{\textup{th}}\xspace}

\newacronym{3gpp}{3GPP}{3rd Generation Partnership Project}
\newacronym{adc}{ADC}{Analog to Digital Converter}
\newacronym{afbw}{AFBW}{Average Fading Bandwidth}
\newacronym{5g}{5G}{5th generation}
\newacronym{4g}{4G}{4th generation}
\newacronym{aimd}{AIMD}{Additive Increase Multiplicative Decrease}
\newacronym{am}{AM}{Acknowledged Mode}
\newacronym{amf}{AMF}{Access and Mobility Management Function}
\newacronym{an}{AN}{Access Network}
\newacronym{amc}{AMC}{Adaptive Modulation and Coding}
\newacronym{aqm}{AQM}{Active Queue Management}
\newacronym{awgn}{AGWN}{Additive White Gaussian Noise}
\newacronym{balia}{BALIA}{Balanced Link Adaptation}
\newacronym{bsr}{BSR}{Buffer Status Report}
\newacronym{msr}{MSR}{Max Sum-Rate}
\newacronym{ba}{BA}{Backlog Avoidance}
\newacronym{mrba}{MRBA}{Max-Rate Backlog Avoidance}
\newacronym{bdp}{BDP}{Bandwidth-Delay Product}
\newacronym{bf}{BF}{Beamforming}
\newacronym{v2x}{V2X}{Vehicle-to-Everything}
\newacronym{vr}{VR}{Virtual Reality}
\newacronym{mdp}{MDP}{Markov Decision Process}
\newacronym{mwm}{MWM}{Maximum Weighted Matching}
\newacronym{tdm}{TDM}{Time Division Multiplexing}
\newacronym{fdm}{FDM}{Frequency Division Multiplexing}
\newacronym{sdm}{SDM}{Space Division Multiplexing}
\newacronym{st}{ST}{Spanning Tree}
\newacronym{rl}{RL}{Reinforcement Learning}
\newacronym{lp}{LP}{Linear Programming}
\newacronym[firstplural=Deep Neural Networks (DNNs)]{dnn}{DNN}{Deep Neural Network}
\newacronym{dag}{DAG}{Directed Acyclic Graph}
\newacronym{gtp}{GTP}{GPRS Tunneling Protocol}
\newacronym{nas}{NAS}{Non-Access Stratum}
\newacronym{isp}{ISP}{Internet Service Provider}
\newacronym{ngnm}{NGNM}{Next Generation Mobile Networks Alliance}
\newacronym{cc}{CC}{Component Carrier}
\newacronym{drb}{DRB}{Data Radio Bearer}
\newacronym{qos}{QoS}{Quality of Service}
\newacronym{ca}{CA}{Carrier Aggregation}
\newacronym{sdap}{SDAP}{Service Data Adaptation Protocol}
\newacronym{lc}{LC}{Logical Channel}
\newacronym{rnti}{RNTI}{Radio Network Temporary Identifier}
\newacronym{qci}{QCI}{Quality Class Identifier}
\newacronym{cdf}{CDF}{Cumulative Distribution Function}
\newacronym{cmos}{CMOS}{Complementary Metal-Oxide Semiconductor}
\newacronym{cn}{CN}{Core Network}
\newacronym{cqi}{CQI}{Channel Quality Information}
\newacronym{cir}{CIR}{Channel Impulse Response}
\newacronym{cp}{CP}{Control Plane}
\newacronym{cu}{CU}{Central Unit}
\newacronym{du}{DU}{Distributed Unit}
\newacronym{csirs}{CSI-RS}{Channel State Information - Reference Signal}
\newacronym{dc}{DC}{Dual Connectivity}
\newacronym{imsi}{IMSI}{International Mobile Subscriber Identity}
\newacronym{dce}{DCE}{Direct Code Execution}
\newacronym{dci}{DCI}{Downlink Control Information}
\newacronym{uci}{UCI}{Uplink Control Information}
\newacronym{dl}{DL}{Downlink}
\newacronym{dmr}{DMR}{Deadline Miss Ratio}
\newacronym{dmrs}{DMRS}{DeModulation Reference Signal}
\newacronym{e2e}{E2E}{End-to-End}
\newacronym{ecn}{ECN}{Explicit Congestion Notification}
\newacronym{edf}{EDF}{Earliest Deadline First}
\newacronym{enb}{eNB}{evolved Node Base}
\newacronym{embb}{eMBB}{enhanced Mobile Broadband}
\newacronym{epc}{EPC}{Evolved Packet Core}
\newacronym{es}{ES}{Edge Server}
\newacronym{fdma}{FDMA}{Frequency Division Multiple Access}
\newacronym{fdd}{FDD}{Frequency Division Duplexing}
\newacronym[firstplural=Radio Access Technologies (RATs)]{rat}{RAT}{Radio Access Technology}
\newacronym[firstplural=Markov Chains (MCs)]{mc}{MC}{Markov Chain}
\newacronym{milp}{MILP}{Mixed-Integer Linear Programming}
\newacronym{fs}{FS}{Fast Switching}
\newacronym{ip}{IP}{Internet Protocol}
\newacronym{fr}{FR}{Frequency Range}
\newacronym{ftp}{FTP}{File Transfer Protocol}
\newacronym{gnb}{gNB}{Next Generation Node Base}
\newacronym{arq}{ARQ}{Automatic Repeat reQuest}
\newacronym{harq}{HARQ}{Hybrid Automatic Repeat reQuest}
\newacronym{hetnet}{HetNet}{Heterogeneous Network}
\newacronym{hh}{HH}{Hard Handover}
\newacronym{hol}{HOL}{Head-of-Line}
\newacronym{ia}{IA}{Initial Access}
\newacronym{imt}{IMT}{International Mobile Telecommunication}
\newacronym{iot}{IoT}{Internet of Things}
\newacronym{lcr}{LCR}{Level Crossing Rate}
\newacronym{lcf}{LCF}{Level Crossing Frequency}
\newacronym{los}{LoS}{Line-of-Sight}
\newacronym{lte}{LTE}{Long Term Evolution}
\newacronym{m2m}{M2M}{Machine to Machine}
\newacronym{mac}{MAC}{Medium Access Control}
\newacronym{num}{NUM}{Network Utility Maximization}
\newacronym{ri}{RI}{Rank Index}
\newacronym{pmi}{PMI}{Precoding Matrix Index}
\newacronym{mcs}{MCS}{Modulation and Coding Scheme}
\newacronym{mec}{MEC}{Mobile Edge Cloud}
\newacronym{mi}{MI}{Mutual Information}
\newacronym{mimo}{MIMO}{Multiple Input, Multiple Output}
\newacronym{mmwave}{mmWave}{millimeter wave}
\newacronym{mptcp}{MPTCP}{Multipath TCP}
\newacronym{mr}{MR}{Maximum Rate}
\newacronym{mt}{MT}{Mobile Termination}
\newacronym{mss}{MSS}{Maximum Segment Size}
\newacronym{mtd}{MTD}{Machine-Type Device}
\newacronym{mtu}{MTU}{Maximum Transmission Unit}
\newacronym{nfv}{NFV}{Network Function Virtualization}
\newacronym{nf}{NF}{Network Function}
\newacronym{nlos}{NLoS}{Non-Line-of-Sight}
\newacronym{nr}{NR}{New Radio}
\newacronym{csi}{CSI}{Channel State Information}
\newacronym{o2i}{O2I}{Outdoor-to-Indoor}
\newacronym{ofdm}{OFDM}{Orthogonal Frequency Division Multiplexing}
\newacronym{pdcch}{PDCCH}{Physical Downlink Control Channel}
\newacronym{pdcp}{PDCP}{Packet-Data Convergence Protocol}
\newacronym{pdsch}{PDSCH}{Physical Downlink Shared Channel}
\newacronym{pdu}{PDU}{Packet Data Unit}
\newacronym{sdu}{SDU}{Service Data Unit}
\newacronym{pf}{PF}{Proportional Fair}
\newacronym{pgw}{PGW}{Packet Gateway}
\newacronym{phy}{PHY}{Physical}
\newacronym{pbch}{PBCH}{Physical Broadcast Channel}
\newacronym[plural=\gls{mme}s,firstplural=Mobility Management Entities (MMEs)]{mme}{MME}{Mobility Management Entity}
\newacronym{prb}{PRB}{Physical Resource Block}
\newacronym{pss}{PSS}{Primary Synchronization Signal}
\newacronym{pucch}{PUCCH}{Physical Uplink Control Channel}
\newacronym{pusch}{PUSCH}{Physical Uplink Shared Channel}
\newacronym{rach}{RACH}{Random Access Channel}
\newacronym{ran}{RAN}{Radio Access Network}
\newacronym{ngran}{NG-RAN}{Next Generation RAN}
\newacronym{red}{RED}{Random Early Detection}
\newacronym{rf}{RF}{Radio Frequency}
\newacronym{rlc}{RLC}{Radio Link Control}
\newacronym{rlf}{RLF}{Radio Link Failure}
\newacronym{rrc}{RRC}{Radio Resource Control}
\newacronym{rrm}{RRM}{Radio Resource Management}
\newacronym{rr}{RR}{Round Robin}
\newacronym{rs}{RS}{Remote Server}
\newacronym{rsrp}{RSRP}{Reference Signal Received Power}
\newacronym{rss}{RSS}{Received Signal Strength}
\newacronym{rtt}{RTT}{Round Trip Time}
\newacronym{rw}{RW}{Receive Window}
\newacronym{rx}{RX}{Receiver}
\newacronym{sa}{SA}{standalone}
\newacronym{sack}{SACK}{Selective Acknowledgment}
\newacronym{sap}{SAP}{Service Access Point}
\newacronym{sch}{SCH}{Secondary Cell Handover}
\newacronym{scoot}{SCOOT}{Split Cycle Offset Optimization Technique}
\newacronym{sdma}{SDMA}{Spatial Division Multiple Access}
\newacronym{sdn}{SDN}{Software Defined Networking}
\newacronym{sinr}{SINR}{Signal-to-Interference-plus-Noise Ratio}
\newacronym{sir}{SIR}{Signal-to-Interference Ratio}
\newacronym{sm}{SM}{Saturation Mode}
\newacronym{snr}{SNR}{Signal-to-Noise Ratio}
\newacronym{son}{SON}{Self-Organizing Network}
\newacronym{ss}{SS}{Synchronization Signal}
\newacronym{srs}{SRS}{Sounding Reference Signal}
\newacronym{sss}{SSS}{Secondary Synchronization Signal}
\newacronym{tb}{TB}{Transport Block}
\newacronym{tcp}{TCP}{Transmission Control Protocol}
\newacronym{tdd}{TDD}{Time Division Duplexing}
\newacronym{tdma}{TDMA}{Time Division Multiple Access}
\newacronym{tfl}{TfL}{Transport for London}
\newacronym{tm}{TM}{Transparent Mode}
\newacronym{trp}{TRP}{Transmitter Receiver Pair}
\newacronym{tti}{TTI}{Transmission Time Interval}
\newacronym{ttt}{TTT}{Time-to-Trigger}
\newacronym{tx}{TX}{Transmitter}
\newacronym{qam}{QAM}{Quadrature Amplitude Modulation}
\newacronym{ue}{UE}{User Equipment}
\newacronym{ul}{UL}{Uplink}
\newacronym{uml}{UML}{Unified Modeling Language}
\newacronym{um}{UM}{Unacknowledged Mode}
\newacronym{uma}{UMa}{Urban Macro}
\newacronym{utc}{UTC}{Urban Traffic Control}
\newacronym{vm}{VM}{Virtual Machine}
\newacronym{rsrq}{RSRQ}{Reference Signal Received Quality}
\newacronym{rssi}{RSSI}{Received Signal Strength Indicator}
\newacronym{crs}{CRS}{Cell Reference Signal}
\newacronym{nsa}{NSA}{Non Stand Alone}
\newacronym{mrdc}{MR-DC}{Multi \gls{rat} \gls{dc}}
\newacronym{eutra}{E-UTRA}{Evolved Universal Terrestrial Radio Access}
\newacronym{endc}{EN-DC}{E-UTRAN-\gls{nr} \gls{dc}}
\newacronym{5gc}{5GC}{5G Core}
\newacronym{si}{SI}{Study Item}
\newacronym{iab}{IAB}{Integrated Access and Backhaul}
\newacronym{wf}{WF}{Wired-first}
\newacronym{hqf}{HQF}{Highest-quality-first}
\newacronym{pa}{PA}{Position-aware}
\newacronym{mlr}{MLR}{Maximum-local-rate}
\newacronym{wbf}{WBF}{Wired Bias Function}
\newacronym{mib}{MIB}{Master Information Block}
\newacronym{sib}{SIB}{Secondary Information Block}
\newacronym{kpi}{KPI}{Key Performance Indicator}
\newacronym{ppp}{PPP}{Poisson Point Process}
\newacronym{mpc}{MPC}{Multi Path Component}
\newacronym{rt}{RT}{Ray Tracer}
\newacronym{aoa}{AoA}{Angle of Arrival}
\newacronym{aod}{AoD}{Angle of Departure}
\newacronym{scm}{SCM}{Spatial Channel Model}
\newacronym{inr}{INR}{Interference to Noise Ratio}
\newacronym{qd}{QD}{Quasi Deterministic}
\newacronym{wlan}{WLAN}{Wireless Local Area Network}
\newacronym{cad}{CAD}{Computer-aided Design}
\newacronym{ap}{AP}{Access Point}
\newacronym{sta}{STA}{Station}
\newacronym{urllc}{URLLC}{Ultra-Reliable Low-Latency Communication}
\newacronym{udp}{UDP}{User Datagram Protocol}
\newacronym{upf}{UPF}{User Plane Function}
\newacronym{umi}{UMi}{Urban Microcell}
\newacronym{uma2}{UMa}{Urban Macrocell}
\newacronym{rma}{RMa}{Rural Macrocell}
\newacronym{in}{In}{Indoor Office}
\newacronym{wpans}{WPANs}{Wireless Personal Area Networks}
\newacronym{wpan}{WPAN}{Wireless Personal Area Network}
\newacronym{fd}{FD}{Full Duplex}
\newacronym{crc}{CRC}{Cyclic Redundancy Check}
\newacronym{wb}{WB}{Wideband}
\newacronym{sb}{SB}{Subband}
\newacronym{bap}{BAP}{Backhaul Adaptation Protocol}
\newacronym{lcg}{LCG}{Logical Channel Group}
\newacronym{ecdf}{ECDF}{Empirical Cumulative Distribution Function}
\newacronym{mu}{MU}{Multi-User}
\newacronym{ce}{CE}{Control Element}
\newacronym{ric}{RIC}{RAN Intelligent Controller}
\newacronym{bler}{BLER}{Block Error Rate}
\newacronym{ieee}{IEEE}{Institute of Electrical and Electronics Engineers}
\newacronym{ilp}{ILP}{Integer Linear Program}
\newacronym{ldpc}{LDPC}{Low-Density Parity Check}
\newacronym{nlosv}{NLOSv}{Vehicle Non-Line-of-Sight}
\newacronym{pscch}{PSCCH}{Physical Sidelink Control Channel}
\newacronym{sc}{SC}{Single Carrier}
\newacronym{sl}{SL}{Sidelink}
\newacronym{dft}{DFT}{Discrete Fourier Transform}
\newacronym{v2v}{V2V}{Vehicle-to-Vehicle}
\newacronym{wave}{WAVE}{Wireless Access in Vehicular Environments}
\newacronym{upa}{UPA}{Uniform Planar Array}
\newacronym{fec}{FEC}{Forward Error Correction}
\newacronym{psfch}{PSFCH}{Physical Sidelink Feedback Channel}
\newacronym{pssch}{PSSCH}{Physical Sidelink Shared Channel}
\newacronym{csma}{CSMA}{Carrier Sense Multiple Access}
\newacronym{v2n}{V2N}{Vehicle-to-Network}
\newacronym{cav}{CAV}{Connected and Autonomous Vehicle}
\newacronym{v2i}{V2I}{Vehicle-to-Infrastructure}
\newacronym{d2d}{D2D}{Device-to-Device}
\newacronym{c-its}{C-ITS}{Connected Intelligent Transportation System}
\newacronym{fr2}{FR2}{Frequency Range 2}
\newacronym{bs}{BS}{Base Station}
\newacronym{scs}{SCS}{Subcarrier Spacing}
\newacronym{sumo}{SUMO}{Simulation of Urban MObility}
\newacronym{prr}{PRR}{Packet Reception Ratio}
\newacronym{edca}{EDCA}{Enhanced Distribution Channel Access}
\newacronym{thz}{THz}{terahertz}
\newacronym{6g}{6G}{6th generation}
\newacronym{uav}{UAV}{unmanned aerial vehicles}
\newacronym{quic}{QUIC}{Quick UDP Internet Connections}
\newacronym{cvar}{CVaR}{Conditional Value at Risk}
\newacronym{bbu}{BBU}{Baseband Unit}
\newacronym{ci}{CI}{Close-in free space reference}
\newacronym{rrh}{RRH}{Remote Radio Head}
\newacronym{mno}{MNO}{Mobile Network Operator}
\newacronym{mbs}{MBS}{Macro Base Station}
\newacronym{sbs}{SBS}{Small Base Station}
\newacronym{ssb}{SSB}{Synchronization Signal Block}
\newacronym{nsf}{NSF}{National Science Foundation}
\newacronym{zf}{ZF}{Zero-Forcing}
\newacronym{comp}{CoMP}{Coordinated Multi-Point}
\newacronym{cran}{C-RAN}{Cloud \acrlong{ran}}
\newacronym{cco}{CC}{Carrier Component}
\newacronym{srb}{SRB}{Service Radio Bearer}
\newacronym{sctp}{SCTP}{Stream Control Transmission Protocol}
\newacronym{os}{OS}{Operating System}
\newacronym{tls}{TLS}{Transport Layer Security}
\newacronym{rfc}{RFC}{Request for Comments}
\newacronym{http}{HTTP}{HyperText Transfer Protocol}
\newacronym{nat}{NAT}{Network Address Translation}
\newacronym{api}{API}{Application Programming Interface}
\newacronym{rto}{RTO}{Retransmission Timeout}
\newacronym{psc}{PSC}{Public Safety Communication}
\newacronym{rpgm}{RPGM}{Reference Point Group Mobility}
\newacronym{ic}{IC}{Incident Command}
\newacronym{rsu}{RSU}{Road Side Unit}
\newacronym{usa}{U.S.}{United States}
\newacronym{cots}{COTS}{Commercial Off-the-Shelf}
\newacronym{fpga}{FPGA}{Field Programmable Gate Array}
\newacronym{rcn}{RCN}{Research Coordination Network}
\newacronym{abg}{ABG}{Alpha-Beta-Gamma}
\newacronym{fi}{FI}{Floating Intercept}
\newacronym{uas}{UAS}{Unmanned Aerial System}
\newacronym{gps}{GPS}{Global Positioning System}
\newacronym{a2g}{A2G}{air-to-ground}
\newacronym{a2a}{A2A}{air-to-air}
\newacronym{inoo}{InOo}{Indoor Open Office}
\newacronym{ple}{PLE}{path loss exponent}
\newacronym{toa}{ToA}{Time of Arrival}
\newacronym{tc}{TC}{Time Cluster}
\newacronym{ns3}{ns-3}{Network Simulator 3}
\newacronym{fsc}{FS}{Fully Stochastic}
\newacronym{hbc}{HB}{Hybrid}
\newacronym{hpbw}{HPBW}{Half Power Beamwidth}
\newacronym{hsc}{HSC}{Hybrid Semantic Compression}
\newacronym{per}{PER}{Packet Error Rate}
\newacronym{ici}{ICI}{inter-cell interference}
\newacronym{psd}{PSD}{Power Spectrum Density}
\newacronym{scaros}{SCAROS}{Scalable and Robust Self-backhauling Solution}

\makeindex
\makeglossaries

%%%%%%%%%%%%%%%%%%%%%%%%%%%%%%%%%%%%%%%%%%%%%%%%%%%%%%%%%%%%%%%%%%%%%%%%
% Incipit
%%%%%%%%%%%%%%%%%%%%%%%%%%%%%%%%%%%%%%%%%%%%%%%%%%%%%%%%%%%%%%%%%%%%%%%%

\title{\textcolor{UNIPDRED}{Dissertation title}}
\author{Matteo Pagin}

\begin{document}

\frontmatter
  %\include{Sources/Title_Unipd}
  %\include{Sources/Title_UniPD_background_logo}
  %\include{Sources/Title_UniPD_background_logo_2}
  \begin{titlepage}
  \makeatletter

  %\begin{figure}
  %  \centering
  %  \includegraphics[width=0.4\textwidth]{unipd_logo_and_text.pdf}
  %\end{figure}
  %\vspace*{1cm}
  {
    \sffamily%
    \fontsize{20}{24}\selectfont\par\noindent\textcolor{black}{\textsc{\@author}}%
    \vspace{1cm}%
    \fontsize{34}{60}\selectfont\par\noindent{\MakeUppercase{\@title}}%
  }

  \vspace{4cm}
  {
    \sffamily%
    \fontsize{8}{15}\selectfont\par\noindent\textcolor{black}
    {\MakeUppercase{A thesis submitted in partial fulfillment of the requirements for the}}
    \vspace{0.1cm}%
    \fontsize{12}{15}\selectfont\par\noindent\textcolor{black}
    {\MakeUppercase{Ph.D. Degree in Information Engineering}}
    \vspace{0.1cm}%
    \fontsize{8}{15}\selectfont\par\noindent\textcolor{black}
    {\MakeUppercase{of the}}
    \vspace{0.1cm}
    \fontsize{12}{15}\selectfont\par\noindent\textcolor{black}
    {\MakeUppercase{Department of Information Engineering\\University of Padova}}%
    \vspace{0.1cm}%
    \fontsize{8}{15}\selectfont\par\noindent\textcolor{black}
    {\MakeUppercase{supervised by}}
    \vspace{0.1cm}%
    \fontsize{12}{15}\selectfont\par\noindent\textcolor{black}
    {\MakeUppercase{prof. Michele Zorzi}}%
  }

  \vspace*{\fill}
  {
    \sffamily%
    \fontsize{8}{10}\selectfont\par\noindent{\MakeUppercase{Academic year 2024}}%
    \fontsize{8}{10}\selectfont\par\noindent{\MakeUppercase{Graduation date XXXX}}%
  }
  \makeatother
\end{titlepage}

\newpage
\null
\thispagestyle{empty}
\newpage

  %{\noindent\huge\itshape Abstract}\\
%

Cellular networks are constantly evolving in order to support the ever-increasing number of mobile users, and the corresponding growth in wireless data traffic, coupled with the emergence of new applications. 
Specifically, the last iteration of mobile networks, i.e., 5G, brought high peak performance and extreme flexibility, making it possible to support a diverse set of applications with heterogeneous yet stringent requirements. 
One of 5G main novelties is represented by the support for \gls{mmwave} frequencies, which unlocks an unprecedented amount of previously unused radio resources. In turn, the latter enables extremely high data rates and low latencies. Moreover, it is envisioned that the upcoming generation, i.e., 6G, will unleash additional bandwidth, by further expanding the supported spectrum bands to include \gls{thz} frequencies as well. 
However, despite their theoretical potential, \gls{mmwave} and \gls{thz} frequencies exhibit harsh propagation conditions which make it challenging to provide ubiquitous high speed wireless connectivity. 
To fill this gap, this thesis studies innovative deployment solutions to overcome the unfavorable propagation characteristics of mmWave and THz communications, paving the way for their widespread use in the context of 6G cellular networks. 

In particular, this thesis
\begin{enumerate*}[label=(\roman*)]
    \item presents novel simulation tools, which model innovative coverage enhancement technologies such as \glspl{irs}, \gls{af} relays, and \glspl{ntn};
    \item presents novel simulation models which improve the computational complexity of \gls{mimo} simulations;
    \item introduces schemes for optimizing \gls{iab} networks;
    \item analyzes the potential of mixed \gls{mmwave} and \gls{thz} links for wireless backhauling; and
    \item analyzes the impact of non-ideal control channels in \gls{irs}-aided deployments, and introduces algorithms for mitigating the corresponding perofrmance degradation
\end{enumerate*}.

This thesis adopts a system-level approach, thus characterizing the network behavior in an end-to-end fashion, and capturing the interplay between the physical signal propagation and the different layers of the communications protocol stack. Results demonstrate the effectiveness of the proposed solutions, which pave the way towards ubiquitous high-performance mobile networks.


  \newpage\phantom{justskipthepage}

{
  % Custom link color for TOC
  \hypersetup{linkcolor=black}
  \tableofcontents
}

\mainmatter
  %%%%%%%%%%%%%%%%%%%%%%%%%%%%%%%%%%%%%%%%%%%%%%%%%%%%%%%%%%%%%%%%%%%%%%%%
\chapter{Introduction}
%%%%%%%%%%%%%%%%%%%%%%%%%%%%%%%%%%%%%%%%%%%%%%%%%%%%%%%%%%%%%%%%%%%%%%%%

The significance of mobile networks in modern society is underscored by their paramount role in facilitating social, professional, and educational interactions. The recent COVID-19 pandemic has served as a bitter reminder of this critical importance, identifying the absence of Internet connectivity as a significant hindrance to our daily lives.
However, the current generation of cellular networks, i.e., \gls{5g}, are found wanting in providing adequate broadband coverage to rural regions~\cite{yaacoub2020key}. Furthermore, even in technologically advanced nations, cellular infrastructures often fall short of meeting the stringent reliability, availability, and responsiveness requirements of emerging wireless applications. The vulnerability of mobile networks to natural disasters and cyberattacks underscores the need for solutions which offer reliability from both a technological and a sociopolitical standpoint.
Indeed, recent geopolitical turmoils have expanded the scope of conflict to include the cyberspace, emphasizing the imperative of maintaining uninterrupted connectivity in emergency situations. In these contexts, connectivity outages can compromise the delivery of critical services, inflict significant economic damage, and even result in loss of lives~\cite{internet_ukr_afg}.

In response to the pressing need for reliable wireless connectivity, the \gls{itu} foresees a future where ubiquitous broadband coverage will be achieved by 2030. This vision is driven by the imperative to provide seamless connectivity to both humans and an increasingly vast array of intelligent devices, including wearables, autonomous vehicles, \glspl{uas}, and robots~\cite{mozaffari2018beyond}.
The advent of novel use cases such as holographic communications, \gls{xr}, and tactile applications will further exacerbate the requirements for peak throughput and latency identified for the 5G's \gls{embb} and \gls{urllc} use cases. 
To achieve these ambitions goals, future cellular systems will continue to evolve and enhance the 5G network paradigm, which has revolutionized the wireless landscape by introducing flexible virtualized architectures, the support for \gls{mmwave} communications, and the adoption of \gls{mimo} technologies~\cite{ghosh20195g}. 
Notably, both academic and industry researchers are exploring a more central role for \gls{mmwave} technology, with plans to allocate additional spectrum towards the \gls{thz} band, and the design of \gls{ai}-native networks, with the goal of achieving autonomous data-centric orchestration and management of the network~\cite{polese20216g}, possibly down to the air interface~\cite{hoydis2021toward}.

The THz and mmWave frequency bands hold significant promise as a means to achieve peak data rates exceeding Tb/s, as envisioned by the ITU~\cite{imt2030}. However, this portion of the spectrum is hampered by unfavorable propagation characteristics that make it challenging to realize its full potential.
Specifically, the THz and mmWave bands are plagued by a pronounced free-space propagation loss, which significantly attenuates signal strength over long distances. Furthermore, these frequencies are also susceptible to blockages, for instance due to buildings, foliage, and other obstacles, which can cause significant signal attenuation~\cite{han2018propagation, jornet2011channel}.
With 5G, preliminary support for \gls{mmwave} bands has been introduced thanks to a major redesign not only of the physical layer, but of the whole cellular protocol stack~\cite{shafi2018microwave}. For instance, the intrinsic directionality of the communication requires ad hoc control procedures~\cite{heng2021six}, while the frequent transitions between \gls{los} and \gls{nlos} conditions call for an ad hoc transport layer design, such as novel \gls{tcp} algorithms~\cite{zhang2019will}. 
Albeit the harsh propagation environment can be partially mitigated by using directional links and densifying network deployments~\cite{polese2020toward}, these unfavorable propagation characteristics pose a significant challenge to the successful deployment of THz and mmWave-based 6G wireless systems. Traditional network densification is costly for network operators,
especially in terms of sites acquisition campaigns, rental fees, and fiber optic layout to provide wired backhauling~\cite{lopez2015towards}.
To solve this issue, the 3GPP approved, as part of its 5G NR specifications for Rel-16~\cite{3gpp_38_874}, \gls{iab} as a new paradigm to replace fiber-like infrastructures with self-configuring relays operating through wireless backhaul links. 
However, while lower than that of wired backhaul deployments, the CAPEX costs of \gls{iab} installation may still prove prohibitive for \glspl{mno}~\cite{chaoub20216g}.
In light of this, new technologies based on \glspl{irs} and \gls{af} relays are also emerging as promising alternatives to overcome the coverage issues of mmWave networks with energy and cost efficiency in mind.
An \gls{irs} is a meta-surface whose elements can be programmed to manipulate electromagnetic fields in favor of a specific destination. By passively beamforming incoming signals without amplification, \glspl{irs} can meet the minimum capacity requirements in dead spots with reduced power consumption compared to approaches such as IAB~\cite{bjornson2019intelligent}. In contrast, \gls{af} relays are designed to capture incident electromagnetic waves from a base station, amplify the received signal, and re-radiate it towards a targeted area to be served.
While IRSs offer advantages in terms of lower power consumption, AF relays have the potential to achieve higher capacity by actively amplifying signals. However, similarly to IAB, this comes at the cost of increased complexity, higher system costs, and potential amplification noise issues~\cite{huang2019reconfigurable}.
Nevertheless, to truly achieve ubiquitous connectivity \gls{ntn} integration within the depicted 6G wireless ecosystem will be essential, owing to its capability to deliver services anywhere and anytime. In fact, 3GPP foresee the support for the transparent integration of satellite gateways in the \gls{ran}, thus providing coverage to handheld devices in areas that are unreachable by conventional terrestrial deployment.


TODO: talk about open, virtual and geopolitical impact.
In addition, as the network progressively becomes increasingly complex and heterogeneous, the push for spectrum expansion will be coupled with an \gls{ai}-native design which, thanks to the ongoing virtualization, will not be limited to the radio link level, but will encompass the orchestration of large scale deployments as well~\cite{polese2023understanding}.
Nevertheless, how to design, test and eventually deploy management and orchestration algorithms is an open research challenge~\cite{polese2022colo}.
First, the training data must accurately capture the interplay of the whole protocol stack with the wireless channel. Furthermore, optimization frameworks such as \gls{drl} also call for preliminary testing in isolated yet realistic environments, with the goal of minimizing the performance degradation to actual network deployments~\cite{lacava2022programmable, amir2023safehaul}.

For the next generation of cellular networks to fully achieve the ubiquitous goal, the above innovative deployment solutions will need to be properly studied and optimized, characterizing their impact on the end-to-end systems, i.e., from the signal propagation to the \gls{qoe} perceived by the end users.
However, accurately testing new solutions on real-word testbeds is typically impractical, especially at scale. 
Fortunately, thanks to increasingly precise simulation tools (such as Network
Simulator 2 (ns-2) [13], Network Simulator 3 (ns-3) [14], OMNeT++ [15], etc.), the hardware
availability might turn fundamental only in the final steps of the study, to assess the best so-
lutions chosen from those studied through extensive simulations. Among the main features of
ns-3, one of the main research tools used to obtain the results discussed in this thesis, its mod-
ularity allows researchers to create solutions completely from scratch, as well as add features
on modules openly available and that have been thoroughly used and tested by the community.
An example consists of ns3-mmwave [16], a 5G-NR-compliant and openly available module [17],
that allows the simulation of next-generation networks operating at mmWave.
It has also to be acknowledged that, in the recent years, there has been an effort from gov-
ernment agencies all over the world to create fundings opportunities for research institutions
to support the creation of public platforms to carry out advanced research in wireless com-
munications, to fill the gap described above. An example consists in the US-based Plaftorms
for Advanced Wireless Research (PAWR) initiative, a $100$ million partnership funded by the
National Science Foundation (NSF) and a consortium of 30 companies and associations, for the
creation of 4 city-scale research testbeds [18]. The European Union, instead, to combat the
negative impact of COVID-19, agreed to invest together €750 billion (in 2018 prices) to build a
greener, more digital and more resilient Europe. Part of these funds has already been allocated
towards research and innovation, which will use it, among other things, to sustain the costs of
top-notch instrumentation and open hardware facilities [19].

% To sum up, in this dissertation, besides giving a detailed overview of the aforementioned
% problems, we contribute to the state of the art with improvements that span from the creation
% of novel ns-3 modules, to the design of artificial intelligence algorithms to proactively assist
% teleoperated driving operations, to the setup of a reproducible workflow for carrying out end-
% to-end measurements on a public platform for wireless research. Even though we touched on
% very diverse topics, from indoor network operations for XR applications to intelligent vehicular
% networks, they were all connected by the interest in studying and evaluating different technolo-
% gies that make up the next generation of mobile networks. Where possible, after assessing their
% peculiarities and the problems that might arise for specific use cases, we tried to propose new
% ad hoc solutions, or optimization to existing ones. It is also important to mention that the
% presented work is the fruit of the collaboration with several researchers from world-leading uni-
% versities, industries and research institutions, without whom reaching such relevant outcomes
% would have not been possible. To properly acknowledge this collaborative effort from all the
% authors, besides including at the end of this thesis a list of all the publications from these
% 3-years of Ph.D. (accepted or in the process of review), at the beginning of each chapter we also
% list the papers to refer to for the original work.
% The topics of this thesis are organized as follows
  
% This ensures that the subsequent sections are being included as root
% items in the bookmark structure of your PDF reader.
\bookmarksetup{startatroot}
\backmatter

  \begingroup
    \let\clearpage\relax
    \printglossary[type=\acronymtype]
    \newpage
    \printglossary
  \endgroup

  \printindex
  \printbibliography

\end{document}
