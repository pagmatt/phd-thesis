\section{Improving the scalability of wireless channel simulation in ns-3}
\label{sec:ch-perf-improv}

% Channel modeling is a fundamental task for the design, simulation, and evaluation of current and future wireless networks. It is especially relevant to perform system-level simulations to test new algorithms, procedures, and architectures, before going into real deployment/device implementations. In the recent decades, the challenges for understanding the propagation at \gls{mmwave} and \gls{thz} frequencies with large antenna arrays and the use of \gls{mimo} have further motivated the channel modeling efforts in those frequency ranges~\cite{hemadeh2018millimeter, 9444237}. As a result, multiple channel measurement campaigns have been performed by the academic and industry communities~\cite{rappaport2013millimeter}, 
% leading to different families of channel models. The various channel models differ in their degree of simplicity and accuracy. 
Channel models range from simple models that just consider a propagation loss component combined with Nakagami-m or Rayleigh fading but fail to capture the spatial dimension of the channel and the interactions with beamforming~\cite{andrews2017modeling}, to deterministic models that are very accurate in specific scenarios but are much more complex and require a precise characterization of the environment~\cite{lecci2020simplified}.
To address the complexity-accuracy trade-off, the \gls{3gpp} has adopted a stochastic channel model for simulations of 5G and beyond networks~\cite{TR38901}. Stochastic channel models are generic, thanks to their stochastic nature, and can model interactions with multiple-antenna arrays.
%Specifically, the \gls{3gpp} defined in TR 38.901 the spatial channel model for simulations that address frequency ranges from 0.5 GHz to 100 GHz~\cite{TR38901}, which is parameterized for various simulation scenarios, including indoor office, indoor factory, urban macro, urban micro, and rural macro. 
% However, for system-level simulations of large-scale systems including multiple nodes and large antenna arrays, the \gls{3gpp} spatial channel model introduces a significant overhead in terms of simulation time. 
%In this line, in~\cite{8445856}, a simplified channel model for the system-level simulations of \gls{mimo} wireless networks is proposed. 
% Therein, the end-to-end channel gain is obtained as the sum of several loss and gain terms that account for large-scale phenomena such as path loss and shadowing, small-scale fading, and antenna and beamforming gains. Notably, the latter terms represent a fundamental component for studies concerning modern wireless systems. In particular, an accurate characterization of the antenna radiation pattern and of the effect of the presence of multiple radiating elements becomes extremely important when studying \gls{mmwave} and \gls{thz} frequencies. Following the model in~\cite{8445856}, the combined antenna and beamforming gain can be computed according to~\cite{8422746}, the path loss and shadowing components can follow the \gls{3gpp} model in~\cite{TR38901}, and the small-scale fading can be sampled from various statistical distributions. 
% For the small-scale fading, authors in~\cite{8445856} propose to use a Nakagami-$m$ distribution, which has been shown to provide a good fit with the 3GPP model, provided that the $m$ parameter is appropriately chosen.
% Another option for small-scale fading modeling is the so-called \gls{ftr} fading model presented in~\cite{7917287}, which models more accurately the fading that occurs at mmWaves.
The latter was included in ns-3 thanks to the efforts of Tommaso Zugno in the 2019 Google Summer of Code~\cite{tommaso:20}, and later extended to address vehicular scenarios in~\cite{10.1145/3460797.3460801} and industrial scenarios in~\cite{10.1145/3532577.3532596}. 
As a consequence, the current spatial channel model implemented in ns-3 is very accurate for simulations in line with \gls{3gpp} specifications for a wide range of frequencies. However, it represents the main bottleneck in terms of computational complexity when considering large-scale simulations with many multi-antenna nodes, especially when equipped with large antenna arrays. This is because of the intrinsic complexity in the generation of the channel model according to 3GPP specifications, and the need to deal with inefficient tensor structures. In fact, the channel matrix in the ns-3 implementation of the \gls{3gpp} spatial channel model is currently implemented as a 3D structure made of nested vectors, whose dimensions depend on the number of the transmit antennas, receive antennas, and clusters. %Currently, in ns-3, the 3GPP channel model uses a vector of vectors of vectors to represent 3D arrays, such as the channel matrix. 

The design of computationally efficient yet accurate channel models has been a topic of interest also in the \gls{wlan} space. The authors of~\cite{jin2020efficient, jin2021efficient} present a frequency-selective channel for \glspl{wlan}, and use \gls{eesm} \gls{l2sm} to integrate their model with the ns-3 system-level Wi-Fi implementation. Moreover, they develop a framework which leverages cached statistical channel matrix realizations to directly estimate the effective \gls{snr}, thus further improving the computational efficiency of the model. Specifically, the latter is modeled as a parameterized log-SGN random variable. They extend their work in~\cite{jin2021eesm}, by accounting for the channel correlation over time. 
Moreover,~\cite{liu2021performance} compares statistical channel models for the 60~GHz band with the \gls{qd} \gls{rt} of~\cite{QD}.

In the remainder of this section, we summarize the efforts carried out in the 2022 Google Summer of Code to further optimize the code in ns-3 in two directions: 1) improving the efficiency of the code by allowing the use of the \texttt{Eigen} library, and 2) proposing a new performance-oriented MIMO channel model for reduced complexity in ns-3 large-scale simulations. 

% First, we have improved the efficiency of the \gls{3gpp} spatial channel model in ns-3 by allowing the usage of \texttt{Eigen} to represent matrices, so that when \texttt{Eigen} is available the \gls{3gpp} channel matrix is represented as an \texttt{std::vector} of \texttt{Eigen} matrices. This already improves the performance of current models. Second, we propose an alternative model, based on the FTR channel model~\cite{7917287}, in which the channel is characterized by a single scalar instead of 3D matrices, and we have calibrated such model to align with the \gls{3gpp} TR 38.901 spatial channel model for various scenarios and channel conditions. This model is especially useful to speed up ns-3 large-scale simulations, when simplicity is prioritized.

\subsection{Efficient MIMO modeling with the Eigen library}
\label{sec:opt_code}

The use of multiple antennas both at the transmitter and at the receiver, a fundamental feature of modern wireless systems, makes a scalar representation of the channel impulse response insufficient. Instead, \gls{mimo} channels are usually represented in the form of a complex matrix $\bm{H} \in \mathbb{C}^{U \times S}$, whose elements depict the channel impulse response between the $U$ and $S$ radiating elements of the transmitting and receiving antenna arrays, respectively~\cite{TR38901}. This peculiarity significantly increases the computational complexity of \gls{mimo} channel models, compared to \gls{siso} ones, since the complex gain of the channel must be evaluated for each pair of transmit and receive antennas.
Notably, previous analyses identified in statistical channel models the main bottleneck for system-level \gls{mimo} wireless simulations. In typical \gls{m-mimo} \gls{5g} scenarios, where the devices feature a high number of antennas, the channel matrix generation and the computation of the beamforming gain represent up to 90\% of the simulation time ~\cite{testolina2020scalable}. 

In light of these limitations, as the first of our contributions, we optimized the implementation of the 3GPP TR 38.901 model in ns-3 introduced in~\cite{tommaso:20}. 
First, we observed that, as of ns-3.37, part of the trigonometric operations of the \texttt{GetNewChannel} method of the \texttt{Three\-Gpp\-Channel\-Model} class are unnecessarily repeated for each pair of transmitting and receiving radiating elements. This represents a significant inefficiency, since the inputs of these functions, i.e., the angular parameters of the propagation clusters, depend on the cluster index only. Moreover, the standard library \texttt{sin} and \texttt{cos} functions are particularly demanding to evaluate. Therefore, we cached the trigonometric evaluations of these terms prior to the computation of $\bm{H}$'s coefficients, effectively reducing the complexity of the trigonometric operations from $\mathcal{O}(U \times S \times N)$ to $\mathcal{O}(N)$, where $N$ is the number of propagation clusters. 

Then, we focused on improving the algebra manipulations of the channel matrix performed in the \texttt{Three\-Gpp\-Spectrum\-Propagation\-Loss\-Model} by introducing the support for the open-source library \texttt{Eigen} in ns-3. \texttt{Eigen} is a linear algebra C\texttt{++} template library that offers fast routines for algebra primitives such as matrix multiplication, decomposition and space transformation~\cite{eigenweb}, and is used by many open-source frameworks such as TensorFlow. 

We set \texttt{Eigen} as an optional, external ns-3 dependency, with the goal of minimizing future code maintenance efforts, %. The library has been integrated in ns-3 by
and thus mimicking the support for other third-party libraries. To get \texttt{Eigen}, ns-3 users can either rely on packet managers, i.e., install the package \texttt{libeigen3-dev} (\texttt{eigen}) for Linux (Mac) systems, or manually install the library by following the official instructions\footnote{\url{https://gitlab.com/libeigen/eigen/-/blob/master/INSTALL}}. Then, \texttt{Eigen} can be enabled via a custom flag defined in the \texttt{macros-\-and-\-definitions.cmake} file, and its presence in the system is shown to the user by exposing whether it has been found or not via the \texttt{ns3-\-config-\-table.cmake} file. The latter also defines the preprocessor definition \texttt{HAVE\_EIGEN3}, which is used in the ns-3 source files to discern \texttt{Eigen}'s availability. Finally, the linking of \texttt{Eigen} with the ns-3 source files is taken care of by the \texttt{CMake} configuration file provided by the library itself, as suggested in the related ns-3 guide.

To prevent the need for \texttt{Eigen} to be installed in the host system, we developed a common set of APIs between the \texttt{Eigen}- and the \gls{stl}-based data structures and primitives. Thanks to this choice, the remainder of the \texttt{spectrum} code is completely abstracted with respect to the presence of the library.
Given that most of the needed operators can not be overloaded for \gls{stl} C\texttt{++} vectors (for instance, \texttt{operator()}), the common interface for both Eigen and \gls{stl}’s based vectors and matrices has been implemented by defining ad hoc structs with custom operators. In particular, we defined:

\begin{itemize}
    \item The complex vector type \texttt{PhasedArrayModel::\-Complex\-Vec\-tor}. This data-structure is defined as an \texttt{std::\-vector} of \texttt{std::\-complex<double>} whenever \texttt{Eigen} is not installed, and as an \texttt{Eigen} vector of \texttt{std::\-complex<double>} otherwise. The set of APIs includes operators \texttt{[]} and \texttt{!=}, which can be used to access the vector entries and to compare pairs of vectors, respectively. Additionally, we defined the \gls{stl}-like methods \texttt{size}, \texttt{norm} and \texttt{resize}, which return the vector size, its $\mathcal{L}^2$-norm, and allow the user to resize the underlying container, respectively. These definitions follow the typical \gls{stl} notation, as it is supported by \texttt{Eigen} as well.
    
    \item The complex matrix type \texttt{MatrixBasedChannelModel::\-Com\-plex\-2DVector}. In this case, the underlying type is a nested \texttt{std::vector} of \texttt{std::complex\-<double>} for when \texttt{Eigen} is disabled, and an \texttt{Eigen} matrix whose entries are of type \texttt{std::\-complex\-<double>} otherwise. 
    
    In this case, we aligned the notation to the APIs provided by \texttt{Eigen}. Specifically, the matrix elements can be accessed via the operator \texttt{()}, which takes as arguments the row and column indices of the entry, while the method \texttt{resize} allows users to resize matrices by specifying the number of rows and columns. In turn, these can be accessed via the \texttt{rows} and \texttt{columns} methods, respectively.

    \item The 3D matrix \texttt{Matrix\-Based\-Channel\-Model::\-Complex\-3D\-Vec\-tor}. This data structure is defined, regardless of \texttt{Eigen}'s availability, as an \texttt{std::\-vector} of \texttt{Matrix\-Based\-Channel\-Model::\-Complex2DVector}. In this case, the only method provided is \texttt{Multiply\-MatBy\-Left\-And\-RightVec}, which computes a product of the type $\bm{w}_T \bm{H} \bm{w}_R^T$, where $\bm{H} \in \mathbb{C}^{U \times S}$, $\bm{w}_T \in \mathbb{C}^{1 \times U}$ and $\bm{w}_R \in \mathbb{C}^{1 \times S}$. Notably, this computationally demanding evaluation, which is required for computing the beamforming gain in \texttt{Three\-Gpp\-Spectrum\-Propagation\-Loss\-Model}, leverages \texttt{Eigen}'s optimized algorithms whenever the library is installed in the host system.
\end{itemize}

Finally, we remark that the support for \texttt{Eigen} in the ns-3 codebase can possibly be further extended to improve the efficiency of other linear algebra operations, such as the \gls{svd} which is used in the \texttt{mmwave} and \texttt{nr} modules to compute optimal beamformers, and the matrix-by-matrix multiplications needed for relayed channels \cite{9810370}.

\subsection{A performance-oriented MIMO statistical channel model}
\label{sec:opt_design}

The second approach to reduce computational complexity we propose in this section is a 
% Additionally, we designed and implemented a 
\gls{mimo} channel model for simulating large \gls{m-mimo} scenarios, implemented in the class \texttt{Two\-Ray\-Spectrum\-Propagation\-Loss\-Model}. The goal of this auxiliary model is to offer a faster, albeit slightly less accurate, statistical channel model than the 3GPP TR 38.901 framework of \cite{tommaso:20} by preventing the need for the computation of the complete channel matrix. In line with~\cite{TR38901}, the frequency range of applicability of this model is $0.5 - 100$ GHz, although the framework can be possibly extended to support higher frequencies as well.

The overall channel model design follows the approach of \cite{8445856}, i.e., the end-to-end channel gain is computed by combining several loss and gain terms which account for both large- and small-scale propagation phenomena, and the antenna and beamforming gains.
In particular, let $T$ be a device transmitting a signal $x$ with power $\mathrm{P}_T^x$, and $R$ be another device in the simulation (which may or may not be the intended destination of
$x$). 
The proposed model implements the \texttt{Phased\-Array\-Spectrum\-Propagation\-Loss\-Model} interface by estimating $\mathrm{P}_R^x$, i.e., the power of $x$ received at $R$, as follows:
\begin{align}
\mathrm{P}_R^x[d B m] =& \,\, \mathrm{P}_T^x[d B m] - \mathrm{PL}_{T, R}[d B] \\
        &+ \mathrm{S}_{T, R}[d B] + G_{T, R}[d B] + F_{T, R}[d B], \nonumber
\end{align}
where the terms $\mathrm{PL}_{T, R}$ and $\mathrm{S}_{T, R}$ represent the path loss and the shadowing, respectively, while $G_{T, R}$ and $F_{T, R}$ denote the antenna and beamforming gain and the small-scale fading, respectively. The remainder of this section describes in detail how each of these terms is computed.

\subsection{Path loss, Shadowing, and LoS Condition}
The large-scale propagation phenomena are modeled according to the 3GPP TR 38.901 model~\cite{TR38901}, since its implementation of~\cite{tommaso:20} is not computationally demanding. Nevertheless, the channel model can in principle be coupled with arbitrary classes extending the \texttt{Channel\-Condition\-Model} interface.

Specifically, we first determine the 3GPP scenario. Then, for each link we set the \gls{los} condition in a stochastic manner, using the class extending \texttt{Three\-Gpp\-Channel\-Condition\-Model} which corresponds to the chosen scenario.

Then, we compute the path loss using the 3GPP TR 38.901 formula
\begin{equation}
    PL_{T, R} = A \log_{10} (d) + B + C \log_{10} (f_C) [dB],
\end{equation}
where $d$ is the 3D distance between the transmitter and the receiver, $f_C$ is the carrier frequency, and $A, B$ and $C$ are model parameters which depend on the specific scenario and the \gls{los} condition.

To account for the presence of blockages, an optional log-normal shadowing component $S_{T, R}$ and an outdoor-to-indoor penetration loss term are added to $PL_{T, R}$.

\subsubsection{Antenna and Beamforming Gain}

%\begin{equation}
 % \operatorname{AF}_{\mathrm{v}}(\theta, \varphi)=\frac{1}{\sqrt{N_{\mathrm{v}}}}
 % \sum_{m=0}^{N_{\mathrm{v}}-1} e^{j k d_{\mathrm{v}} m\left(\cos \theta-\cos \theta_0\right)} \\
%\end{equation}
%and
%\begin{equation}
%  \operatorname{AF}_{\mathrm{h}}(\theta, \varphi)=\frac{1}{\sqrt{N_{\mathrm{h}}}}
%  \sum_{n=0}^{N_{\mathrm{h}}-1} e^{j k d_{\mathrm{h}} n\left(\sin \theta
%  \sin \varphi-\sin \theta_0 \sin \varphi_0\right)}.
%\end{equation}


% computed as outlined in using the CalcBeamformingGain function. 
The combined array and beamforming gain is computed using the approach of~\cite{8422746}. 
%That is to say, 
The proposed model supports the presence of multiple antenna elements at the transmitter and at the receiver, and arbitrary analog beamforming vectors and antenna radiation patterns. Therefore, ns-3 users can use this model in conjunction with any class that implements the \texttt{AntennaModel} interface.
In this implementation, we focus on \glspl{upa}, although the methodology is general and can be applied to arbitrary antenna arrays.

Let $\theta$ and $\varphi$ be the relative zenith and azimuth angles between transmitter and receiver, respectively, and let $\bm{w}\left(\theta_0, \varphi_0\right)$ denote the beamforming vector pointing towards the steering direction $\left(\theta_0, \varphi_0\right)$. We denote with $U = U_h U_v $ the total, horizontal, and vertical number of antenna elements, respectively, and with $ d_h, d_v $ their spacing in the horizontal and vertical domains of the array, respectively. %Since we consider analog beamforming vectors only, $\bm{w}\left(\theta_0, \varphi_0\right)$

Considering first isotropic antennas, the gain pattern of a \gls{upa}, in terms of received power relative to a single radiating element, can be expressed as~\cite{ASPLUND202089}
\begin{equation}
  G_{T, R}^{iso}(\theta, \varphi) = \left| \bm{a_i}^{\mathrm{T}}(\theta, \varphi)  \bm{w}\left(\theta_0, \varphi_0\right) \right|^2,
\end{equation}
where $\bm{a_i}(\theta, \varphi)$ is the array response vector, whose generic entry $m,n$ with $m \in \{0, \ldots, U_v - 1 \}, n \in \{0, \ldots, U_h - 1 \}$ reads
\begin{align}
  a_i (\theta, \varphi)_{m, n} = & \exp \left( j\frac{2\pi}{\lambda}m d_v \cos(\theta) \right) \exp\left( j \frac{2\pi}{\lambda} n d_h \sin(\theta)\sin(\varphi) \right). \nonumber
\end{align} 

In this work, which supports arbitrary antennas, each antenna element $(m, n)$ actually exhibits a generic radiation pattern $g(\theta, \varphi)_{m, n}$ towards direction $(\theta, \varphi)$. In particular, we assume that $g(\theta, \varphi)_{m, n}$ is constant for all the elements of the array, i.e., $g(\theta, \varphi)_{m, n} \equiv g(\theta, \varphi)$. 
Accordingly, we compute $G_{T, R}(\theta, \varphi)$ in the \texttt{Compute\-Beamforming\-Gain} function of the \texttt{Two\-Ray\-Spectrum\-Propagation\-Loss\-Model} class as
\begin{equation}
  G_{T, R}(\theta, \varphi) =  G_{T, R}^{iso}(\theta, \varphi) \left| g(\theta, \varphi) \right|^2.
\end{equation}
%
Figures~\ref{fig:pattern_iso} and~\ref{fig:pattern_3gpp} report $G_{T, R} (\theta, \varphi)$ for both the isotropic (\texttt{Isotropic\-Antenna\-Model}) and the 3GPP (\texttt{ThreeGpp\-Antenna\-Model}) radiation patterns, respectively. 

 \begin{figure}
  \centering
  \begin{subfigure}[t]{\columnwidth}
    \centering 
    \setlength\fwidth{0.65\columnwidth}
    \setlength\fheight{0.28\columnwidth}
    % This file was created with tikzplotlib v0.10.1.
\begin{tikzpicture}

  \definecolor{plotColor1}{HTML}{e60049}
  \definecolor{plotColor2}{HTML}{0bb4ff}
  \definecolor{plotColor3}{HTML}{87bc45}
  \definecolor{plotColor4}{HTML}{ffa300}

 \begin{axis}[
    width=\fwidth,
    height=\fheight,
    at={(0\fwidth,0\fheight)},
    scale only axis,
    legend image post style={mark indices={}},
    %axis line style={white!80!black},
    legend style={
        /tikz/every even column/.append style={column sep=0.2cm},
        at={(0.5, 0.95)}, 
        anchor=south, 
        draw=white!80!black, 
        font=\scriptsize
        },
    legend columns=4,
    %tick align=outside,
    %x grid style={white!80!black},
    xlabel style={font=\footnotesize},
    xlabel={Relative azimuth $\varphi$ between transmitter and receiver [deg]},
    %xtick={0, 16, 64, 256},
    xmajorgrids,
    %xmajorticks=false,
    xmin=-90, xmax=90,
    xtick style={color=white!15!black},
    %y grid style={white!80!black},
    ylabel shift = -1 pt,
    ylabel style={font=\footnotesize},
    ylabel={$G_{\mathrm{AA}} (\theta, \varphi)$ [dB]},
    ymajorgrids,
    ymajorticks=true,
    ymin=-10, ymax=30,
    ytick style={color=white!15!black}
]
\addplot [very thick, plotColor4, mark=o]
table {%
-90 0
-60 0
-30 0
0 0
30 0
60 0
90 0
};
\addlegendentry{1x1}
\addplot [very thick, plotColor1, mark=square*, mark repeat = 8, mark options={solid}]
table {%
-64.0999984741211 -10.0547437667847
-62.9000015258789 -9.28906345367432
-61.5999984741211 -8.49977684020996
-60.2999992370605 -7.74889802932739
-58.9000015258789 -6.97978782653809
-57.5 -6.24846887588501
-56 -5.50364255905151
-54.5 -4.79591131210327
-52.9000015258789 -4.07895469665527
-51.2999992370605 -3.3984534740448
-49.5999984741211 -2.71270537376404
-47.9000015258789 -2.06292009353638
-46.0999984741211 -1.41170561313629
-44.2999992370605 -0.796162843704224
-42.4000015258789 -0.182977914810181
-40.5 0.394577503204346
-38.5999984741211 0.938207745552063
-36.5999984741211 1.47543382644653
-34.5999984741211 1.97817921638489
-32.5 2.47034955024719
-30.3999996185303 2.92718935012817
-28.2999992370605 3.34982228279114
-26.2000007629395 3.73915600776672
-24.1000003814697 4.09603452682495
-21.8999996185303 4.4358057975769
-19.7000007629395 4.74137544631958
-17.5 5.01330423355103
-15.3000001907349 5.25206804275513
-13.1000003814697 5.45807933807373
-10.8999996185303 5.63166522979736
-8.60000038146973 5.77875375747681
-6.30000019073486 5.89092445373535
-4 5.96835613250732
-1.70000004768372 6.01116561889648
0.600000023841858 6.01942729949951
2.90000009536743 5.99314165115356
5.19999980926514 5.93227338790894
7.40000009536743 5.84161567687988
9.69999980926514 5.71277809143066
12 5.54891157150269
14.1999998092651 5.35915374755859
16.3999996185303 5.13680934906006
18.6000003814697 4.88151502609253
20.7999992370605 4.59282398223877
23 4.27023315429688
25.2000007629395 3.91310977935791
27.3999996185303 3.52070546150208
29.5 3.11245131492615
31.6000003814697 2.67037892341614
33.5999984741211 2.21701073646545
35.5999984741211 1.73101735115051
37.5999984741211 1.21123027801514
39.5999984741211 0.656207919120789
41.5 0.0949338674545288
43.4000015258789 -0.501149892807007
45.2000007629395 -1.09960341453552
47 -1.73272299766541
48.7000007629395 -2.36436486244202
50.4000015258789 -3.03078603744507
52.0999984741211 -3.73429703712463
53.7000007629395 -4.43270635604858
55.2999992370605 -5.16892194747925
56.7999992370605 -5.89609146118164
58.2000007629395 -6.6096019744873
59.5999984741211 -7.35942840576172
61 -8.14863109588623
62.2999992370605 -8.91979885101318
63.5999984741211 -9.73119640350342
64.0999984741211 -10.0547437667847
};
\addlegendentry{2x2}
\addplot [very thick, plotColor2]
table {%
-71.6999969482422 -10.0770063400269
-70.8000030517578 -9.27555274963379
-69.9000015258789 -8.51647281646729
-68.9000015258789 -7.71907758712769
-67.9000015258789 -6.96678638458252
-66.9000015258789 -6.25673341751099
-65.9000015258789 -5.58645820617676
-64.9000015258789 -4.95393228530884
-63.7999992370605 -4.29974555969238
-62.7000007629395 -3.68738913536072
-61.5999984741211 -3.11541032791138
-60.5 -2.58272624015808
-59.4000015258789 -2.0885694026947
-58.2999992370605 -1.63246238231659
-57.2000007629395 -1.21419608592987
-56.0999984741211 -0.833855152130127
-55 -0.491772413253784
-54 -0.214520215988159
-53 0.0299065113067627
-52 0.240626573562622
-51 0.416452884674072
-50 0.556000113487244
-49.0999984741211 0.649182558059692
-48.2000007629395 0.710014820098877
-47.2999992370605 0.736743688583374
-46.4000015258789 0.727240085601807
-45.5999984741211 0.686347723007202
-44.7999992370605 0.612714767456055
-44 0.503874897956848
-43.2999992370605 0.377411484718323
-42.5999984741211 0.219455480575562
-41.9000015258789 0.0272747278213501
-41.2000007629395 -0.202440857887268
-40.5999984741211 -0.432084321975708
-40 -0.695007681846619
-39.4000015258789 -0.994657397270203
-38.7999992370605 -1.3351594209671
-38.2999992370605 -1.65370142459869
-37.7999992370605 -2.00756669044495
-37.2999992370605 -2.40094304084778
-36.7999992370605 -2.83890604972839
-36.2999992370605 -3.32770323753357
-35.9000015258789 -3.76052260398865
-35.5 -4.23575735092163
-35.0999984741211 -4.75945472717285
-34.7000007629395 -5.33915710449219
-34.2999992370605 -5.98440933227539
-33.9000015258789 -6.70756721496582
-33.5999984741211 -7.31071186065674
-33.2999992370605 -7.97571611404419
-33 -8.71419811248779
-32.7000007629395 -9.541335105896
-32.5 -10.1518659591675
nan nan
-27.7999992370605 -10.0768070220947
-27.5 -8.90235900878906
-27.2000007629395 -7.85529708862305
-26.8999996185303 -6.90991926193237
-26.5 -5.77625226974487
-26.1000003814697 -4.75928592681885
-25.7000007629395 -3.83669447898865
-25.2999992370605 -2.99211502075195
-24.8999996185303 -2.21320676803589
-24.3999996185303 -1.31759488582611
-23.8999996185303 -0.495480060577393
-23.3999996185303 0.264065980911255
-22.8999996185303 0.969621181488037
-22.2999992370605 1.75448024272919
-21.7000007629395 2.48059129714966
-21.1000003814697 3.15531635284424
-20.5 3.78459668159485
-19.7999992370605 4.46766996383667
-19.1000003814697 5.1016149520874
-18.3999996185303 5.69147920608521
-17.7000007629395 6.24141788482666
-16.8999996185303 6.82543087005615
-16.1000003814697 7.36604452133179
-15.3000001907349 7.86681175231934
-14.5 8.33067417144775
-13.6000003814697 8.81149482727051
-12.6999998092651 9.25171756744385
-11.8000001907349 9.65383720397949
-10.8999996185303 10.0199317932129
-10 10.3518152236938
-9.10000038146973 10.6509027481079
-8.10000038146973 10.9463882446289
-7.09999990463257 11.2044563293457
-6.09999990463257 11.426365852356
-5.09999990463257 11.6130208969116
-4.09999990463257 11.7653398513794
-3.09999990463257 11.8838558197021
-2.09999990463257 11.9691095352173
-1.10000002384186 12.0214490890503
-0.100000023841858 12.041036605835
0.899999976158142 12.0279884338379
1.89999997615814 11.9821977615356
2.90000009536743 11.9035692214966
3.90000009536743 11.7917251586914
4.90000009536743 11.6462345123291
5.90000009536743 11.4664888381958
6.90000009536743 11.2517213821411
7.90000009536743 11.0009469985962
8.89999961853027 10.7130746841431
9.80000019073486 10.421067237854
10.6999998092651 10.0965909957886
11.6000003814697 9.73824024200439
12.5 9.34429264068604
13.3999996185303 8.9127368927002
14.3000001907349 8.4411792755127
15.1000003814697 7.98613357543945
15.8999996185303 7.49485111236572
16.7000007629395 6.96449899673462
17.5 6.39170598983765
18.2000007629395 5.85251998901367
18.8999996185303 5.27445554733276
19.6000003814697 4.6535758972168
20.2999992370605 3.98511171340942
20.8999996185303 3.36987781524658
21.5 2.71088480949402
22.1000003814697 2.00264835357666
22.7000007629395 1.23828780651093
23.2000007629395 0.552368640899658
23.7000007629395 -0.184672474861145
24.2000007629395 -0.980602502822876
24.7000007629395 -1.8453117609024
25.1000003814697 -2.59510207176208
25.5 -3.40550661087036
25.8999996185303 -4.28732681274414
26.2999992370605 -5.25474882125854
26.7000007629395 -6.32679510116577
27 -7.21500205993652
27.2999992370605 -8.19191932678223
27.6000003814697 -9.27812576293945
27.7999992370605 -10.0768070220947
nan nan
32.5 -10.1518659591675
32.7999992370605 -9.25461483001709
33.0999984741211 -8.45901775360107
33.4000015258789 -7.74651956558228
33.7000007629395 -7.10326337814331
34.0999984741211 -6.33534097671509
34.5 -5.65290594100952
34.9000015258789 -5.0417685508728
35.2999992370605 -4.4911150932312
35.7000007629395 -3.99249196052551
36.0999984741211 -3.53914022445679
36.5999984741211 -3.02792453765869
37.0999984741211 -2.57045602798462
37.5999984741211 -2.15991425514221
38.0999984741211 -1.79079031944275
38.5999984741211 -1.45857381820679
39.2000007629395 -1.10338687896729
39.7999992370605 -0.790620923042297
40.4000015258789 -0.515870332717896
41 -0.275481343269348
41.7000007629395 -0.0343842506408691
42.4000015258789 0.168161392211914
43.0999984741211 0.335644721984863
43.7999992370605 0.470800757408142
44.5999984741211 0.588889241218567
45.4000015258789 0.671080708503723
46.2000007629395 0.719960927963257
47 0.737733125686646
47.9000015258789 0.722793102264404
48.7999992370605 0.673126816749573
49.7000007629395 0.59055757522583
50.5999984741211 0.476719737052917
51.5999984741211 0.315215706825256
52.5999984741211 0.118283271789551
53.5999984741211 -0.1127610206604
54.5999984741211 -0.376983880996704
55.5999984741211 -0.673588395118713
56.7000007629395 -1.03660106658936
57.7999992370605 -1.43764877319336
58.9000015258789 -1.87654936313629
60 -2.35336780548096
61.0999984741211 -2.86846446990967
62.2000007629395 -3.42246389389038
63.2999992370605 -4.01630878448486
64.4000015258789 -4.65128993988037
65.4000015258789 -5.26558303833008
66.4000015258789 -5.91675233840942
67.4000015258789 -6.60663843154907
68.4000015258789 -7.33747148513794
69.4000015258789 -8.11192417144775
70.3000030517578 -8.84882259368896
71.1999969482422 -9.62633228302002
71.6999969482422 -10.0770063400269
};
\addlegendentry{4x4}
\addplot [very thick, plotColor3, densely dotted]
table {%
-77.0999984741211 -10.1137104034424
-76.4000015258789 -9.23261260986328
-75.6999969482422 -8.40327548980713
-75 -7.62189912796021
-74.3000030517578 -6.88534355163574
-73.5 -6.09502935409546
-72.6999969482422 -5.3565502166748
-71.9000015258789 -4.66736459732056
-71.0999984741211 -4.02546167373657
-70.3000030517578 -3.42930746078491
-69.5 -2.87783122062683
-68.6999969482422 -2.37035751342773
-67.9000015258789 -1.90657150745392
-67.1999969482422 -1.53662264347076
-66.5 -1.2003983259201
-65.8000030517578 -0.898357391357422
-65.0999984741211 -0.631174564361572
-64.4000015258789 -0.399746894836426
-63.7000007629395 -0.205208301544189
-63 -0.0489813089370728
-62.4000015258789 0.0531352758407593
-61.7999992370605 0.124575734138489
-61.2000007629395 0.163814067840576
-60.5999984741211 0.169122219085693
-60 0.138437867164612
-59.4000015258789 0.0693217515945435
-58.9000015258789 -0.0196353197097778
-58.4000015258789 -0.139133810997009
-57.9000015258789 -0.291386961936951
-57.4000015258789 -0.478980422019958
-56.9000015258789 -0.704947352409363
-56.4000015258789 -0.972880125045776
-55.9000015258789 -1.28707349300385
-55.5 -1.57518696784973
-55.0999984741211 -1.89939618110657
-54.7000007629395 -2.26347374916077
-54.2999992370605 -2.67197060585022
-53.9000015258789 -3.1303858757019
-53.5 -3.64553451538086
-53.0999984741211 -4.22592067718506
-52.7000007629395 -4.88242244720459
-52.4000015258789 -5.43312168121338
-52.0999984741211 -6.04217720031738
-51.7999992370605 -6.71932458877563
-51.5 -7.47715091705322
-51.2000007629395 -8.33229541778564
-50.9000015258789 -9.30757522583008
-50.7000007629395 -10.0398941040039
nan nan
-46.7000007629395 -10.3717527389526
-46.4000015258789 -9.08484268188477
-46.0999984741211 -7.96844673156738
-45.7999992370605 -6.98514223098755
-45.5 -6.10910129547119
-45.2000007629395 -5.32172966003418
-44.9000015258789 -4.60922622680664
-44.5 -3.75780844688416
-44.0999984741211 -3.00226426124573
-43.7000007629395 -2.32875323295593
-43.2999992370605 -1.72686958312988
-42.9000015258789 -1.18861377239227
-42.5 -0.707793593406677
-42.0999984741211 -0.279532194137573
-41.7000007629395 0.099987268447876
-41.2999992370605 0.433740615844727
-40.9000015258789 0.724079966545105
-40.5 0.972781419754028
-40.0999984741211 1.18105924129486
-39.7000007629395 1.34983086585999
-39.2999992370605 1.47948253154755
-38.9000015258789 1.57018327713013
-38.5 1.62158000469208
-38.0999984741211 1.63310492038727
-37.7000007629395 1.60366475582123
-37.2999992370605 1.53180480003357
-36.9000015258789 1.41559505462646
-36.5999984741211 1.29783308506012
-36.2999992370605 1.15240955352783
-36 0.977812051773071
-35.7000007629395 0.772152423858643
-35.4000015258789 0.533320307731628
-35.0999984741211 0.25862717628479
-34.7999992370605 -0.0550247430801392
-34.5 -0.411385655403137
-34.2000007629395 -0.815052032470703
-33.9000015258789 -1.27168035507202
-33.5999984741211 -1.78831160068512
-33.2999992370605 -2.37385201454163
-33 -3.03981041908264
-32.7000007629395 -3.80135846138
-32.4000015258789 -4.67905855178833
-32.0999984741211 -5.70178318023682
-31.8999996185303 -6.48425912857056
-31.7000007629395 -7.3671669960022
-31.5 -8.37480354309082
-31.2999992370605 -9.54188251495361
-31.2000007629395 -10.2005662918091
nan nan
-28.8999996185303 -10.3718099594116
-28.7000007629395 -8.89373302459717
-28.5 -7.62760496139526
-28.2999992370605 -6.5212574005127
-28 -5.08835887908936
-27.7000007629395 -3.86225819587708
-27.3999996185303 -2.79450416564941
-27.1000003814697 -1.85263764858246
-26.7999992370605 -1.01389443874359
-26.5 -0.261704444885254
-26.2000007629395 0.416334509849548
-25.8999996185303 1.02968525886536
-25.5 1.75949370861053
-25.1000003814697 2.40072250366211
-24.7000007629395 2.96340441703796
-24.2999992370605 3.4550507068634
-23.8999996185303 3.88131403923035
-23.5 4.24648094177246
-23.1000003814697 4.55358076095581
-22.7999992370605 4.74712944030762
-22.5 4.9098048210144
-22.2000007629395 5.04196214675903
-21.8999996185303 5.1436710357666
-21.6000003814697 5.21480655670166
-21.2999992370605 5.25499582290649
-21 5.26358652114868
-20.7000007629395 5.23968744277954
-20.3999996185303 5.18209838867188
-20.1000003814697 5.08928155899048
-19.7999992370605 4.95927143096924
-19.5 4.78967428207397
-19.2000007629395 4.57748699188232
-18.8999996185303 4.31903886795044
-18.6000003814697 4.00974655151367
-18.2999992370605 3.64385867118835
-18 3.21414065361023
-17.7000007629395 2.7113037109375
-17.3999996185303 2.12320804595947
-17.1000003814697 1.43370842933655
-16.7999992370605 0.620627760887146
-16.6000003814697 -0.00514078140258789
-16.3999996185303 -0.712309598922729
-16.2000007629395 -1.51788091659546
-16 -2.44517707824707
-15.8000001907349 -3.52746152877808
-15.6000003814697 -4.81443929672241
-15.3999996185303 -6.38532829284668
-15.1999998092651 -8.37806129455566
-15.1000003814697 -9.60847759246826
-15 -11.0668716430664
nan nan
-14 -11.2703714370728
-13.8000001907349 -8.12642860412598
-13.6000003814697 -5.7777795791626
-13.3999996185303 -3.8958580493927
-13.1999998092651 -2.32179474830627
-13 -0.96657657623291
-12.6999998092651 0.770734310150146
-12.3999996185303 2.24967908859253
-12.1000003814697 3.53833556175232
-11.8000001907349 4.68035507202148
-11.5 5.7054762840271
-11.1999998092651 6.63487005233765
-10.8000001907349 7.75149345397949
-10.3999996185303 8.75226306915283
-10 9.65663051605225
-9.60000038146973 10.4790544509888
-9.19999980926514 11.230541229248
-8.80000019073486 11.9197607040405
-8.39999961853027 12.5534973144531
-7.90000009536743 13.2759799957275
-7.40000009536743 13.9290266036987
-6.90000009536743 14.5194597244263
-6.40000009536743 15.0527753829956
-5.90000009536743 15.5334405899048
-5.40000009536743 15.9651021957397
-4.90000009536743 16.3507843017578
-4.40000009536743 16.6929836273193
-3.90000009536743 16.9937801361084
-3.40000009536743 17.2548770904541
-2.90000009536743 17.4777050018311
-2.40000009536743 17.6634216308594
-1.89999997615814 17.8129577636719
-1.39999997615814 17.9270401000977
-0.899999976158142 18.0062084197998
-0.399999976158142 18.0508270263672
0.100000023841858 18.0611152648926
0.600000023841858 18.0371112823486
1.10000002384186 17.9787044525146
1.60000002384186 17.8856220245361
2.09999990463257 17.7574367523193
2.59999990463257 17.59352684021
3.09999990463257 17.3930892944336
3.59999990463257 17.1551036834717
4.09999990463257 16.8783111572266
4.59999990463257 16.5611839294434
5.09999990463257 16.2018604278564
5.59999990463257 15.7981100082397
6.09999990463257 15.3472394943237
6.59999990463257 14.846001625061
7.09999990463257 14.2904291152954
7.59999990463257 13.6756839752197
8.10000038146973 12.9957704544067
8.5 12.3999481201172
8.89999961853027 11.7528848648071
9.30000019073486 11.0488300323486
9.69999980926514 10.280463218689
10.1000003814697 9.43871784210205
10.5 8.51176357269287
10.8999996185303 7.48409748077393
11.1999998092651 6.63487005233765
11.5 5.7054762840271
11.8000001907349 4.68035507202148
12.1000003814697 3.53833556175232
12.3999996185303 2.24967908859253
12.6999998092651 0.770734310150146
12.8999996185303 -0.353034615516663
13.1000003814697 -1.62082183361053
13.3000001907349 -3.07697820663452
13.5 -4.79071283340454
13.6999998092651 -6.87924242019653
13.8999996185303 -9.56574821472168
14 -11.2703714370728
nan nan
15 -11.0668716430664
15.1999998092651 -8.37806129455566
15.3999996185303 -6.38532829284668
15.6000003814697 -4.81443929672241
15.8000001907349 -3.52746152877808
16 -2.44517707824707
16.2000007629395 -1.51788091659546
16.3999996185303 -0.712309598922729
16.6000003814697 -0.00514078140258789
16.8999996185303 0.90709400177002
17.2000007629395 1.67603814601898
17.5 2.32960557937622
17.7999992370605 2.88772201538086
18.1000003814697 3.36499762535095
18.3999996185303 3.77251601219177
18.7000007629395 4.11882209777832
19 4.4106011390686
19.2999992370605 4.65315914154053
19.6000003814697 4.8507661819458
19.8999996185303 5.00687837600708
20.2000007629395 5.12424230575562
20.5 5.20513677597046
20.7999992370605 5.25132942199707
21.1000003814697 5.26428461074829
21.3999996185303 5.24507761001587
21.7000007629395 5.19451284408569
22 5.11315059661865
22.2999992370605 5.00129413604736
22.6000003814697 4.85898780822754
22.8999996185303 4.68605756759644
23.2000007629395 4.48211479187012
23.5 4.24648094177246
23.7999992370605 3.97820115089417
24.2000007629395 3.56756091117859
24.6000003814697 3.09272360801697
25 2.54842829704285
25.2999992370605 2.09048295021057
25.6000003814697 1.58579587936401
25.8999996185303 1.02968525886536
26.2000007629395 0.416334509849548
26.5 -0.261704444885254
26.7999992370605 -1.01389443874359
27.1000003814697 -1.85263764858246
27.3999996185303 -2.79450416564941
27.7000007629395 -3.86225819587708
28 -5.08835887908936
28.2000007629395 -6.01669311523438
28.3999996185303 -7.05694723129272
28.6000003814697 -8.23795795440674
28.7999992370605 -9.60201358795166
28.8999996185303 -10.3718099594116
nan nan
31.2000007629395 -10.2005662918091
31.3999996185303 -8.93567752838135
31.6000003814697 -7.8535213470459
31.7999992370605 -6.91183853149414
32.0999984741211 -5.70178318023682
32.4000015258789 -4.67905855178833
32.7000007629395 -3.80135846138
33 -3.03981041908264
33.2999992370605 -2.37385201454163
33.5999984741211 -1.78831160068512
33.9000015258789 -1.27168035507202
34.2000007629395 -0.815052032470703
34.5 -0.411385655403137
34.7999992370605 -0.0550247430801392
35.0999984741211 0.25862717628479
35.4000015258789 0.533320307731628
35.7000007629395 0.772152423858643
36 0.977812051773071
36.2999992370605 1.15240955352783
36.5999984741211 1.29783308506012
37 1.4489164352417
37.4000015258789 1.55384624004364
37.7999992370605 1.61490786075592
38.2000007629395 1.63399934768677
38.5999984741211 1.61245179176331
39 1.55117321014404
39.4000015258789 1.45072042942047
39.7999992370605 1.31130850315094
40.2000007629395 1.13271343708038
40.5999984741211 0.914418458938599
41 0.655461072921753
41.4000015258789 0.354457497596741
41.7999992370605 0.00950062274932861
42.2000007629395 -0.381893992424011
42.5999984741211 -0.82289731502533
43 -1.31757724285126
43.4000015258789 -1.87109363079071
43.7999992370605 -2.49005889892578
44.2000007629395 -3.18298482894897
44.5999984741211 -3.96104645729065
44.9000015258789 -4.60922622680664
45.2000007629395 -5.32172966003418
45.5 -6.10910129547119
45.7999992370605 -6.98514223098755
46.0999984741211 -7.96844673156738
46.4000015258789 -9.08484268188477
46.7000007629395 -10.3717527389526
nan nan
50.7000007629395 -10.0398941040039
51 -8.96740341186523
51.2999992370605 -8.03517055511475
51.5999984741211 -7.21465873718262
51.9000015258789 -6.48534154891968
52.2000007629395 -5.83213090896606
52.5 -5.24347686767578
52.9000015258789 -4.54386138916016
53.2999992370605 -3.92696213722229
53.7000007629395 -3.38038802146912
54.0999984741211 -2.89454483985901
54.5 -2.46184921264648
54.9000015258789 -2.07618808746338
55.2999992370605 -1.73255479335785
55.7000007629395 -1.42682588100433
56.0999984741211 -1.15553569793701
56.5999984741211 -0.860415816307068
57.0999984741211 -0.609737277030945
57.5999984741211 -0.399518251419067
58.0999984741211 -0.22639524936676
58.5999984741211 -0.0875353813171387
59.0999984741211 0.0194994211196899
59.5999984741211 0.0968030691146851
60.2000007629395 0.152801036834717
60.7999992370605 0.171252012252808
61.4000015258789 0.154435873031616
62 0.104271769523621
62.5999984741211 0.0224384069442749
63.2000007629395 -0.0896141529083252
63.9000015258789 -0.256951451301575
64.5999984741211 -0.462156772613525
65.3000030517578 -0.70391047000885
66 -0.981130361557007
66.6999969482422 -1.29299449920654
67.4000015258789 -1.63889288902283
68.0999984741211 -2.01842617988586
68.9000015258789 -2.49312686920166
69.6999969482422 -3.01155948638916
70.5 -3.5741183757782
71.3000030517578 -4.18159532546997
72.0999984741211 -4.8351616859436
72.9000015258789 -5.536461353302
73.6999969482422 -6.28762149810791
74.4000015258789 -6.98791980743408
75.0999984741211 -7.73072862625122
75.8000030517578 -8.51872634887695
76.5 -9.35519504547119
77.0999984741211 -10.1137104034424
};
\addlegendentry{8x8}
\end{axis}

\end{tikzpicture}

    %\vspace*{-4mm}
    \caption{Isotropic radiating elements.}
    \vspace*{2mm}
    \label{fig:pattern_iso}
  \end{subfigure}
 \hfill
  \begin{subfigure}[t]{\columnwidth}
    \centering 
    \setlength\fwidth{0.65\columnwidth}
    \setlength\fheight{0.28\columnwidth}
    % This file was created with tikzplotlib v0.10.1.
\begin{tikzpicture}

  \definecolor{plotColor1}{HTML}{e60049}
  \definecolor{plotColor2}{HTML}{0bb4ff}
  \definecolor{plotColor3}{HTML}{87bc45}
  \definecolor{plotColor4}{HTML}{ffa300}

 \begin{axis}[
    width=\fwidth,
    height=\fheight,
    at={(0\fwidth,0\fheight)},
    scale only axis,
    legend image post style={mark indices={}},
    %axis line style={white!80!black},
    legend style={
        /tikz/every even column/.append style={column sep=0.2cm},
        at={(0.5, 0.95)}, 
        anchor=south, 
        draw=white!80!black, 
        font=\scriptsize
        },
    legend columns=4,
    %tick align=outside,
    %x grid style={white!80!black},
    xlabel style={font=\footnotesize},
    xlabel={Relative azimuth $\varphi$ between transmitter and receiver [deg]},
    %xtick={0, 16, 64, 256},
    xmajorgrids,
    %xmajorticks=false,
    xmin=-90, xmax=90,
    xtick style={color=white!15!black},
    %y grid style={white!80!black},
    ylabel shift = -1 pt,
    ylabel style={font=\footnotesize},
    ylabel={$G_{\mathrm{AA}} (\theta, \varphi)$ [dB]},
    ymajorgrids,
    ymajorticks=true,
    ymin=-10, ymax=30,
    ytick style={color=white!15!black}
]

\addplot [very thick, plotColor4, mark=o, mark repeat = 12]
table {%
-79.6999969482422 -10.0414381027222
-77.1999969482422 -8.92735767364502
-74.5999984741211 -7.8063817024231
-72.0999984741211 -6.76471519470215
-69.5999984741211 -5.75855350494385
-67.0999984741211 -4.78791284561157
-64.5999984741211 -3.8527569770813
-62.0999984741211 -2.95311665534973
-59.5999984741211 -2.08897686004639
-57.0999984741211 -1.260333776474
-54.5999984741211 -0.467197895050049
-52.0999984741211 0.290426969528198
-49.7000007629395 0.984360456466675
-47.2000007629395 1.67240512371063
-44.7000007629395 2.32495498657227
-42.2000007629395 2.94199395179749
-39.7000007629395 3.52352333068848
-37.2999992370605 4.04840564727783
-34.9000015258789 4.54056310653687
-32.5 5.00000333786011
-30.1000003814697 5.4267110824585
-27.7000007629395 5.82071304321289
-25.2999992370605 6.18199825286865
-22.8999996185303 6.51055002212524
-20.5 6.80639123916626
-18.1000003814697 7.06950664520264
-15.6999998092651 7.29991245269775
-13.3000001907349 7.49758720397949
-10.8999996185303 7.66255140304565
-8.5 7.79479598999023
-6.09999990463257 7.89431238174438
-3.70000004768372 7.96112060546875
-1.29999995231628 7.99519729614258
1.10000002384186 7.9965615272522
3.5 7.96520900726318
5.90000009536743 7.90113401412964
8.30000019073486 7.8043327331543
10.6999998092651 7.67482376098633
13.1000003814697 7.51259088516235
15.5 7.31763553619385
17.8999996185303 7.08995914459229
20.2999992370605 6.8295693397522
22.7000007629395 6.53645610809326
25.1000003814697 6.21061992645264
27.5 5.85206842422485
29.8999996185303 5.46079730987549
32.2999992370605 5.03680372238159
34.7000007629395 4.58010482788086
37.0999984741211 4.09067058563232
39.5 3.56851577758789
41.9000015258789 3.01364278793335
44.2999992370605 2.42606019973755
46.7999992370605 1.77920126914978
49.2999992370605 1.0968314409256
51.7999992370605 0.378963947296143
54.2999992370605 -0.374407887458801
56.7999992370605 -1.16328346729279
59.2999992370605 -1.98766469955444
61.7999992370605 -2.84754824638367
64.3000030517578 -3.74292540550232
66.8000030517578 -4.67381906509399
69.3000030517578 -5.64020347595215
71.8000030517578 -6.64209222793579
74.3000030517578 -7.67949676513672
76.8000030517578 -8.75238227844238
79.3000030517578 -9.86080455780029
79.6999969482422 -10.0414381027222
};
\addlegendentry{1x1}
\addplot [very thick, plotColor1, mark=square*, mark repeat = 8, mark options={solid}]
table {%
-60.2999992370605 -10.076268196106
-59 -8.92030715942383
-57.7000007629395 -7.80672264099121
-56.2999992370605 -6.65222644805908
-54.9000015258789 -5.54167032241821
-53.5 -4.47286796569824
-52.0999984741211 -3.44387173652649
-50.5999984741211 -2.38360786437988
-49.0999984741211 -1.36523079872131
-47.5999984741211 -0.38710880279541
-46.0999984741211 0.552177429199219
-44.5999984741211 1.4539532661438
-43.0999984741211 2.31930661201477
-41.5 3.2033326625824
-39.9000015258789 4.04816627502441
-38.2999992370605 4.85472965240479
-36.7000007629395 5.6239161491394
-35.0999984741211 6.35643577575684
-33.5 7.05299091339111
-31.8999996185303 7.7141695022583
-30.2999992370605 8.34049224853516
-28.7000007629395 8.93243980407715
-27.1000003814697 9.49042415618896
-25.5 10.0148277282715
-23.7999992370605 10.5355863571167
-22.1000003814697 11.0191535949707
-20.3999996185303 11.465838432312
-18.7000007629395 11.8759117126465
-17.1000003814697 12.2286472320557
-15.3999996185303 12.5683317184448
-13.8000001907349 12.8551683425903
-12.1000003814697 13.1251373291016
-10.5 13.3466281890869
-8.80000019073486 13.547402381897
-7.09999990463257 13.7126741409302
-5.40000009536743 13.8425302505493
-3.70000004768372 13.9370155334473
-2 13.9961833953857
-0.299999952316284 14.0200500488281
1.29999995231628 14.0102815628052
2.90000009536743 13.9692602157593
4.59999990463257 13.8913946151733
6.19999980926514 13.7858324050903
7.90000009536743 13.6393146514893
9.60000038146973 13.4573431015015
11.3000001907349 13.2398386001587
13 12.9866752624512
14.6999998092651 12.6977434158325
16.3999996185303 12.3728981018066
18.1000003814697 12.0119495391846
19.7000007629395 11.6390943527222
21.2999992370605 11.2339096069336
22.8999996185303 10.7961893081665
24.5 10.3256635665894
26.1000003814697 9.82209205627441
27.7000007629395 9.28513622283936
29.2999992370605 8.71445655822754
30.8999996185303 8.10967350006104
32.5 7.47034072875977
34.0999984741211 6.79595947265625
35.7000007629395 6.08598756790161
37.2999992370605 5.33980369567871
38.9000015258789 4.55669975280762
40.5 3.73588681221008
42.0999984741211 2.87647247314453
43.7000007629395 1.97746312618256
45.2999992370605 1.03772938251495
46.7999992370605 0.118579030036926
48.2999992370605 -0.838651776313782
49.7999992370605 -1.83536636829376
51.2999992370605 -2.87307381629944
52.7999992370605 -3.95350670814514
54.2000007629395 -5.00218105316162
55.5999984741211 -6.09160232543945
57 -7.22381639480591
58.4000015258789 -8.40123176574707
59.7000007629395 -9.53731250762939
60.2999992370605 -10.076268196106
};
\addlegendentry{2x2}
\addplot [very thick, plotColor2]
table {%
-66 -10.0238256454468
-64.9000015258789 -8.91702270507812
-63.7999992370605 -7.86076021194458
-62.7000007629395 -6.85317659378052
-61.5999984741211 -5.89285182952881
-60.5 -4.97870683670044
-59.4000015258789 -4.10994815826416
-58.2999992370605 -3.28611445426941
-57.2000007629395 -2.50699734687805
-56.2000007629395 -1.83757376670837
-55.2000007629395 -1.2054169178009
-54.2000007629395 -0.6109619140625
-53.2000007629395 -0.054857611656189
-52.2000007629395 0.462024211883545
-51.2000007629395 0.938627600669861
-50.2999992370605 1.33200287818909
-49.4000015258789 1.69042217731476
-48.5 2.01246881484985
-47.5999984741211 2.29646015167236
-46.7999992370605 2.51528716087341
-46 2.7008228302002
-45.2000007629395 2.85104775428772
-44.4000015258789 2.9637336730957
-43.7000007629395 3.02941012382507
-43 3.06214809417725
-42.2999992370605 3.05946493148804
-41.7000007629395 3.02675032615662
-41.0999984741211 2.96371173858643
-40.5 2.86779713630676
-39.9000015258789 2.73600769042969
-39.2999992370605 2.56484413146973
-38.7999992370605 2.3890380859375
-38.2999992370605 2.17999529838562
-37.7999992370605 1.93419599533081
-37.2999992370605 1.64745330810547
-36.7999992370605 1.31474304199219
-36.4000015258789 1.01128172874451
-36 0.670485258102417
-35.5999984741211 0.287745237350464
-35.2000007629395 -0.142772078514099
-34.7999992370605 -0.628142237663269
-34.4000015258789 -1.17736232280731
-34.0999984741211 -1.6379998922348
-33.7999992370605 -2.14713597297668
-33.5 -2.7124457359314
-33.2000007629395 -3.34368348121643
-32.9000015258789 -4.05346727371216
-32.5999984741211 -4.85857534408569
-32.2999992370605 -5.78198719024658
-32 -6.85631322860718
-31.7999992370605 -7.67942047119141
-31.6000003814697 -8.61125946044922
-31.3999996185303 -9.68070697784424
-31.2999992370605 -10.2795505523682
nan nan
-28.8999996185303 -10.7165994644165
-28.7000007629395 -9.18656253814697
-28.5 -7.86543321609497
-28.2000007629395 -6.16611957550049
-27.8999996185303 -4.7134165763855
-27.6000003814697 -3.44172239303589
-27.2999992370605 -2.30872392654419
-27 -1.28553748130798
-26.6000003814697 -0.057321310043335
-26.2000007629395 1.0464072227478
-25.7999992370605 2.04995822906494
-25.3999996185303 2.97092700004578
-24.8999996185303 4.02581167221069
-24.3999996185303 4.99143886566162
-23.8999996185303 5.88214540481567
-23.3999996185303 6.70888471603394
-22.7999992370605 7.62842035293579
-22.2000007629395 8.47957420349121
-21.6000003814697 9.27129554748535
-21 10.0106706619263
-20.2999992370605 10.814697265625
-19.6000003814697 11.5624618530273
-18.8999996185303 12.2598867416382
-18.2000007629395 12.9117116928101
-17.5 13.5218830108643
-16.7000007629395 14.1723909378052
-15.8999996185303 14.7768115997314
-15.1000003814697 15.3385334014893
-14.3000001907349 15.8603792190552
-13.5 16.3447322845459
-12.6999998092651 16.7936153411865
-11.8000001907349 17.2583656311035
-10.8999996185303 17.6824951171875
-10 18.0677814483643
-9.10000038146973 18.4157161712646
-8.19999980926514 18.727575302124
-7.30000019073486 19.004430770874
-6.40000009536743 19.247184753418
-5.5 19.4566116333008
-4.59999990463257 19.6333389282227
-3.70000004768372 19.7778797149658
-2.79999995231628 19.8906478881836
-1.89999997615814 19.9719581604004
-1 20.0220489501953
-0.100000023841858 20.0410194396973
0.799999952316284 20.0289134979248
1.70000004768372 19.9857788085938
2.59999990463257 19.9114284515381
3.5 19.8056735992432
4.40000009536743 19.6682243347168
5.30000019073486 19.4986896514893
6.19999980926514 19.2965850830078
7.09999990463257 19.0612907409668
8 18.7920894622803
8.89999961853027 18.4880962371826
9.80000019073486 18.1482810974121
10.6999998092651 17.7714176177979
11.6000003814697 17.3560600280762
12.5 16.9005069732666
13.3000001907349 16.4602031707764
14.1000003814697 15.9848957061768
14.8999996185303 15.4726257324219
15.6999998092651 14.9211196899414
16.5 14.327672958374
17.2999992370605 13.6890392303467
18 13.090145111084
18.7000007629395 12.4505844116211
19.3999996185303 11.7666416168213
20.1000003814697 11.0337944030762
20.7999992370605 10.2465391159058
21.3999996185303 9.52326393127441
22 8.7497034072876
22.6000003814697 7.91924905776978
23.2000007629395 7.02364158630371
23.7000007629395 6.21999406814575
24.2000007629395 5.35604238510132
24.7000007629395 4.4218807220459
25.2000007629395 3.40469884872437
25.6000003814697 2.51983571052551
26 1.55949079990387
26.3999996185303 0.508397102355957
26.7999992370605 -0.654005646705627
27.1000003814697 -1.61580383777618
27.3999996185303 -2.67296719551086
27.7000007629395 -3.84846711158752
28 -5.17492294311523
28.2999992370605 -6.70093870162964
28.5 -7.86543321609497
28.7000007629395 -9.18656253814697
28.8999996185303 -10.7165994644165
nan nan
31.2999992370605 -10.2795505523682
31.5 -9.12655258178711
31.7000007629395 -8.13018226623535
31.8999996185303 -7.25571966171265
32.2000007629395 -6.12133407592773
32.5 -5.15186452865601
32.7999992370605 -4.31026601791382
33.0999984741211 -3.57081627845764
33.4000015258789 -2.91498374938965
33.7000007629395 -2.32890152931213
34 -1.80198192596436
34.4000015258789 -1.17736232280731
34.7999992370605 -0.628142237663269
35.2000007629395 -0.142772078514099
35.5999984741211 0.287745237350464
36 0.670485258102417
36.4000015258789 1.01128172874451
36.7999992370605 1.31474304199219
37.2999992370605 1.64745330810547
37.7999992370605 1.93419599533081
38.2999992370605 2.17999529838562
38.7999992370605 2.3890380859375
39.2999992370605 2.56484413146973
39.9000015258789 2.73600769042969
40.5 2.86779713630676
41.0999984741211 2.96371173858643
41.7000007629395 3.02675032615662
42.2999992370605 3.05946493148804
43 3.06214809417725
43.7000007629395 3.02941012382507
44.4000015258789 2.9637336730957
45.0999984741211 2.86723613739014
45.9000015258789 2.72157335281372
46.7000007629395 2.54035425186157
47.5 2.32556509971619
48.2999992370605 2.07892775535583
49.2000007629395 1.76519548892975
50.0999984741211 1.41471719741821
51 1.02899992465973
51.9000015258789 0.60931921005249
52.9000015258789 0.104356527328491
53.9000015258789 -0.440069556236267
54.9000015258789 -1.02309691905975
55.9000015258789 -1.64399468898773
56.9000015258789 -2.30227708816528
57.9000015258789 -2.99762845039368
59 -3.80519533157349
60.0999984741211 -4.65756797790527
61.2000007629395 -5.55514907836914
62.2999992370605 -6.4985671043396
63.4000015258789 -7.4888277053833
64.5 -8.52719497680664
65.5999984741211 -9.61538791656494
66 -10.0238256454468
};
\addlegendentry{4x4}
\addplot [very thick, plotColor3, densely dotted]
table {%
-70.8000030517578 -10.0336885452271
-70 -9.13447952270508
-69.1999969482422 -8.28326225280762
-68.4000015258789 -7.47953224182129
-67.5999984741211 -6.7231011390686
-66.8000030517578 -6.01415967941284
-66 -5.35319709777832
-65.3000030517578 -4.81493949890137
-64.5999984741211 -4.31492042541504
-63.9000015258789 -3.85423350334167
-63.2000007629395 -3.4341983795166
-62.5 -3.05645799636841
-61.9000015258789 -2.7678165435791
-61.2999992370605 -2.51314377784729
-60.7000007629395 -2.29414534568787
-60.0999984741211 -2.11283087730408
-59.5 -1.97153103351593
-59 -1.88633477687836
-58.5 -1.83268988132477
-58 -1.81275415420532
-57.5 -1.82902884483337
-57 -1.88445103168488
-56.5999984741211 -1.95928072929382
-56.2000007629395 -2.06345081329346
-55.7999992370605 -2.1993772983551
-55.4000015258789 -2.36985540390015
-55 -2.57822608947754
-54.5999984741211 -2.82843971252441
-54.2000007629395 -3.12524557113647
-53.7999992370605 -3.47445225715637
-53.5 -3.77499985694885
-53.2000007629395 -4.11271286010742
-52.9000015258789 -4.49201917648315
-52.5999984741211 -4.91827297210693
-52.2999992370605 -5.39809036254883
-52 -5.93977880477905
-51.7000007629395 -6.55389642715454
-51.4000015258789 -7.25419616699219
-51.0999984741211 -8.05913257598877
-50.7999992370605 -8.99419212341309
-50.5999984741211 -9.70729923248291
-50.5 -10.0960254669189
nan nan
-47 -10.1601896286011
-46.7000007629395 -8.56601047515869
-46.4000015258789 -7.19977378845215
-46.0999984741211 -6.00454330444336
-45.7999992370605 -4.94292593002319
-45.5 -3.98910665512085
-45.2000007629395 -3.12445187568665
-44.7999992370605 -2.08693289756775
-44.4000015258789 -1.15968012809753
-44 -0.325348615646362
-43.5999984741211 0.428903937339783
-43.2000007629395 1.11282682418823
-42.7999992370605 1.73390257358551
-42.4000015258789 2.29796981811523
-42 2.80960607528687
-41.5999984741211 3.27240371704102
-41.2000007629395 3.68920660018921
-40.7999992370605 4.06214761734009
-40.4000015258789 4.3928689956665
-40 4.68255758285522
-39.5999984741211 4.93194484710693
-39.2000007629395 5.14139842987061
-38.7999992370605 5.31091690063477
-38.4000015258789 5.44014692306519
-38 5.52832412719727
-37.5999984741211 5.57431411743164
-37.2999992370605 5.5802059173584
-37 5.56063556671143
-36.7000007629395 5.51459741592407
-36.4000015258789 5.44084167480469
-36.0999984741211 5.33793640136719
-35.7999992370605 5.2041277885437
-35.5 5.03736209869385
-35.2000007629395 4.83517408370972
-34.9000015258789 4.59463357925415
-34.5999984741211 4.31218957901001
-34.2999992370605 3.983567237854
-34 3.60351014137268
-33.7000007629395 3.16544818878174
-33.4000015258789 2.66118764877319
-33.0999984741211 2.08016538619995
-32.7999992370605 1.40859866142273
-32.5 0.628000140190125
-32.2000007629395 -0.287380695343018
-32 -0.990100145339966
-31.7999992370605 -1.78399360179901
-31.6000003814697 -2.6896755695343
-31.3999996185303 -3.73603177070618
-31.2000007629395 -4.96536159515381
-31 -6.44328212738037
-30.7999992370605 -8.27969455718994
-30.6000003814697 -10.6805772781372
nan nan
-29.3999996185303 -10.17626953125
-29.2000007629395 -7.60745573043823
-29 -5.60331058502197
-28.7999992370605 -3.95781517028809
-28.6000003814697 -2.56115674972534
-28.3999996185303 -1.34775769710541
-28.1000003814697 0.217349290847778
-27.7999992370605 1.55357301235199
-27.5 2.71699905395508
-27.2000007629395 3.74451375007629
-26.8999996185303 4.66162967681885
-26.6000003814697 5.48670673370361
-26.2999992370605 6.23336124420166
-25.8999996185303 7.12443447113037
-25.5 7.91261768341064
-25.1000003814697 8.6113395690918
-24.7000007629395 9.23059463500977
-24.2999992370605 9.77790832519531
-23.8999996185303 10.2589540481567
-23.5 10.6779594421387
-23.1000003814697 11.0380029678345
-22.7000007629395 11.3412294387817
-22.3999996185303 11.5321102142334
-22.1000003814697 11.6920404434204
-21.7999992370605 11.8210067749023
-21.5 11.9187841415405
-21.2000007629395 11.9848937988281
-20.8999996185303 12.0186395645142
-20.6000003814697 12.0190210342407
-20.2999992370605 11.9847288131714
-20 11.9140911102295
-19.7000007629395 11.8050146102905
-19.3999996185303 11.6548767089844
-19.1000003814697 11.4605360031128
-18.7999992370605 11.2179937362671
-18.5 10.9223279953003
-18.2000007629395 10.5673341751099
-17.8999996185303 10.1451635360718
-17.6000003814697 9.64565563201904
-17.2999992370605 9.05554676055908
-17 8.35691833496094
-16.7999992370605 7.81899642944336
-16.6000003814697 7.2122015953064
-16.3999996185303 6.52377843856812
-16.2000007629395 5.73673248291016
-16 4.82772159576416
-15.8000001907349 3.76350617408752
-15.6000003814697 2.49435615539551
-15.3999996185303 0.941076040267944
-15.1999998092651 -1.0342583656311
-15.1000003814697 -2.25606417655945
-15 -3.70592451095581
-14.8999996185303 -5.48089265823364
-14.8000001907349 -7.75757074356079
-14.6999998092651 -10.9133214950562
nan nan
-14.3000001907349 -12.6106576919556
-14.1999998092651 -8.66582489013672
-14.1000003814697 -5.93003606796265
-14 -3.82706022262573
-13.8000001907349 -0.667314291000366
-13.6000003814697 1.69689130783081
-13.3999996185303 3.59414267539978
-13.1999998092651 5.18332290649414
-13 6.55342102050781
-12.6999998092651 8.31262969970703
-12.3999996185303 9.81297016143799
-12.1000003814697 11.1224966049194
-11.8000001907349 12.284875869751
-11.5 13.32985496521
-11.1999998092651 14.2785921096802
-10.8000001907349 15.4202079772949
-10.3999996185303 16.4450645446777
-10 17.3726024627686
-9.60000038146973 18.2172966003418
-9.19999980926514 18.9901561737061
-8.80000019073486 19.6998081207275
-8.39999961853027 20.3530960083008
-7.90000009536743 21.0987205505371
-7.40000009536743 21.7734889984131
-6.90000009536743 22.3842506408691
-6.40000009536743 22.9364528656006
-5.90000009536743 23.4345779418945
-5.40000009536743 23.8822917938232
-4.90000009536743 24.2825946807861
-4.40000009536743 24.6380043029785
-3.90000009536743 24.9505710601807
-3.40000009536743 25.2220420837402
-2.90000009536743 25.4538249969482
-2.40000009536743 25.6470623016357
-1.89999997615814 25.8027019500732
-1.39999997615814 25.9214782714844
-0.899999976158142 26.0039024353027
-0.399999976158142 26.0503768920898
0.100000023841858 26.0610828399658
0.600000023841858 26.0360870361328
1.10000002384186 25.9752616882324
1.60000002384186 25.8783550262451
2.09999990463257 25.7449073791504
2.59999990463257 25.5743255615234
3.09999990463257 25.3657989501953
3.59999990463257 25.1182994842529
4.09999990463257 24.8305644989014
4.59999990463257 24.5010890960693
5.09999990463257 24.1279811859131
5.59999990463257 23.7090339660645
6.09999990463257 23.241569519043
6.59999990463257 22.7222690582275
7.09999990463257 22.1472473144531
7.59999990463257 21.5116138458252
8.10000038146973 20.8094024658203
8.5 20.1947364807129
8.89999961853027 19.5279235839844
9.30000019073486 18.8031673431396
9.69999980926514 18.013240814209
10.1000003814697 17.1489868164062
10.5 16.1986312866211
10.8999996185303 15.1466464996338
11.1999998092651 14.2785921096802
11.5 13.32985496521
11.8000001907349 12.284875869751
12.1000003814697 11.1224966049194
12.3999996185303 9.81297016143799
12.6999998092651 8.31262969970703
12.8999996185303 7.17432498931885
13.1000003814697 5.89176321029663
13.3000001907349 4.42060947418213
13.5 2.69164896011353
13.6999998092651 0.58767557144165
13.8999996185303 -2.11452674865723
14 -3.82706022262573
14.1000003814697 -5.93003606796265
14.1999998092651 -8.66582489013672
14.3000001907349 -12.6106576919556
nan nan
14.6999998092651 -10.9133214950562
14.8000001907349 -7.75757074356079
14.8999996185303 -5.48089265823364
15 -3.70592451095581
15.1999998092651 -1.0342583656311
15.3999996185303 0.941076040267944
15.6000003814697 2.49435615539551
15.8000001907349 3.76350617408752
16 4.82772159576416
16.2000007629395 5.73673248291016
16.3999996185303 6.52377843856812
16.6000003814697 7.2122015953064
16.8999996185303 8.09589385986328
17.2000007629395 8.83578586578369
17.5 9.45979404449463
17.7999992370605 9.98782253265381
18.1000003814697 10.4345264434814
18.3999996185303 10.8109159469604
18.7000007629395 11.1256141662598
19 11.3852691650391
19.2999992370605 11.5951910018921
19.6000003814697 11.7596673965454
19.8999996185303 11.8821105957031
20.2000007629395 11.9653263092041
20.5 12.0115118026733
20.7999992370605 12.0225391387939
21.1000003814697 11.9997720718384
21.3999996185303 11.9443664550781
21.7000007629395 11.8570642471313
22 11.7384738922119
22.2999992370605 11.5888690948486
22.6000003814697 11.4083156585693
22.8999996185303 11.1966161727905
23.2000007629395 10.9534025192261
23.6000003814697 10.5788278579712
24 10.1446590423584
24.3999996185303 9.64748859405518
24.7999992370605 9.08278560638428
25.2000007629395 8.44444847106934
25.6000003814697 7.72441291809082
25.8999996185303 7.12443447113037
26.2000007629395 6.46667337417603
26.5 5.74373912811279
26.7999992370605 4.94613695144653
27.1000003814697 4.06146574020386
27.3999996185303 3.07316303253174
27.7000007629395 1.9584709405899
28 0.684901237487793
28.2999992370605 -0.795974969863892
28.5 -1.93459165096283
28.7000007629395 -3.2331919670105
28.8999996185303 -4.7440071105957
29.1000003814697 -6.55093145370483
29.2999992370605 -8.80180168151855
29.3999996185303 -10.17626953125
nan nan
30.6000003814697 -10.6805772781372
30.7999992370605 -8.27969455718994
31 -6.44328212738037
31.2000007629395 -4.96536159515381
31.3999996185303 -3.73603177070618
31.6000003814697 -2.6896755695343
31.7999992370605 -1.78399360179901
32 -0.990100145339966
32.2999992370605 0.0346914529800415
32.5999984741211 0.901698470115662
32.9000015258789 1.64352858066559
33.2000007629395 2.28313088417053
33.5 2.8372175693512
33.7999992370605 3.3183810710907
34.0999984741211 3.73629093170166
34.4000015258789 4.09855365753174
34.7000007629395 4.41124582290649
35 4.67927598953247
35.2999992370605 4.90667676925659
35.5999984741211 5.09675550460815
35.9000015258789 5.25228834152222
36.2000007629395 5.37558460235596
36.5 5.46859645843506
36.7999992370605 5.53295469284058
37.0999984741211 5.57004022598267
37.4000015258789 5.5810227394104
37.7999992370605 5.55667495727539
38.2000007629395 5.48941898345947
38.5999984741211 5.38060712814331
39 5.23116683959961
39.4000015258789 5.04164934158325
39.7999992370605 4.8122615814209
40.2000007629395 4.54279136657715
40.5999984741211 4.23270463943481
41 3.88102984428406
41.4000015258789 3.48641657829285
41.7999992370605 3.04692983627319
42.2000007629395 2.56010174751282
42.5999984741211 2.0227575302124
43 1.43083536624908
43.4000015258789 0.779163360595703
43.7999992370605 0.0611076354980469
44.2000007629395 -0.7317875623703
44.5999984741211 -1.61072874069214
45 -2.59059691429138
45.2999992370605 -3.40368986129761
45.5999984741211 -4.29633378982544
45.9000015258789 -5.28372430801392
46.2000007629395 -6.38653755187988
46.5 -7.63387155532837
46.7999992370605 -9.06844615936279
47 -10.1601896286011
nan nan
50.5 -10.0960254669189
50.7999992370605 -8.99419212341309
51.0999984741211 -8.05913257598877
51.4000015258789 -7.25419616699219
51.7000007629395 -6.55389642715454
52 -5.93977880477905
52.2999992370605 -5.39809036254883
52.5999984741211 -4.91827297210693
52.9000015258789 -4.49201917648315
53.2000007629395 -4.11271286010742
53.5 -3.77499985694885
53.9000015258789 -3.38187313079834
54.2999992370605 -3.04637432098389
54.7000007629395 -2.7617175579071
55.0999984741211 -2.52237939834595
55.5 -2.32382607460022
55.9000015258789 -2.16227149963379
56.2999992370605 -2.03453421592712
56.7000007629395 -1.93790853023529
57.0999984741211 -1.87007808685303
57.5999984741211 -1.82274031639099
58.0999984741211 -1.81392776966095
58.5999984741211 -1.8407906293869
59.0999984741211 -1.9009096622467
59.5999984741211 -1.99219024181366
60.0999984741211 -2.11283087730408
60.7000007629395 -2.29414534568787
61.2999992370605 -2.51314377784729
61.9000015258789 -2.7678165435791
62.5 -3.05645799636841
63.2000007629395 -3.4341983795166
63.9000015258789 -3.85423350334167
64.5999984741211 -4.31492042541504
65.3000030517578 -4.81493949890137
66 -5.35319709777832
66.8000030517578 -6.01415967941284
67.5999984741211 -6.7231011390686
68.4000015258789 -7.47953224182129
69.1999969482422 -8.28326225280762
70 -9.13447952270508
70.8000030517578 -10.0336885452271
};
\addlegendentry{8x8}
\end{axis}

\end{tikzpicture}

    %\vspace*{-4mm}
    \caption{Directional antenna radiation pattern of~\cite[Section~7.3]{TR38901}.}
    \label{fig:pattern_3gpp}
  \end{subfigure}
  % \setlength\belowcaptionskip{.1cm}
  \caption{Overall array and beamforming gain of a \gls{upa}, for isotropic and 3GPP~\cite[Section~7.3]{TR38901} radiating elements and \{1x1, 2x2, 4x4, 8x8\} antenna configurations. The steering direction is fixed to $\left(\theta_0, \varphi_0\right) = (0^{\circ}, 0^{\circ})$, and $\theta \equiv 0^{\circ}$.}
  \label{Fig:rad_pattern}
\end{figure}

It can be noted that our model abstracts the computation of the received signal power as a \gls{siso} keyhole channel~\cite{chizhik2000capacities}, which is then combined with the spatial antenna gain
patterns at the transmitter/receiver to obtain the received power. This approximation is possibly imprecise when considering \gls{nlos} links, due to the lack of a dominant multipath component. To account for this limitation, we introduce a multiplicative correction factor $\eta$ which scales the beamforming gain as $G^{'}_{T, R}(\theta, \varphi) \equiv \eta G_{T, R}(\theta, \varphi) $. In line with~\cite{kulkarni2018correction}, we set $\eta = 1 / 19$.

\subsubsection{Fast Fading}

The widely used Rayleigh and Rician distributions fail, even in their generalized forms, to capture the intrinsic bimodality exhibited by \gls{mmwave} scenarios~\cite{yacoub2007kappa, cotton2014human, mavridis2015near}.
Therefore, in our implementation we model fast fading using the more general \gls{ftr} model of~\cite{7917287}. %The latter is a fading model which is more general than typical ones. 
This fading model assumes that the received signal comprises two dominant specular components and a mixture of scattered paths, thus modeling the amplitude of the received signal $V_r$ as
\begin{equation}
   V_r = V_1 \sqrt{\xi} \exp(j \phi_1) + V_2 \sqrt{\xi} \exp(j \phi_2) + X + jY,
\end{equation}
where $\phi_1$, $\phi_2$ are statistically independent random phases, distributed as $\phi_{i} \sim \mathcal{U} \left[ 0, 2\pi\right]$. $X$ and $Y$ are independent Gaussian random variables, i.e., $X, Y \sim \mathcal{N} (0, \sigma^2)$, which represent the diffuse component of the received signal, which is assumed to be the superposition of multiple weak scattered waves with independent phase. Finally, $\xi$ is a unit-mean Gamma distributed random variable with rate $m$ and \gls{pdf}
\begin{equation}
   f_{\xi} (u) = \frac{m^m u^{m-1}}{\Gamma (m)} exp(-m u).
\end{equation}
In our implementation, $F_{T, R} = \left| V_r \right|^2$ is sampled via the \texttt{Get\-Ftr\-Fast\-Fading} function of the \texttt{Two\-Ray\-Spectrum\-Propagation\-Loss\-Model} class.

The \gls{ftr} fading model is usually expressed as a function of the Gamma rate $m$ and the auxiliary parameters
\begin{align}
    K &\doteq \frac{V_1^2 + V_2^2}{2 \sigma^2} \\
    \Delta &\doteq \frac{2 V_1 V_2}{V_1^2 + V_2^2} \in \left[ 0, 1 \right],
\end{align}
where $K$ represents the ratio of the power of the specular components with respect to the diffuse ones, while $\Delta$ denotes how similar the received powers of the specular components are. By tuning these parameters, a high degree of flexibility can be achieved. Notably, a choice of $\Delta = 0$ effectively yields a Rician-distributed signal amplitude~\cite{7917287}.

\paragraph{Calibration.}

In our work, we calibrated the $V_1, V_2$ and $m$ parameters of the \gls{ftr} fading model using the full 3GPP TR 38.901 channel model as a reference. 
In particular, we first obtained the statistics of the small-scale fading of the 3GPP model, using an ad hoc calibration script (\texttt{three-\-gpp-\-two-\-ray-\-channel-\-calibration.cc}). The script produces a collection of channel gain samples obtained by using the \texttt{Three\-Gpp\-Spectrum\-Propagation\-Loss\-Model} and the \texttt{Three\-Gpp\-Channel\-Model} classes, and neglecting the beamforming gain, path-loss, shadowing and blockages. Accordingly, we isolate the variation around the mean received power caused by the small-scale fading only. 
A separate set of these samples has been retrieved for both \gls{los} and \gls{nlos} channel conditions, the different propagation scenarios of~\cite{TR38901}, and a set of carrier frequencies ranging from $0.5$ to $100$ GHz. However, a preliminary evaluation of the obtained data showed a negligible dependence of the small-scale fading with respect to the carrier frequency, as can be observed in Figure~\ref{fig:fading_vs_fc}. Therefore, we calibrated the \gls{ftr} parameters considering only the channel condition and the propagation scenario.

\begin{figure}
    \centering 
    \setlength\fwidth{0.95\columnwidth}
    \setlength\fheight{0.25\columnwidth}
    % This file was created with tikzplotlib v0.10.1.
\begin{tikzpicture}[spy using outlines={circle, connect spies}]

\definecolor{plotColor1}{HTML}{e60049}
\definecolor{plotColor2}{HTML}{0bb4ff}
\definecolor{plotColor3}{HTML}{87bc45}
\definecolor{plotColor4}{HTML}{ffa300}

 \begin{groupplot}[
  group style={
    group size=2 by 1,
    group name=plots,
    horizontal sep= 0.1 cm
    %ylabels at=edge
  },
    width=0.45\fwidth,
    height=\fheight,
    at={(0\fwidth,0\fheight)},
    scale only axis,
    legend image post style={mark indices={}},
    %axis line style={white!80!black},
    legend style={
        /tikz/every even column/.append style={column sep=0.2cm},
        at={(1, 1.05)}, 
        anchor=south, 
        draw=white!80!black, 
        font=\scriptsize
        },
    legend columns=4,
    %tick align=outside,
    %x grid style={white!80!black},
    xlabel style={font=\footnotesize},
    xlabel={Small-scale fading gain [dB]},
    %xtick={0, 16, 64, 256},
    xmajorgrids,
    %xmajorticks=false,
    xmin=-4.5, xmax=2.5,
    xtick style={color=white!15!black},
    %y grid style={white!80!black},
    ylabel shift = -1 pt,
    ylabel style={font=\footnotesize},
    %ylabel={ECDF},
    ymajorgrids,
    %ymajorticks=true,
    ymin=0, ymax=1.0,
    ytick style={color=white!15!black},
    yticklabels=\empty
]

\nextgroupplot[title={a) \gls{los}}, 
               every axis title/.style={below, at={(0.5, -0.38)}},
               ylabel={ECDF}, 
               yticklabels={0, 0.2, 0.4, 0.6, 0.8, 1},
               ytick={0, 0.2, 0.4, 0.6, 0.8, 1}]

\addplot [plotColor4, mark=*, mark repeat = 12]
table {%
-7.16952180862427 0
-3.02245116233826 0.00580120086669922
-2.07791876792908 0.0234047174453735
-1.88264846801758 0.0326064825057983
-1.68216586112976 0.0434086322784424
-1.57704734802246 0.0492098331451416
-1.49631953239441 0.0564112663269043
-1.321537733078 0.0716142654418945
-1.23770606517792 0.0794159173965454
-1.08797168731689 0.103420734405518
-1.01527678966522 0.113622665405273
-0.793648958206177 0.158631682395935
-0.548180818557739 0.23804759979248
-0.524192571640015 0.248849749565125
-0.498615980148315 0.259251832962036
-0.441958785057068 0.282856583595276
-0.310596466064453 0.34186840057373
-0.276281952857971 0.358471632003784
-0.242456912994385 0.379075765609741
-0.175700068473816 0.41948390007019
-0.0584260225296021 0.495499134063721
-0.00356698036193848 0.530706167221069
0.0582375526428223 0.572514533996582
0.290467619895935 0.703140616416931
0.345161557197571 0.730346083641052
0.399967551231384 0.753350734710693
0.615316390991211 0.838167667388916
0.687608242034912 0.859171867370605
1.01973783969879 0.931186199188232
1.10434818267822 0.93918776512146
1.33452677726746 0.963192701339722
1.53223216533661 0.975595116615295
1.76534008979797 0.985997200012207
3.35465931892395 1
};
\addlegendentry{0.5 GHz}

\addplot [plotColor1, mark=square*, mark repeat = 8, mark options={solid}]
table {%
-6.11087465286255 0
-2.31923818588257 0.0130026340484619
-2.0123131275177 0.0234047174453735
-1.54869079589844 0.0492098331451416
-1.47585272789001 0.0540107488632202
-1.34296834468842 0.0666133165359497
-0.962422609329224 0.12322461605072
-0.924833059310913 0.13002598285675
-0.847009658813477 0.145629167556763
-0.769151210784912 0.16623330116272
-0.60158896446228 0.21604323387146
-0.486804485321045 0.260452032089233
-0.437153220176697 0.27845573425293
-0.35867702960968 0.316463232040405
-0.325161457061768 0.332266449928284
-0.211377501487732 0.398079633712769
-0.124695897102356 0.44989001750946
-0.0733383893966675 0.489297866821289
-0.0418848991394043 0.510102033615112
0.0911506414413452 0.596519231796265
0.23384165763855 0.679335832595825
0.270069360733032 0.697739601135254
0.33061695098877 0.727945566177368
0.536468744277954 0.813962817192078
0.620778322219849 0.839767932891846
0.673089623451233 0.856171250343323
0.753498435020447 0.878375649452209
0.861141443252563 0.899179816246033
0.903180241584778 0.908181667327881
0.96204137802124 0.919383883476257
1.01055908203125 0.927385449409485
1.0606164932251 0.93498706817627
1.63912391662598 0.980996131896973
2.92333626747131 0.999399900436401
3.59355401992798 1
};
\addlegendentry{5.5 GHz}

\addplot [plotColor2]
table {%
-5.81224870681763 0
-3.01014351844788 0.00640130043029785
-2.68981266021729 0.00980198383331299
-2.28117942810059 0.0168033838272095
-2.08844304084778 0.0240048170089722
-1.58313989639282 0.0504100322723389
-1.42032206058502 0.0642127990722656
-0.920342922210693 0.14262855052948
-0.744708180427551 0.183236598968506
-0.68340802192688 0.203240633010864
-0.656758069992065 0.212442517280579
-0.615943670272827 0.226845383644104
-0.491952300071716 0.273254632949829
-0.47163987159729 0.282056450843811
-0.412240386009216 0.310062050819397
-0.18245255947113 0.429685950279236
-0.0817031860351562 0.489897966384888
-0.00385391712188721 0.537107467651367
0.0970579385757446 0.599720001220703
0.14398717880249 0.623724699020386
0.195233821868896 0.652730464935303
0.379083395004272 0.743148565292358
0.510753154754639 0.797159433364868
0.550035119056702 0.810562133789062
0.582399606704712 0.819763898849487
0.651538014411926 0.842368483543396
0.720733404159546 0.861972332000732
0.765894770622253 0.874774932861328
0.8250892162323 0.885577201843262
1.1613267660141 0.942188501358032
1.26683259010315 0.951390266418457
1.4355810880661 0.966793298721313
1.74795126914978 0.982796549797058
2.63998031616211 0.998199701309204
3.13217401504517 1
};
\addlegendentry{30.5 GHz}

\addplot [plotColor3, dashed]
table {%
-6.61107444763184 0
-2.94158506393433 0.00600123405456543
-2.68199372291565 0.00980198383331299
-2.32327008247375 0.0168033838272095
-2.06865048408508 0.0250049829483032
-1.9904522895813 0.0296058654785156
-1.80777955055237 0.0382076501846313
-1.76482737064362 0.0416083335876465
-1.54088735580444 0.0568113327026367
-1.15443646907806 0.0976195335388184
-1.07932674884796 0.110222101211548
-0.927203416824341 0.139827966690063
-0.846091270446777 0.158631682395935
-0.566835880279541 0.235647082328796
-0.524971723556519 0.251450300216675
-0.374649286270142 0.311262249946594
-0.316118597984314 0.339067816734314
-0.171103358268738 0.41528308391571
-0.144356727600098 0.431086182594299
-0.100245475769043 0.463892817497253
-0.0593284368515015 0.491898417472839
-0.0401619672775269 0.505901098251343
-0.00453639030456543 0.529706001281738
0.0138752460479736 0.54170835018158
0.232606887817383 0.67333459854126
0.390649557113647 0.744748950004578
0.509015321731567 0.789958000183105
0.553706884384155 0.8033607006073
0.630780458450317 0.827365398406982
0.671303987503052 0.840568065643311
0.770843386650085 0.865373134613037
0.79900860786438 0.871774435043335
1.20728540420532 0.941388249397278
1.37189483642578 0.956191301345825
1.6975964307785 0.97639524936676
1.88776659965515 0.984997034072876
2.11104512214661 0.99159836769104
4.47117710113525 1
};
\addlegendentry{60 GHz}

%%%%%%%%
% NLOS %
%%%%%%%%
\nextgroupplot[title={b) \gls{nlos}},
               every axis title/.style={below, at={(0.5, -0.38)}}]
\addplot [plotColor4, mark=*, mark repeat = 12]
table {%
-5.58383417129517 0
-3.43877840042114 0.00600123405456543
-2.77724838256836 0.0192039012908936
-2.45355868339539 0.0344069004058838
-2.25332689285278 0.0462092161178589
-2.02714443206787 0.0676134824752808
-1.78531670570374 0.0966193675994873
-1.72256052494049 0.105621099472046
-1.60946393013 0.124825000762939
-1.53571331501007 0.133226633071899
-1.43572020530701 0.149029850959778
-1.27628672122955 0.178035616874695
-1.17881488800049 0.198439717292786
-1.14884841442108 0.207241415977478
-1.10863077640533 0.214843034744263
-0.963902592658997 0.25225043296814
-0.877635955810547 0.275855183601379
-0.810600638389587 0.294858932495117
-0.522590398788452 0.374274849891663
-0.373787760734558 0.422684550285339
-0.34393835067749 0.435086965560913
-0.318351864814758 0.442288398742676
-0.283211350440979 0.456491231918335
-0.232416033744812 0.472894549369812
-0.1290602684021 0.504700899124146
0.497256994247437 0.708541631698608
0.573503494262695 0.730146050453186
0.644916772842407 0.751150250434875
0.735460638999939 0.773554682731628
0.778981924057007 0.783556699752808
0.819374918937683 0.794759035110474
0.871351718902588 0.808561682701111
1.00101613998413 0.843168616294861
1.21323430538177 0.882976531982422
1.25286030769348 0.889777898788452
1.36209893226624 0.90458083152771
1.46590006351471 0.920984268188477
1.60987448692322 0.939587831497192
1.86164963245392 0.961992383003235
2.04700469970703 0.972994565963745
2.19547843933105 0.98059606552124
2.78952407836914 0.995599031448364
3.92290592193604 1
};

\addplot [plotColor1, mark=square*, mark repeat = 8, mark options={solid}]
table {%
-7.44770908355713 0
-3.75716876983643 0.00620126724243164
-3.38516712188721 0.0124025344848633
-2.71328926086426 0.0310062170028687
-1.92704510688782 0.0870174169540405
-1.88284945487976 0.092618465423584
-1.35623621940613 0.173434734344482
-1.26709628105164 0.188637733459473
-0.963685750961304 0.254250884056091
-0.890411972999573 0.269853949546814
-0.838535904884338 0.282256484031677
-0.79194450378418 0.297459483146667
-0.744202971458435 0.307861566543579
-0.700735092163086 0.319263815879822
-0.585591316223145 0.352470517158508
-0.502573251724243 0.379275798797607
-0.377287149429321 0.420684099197388
-0.347954750061035 0.431286215782166
0.192201733589172 0.597319483757019
0.676212787628174 0.742748498916626
0.717347621917725 0.752350449562073
0.799983978271484 0.771154165267944
0.831835985183716 0.777755498886108
0.87568473815918 0.78815770149231
1.07050478458405 0.83116626739502
1.2114931344986 0.857971668243408
1.25569093227386 0.866373300552368
1.30430269241333 0.875775098800659
1.35132646560669 0.883776783943176
1.54542648792267 0.913782835006714
1.65665650367737 0.924984931945801
1.80994164943695 0.941988468170166
1.91070771217346 0.95019006729126
1.96159601211548 0.953790783882141
2.12030386924744 0.964392900466919
2.2762451171875 0.973594665527344
2.37244558334351 0.978595733642578
2.95338344573975 0.990998268127441
3.16475677490234 0.993598699569702
4.37151002883911 0.999799966812134
5.41445684432983 1
};


\addplot [plotColor2]
table {%
-7.33479022979736 0
-4.32422018051147 0.00640130043029785
-3.7291898727417 0.0136027336120605
-3.27055621147156 0.0236047506332397
-3.10557055473328 0.0300060510635376
-2.89286208152771 0.0374075174331665
-2.12545895576477 0.0876175165176392
-1.84410297870636 0.119223833084106
-1.36818158626556 0.189037799835205
-1.10382354259491 0.248449683189392
-1.00924015045166 0.271454334259033
-0.956583380699158 0.285057067871094
-0.904273986816406 0.298659801483154
-0.865824937820435 0.308261632919312
-0.763674736022949 0.333866834640503
-0.672462701797485 0.359271883964539
-0.611546754837036 0.375675201416016
-0.565569281578064 0.388277649879456
-0.452761769294739 0.417083382606506
-0.268411874771118 0.468493700027466
-0.221772074699402 0.482896566390991
-0.155807137489319 0.500300049781799
-0.102094650268555 0.51810359954834
-0.0445518493652344 0.5375075340271
0.0358977317810059 0.56111216545105
0.0793545246124268 0.575114965438843
0.8377845287323 0.781156301498413
0.870854139328003 0.787557482719421
0.957480192184448 0.804560899734497
1.01148819923401 0.818163633346558
1.16464948654175 0.848569631576538
1.19410967826843 0.856571316719055
1.30861032009125 0.875775098800659
1.43036675453186 0.893578767776489
1.59028506278992 0.916983366012573
1.66640257835388 0.925384998321533
1.75763964653015 0.932986617088318
1.8655800819397 0.942388534545898
2.00240206718445 0.954190850257874
2.25198101997375 0.967593431472778
2.81094765663147 0.985397100448608
3.61155319213867 0.997799634933472
5.2105827331543 1
};

\addplot [plotColor3, dashed]
table {%
-10.0615301132202 0
-4.63907670974731 0.00620126724243164
-4.33788824081421 0.00920188426971436
-3.61091637611389 0.0186036825180054
-3.48463225364685 0.0212042331695557
-3.22176861763 0.0284056663513184
-2.93201160430908 0.0390077829360962
-2.51739859580994 0.0640127658843994
-1.78021633625031 0.139427900314331
-1.58516013622284 0.169233798980713
-1.3883193731308 0.20084023475647
-1.26505601406097 0.226645350456238
-1.19077849388123 0.239247798919678
-0.766751408576965 0.335867166519165
-0.695323705673218 0.358871817588806
-0.605316638946533 0.383076667785645
-0.462494850158691 0.423884749412537
-0.405903339385986 0.442488431930542
-0.0142900943756104 0.55431079864502
0.0919159650802612 0.586317300796509
0.164441347122192 0.608521699905396
0.205601453781128 0.61932384967804
0.320261359214783 0.64812970161438
0.372431755065918 0.661932468414307
0.500702619552612 0.69953989982605
0.601347804069519 0.722944617271423
0.627511501312256 0.730346083641052
0.754082560539246 0.758751749992371
0.809149503707886 0.77295458316803
0.911285877227783 0.793958783149719
1.02947962284088 0.81856369972229
1.48442232608795 0.894378900527954
1.5636134147644 0.903780698776245
1.63572800159454 0.913782835006714
1.71624088287354 0.924384832382202
1.95060682296753 0.945389032363892
2.07755517959595 0.954590916633606
2.40788793563843 0.970794200897217
2.82563209533691 0.985397100448608
2.93192338943481 0.987597465515137
3.51906418800354 0.996799349784851
6.17277193069458 1
};

\spy[size=1.7cm, magnification=10] on (1.65, 0.17) in node at (1.1, 1.1);
\spy[size=1.7cm, magnification=6] on (4.8, 0.14) in node at (4.7, 1.15);

\end{groupplot}
\end{tikzpicture}

    \caption{Small-scale fading gain statistics for the UMi propagation scenario versus the carrier frequency $f_C$, for both \gls{los} and \gls{nlos} channel conditions.}
    \label{fig:fading_vs_fc}
\end{figure}

%$\mathcal{P} \doteq \mathcal{M} \times \mathcal{\Delta}$, where $\times$ denotes the cartesian product, $ \mathcal{C} \doteq \{ \mathrm{LoS, NLoS} \} $, and $\mathcal{S}$ comprises all the 3GPP scenarios currently implmented in ns-3

The small-scale fading samples have been used to estimate the $\Delta, K$ and $m$ \gls{ftr} parameters, and then derive analytically the values of $V_1$ and $V_2$ yielding the fading realizations that are the closest (in a goodness-of-fit sense) to the TR 38.901 model.
To this end, we defined a discrete grid of \gls{ftr} parameters, spanning their whole domain, and considered the corresponding set of parameterized \gls{ftr} distributions. To find the best matching one, we measured the distance between each of these distributions and the 3GPP reference curves by using the Anderson-Darling goodness-of-fit test~\cite{anderson1954test}. This test is used to discern whether a sorted collection of $n$ samples $\{Y_{1}, \ldots, Y_n\}$ originates from a specific distribution, by evaluating the test statistic~\cite{anderson1954test}
\begin{equation}
    A^2 = -n -S(\mathcal{F}),
\end{equation}
where
\begin{equation}
    S(\mathcal{F}) = \sum _{i=1}^{n} \frac {2i-1}{n} \left[ \ln (\mathcal{F}(Y_{i}))+\ln \left(1-(\mathcal{F}(Y_{n+1-i})\right)\right], 
\end{equation}
and $\mathcal{F}$ is the \gls{cdf} of the target distribution.
In the standard Anderson-Darling test, $A^2$ is then compared to a pre-defined critical value to validate the hypothesis. Instead, in our work we find the \gls{ftr} distribution $\mathcal{F}_{m, K, \Delta}$ which yields the lowest $S$. 
Specifically, for each combination of propagation scenario, \gls{los} condition and corresponding samples $\{Y_{1}, \ldots, Y_n\}$ we find
\begin{equation}
    \mathcal{F}_{m^*, K^*, \Delta^*} \doteq \argmin_{m, K, \Delta} S (\mathcal{F}_{m, K, \Delta}).
\end{equation}
%
Finally, we exported the calibrated \gls{ftr} parameters into ns-3, by storing them in \texttt{SIM\_\-PARAMS\_\-TO\_FTR\_\-PARAMS\_\-TABLE}, i.e., an \texttt{std::map} which associates the propagation scenario and condition to the corresponding best fitting \gls{ftr} parameters.
We remark that this calibration process represents a pre-computation step which needs to be done only once. Indeed, when running a simulation with this channel model, the \gls{ftr} parameters get simply retrieved from the pre-computed lookup table by the \texttt{Get\-Ftr\-Parameters} function. Nevertheless, for the sake of reproducibility and maintainability of the code, we provide this functionality in the Python script \texttt{two-\-ray-\-to-\-three-\-gpp-\-ch-\-calibration.py}.

\subsection{Benchmarks, examples and use cases}
\label{sec:results-ch-perf}

In this section, we provide an example on how to use the performance-oriented channel model presented above, in conjunction with the \gls{nr}~\cite{patriciello2019e2e} module, to simulate \gls{5g} \gls{mimo} networks. Moreover, we present benchmarks which quantify the simulation time reduction achieved with this work, and we outline some possible use cases.

\subsubsection{Examples and Benchmarks}

We demonstrate how to use the performance-oriented channel model in the \texttt{cttc-nr-demo-two-ray} script, i.e., a custom version of the \texttt{cttc-nr-demo} example which is included in the \gls{nr} module.
The script deploys $N_{gNB}$ \gls{5g} \gls{nr} base stations, along with $N_{UE}$ users in each cell. Each \gls{ue} uploads data using two \glspl{bwp} operating at 28 and 30~GHz, respectively. Both base stations and user terminals feature \glspl{upa} with multiple radiating elements.


Most simulation parameters can be tuned by ns-3 users. Notably, the script provides the possibility to choose whether to use the 3GPP TR 38.901 channel model of~\cite{tommaso:20} or the \gls{ftr}-based channel model proposed in this work.
In such regard, the use of the \texttt{Two\-Ray\-Spectrum\-Propagation\-Loss\-Model}, instead of the TR 38.901 one, is achieved by:
\begin{enumerate}
\item Setting the \texttt{TypeId} of the \texttt{Spectrum\-Propagation\-Loss\-Model} factory to \texttt{Two\-Ray\-Spectrum\-Propagation\-Loss\-Model}; 
\item Creating an instance of the \texttt{Two\-Ray\-Spectrum\-Propagation\-Loss\-Model} class using the above factory, and setting the corresponding pointer as the \texttt{Spectrum\-Propagation\-Loss\-Model} of both \glspl{bwp}; 
\item  Setting the attribute \texttt{Frequency} of the \texttt{Two\-Ray\-Spectrum\-Propagation\-Loss\-Model} instance as the \gls{bwp} carrier frequency; 
\item  Specifying the 3GPP propagation scenario by setting the attribute \texttt{Scenario}; and 
\item  Creating and setting the \texttt{Channel\-Condition\-Model} by using the \texttt{Two\-Ray\-Spectrum\-Propagation\-Loss\-Model} class \\ \texttt{ChannelConditionModel} attribute. 
\end{enumerate}

On the other hand, the \texttt{Eigen} optimizations simply require users to have the corresponding library installed in their system, and to enable \texttt{Eigen} when configuring ns-3, using the flag \texttt{enable\--eigen}.

\begin{figure}
    \centering 
    \setlength\fwidth{0.95\columnwidth}
    \setlength\fheight{0.25\columnwidth}
    \input{Figures/ChannelPerformance/Bench3gppOpt}
    \vspace*{-0.2cm}
    \caption{Ratio of the median simulation times after the merge of this work with the Eigen integration ($T_{A}^{3GPP}$) and as per ns-3.37 ($T_{B}$), when using the 3GPP channel model of~\cite{TR38901}.}
    \label{fig:bench_eigen}
\end{figure}

\begin{figure}
    \centering 
    \setlength\fwidth{0.95\columnwidth}
    \setlength\fheight{0.25\columnwidth}
    \begin{tikzpicture}

\definecolor{plotColor1}{HTML}{e60049}
\definecolor{plotColor2}{HTML}{0bb4ff}
\definecolor{plotColor3}{HTML}{87bc45}
\definecolor{plotColor4}{HTML}{ffa300}

 \begin{groupplot}[
  group style={
    group size=2 by 1,
    group name=plots,
    horizontal sep= 0.1 cm
    %ylabels at=edge
  },
    width=0.45\fwidth,
    height=\fheight,
    at={(0\fwidth,0\fheight)},
    scale only axis,
    legend image post style={mark indices={}},
    %axis line style={white!80!black},
    legend style={
        /tikz/every even column/.append style={column sep=0.2cm},
        at={(1, 1.05)}, 
        anchor=south, 
        draw=white!80!black, 
        font=\scriptsize
        },
    legend columns=4,
    %tick align=outside,
    %x grid style={white!80!black},
    xlabel style={font=\footnotesize},
    %xtick={0, 16, 64, 256},
    xmajorgrids,
    %xmajorticks=false,
    xtick style={color=white!15!black},
    %y grid style={white!80!black},
    ylabel shift = -1 pt,
    ylabel style={font=\footnotesize},
    %ylabel={ECDF},
    ymajorgrids,
    ybar,
    /pgf/bar width=0.6cm,
    %ymajorticks=true,
    ymin=0, ymax=0.8,
    ytick style={color=white!15!black},
    yticklabels=\empty
]

\nextgroupplot[%title=vs. Number of antenna elements, 
               every axis title/.style={below, at={(0.5, -0.38)}},
               ylabel={$T_{A}^{TR} / T_{B}$}, 
               yticklabels={0, 0.2, 0.4, 0.6, 0.8, 1},
               ytick={0, 0.2, 0.4, 0.6, 0.8, 1},
               xticklabels={4, 16, 64, 256},
               xtick={1, 2, 3, 4},
               xlabel={Number of antennas at the gNB},
               xmin=0.5, xmax=4.5,]

\addplot[fill=plotColor1] coordinates {
    (1, 0.408498287200928) 
    (2, 0.233031988143921) 
    (3, 0.114968657493591) 
    (4, 0.0595511198043823)
};

\nextgroupplot[%title=vs. number of gNBs, 
               every axis title/.style={below, at={(0.5, -0.38)}},
               xlabel={Number of UEs},
               xtick={1, 2, 3},
               xticklabels={2, 4, 8},
               xmin=0.5, xmax=3.5,]

\addplot[fill=plotColor2] coordinates {
    (1, 0.408498287200928) 
    (2, 0.296834230422974) 
    (3, 0.142506241798401) 
};

\end{groupplot}
\end{tikzpicture}
    \vspace*{-0.2cm}
    \caption{Ratio of the median simulation times using the performance-oriented channel model presented in this work ($T_{A}^{TR}$) and the 3GPP channel model of~\cite{TR38901} after the merge of this work. In this case, \texttt{Eigen} is disabled.}
    \label{fig:bench_tworay}
\end{figure}
 
We validated our contributions by benchmarking the simulation times exhibited by the above simulation script, which depicts a typical \gls{mimo} \gls{5g} \gls{nr} scenario. To such end, we varied the number of \gls{gnb} antennas and \glspl{ue} deployed, and we timed $100$ simulation runs for each parameter combination. Figure~\ref{fig:bench_eigen} reports the ratio of the median simulation time achieved when using the \texttt{Eigen}-based optimizations, and of the same metric obtained using the vanilla ns-3.37. It can be seen that the matrix multiplication routines offered by \texttt{Eigen} can significantly reduce simulation times. For instance, a reduction of $5$ times in the simulation time is achieved when equipping \glspl{gnb} with $256$ radiating elements.
Similarly, Figure~\ref{fig:bench_tworay} depicts the ratio of the median simulation time obtained by using the \gls{ftr}-based channel model, and the 3GPP TR 38.901 with \texttt{Eigen} disabled. In this case the computational complexity improvement is even more dramatic, with simulations taking as low as 6~\% of the time to complete, with respect to the 3GPP model implementation of~\cite{tommaso:20}. As a reference, the median simulation time obtained on an Intel\textsuperscript{\textcopyright} i5-6700 processor system, before the merge of this work and for $\{2, 4, 8\}$ users is $\{64.7, 210.5, 666.6\}$~[s], respectively.


\begin{figure}
    \centering 
    \setlength\fwidth{0.95\columnwidth}
    \setlength\fheight{0.25\columnwidth}
    \begin{tikzpicture}

\definecolor{plotColor1}{HTML}{e60049}
\definecolor{plotColor2}{HTML}{0bb4ff}
\definecolor{plotColor3}{HTML}{87bc45}
\definecolor{plotColor4}{HTML}{ffa300}

 \begin{groupplot}[
  group style={
    group size=2 by 2,
    group name=plots,
    horizontal sep= 0.2 cm,
    vertical sep= 2.0 cm
    %ylabels at=edge
  },
    width=0.45\fwidth,
    height=\fheight,
    at={(0\fwidth,0\fheight)},
    scale only axis,
    legend image post style={mark indices={}},
    %axis line style={white!80!black},
    legend style={
        /tikz/every even column/.append style={column sep=0.2cm},
        at={(1, 1.05)}, 
        anchor=south, 
        draw=white!80!black, 
        font=\scriptsize
        },
    legend columns=4,
    %tick align=outside,
    %x grid style={white!80!black},
    xlabel style={font=\footnotesize},
    %xtick={0, 16, 64, 256},
    xmajorgrids,
    %xmajorticks=false,
    xtick style={color=white!15!black},
    %y grid style={white!80!black},
    ylabel shift = -1 pt,
    ylabel style={font=\footnotesize},
    %ylabel={ECDF},
    ymajorgrids,
    %ymajorticks=true,
    ymin=0, ymax=1.0,
    ytick style={color=white!15!black},
    yticklabels=\empty,
    xlabel={SINR [dB]}
]

\nextgroupplot[title={a) InH-OfficeMixed}, 
               every axis title/.style={below, at={(0.5, -0.4)}},
               ylabel={ECDF}, 
               yticklabels={0, 0.2, 0.4, 0.6, 0.8, 1},
               ytick={0, 0.2, 0.4, 0.6, 0.8, 1},
               xmin=-50, xmax=30]

\addplot [very thick, plotColor1]
table {%
-45.4739990234375 0
-40.6753005981445 0.00277209281921387
-39.1688003540039 0.0101547241210938
-36.8905982971191 0.0340715646743774
-36.0579986572266 0.0453680753707886
-35.5584983825684 0.0524269342422485
-34.9715003967285 0.0612345933914185
-34.3119010925293 0.0718749761581421
-31.5578994750977 0.112768411636353
-30.4475994110107 0.123775839805603
-29.8745994567871 0.130074501037598
-27.6023006439209 0.165666580200195
-27.1711006164551 0.173401832580566
-26.7413005828857 0.18103301525116
-25.8661003112793 0.196359038352966
-25.3099994659424 0.207146763801575
-24.672399520874 0.221513032913208
-23.9913005828857 0.239729642868042
-22.6527004241943 0.283823013305664
-22.1881999969482 0.301380515098572
-21.636999130249 0.322909832000732
-20.2591991424561 0.38132905960083
-15.5363998413086 0.568692445755005
-15.1956996917725 0.581072807312012
-13.5747995376587 0.638569951057434
-13.2358999252319 0.650447487831116
-11.398099899292 0.712211132049561
-10.8994998931885 0.726964712142944
-9.98820972442627 0.75054919719696
-4.94411993026733 0.839574098587036
-4.18298006057739 0.848037838935852
-2.78027009963989 0.85992693901062
0.765547037124634 0.882488131523132
4.09602022171021 0.909321546554565
4.84363985061646 0.915860176086426
5.32557010650635 0.919519662857056
6.24712991714478 0.924942493438721
8.80624961853027 0.936938524246216
13.1233997344971 0.970267295837402
16.2162990570068 0.981925010681152
17.1676006317139 0.988044500350952
19.3040008544922 0.991917848587036
28.786600112915 1
};
\addlegendentry{Two-Ray model}

\addplot [very thick, plotColor2, dashed]
table {%
-47.6212997436523 0
-46.793399810791 0.00453448295593262
-46.6987991333008 0.00649058818817139
-45.4324989318848 0.00925076007843018
-45.1069984436035 0.0133739709854126
-42.7867012023926 0.0161367654800415
-42.7803001403809 0.0181931257247925
-42.657398223877 0.0209664106369019
-42.2151985168457 0.0237820148468018
-42.2122993469238 0.0259965658187866
-42.0999984741211 0.0287541151046753
-41.8367004394531 0.0295871496200562
-41.597599029541 0.0323420763015747
-41.5612983703613 0.0347305536270142
-41.4659996032715 0.0374881029129028
-41.1688995361328 0.0433117151260376
-40.7485008239746 0.0460718870162964
-40.7454986572266 0.0464568138122559
-40.6446990966797 0.0492091178894043
-40.6408996582031 0.052739143371582
-40.6327018737793 0.0541601181030273
-40.4177017211914 0.0569229125976562
-40.4067993164062 0.0573025941848755
-39.7728004455566 0.0600706338882446
-39.7728004455566 0.0636111497879028
-39.2282981872559 0.0663951635360718
-39.1934013366699 0.0692212581634521
-39.1180992126465 0.0719814300537109
-39.0964012145996 0.0725376605987549
-38.9788017272949 0.0752979516983032
-38.7613983154297 0.0773226022720337
-38.3129005432129 0.0801882743835449
-38.2293014526367 0.0807998180389404
-37.9886016845703 0.083554744720459
-37.8941993713379 0.0865812301635742
-37.5946998596191 0.089341402053833
-37.5550003051758 0.0927027463912964
-36.8230018615723 0.096027135848999
-36.6624984741211 0.0987794399261475
-36.5913009643555 0.0999103784561157
-36.2827987670898 0.102686405181885
-36.2793006896973 0.10529899597168
-36.2302017211914 0.108056545257568
-36.0744018554688 0.109068870544434
-35.9659004211426 0.111860752105713
-35.9556999206543 0.113052248954773
-35.6944007873535 0.115807294845581
-35.4617004394531 0.118638634681702
-35.4608993530273 0.123241543769836
-35.3409996032715 0.126017570495605
-35.3409996032715 0.129276037216187
-35.2554016113281 0.132054686546326
-35.2396011352539 0.134379982948303
-35.0358009338379 0.137150645256042
-35.0149002075195 0.138455629348755
-34.9384994506836 0.141207933425903
-34.8009986877441 0.144202828407288
-34.0023002624512 0.150972843170166
-33.4650001525879 0.153743505477905
-33.4650001525879 0.156511664390564
-33.4290008544922 0.159263968467712
-33.3712005615234 0.159907221794128
-33.2421989440918 0.162675380706787
-33.225399017334 0.164715766906738
-33.1568984985352 0.167468070983887
-33.1497993469238 0.168164134025574
-32.7521018981934 0.170937418937683
-32.5457992553711 0.17403507232666
-32.1589012145996 0.178145051002502
-32.0514984130859 0.1832594871521
-31.869499206543 0.191218495368958
-31.8661994934082 0.191479444503784
-31.4423007965088 0.194247603416443
-31.3257999420166 0.196295976638794
-31.1588001251221 0.199064135551453
-30.6082992553711 0.208270072937012
-30.1429004669189 0.211038112640381
-30.1004009246826 0.211707830429077
-29.8603992462158 0.214470624923706
-29.739200592041 0.218034863471985
-29.7164001464844 0.222139596939087
-29.5723991394043 0.224926233291626
-29.5121994018555 0.226442098617554
-29.2744998931885 0.229210138320923
-28.8232002258301 0.234609365463257
-28.7422008514404 0.240132331848145
-28.6702003479004 0.240656971931458
-28.5620002746582 0.243419766426086
-28.0408000946045 0.250746011734009
-27.9706001281738 0.251900792121887
-27.7639007568359 0.254666209220886
-27.4596996307373 0.258082866668701
-27.4596996307373 0.261291265487671
-27.0669002532959 0.26405668258667
-26.8938007354736 0.268973469734192
-26.8262996673584 0.271375060081482
-26.7327003479004 0.27412736415863
-26.7278003692627 0.275893688201904
-26.6868000030518 0.278648614883423
-26.6466999053955 0.282444953918457
-26.6427001953125 0.282769203186035
-26.4330005645752 0.285526752471924
-26.3411998748779 0.288286924362183
-26.017599105835 0.291055083274841
-25.9736995697021 0.293430328369141
-25.7868003845215 0.296187877655029
-25.7856006622314 0.298929691314697
-25.4139995574951 0.301797986030579
-25.4139995574951 0.304597735404968
-25.278600692749 0.307360529899597
-25.0881004333496 0.312047839164734
-24.7728996276855 0.314821243286133
-24.7469005584717 0.318164110183716
-24.6879997253418 0.318944454193115
-24.5205993652344 0.321778416633606
-24.4853000640869 0.323618650436401
-24.3778991699219 0.326410412788391
-24.3453998565674 0.329215407371521
-24.2213001251221 0.331978321075439
-24.2159996032715 0.337198138237
-24.1492004394531 0.341089248657227
-24.1457996368408 0.345805644989014
-23.9090995788574 0.360075950622559
-23.87619972229 0.360777139663696
-23.7880992889404 0.363587498664856
-23.1247997283936 0.375540494918823
-23.0785999298096 0.376950860023499
-22.930700302124 0.379761219024658
-22.9120006561279 0.380493998527527
-22.8411998748779 0.383264780044556
-22.8411998748779 0.386343955993652
-22.737699508667 0.389151573181152
-22.7248992919922 0.39308762550354
-22.6973991394043 0.394909381866455
-22.4491996765137 0.397677421569824
-22.0041007995605 0.404194355010986
-21.9622993469238 0.406946659088135
-21.9559993743896 0.408554792404175
-21.831600189209 0.411317586898804
-21.5967998504639 0.415485620498657
-21.5953006744385 0.421696662902832
-21.5356006622314 0.424470067024231
-21.3966999053955 0.428182005882263
-21.3673000335693 0.430934309959412
-21.1774005889893 0.4368896484375
-21.1599998474121 0.439641952514648
-21.1569004058838 0.440417051315308
-21.0501003265381 0.443185210227966
-20.9776000976562 0.444350481033325
-20.8700008392334 0.447113275527954
-20.8528995513916 0.44936203956604
-20.7434005737305 0.452122211456299
-20.7362003326416 0.454326152801514
-20.6830005645752 0.457083702087402
-20.6805992126465 0.458968639373779
-20.5494995117188 0.461768388748169
-20.4025993347168 0.470990180969238
-20.3213005065918 0.473750352859497
-20.2987003326416 0.477770805358887
-20.1235008239746 0.485139131546021
-20.0960006713867 0.486512660980225
-19.9547996520996 0.489272952079773
-19.9400997161865 0.491917133331299
-19.9309997558594 0.494669437408447
-19.7658996582031 0.49998152256012
-19.7479000091553 0.501505374908447
-19.5809001922607 0.504262924194336
-19.5776996612549 0.504911422729492
-19.4463996887207 0.507684826850891
-19.4309997558594 0.508502006530762
-19.3906002044678 0.51127016544342
-19.3875007629395 0.515498876571655
-19.3719997406006 0.518251061439514
-19.0492000579834 0.533723592758179
-19.005500793457 0.539309740066528
-18.7584991455078 0.542085886001587
-18.736499786377 0.543572664260864
-18.4617004394531 0.546390891075134
-18.4057006835938 0.549760103225708
-18.3950004577637 0.553527355194092
-18.2590007781982 0.556574940681458
-18.2254009246826 0.559329867362976
-18.2236995697021 0.562422275543213
-18.2005996704102 0.565182447433472
-18.1620006561279 0.567341566085815
-18.1051998138428 0.570099115371704
-18.0611000061035 0.576518535614014
-17.8497009277344 0.579315662384033
-17.8426990509033 0.586135745048523
-17.6166000366211 0.588911771774292
-17.550500869751 0.594366312026978
-17.5272006988525 0.596844434738159
-17.5205993652344 0.599596619606018
-17.5130004882812 0.60409152507782
-17.3082008361816 0.608821034431458
-17.0765991210938 0.612401127815247
-17.0765991210938 0.615796685218811
-16.9361000061035 0.618635892868042
-16.8320999145508 0.622229337692261
-16.8255996704102 0.62498152256012
-16.824499130249 0.627936840057373
-16.6142997741699 0.630739212036133
-16.6081008911133 0.633715629577637
-16.578800201416 0.636483669281006
-16.5298004150391 0.640593767166138
-16.2488994598389 0.643446207046509
-16.1562004089355 0.645022630691528
-15.9151000976562 0.647801399230957
-15.8965997695923 0.651384115219116
-15.6806001663208 0.654168009757996
-15.6759004592896 0.657632112503052
-15.5720996856689 0.662609338760376
-15.3655996322632 0.665422320365906
-15.2554998397827 0.668728232383728
-15.2461004257202 0.672345161437988
-15.147500038147 0.675479888916016
-15.1312999725342 0.677283048629761
-14.9371004104614 0.680103898048401
-14.9031000137329 0.682803392410278
-14.789999961853 0.685574173927307
-14.7835998535156 0.687419652938843
-14.5874996185303 0.690264225006104
-14.5221004486084 0.694785356521606
-14.5043001174927 0.696351289749146
-14.4372997283936 0.699111580848694
-14.4267997741699 0.700432300567627
-14.2968997955322 0.703234672546387
-14.0559997558594 0.708828926086426
-14.0434999465942 0.711697220802307
-13.9729995727539 0.714449524879456
-13.9277000427246 0.715074300765991
-13.7215003967285 0.717884659767151
-13.600700378418 0.720020055770874
-13.5291996002197 0.722777605056763
-13.5124998092651 0.724986791610718
-13.4561996459961 0.727744340896606
-13.4153003692627 0.730813026428223
-13.2875003814697 0.733573198318481
-13.1913995742798 0.741787910461426
-13.1232995986938 0.744173765182495
-13.1038999557495 0.746926069259644
-13.0875997543335 0.747516632080078
-13.0574998855591 0.750274181365967
-13.0574998855591 0.753788352012634
-12.8814001083374 0.756603956222534
-12.7177000045776 0.759778022766113
-12.6565999984741 0.766250133514404
-12.5255002975464 0.769018173217773
-12.4750995635986 0.772223949432373
-12.4418001174927 0.773658037185669
-12.3712997436523 0.776428937911987
-12.3346996307373 0.782856225967407
-12.1560001373291 0.785692811012268
-11.8842000961304 0.79220175743103
-11.8842000961304 0.795180797576904
-11.8308000564575 0.797938346862793
-11.8308000564575 0.800777673721313
-11.7645998001099 0.803543210029602
-11.7019996643066 0.80504846572876
-11.637900352478 0.807811260223389
-11.1338996887207 0.816838026046753
-11.1276998519897 0.819579839706421
-10.953200340271 0.82233738899231
-9.93025970458984 0.840356349945068
-9.83695983886719 0.845779299736023
-9.4212703704834 0.848547458648682
-9.413330078125 0.850514054298401
-8.83522033691406 0.853324413299561
-8.48256969451904 0.857418537139893
-8.36312961578369 0.862242937088013
-7.8860502243042 0.865018963813782
-7.84246015548706 0.867338895797729
-7.67691993713379 0.870099067687988
-7.66525983810425 0.871401429176331
-7.08432006835938 0.874164342880249
-7.08432006835938 0.87701153755188
-6.54679012298584 0.879787445068359
-6.28539991378784 0.880654811859131
-5.89286994934082 0.883486270904541
-5.83372020721436 0.886494278907776
-5.81855010986328 0.887733340263367
-5.75659990310669 0.890485525131226
-5.65750980377197 0.891975164413452
-5.17423009872437 0.894806504249573
-5.07011985778809 0.897780179977417
-5.066810131073 0.899209022521973
-4.11230993270874 0.902312040328979
-4.11230993270874 0.905201435089111
-3.53975009918213 0.908030152320862
-3.17685008049011 0.910903692245483
-2.9661500453949 0.918649196624756
-0.165730953216553 0.921448945999146
-0.038481593132019 0.92443323135376
1.10673999786377 0.929315567016602
1.8716299533844 0.93207311630249
1.89564001560211 0.93407678604126
2.45919990539551 0.936831712722778
2.67911005020142 0.940562009811401
8.37812995910645 0.948874235153198
9.25216960906982 0.951908707618713
10.2245998382568 0.960832595825195
10.8753004074097 0.96423602104187
11.5426998138428 0.970238924026489
11.7455997467041 0.973760962486267
11.7756996154785 0.975187182426453
11.9947004318237 0.977942109107971
13.3636999130249 0.9815354347229
14.2082004547119 0.984292984008789
14.2839002609253 0.984864950180054
15.3912000656128 0.987619876861572
16.1329002380371 0.996422529220581
16.951000213623 0.999333024024963
23.569299697876 0.99999737739563
23.8798007965088 1
};
\addlegendentry{3GPP TR 38.901 model}

\nextgroupplot[title={b) RMa}, 
               every axis title/.style={below, at={(0.5, -0.4)}},
               xmin=-50, xmax=30]

\addplot [very thick, plotColor1]
table {%
-58.4241981506348 0
-49.8932991027832 0.00292646884918213
-46.6567001342773 0.00893771648406982
-42.3497009277344 0.012831449508667
-38.9543991088867 0.0236983299255371
-37.2331008911133 0.0330722332000732
-35.0691986083984 0.0432584285736084
-31.7752990722656 0.0519206523895264
-29.2413005828857 0.056199312210083
-25.6975994110107 0.0570739507675171
-12.8933000564575 0.0598253011703491
-11.4333000183105 0.0633151531219482
-9.58489990234375 0.0760228633880615
-7.48761987686157 0.105137705802917
-7.06300020217896 0.116045832633972
-6.62382984161377 0.129128336906433
-5.791100025177 0.156539559364319
-5.11824989318848 0.178672552108765
-4.25802993774414 0.20904266834259
-2.70146989822388 0.268002152442932
-2.34332990646362 0.281754016876221
-1.96256995201111 0.296666383743286
-0.924432992935181 0.344414353370667
2.30526995658875 0.531868457794189
2.84188008308411 0.55970025062561
4.91652011871338 0.663451194763184
5.82232999801636 0.70700204372406
7.28779983520508 0.769652247428894
7.83456993103027 0.789007186889648
8.95771980285645 0.823052287101746
11.5017004013062 0.892273902893066
12.5813999176025 0.913612484931946
15.7042999267578 0.961414098739624
16.2584991455078 0.966899871826172
18.0422992706299 0.979759216308594
19.1798000335693 0.984792113304138
26.6352005004883 0.999830484390259
28.213399887085 1
};
\addplot [very thick, plotColor2, dashed]
table {%
-50.9659996032715 0
-50.4021987915039 0.00184690952301025
-48.8381004333496 0.00460243225097656
-42.7760009765625 0.0150651931762695
-41.3692016601562 0.0178284645080566
-40.8376998901367 0.0195218324661255
-39.9561004638672 0.0222872495651245
-39.2643013000488 0.0253058671951294
-38.7924003601074 0.0285885334014893
-38.1789016723633 0.0314159393310547
-37.5993003845215 0.0333513021469116
-37.3193016052246 0.0361123085021973
-33.1436996459961 0.0524519681930542
-31.869499206543 0.0552053451538086
-28.722900390625 0.058049201965332
-27.5632991790771 0.0602631568908691
-14.4788999557495 0.0630717277526855
-10.0645999908447 0.0797485113143921
-9.79870986938477 0.0833418369293213
-9.55253028869629 0.0860950946807861
-9.51585006713867 0.0883764028549194
-9.15447998046875 0.0911440849304199
-8.80432987213135 0.0948159694671631
-8.54833984375 0.0975725650787354
-7.36753988265991 0.118464946746826
-7.32486009597778 0.119724988937378
-7.0983099937439 0.122480511665344
-7.0208101272583 0.124686598777771
-6.94729995727539 0.127442121505737
-6.68981981277466 0.131300806999207
-6.62122011184692 0.135174870491028
-6.53788995742798 0.142999291419983
-6.47425985336304 0.146379232406616
-6.3654899597168 0.152297019958496
-6.1628999710083 0.162892460823059
-6.08213996887207 0.165645718574524
-6.05484008789062 0.166379690170288
-5.88591003417969 0.169140696525574
-5.81043004989624 0.172798156738281
-5.77866983413696 0.177757620811462
-5.70755004882812 0.180523037910461
-5.68997001647949 0.182933688163757
-5.60559988021851 0.185689210891724
-5.60375022888184 0.18941855430603
-5.48010015487671 0.192177295684814
-5.31640005111694 0.200553297996521
-5.19918012619019 0.203308820724487
-5.12245988845825 0.208600997924805
-4.96643018722534 0.218319892883301
-4.80440998077393 0.221084237098694
-4.71105003356934 0.225092053413391
-4.63222980499268 0.227028608322144
-4.62265014648438 0.229779601097107
-4.45223999023438 0.238526940345764
-4.44140005111694 0.240621566772461
-4.35048007965088 0.243374824523926
-4.26680994033813 0.246366858482361
-4.19108009338379 0.250108361244202
-4.08849000930786 0.254810333251953
-4.03008985519409 0.25756573677063
-3.84674000740051 0.265047550201416
-3.7260000705719 0.271075844764709
-3.63474011421204 0.27383029460907
-3.52435994148254 0.277897834777832
-3.52007007598877 0.278606295585632
-3.45712995529175 0.281359553337097
-3.44497990608215 0.283056259155273
-3.37649011611938 0.28581166267395
-3.33191990852356 0.289091110229492
-3.2314600944519 0.294009685516357
-3.11728000640869 0.299149394035339
-2.9994900226593 0.30190372467041
-2.8218400478363 0.307631492614746
-2.78885006904602 0.311957597732544
-2.67360997200012 0.315018177032471
-2.63050007820129 0.319443702697754
-2.54071998596191 0.322836995124817
-2.12392997741699 0.336241006851196
-2.05392003059387 0.339727163314819
-2.00805997848511 0.345080137252808
-1.91630005836487 0.347833395004272
-1.88935005664825 0.352692246437073
-1.57562005519867 0.37048876285553
-1.5460000038147 0.374152779579163
-1.5375599861145 0.377047538757324
-1.47333002090454 0.379799723625183
-1.37747001647949 0.384752631187439
-1.36968994140625 0.386403918266296
-1.34267997741699 0.389155030250549
-1.29263997077942 0.392378091812134
-1.28630995750427 0.394898176193237
-1.19221997261047 0.397650361061096
-1.16118001937866 0.402511477470398
-1.13093996047974 0.40575110912323
-1.12793004512787 0.407942891120911
-1.09247004985809 0.410698413848877
-1.0650600194931 0.411525249481201
-1.02002000808716 0.414277315139771
-1.01110005378723 0.420140981674194
-0.989364981651306 0.422395825386047
-0.887906074523926 0.42515242099762
-0.874979019165039 0.428248405456543
-0.819818019866943 0.431002736091614
-0.69010591506958 0.438047885894775
-0.635810017585754 0.443631887435913
-0.552250027656555 0.446388483047485
-0.49685001373291 0.449653625488281
-0.487515926361084 0.452404737472534
-0.451372027397156 0.456612586975098
-0.37108302116394 0.459953904151917
-0.340479016304016 0.463571548461914
-0.189784049987793 0.475393772125244
-0.161568999290466 0.477960348129272
-0.0334948301315308 0.48072361946106
0.0414415597915649 0.485772609710693
0.0880346298217773 0.487779855728149
0.152194023132324 0.490536451339722
0.230947017669678 0.500043153762817
0.262408018112183 0.50068199634552
0.332131028175354 0.503437519073486
0.372614026069641 0.51108717918396
0.527071952819824 0.513852715492249
0.563940048217773 0.514214158058167
0.635549068450928 0.51697301864624
0.640614032745361 0.518090486526489
0.679497003555298 0.520843744277954
0.685369968414307 0.522444248199463
0.714874982833862 0.525196433067322
0.753136992454529 0.528598546981812
0.802029013633728 0.531460165977478
0.823423981666565 0.532903671264648
0.855869054794312 0.535656929016113
0.87897002696991 0.538803815841675
0.905153036117554 0.540411949157715
0.937005996704102 0.543164134025574
1.01874005794525 0.546598315238953
1.1470799446106 0.549352765083313
1.20359003543854 0.552385687828064
1.27945005893707 0.555139064788818
1.31082999706268 0.558619499206543
1.35362005233765 0.561371803283691
1.37881004810333 0.565202713012695
1.4543399810791 0.567975997924805
1.80118000507355 0.5857834815979
1.89962995052338 0.589838743209839
1.91898000240326 0.592593193054199
2.04596996307373 0.602057933807373
2.06329011917114 0.603206276893616
2.07985997200012 0.605958461761475
2.1174099445343 0.609998345375061
2.19837999343872 0.61539888381958
2.28719997406006 0.619488477706909
2.30788993835449 0.623623371124268
2.39478993415833 0.626375555992126
2.53241991996765 0.632019281387329
2.61799001693726 0.636014938354492
2.7939600944519 0.642764925956726
2.80258989334106 0.645965814590454
2.83927989006042 0.649569153785706
2.87779998779297 0.653648853302002
2.98864006996155 0.65896201133728
3.30221009254456 0.680735349655151
3.34881997108459 0.68349301815033
3.38190007209778 0.684031248092651
3.45960998535156 0.686784505844116
3.62298989295959 0.698038816452026
3.69097995758057 0.700790882110596
3.73451995849609 0.704175353050232
3.82296991348267 0.707312226295471
3.84221005439758 0.711777687072754
3.89553999900818 0.712749242782593
4.00871992111206 0.715506911277771
4.04549980163574 0.718697905540466
4.05563020706177 0.720927238464355
4.20059013366699 0.723690509796143
4.31102991104126 0.730031728744507
4.35987997055054 0.731852054595947
4.42279005050659 0.734607577323914
4.56692981719971 0.739394664764404
4.60377979278564 0.742146849632263
4.96318006515503 0.758758306503296
5.02218008041382 0.764482736587524
5.15709018707275 0.769207954406738
5.26599979400635 0.771962285041809
5.29262018203735 0.77498197555542
5.3167200088501 0.775692701339722
5.35607004165649 0.778452634811401
5.36941003799438 0.779154539108276
5.42959022521973 0.781915545463562
5.78860998153687 0.800768613815308
5.83735990524292 0.805437326431274
5.93683004379272 0.808190703392029
5.95499992370605 0.809016346931458
6.05815982818604 0.811789512634277
6.13186979293823 0.815056800842285
6.19983005523682 0.81972336769104
6.29650020599365 0.823154211044312
6.389319896698 0.82590651512146
6.50393009185791 0.829855680465698
6.51413011550903 0.831959009170532
6.60607004165649 0.834717988967896
6.60964012145996 0.836855530738831
6.67682981491089 0.839609980583191
6.68046998977661 0.840119481086731
6.82298994064331 0.84287166595459
6.94294023513794 0.848851442337036
7.01065015792847 0.851606845855713
7.38243007659912 0.865309238433838
8.03359031677246 0.880825519561768
8.2101001739502 0.884645462036133
8.24102973937988 0.886417150497437
8.45075988769531 0.889240145683289
8.6135196685791 0.892474174499512
8.799880027771 0.89551830291748
9.17061042785645 0.900190353393555
9.44515037536621 0.907512903213501
9.46395969390869 0.911099672317505
9.65888977050781 0.914782524108887
9.76014995574951 0.917542457580566
9.78670978546143 0.920548915863037
9.87034034729004 0.923302173614502
10.249400138855 0.9318528175354
10.296199798584 0.932384371757507
10.4107999801636 0.9351487159729
10.4725999832153 0.936022996902466
10.6022996902466 0.938779592514038
10.6978998184204 0.942935585975647
10.8538999557495 0.945691108703613
11.454400062561 0.95275616645813
11.5804996490479 0.954817533493042
11.8708000183105 0.957577466964722
11.9554004669189 0.958975672721863
12.2978000640869 0.961740016937256
12.3662004470825 0.963695287704468
12.9359998703003 0.966456413269043
13.5927000045776 0.97378671169281
13.6562995910645 0.975575089454651
13.7693004608154 0.978349447250366
15.0324001312256 0.993057608604431
16.8943996429443 0.995844125747681
25.6613006591797 0.999226331710815
26.0673999786377 1
};

\nextgroupplot[title={c) UMa}, 
               every axis title/.style={below, at={(0.5, -0.38)}},
               ylabel={ECDF}, 
               yticklabels={0, 0.2, 0.4, 0.6, 0.8, 1},
               ytick={0, 0.2, 0.4, 0.6, 0.8, 1},
               xmin=-45, xmax=38]

\addplot [very thick, plotColor1]
table {%
-39.4786987304688 0
-32.8926010131836 0.00276100635528564
-30.2686004638672 0.0092475414276123
-28.4372005462646 0.0139545202255249
-23.8812007904053 0.0213416814804077
-21.0468006134033 0.0249406099319458
-19.1611003875732 0.0319890975952148
-17.7187995910645 0.0418624877929688
-16.5771007537842 0.0513460636138916
-15.4484996795654 0.063957691192627
-14.7200002670288 0.0744386911392212
-13.7468004226685 0.09145188331604
-13.0558996200562 0.105454206466675
-12.2981996536255 0.122702240943909
-11.1494998931885 0.152092218399048
-10.5424995422363 0.16818642616272
-9.70034980773926 0.190482258796692
-7.96398019790649 0.233806252479553
-7.34973001480103 0.245786309242249
-6.91192007064819 0.254706740379333
-6.43833017349243 0.266777992248535
-6.10129976272583 0.277488231658936
-5.51923990249634 0.299781680107117
-5.14356994628906 0.314746260643005
-2.43891000747681 0.408248901367188
-0.924937009811401 0.467009425163269
0.397819995880127 0.510813236236572
4.87529993057251 0.660893440246582
6.53020000457764 0.736389994621277
7.19152021408081 0.765990853309631
7.58550977706909 0.779619336128235
8.30426979064941 0.797728061676025
9.02287006378174 0.811772584915161
11.0443000793457 0.851796388626099
12.1319999694824 0.882447004318237
12.6979999542236 0.897444725036621
13.6245002746582 0.914726972579956
14.5567998886108 0.92638897895813
15.0443000793457 0.932916402816772
17.6961994171143 0.970134735107422
18.6513996124268 0.975949764251709
25.0723991394043 0.996044278144836
29.5380001068115 1
};
\addplot [very thick, plotColor2, dashed]
table {%
-37.4415016174316 0
-34.6418991088867 0.00280141830444336
-29.550500869751 0.0146421194076538
-26.9176006317139 0.0185360908508301
-26.7423000335693 0.0188848972320557
-26.4561996459961 0.0216383934020996
-26.4256992340088 0.0222846269607544
-26.0548000335693 0.0250426530838013
-25.7290992736816 0.0279425382614136
-25.614200592041 0.0295428037643433
-24.2124004364014 0.0323100090026855
-23.5431995391846 0.0372231006622314
-22.4682006835938 0.0399891138076782
-20.6721000671387 0.0466307401657104
-20.6296005249023 0.0468343496322632
-20.3402996063232 0.0495979785919189
-20.3285999298096 0.0503931045532227
-19.593599319458 0.0531522035598755
-18.4853992462158 0.0588740110397339
-18.2000999450684 0.0616366863250732
-18.136999130249 0.0629830360412598
-17.8868999481201 0.0657376050949097
-17.6431999206543 0.0691463947296143
-17.5408000946045 0.071378231048584
-17.2908992767334 0.0741350650787354
-17.1177005767822 0.0769777297973633
-16.9712009429932 0.0790024995803833
-16.5911998748779 0.0817673206329346
-16.5797996520996 0.0824112892150879
-16.4591999053955 0.0851705074310303
-15.2978000640869 0.101204752922058
-15.272500038147 0.10155713558197
-15.012900352478 0.104316234588623
-14.9540004730225 0.106723070144653
-14.8554000854492 0.109493613243103
-14.6471996307373 0.113235354423523
-14.4746999740601 0.118647217750549
-14.3810997009277 0.121407508850098
-14.2802000045776 0.124804973602295
-14.2087001800537 0.127569794654846
-14.167200088501 0.128949403762817
-13.9575004577637 0.131773710250854
-12.9713001251221 0.156944513320923
-12.9120998382568 0.157596588134766
-12.692099571228 0.160390019416809
-12.2810001373291 0.172635674476624
-11.5620002746582 0.188386201858521
-11.5043001174927 0.189842462539673
-11.3867998123169 0.192607283592224
-10.7680997848511 0.203042149543762
-10.4056997299194 0.215671062469482
-10.3231000900269 0.218432426452637
-10.2532997131348 0.219653010368347
-9.96891975402832 0.222455620765686
-9.92072010040283 0.225718021392822
-9.74792957305908 0.228478312492371
-9.73810005187988 0.229401469230652
-9.68677997589111 0.232160568237305
-9.264479637146 0.245216131210327
-9.13978958129883 0.247970700263977
-9.05214977264404 0.250777840614319
-8.97402000427246 0.25355076789856
-8.95633029937744 0.25493597984314
-8.75650978088379 0.257731676101685
-8.60453987121582 0.261539816856384
-8.50152015686035 0.262915968894958
-8.26716995239258 0.265684247016907
-7.88964986801147 0.274125218391418
-6.48802995681763 0.303097248077393
-6.4207501411438 0.30733323097229
-6.36444997787476 0.309767484664917
-6.28544998168945 0.312524318695068
-6.2510199546814 0.314881920814514
-6.11001014709473 0.317646741867065
-6.09533023834229 0.318826198577881
-6.04491996765137 0.32157838344574
-6.02301979064941 0.323049545288086
-5.95650005340576 0.32580292224884
-5.89019012451172 0.328992128372192
-5.79235982894897 0.331755876541138
-5.74218988418579 0.334954261779785
-5.73230981826782 0.336603760719299
-5.68632984161377 0.339357137680054
-5.36496019363403 0.354718923568726
-5.21649980545044 0.361667037010193
-5.1911301612854 0.364427328109741
-5.11529016494751 0.371962308883667
-5.06044006347656 0.375968337059021
-4.9608302116394 0.378728628158569
-4.95199012756348 0.380018949508667
-4.86016988754272 0.382775783538818
-4.60867023468018 0.386249899864197
-4.56633996963501 0.38900101184845
-4.54554986953735 0.390352010726929
-4.51279020309448 0.393103122711182
-4.40248012542725 0.401461720466614
-4.23443984985352 0.404232263565063
-4.16966009140015 0.405933260917664
-4.07239007949829 0.408686637878418
-4.06918001174927 0.413606643676758
-4.02384996414185 0.416360139846802
-3.70341992378235 0.430945038795471
-3.59283995628357 0.435567617416382
-3.49324989318848 0.43831992149353
-3.44758009910583 0.439393997192383
-3.39208006858826 0.442146301269531
-3.18980002403259 0.448813080787659
-3.14805006980896 0.451564192771912
-3.12339997291565 0.455402016639709
-3.04986000061035 0.458153128623962
-3.02305006980896 0.460993528366089
-2.99435997009277 0.4622483253479
-2.93557000160217 0.46500301361084
-2.90111994743347 0.470841526985168
-2.73887991905212 0.473597168922424
-2.72316002845764 0.474847555160522
-2.58741998672485 0.477600932121277
-2.57021999359131 0.482261180877686
-2.44525003433228 0.485706806182861
-2.44333004951477 0.488862752914429
-2.41534996032715 0.491613864898682
-2.4073600769043 0.492569088935852
-2.30225992202759 0.495323657989502
-2.18653988838196 0.500617742538452
-2.13968992233276 0.505578875541687
-2.06703996658325 0.508332252502441
-2.05022001266479 0.512822151184082
-1.80066001415253 0.518326759338379
-1.75436997413635 0.522045612335205
-1.72091996669769 0.526023030281067
-1.62404000759125 0.528775215148926
-1.09297001361847 0.537286043167114
-1.06152999401093 0.540538191795349
-0.976733922958374 0.543289303779602
-0.926949024200439 0.546166181564331
-0.344895958900452 0.561399817466736
-0.305698037147522 0.564357995986938
-0.241829037666321 0.568294286727905
-0.170402050018311 0.572078227996826
-0.021994948387146 0.579071044921875
0.0358850955963135 0.582390785217285
0.0917273759841919 0.585203647613525
0.137346982955933 0.587536096572876
0.37601900100708 0.590296387672424
0.562067031860352 0.596903562545776
0.663299083709717 0.600273609161377
0.817605018615723 0.603712320327759
0.822067022323608 0.606847763061523
1.07721996307373 0.609616041183472
1.12170994281769 0.610794305801392
1.22748994827271 0.61354649066925
1.50346004962921 0.621694564819336
1.66396999359131 0.624449253082275
1.76621997356415 0.628277897834778
1.96609997749329 0.631033658981323
1.99215996265411 0.632871866226196
2.04622006416321 0.635626435279846
2.05353999137878 0.638672709465027
2.15040993690491 0.641430616378784
2.52495002746582 0.65289044380188
2.71029996871948 0.656007647514343
2.89335989952087 0.662022352218628
3.19167995452881 0.664786100387573
3.34090995788574 0.668853759765625
3.40387010574341 0.672559022903442
3.52565002441406 0.675320386886597
4.00820016860962 0.687870264053345
4.18964004516602 0.693744421005249
4.29778003692627 0.6965012550354
4.30902004241943 0.697692036628723
4.37258005142212 0.700443148612976
4.57655000686646 0.707847714424133
4.7548999786377 0.717378854751587
4.81091976165771 0.721857309341431
4.91706991195679 0.724611878395081
5.01006984710693 0.729904770851135
5.22018003463745 0.740746855735779
5.26666021347046 0.743907451629639
5.41473007202148 0.752188324928284
5.60019016265869 0.757997155189514
5.71787977218628 0.762515544891357
5.7943000793457 0.766402721405029
5.81519985198975 0.768280982971191
5.87426996231079 0.771043539047241
5.8932900428772 0.772214889526367
5.93581008911133 0.774967193603516
6.01294994354248 0.77777886390686
6.24042987823486 0.789224982261658
6.32308006286621 0.791977167129517
6.38739013671875 0.794535040855408
6.42944002151489 0.797291874885559
6.46678018569946 0.803121328353882
6.52338981628418 0.805875778198242
6.5422101020813 0.807961225509644
6.67819023132324 0.810714602470398
6.72669982910156 0.812280654907227
6.82743978500366 0.815032958984375
6.87639999389648 0.818500161170959
6.97458982467651 0.823023200035095
6.99035978317261 0.824493169784546
7.0726900100708 0.8272465467453
7.24381017684937 0.834550380706787
7.25694990158081 0.835816860198975
7.41174983978271 0.838569045066833
7.43344020843506 0.841411709785461
7.51516008377075 0.844167470932007
7.52059984207153 0.846148729324341
7.67204999923706 0.848907709121704
7.93291997909546 0.853198647499084
8.86886978149414 0.871249675750732
9.07705974578857 0.874003052711487
9.37397956848145 0.880160808563232
9.37977027893066 0.88081169128418
9.58530044555664 0.883569717407227
9.58782958984375 0.88471245765686
9.94871044158936 0.887520790100098
10.4249000549316 0.893037796020508
10.5908002853394 0.896875739097595
10.964599609375 0.901128768920898
11.0282001495361 0.903883457183838
11.3998003005981 0.914476037025452
11.5268001556396 0.92024827003479
11.5586996078491 0.922016859054565
11.6468000411987 0.924775838851929
11.6641998291016 0.928956985473633
11.8662004470825 0.931713819503784
11.8692998886108 0.932724952697754
11.9469995498657 0.935479640960693
11.9762001037598 0.938712239265442
12.0736999511719 0.941465616226196
12.0978002548218 0.942173719406128
12.4385004043579 0.944945454597473
12.444299697876 0.94687294960022
12.7167997360229 0.949634432792664
12.9028997421265 0.953300714492798
12.9392995834351 0.954417109489441
13.1651000976562 0.957170486450195
13.1691999435425 0.958046793937683
13.5699996948242 0.960805892944336
13.7115001678467 0.962361574172974
14.0141000747681 0.965116262435913
15.003999710083 0.973351240158081
15.1162996292114 0.974155426025391
15.3997001647949 0.976909995079041
15.5879001617432 0.978629350662231
15.8931999206543 0.981383800506592
15.9921998977661 0.983112335205078
16.3703994750977 0.985878348350525
18.195499420166 0.996044397354126
22.4463005065918 0.998826265335083
22.9507007598877 0.999249577522278
26.8600997924805 1
};

\nextgroupplot[title={d) UMi-Street Canyon}, 
               every axis title/.style={below, at={(0.5, -0.38)}},
               xmin=-45, xmax=35]
               
\addplot [very thick, plotColor1]
table {%
-41.4580001831055 0
-36.3513984680176 0.00299692153930664
-33.2946014404297 0.0146913528442383
-31.2292003631592 0.0312420129776001
-30.7192993164062 0.0360467433929443
-27.4433002471924 0.0691410303115845
-26.2604007720947 0.0776911973953247
-23.363899230957 0.104468703269958
-21.4120006561279 0.130834579467773
-18.3808002471924 0.181688547134399
-17.5382995605469 0.197242975234985
-16.7485008239746 0.213186740875244
-13.5452995300293 0.290858030319214
-12.8002004623413 0.310103893280029
-11.2638998031616 0.353930234909058
-10.600700378418 0.373579502105713
-9.60914993286133 0.399398326873779
-7.94122982025146 0.435102462768555
-6.45549011230469 0.473974347114563
-5.7975001335144 0.493845820426941
-4.26134014129639 0.533821582794189
-3.42125010490417 0.565641164779663
-2.64226007461548 0.600657939910889
-1.65195000171661 0.634684085845947
0.129503011703491 0.685104131698608
1.3973400592804 0.722796201705933
1.87323999404907 0.738181114196777
3.57693004608154 0.796469569206238
6.47046995162964 0.871368408203125
7.30600023269653 0.886711120605469
8.25782012939453 0.907030463218689
9.2252197265625 0.924150466918945
13.1393995285034 0.970296859741211
13.8284997940063 0.974733233451843
19.3164005279541 0.994880676269531
25.074800491333 1
};
\addplot [very thick, plotColor2, dashed]
table {%
-46.0313987731934 0
-39.8997993469238 0.00288355350494385
-37.515998840332 0.0106054544448853
-36.9576988220215 0.0138105154037476
-35.9042015075684 0.0165715217590332
-35.8978004455566 0.0176483392715454
-35.0415992736816 0.0204312801361084
-35.0046997070312 0.02088463306427
-34.8599014282227 0.0236375331878662
-34.7481994628906 0.0262296199798584
-34.4382019042969 0.0289825201034546
-33.7097015380859 0.0363689661026001
-33.5942001342773 0.0381420850753784
-33.3434982299805 0.0408960580825806
-32.9823989868164 0.0447547435760498
-30.5209999084473 0.0676323175430298
-30.4225997924805 0.0703862905502319
-30.3211994171143 0.0732028484344482
-29.7879009246826 0.0759649276733398
-29.5312004089355 0.0798790454864502
-29.5191993713379 0.081498384475708
-29.2598991394043 0.0842512845993042
-29.1086006164551 0.0848145484924316
-28.980899810791 0.0875697135925293
-28.888500213623 0.091028094291687
-28.8043994903564 0.092456579208374
-28.5718002319336 0.0952163934707642
-28.2996997833252 0.100958108901978
-28.2674999237061 0.102716207504272
-28.0284004211426 0.105473637580872
-27.3533992767334 0.112141847610474
-26.7742004394531 0.119036674499512
-26.4626007080078 0.121798753738403
-26.4097003936768 0.122432589530945
-26.1149005889893 0.125213265419006
-26.0443000793457 0.130186915397644
-25.767599105835 0.132951259613037
-25.6336994171143 0.136015295982361
-25.6032009124756 0.136615633964539
-25.2936000823975 0.139380097389221
-25.2901992797852 0.140001177787781
-25.0977993011475 0.142760992050171
-25.0324993133545 0.142940282821655
-24.706600189209 0.145701169967651
-24.558500289917 0.14759349822998
-24.2541007995605 0.150352120399475
-24.100700378418 0.1535804271698
-23.4862003326416 0.159283876419067
-23.1070003509521 0.162064552307129
-23.0941009521484 0.163259267807007
-22.8346996307373 0.166022539138794
-22.2994003295898 0.169198751449585
-22.0916004180908 0.172139048576355
-21.3712005615234 0.18372642993927
-20.6779003143311 0.197980046272278
-20.3567008972168 0.202524542808533
-20.2796993255615 0.203623294830322
-20.1399002075195 0.206414341926575
-19.9762001037598 0.210121512413025
-19.7455997467041 0.212883591651917
-18.6424007415771 0.226411938667297
-18.5368003845215 0.229164719581604
-18.2345008850098 0.232394218444824
-17.9382991790771 0.237333059310913
-17.8649997711182 0.240087151527405
-15.9330997467041 0.279587149620056
-15.7975997924805 0.282342314720154
-15.7946996688843 0.28371524810791
-15.5974998474121 0.286484241485596
-15.3965997695923 0.289518237113953
-15.1597995758057 0.293937802314758
-15.0712003707886 0.296810984611511
-15.058500289917 0.298282265663147
-14.9799995422363 0.301040887832642
-14.8559999465942 0.305397987365723
-14.5768003463745 0.308178663253784
-14.5146999359131 0.310608744621277
-14.4075002670288 0.31336510181427
-13.9457998275757 0.321808815002441
-13.8767004013062 0.324566245079041
-13.3880996704102 0.33562970161438
-13.2531003952026 0.338383674621582
-13.0364999771118 0.341687202453613
-12.7572002410889 0.344446897506714
-12.6905002593994 0.345527291297913
-12.4654998779297 0.34829044342041
-12.3016996383667 0.353143930435181
-12.1395998001099 0.355922222137451
-11.8387002944946 0.365804672241211
-11.5896997451782 0.370434761047363
-11.5426998138428 0.37090790271759
-11.497200012207 0.373663067817688
-11.0453996658325 0.385620594024658
-10.9646997451782 0.388809561729431
-10.8802995681763 0.392144203186035
-10.6766996383667 0.394907474517822
-10.6414003372192 0.396605491638184
-10.5846996307373 0.399359464645386
-10.5286998748779 0.40253210067749
-10.3970003128052 0.405436515808105
-10.2397003173828 0.40821361541748
-10.0580997467041 0.412022590637207
-10.0274000167847 0.4132000207901
-9.90007019042969 0.415962219238281
-9.84902000427246 0.41877281665802
-9.72109985351562 0.42196524143219
-9.59097003936768 0.424748182296753
-8.94359016418457 0.440560817718506
-8.77958011627197 0.44334614276886
-8.66703033447266 0.445314764976501
-8.56585025787354 0.448068737983704
-8.52196979522705 0.448435425758362
-8.36353969573975 0.451195240020752
-8.2965202331543 0.451908826828003
-8.06462955474854 0.454704523086548
-7.93067979812622 0.457847237586975
-7.8828501701355 0.461074352264404
-7.7972297668457 0.463916182518005
-7.74197006225586 0.469635963439941
-7.51878023147583 0.472409605979919
-7.44442987442017 0.476453304290771
-7.42505979537964 0.478242635726929
-7.22441005706787 0.481003522872925
-7.21062994003296 0.48179817199707
-6.98426008224487 0.484595060348511
-6.88021993637085 0.487715721130371
-6.69883012771606 0.491743206977844
-6.49357986450195 0.49450409412384
-6.4737401008606 0.49563992023468
-6.41906976699829 0.498392820358276
-6.32357978820801 0.502156615257263
-6.190110206604 0.504916429519653
-6.18189001083374 0.505754947662354
-6.02124977111816 0.508512496948242
-5.99909019470215 0.511347413063049
-5.9539999961853 0.514100313186646
-5.94348001480103 0.515984535217285
-5.89594984054565 0.518737316131592
-5.87410020828247 0.519947290420532
-5.76550006866455 0.522701263427734
-5.73570013046265 0.523402214050293
-5.67646980285645 0.5261549949646
-5.61391019821167 0.529200553894043
-5.57807016372681 0.531598329544067
-5.51769018173218 0.534351110458374
-5.46011018753052 0.53865385055542
-5.29777002334595 0.544729828834534
-5.23770999908447 0.547481536865234
-5.16074991226196 0.550833463668823
-4.99162006378174 0.553594589233398
-4.76663017272949 0.557291269302368
-4.64948987960815 0.56034255027771
-4.49037981033325 0.563104629516602
-4.43736982345581 0.567532300949097
-4.3518500328064 0.570299029350281
-4.3100700378418 0.572355508804321
-4.2433500289917 0.575119972229004
-3.86088991165161 0.588860034942627
-3.81626009941101 0.591612815856934
-3.78898000717163 0.592530012130737
-3.73791003227234 0.595282912254333
-3.65746998786926 0.601698875427246
-3.54395008087158 0.611406803131104
-3.50952005386353 0.614160776138306
-3.28170990943909 0.62702751159668
-3.23337006568909 0.630067110061646
-3.23091006278992 0.631441354751587
-3.20642995834351 0.634192943572998
-3.11998009681702 0.639674425125122
-2.96805000305176 0.642428398132324
-2.95144009590149 0.645721435546875
-2.90019989013672 0.647432088851929
-2.87302994728088 0.650183916091919
-2.86858010292053 0.652291297912598
-2.7878201007843 0.655044078826904
-2.31567001342773 0.66317892074585
-1.95132994651794 0.665935277938843
-1.88363003730774 0.668943762779236
-1.74026000499725 0.671752095222473
-1.6403900384903 0.674508452415466
-1.35939002037048 0.683960676193237
-1.21850001811981 0.68751049041748
-1.13590002059937 0.688132762908936
-0.998072028160095 0.690892577171326
-0.996742010116577 0.69456148147583
-0.790364980697632 0.697359442710876
-0.72720193862915 0.701120853424072
-0.503114938735962 0.703873753547668
-0.492006063461304 0.705402851104736
-0.388393044471741 0.708156824111938
0.221850037574768 0.727968215942383
0.440104007720947 0.731752872467041
0.53612494468689 0.734503269195557
0.705673933029175 0.739788055419922
0.885710000991821 0.742608070373535
0.951401948928833 0.746651768684387
0.976479053497314 0.749402284622192
1.3309999704361 0.766732573509216
1.43379998207092 0.769485473632812
1.57098996639252 0.773643612861633
1.58404004573822 0.77453887462616
1.715980052948 0.777291774749756
1.79333996772766 0.782872676849365
1.89711999893188 0.785624265670776
1.99521994590759 0.792453289031982
2.12202000617981 0.795206069946289
2.16201996803284 0.799380540847778
2.26955008506775 0.802057027816772
2.42289996147156 0.804809808731079
2.4770200252533 0.80785071849823
2.53858995437622 0.810602426528931
2.63651990890503 0.813639879226685
2.6400899887085 0.815244078636169
2.68345999717712 0.817996978759766
2.76798009872437 0.821272611618042
2.82927989959717 0.824517011642456
2.86185002326965 0.827395915985107
3.05834007263184 0.835151433944702
3.22791004180908 0.839633464813232
3.29106998443604 0.84514844417572
3.53485989570618 0.847912788391113
3.56922006607056 0.849090337753296
3.72145009040833 0.851844310760498
3.81380009651184 0.855336308479309
4.01025009155273 0.859923601150513
4.13062000274658 0.862677574157715
4.33232021331787 0.865461707115173
4.66383981704712 0.873146533966064
4.80281019210815 0.87812602519989
4.86128997802734 0.881360054016113
5.01436996459961 0.88411283493042
5.14535999298096 0.887835025787354
5.28491020202637 0.89058792591095
5.54118013381958 0.893484115600586
5.60909986495972 0.895297765731812
5.74154996871948 0.898050665855408
5.89589023590088 0.902463316917419
6.00586986541748 0.906792640686035
6.18053007125854 0.909546732902527
7.15086984634399 0.924909591674805
7.17620992660522 0.925530672073364
7.24360990524292 0.928284645080566
7.37536001205444 0.931264162063599
7.4756498336792 0.934694886207581
7.63982009887695 0.938194990158081
7.88827991485596 0.94256591796875
8.50098037719727 0.945339679718018
8.59768009185791 0.946276545524597
8.70409965515137 0.949030518531799
9.39048957824707 0.954936504364014
9.92469024658203 0.957693934440613
10.3067998886108 0.960618019104004
10.6338996887207 0.963396310806274
12.0052995681763 0.98605751991272
12.4492998123169 0.990302562713623
13.1153001785278 0.993058919906616
15.1802997589111 0.998584270477295
23.3670997619629 1
};

\end{groupplot}
\end{tikzpicture}
    \caption{ECDF of the \gls{sinr} obtained using the 3GPP channel model of~\cite{TR38901}, and the performance-oriented channel model presented in this work, for different propagation scenarios.}
    \label{fig:accuracy}
\end{figure}

Finally, we also computed (using the same simulation script, i.e.,~\texttt{cttc-\-nr-\-demo-\-two-\-ray}) the \gls{sinr} statistics achieved by the proposed \gls{ftr}-based model, and compared them to those obtained using the model of~\cite{tommaso:20}. As can be seen in Figure~\ref{fig:accuracy}, the two models provide similar results. Indeed, a non-negligible difference can be found only in the case of the \texttt{InH-OfficeMixed} propagation scenario.

We remark that all the results presented in this section can be reproduced by using the SEM~\cite{magrin2019simulation} scripts which we provide\footnote{\url{https://gitlab.com/pagmatt/ns-3-dev/-/tree/gsoc-wns3}}.

\subsubsection{Use Cases}

The main goal of both the performance oriented channel model and the optimizations to the 3GPP TR 38.901 model is to enable system-level
simulations of large-scale \gls{mimo} scenarios for which the implementation of~\cite{tommaso:20} exhibits prohibitive computational complexity. Specifically, our contributions allow ns-3 users to simulate wireless deployments where the devices feature antenna arrays with more than hundreds of radiating elements, and/or the number of communication endpoints is particularly high. For example, the modifications presented in this work can be used in the~\gls{nr} and \texttt{mmwave}~\cite{mezzavilla2018end} modules (which both already support the proposed channel models) to simulate massive MIMO 5G NR networks. 
Notably, a preliminary version of the \texttt{Eigen} port has been used in conjunction with the \texttt{mmwave}~\cite{mezzavilla2018end} module to simulate \gls{5g} networks aided by \glspl{irs}, i.e., devices which feature up to $100 \times 100$ reflecting elements~\cite{pagin2022}. 

Moreover, since the supported frequency range is $0.5 - 100$~GHz, this encompasses not only terrestrial \gls{5g} and \gls{lte} deployments, but also most non-terrestrial networks and IEEE \glspl{rat}. Finally, the proposed \texttt{Two\-Ray\-Spectrum\-Propagation\-Loss\-Model} can be further extended to support frequencies above $100$~GHz using reference fading and path loss statistics.

\section{Conclusions and future work}
\label{sec:conc}

In this chapter, we proposed a signal model for \glspl{irs} and \gls{af} relays based on the 3GPP TR 38.901 channel for 5G NR networks, and explained the methodology we used to perform network-level simulations of 5G and beyond scenarios with IRS and AF relay nodes.
Based on this framework, we performed simulations to provide numerical guidelines to dimension IRS/AF-assisted networks. 
Moreover, we presented an ns-3 implementation of the \gls{3gpp} channel model for NTNs, developed following the specifications provided in \cite{38811}. The code, which is publicly available at~\cite{ntngitlab}, well integrates with the rest of the ns-3 framework, and enables full-stack end-to-end simulations in different \gls{ntn} scenarios. We validated the link-level and end-to-end accuracy of our module against 3GPP calibration results reported in~\cite{38821}. 
Finally, we presented a set of optimizations concerning the simulation of \gls{mimo} wireless channels in ns-3. These improvements comprise the optimization of the related linear algebra routines, and the design and implementation in ns-3 of a performance-oriented statistical channel model based on the \gls{ftr} fading model, which further reduces the simulation time of \gls{mimo} scenarios. %Th


As part of our future work, we plan to extend our smart relays simulator by considering more sophisticated
scenarios in which heterogeneous types of relays are deployed, and compare the numerical performance of
IRS/AF relays with that of IAB. Moreover, we will also relax some of our assumptions by considering quantization of the
relay phase shifters.

Furthermore, we foresee to further extend our NTN module to simulate end-to-end NTN 5G NR networks. To this end, we will incorporate additional functionalities, such as a delay model, along with the adaptations to the terrestrial 5G NR protocol stack which it entails, and the support for satellite mobility.

Additionally, we plan to further improve the scalability of the wireless channel simulation framework in ns-3 by studying more refined beamforming gain correction factors, and possibly making the estimation of such term scenario-dependent. Moreover, we envision to design more efficient storage/access data structures and linear algebra operations for 3D matrices, by better leveraging \texttt{Eigen} also in this context.
Finally, we will consider using \gls{simd} for speeding up the evaluation of trigonometric functions, and caching the beamforming gain in the \texttt{Two\-Ray\-Spectrum\-Propagation\-Loss\-Model} class to further reduce the simulation time of \gls{mimo} scenarios in ns-3.

