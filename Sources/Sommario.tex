{\noindent\huge\itshape Sommario}\\
%

% Cellular networks are constantly evolving in order to support the ever-increasing number of mobile users, and the corresponding growth in wireless data traffic, coupled with the emergence of new applications. 
% Specifically, the last iteration of mobile networks, i.e., 5G, brought high peak performance and extreme flexibility, making it possible to support a diverse set of applications with heterogeneous yet stringent requirements. 
Le reti cellulari sono in constante evoluzione, sia per sostenere il numero crescente di utenti mobili e la corrispondente crescita del traffico di dati, sia per far fronte all'emergenza di nuovi scenari d'uso.
In particolare, l'ultima versione delle reti mobili, ovvero 5G, offre alte prestazioni di picco ed estrema flessibilità, permettendo di sostenere un diverso insieme di applicazioni con caratteristiche eterogenee.
% One of 5G main novelties is represented by the support for \gls{mmwave} frequencies, which unlocks an unprecedented amount of previously unused radio resources. In turn, the latter enables extremely high data rates and low latencies. Moreover, it is envisioned that the upcoming generation, i.e., 6G, will unleash additional bandwidth, by further expanding the supported spectrum bands to include \gls{thz} frequencies as well. 
Una delle principali novità della tecnologia 5G è rappresentata dal supporto alle frequenze \gls{mmwave}, che forniscono l'accesso ad una quantità inedita di risorse radio precedentemente non utilizzate. Queste ultime consentono trasmissioni dati a velocità estremamente elevate e latenze particolarmente basse. 
Inoltre, si prevede che la prossima generazione di reti cellulari, ovvero 6G, supporterà ulteriori bande includendo anche le frequenze \gls{thz}.

% However, despite their theoretical potential, \gls{mmwave} and \gls{thz} frequencies exhibit harsh propagation conditions which make it challenging to provide ubiquitous high speed wireless connectivity. 
% To fill this gap, this thesis studies innovative deployment solutions to overcome the unfavorable propagation characteristics of mmWave and THz communications, paving the way for their widespread use in the context of 6G cellular networks. 
Tuttavia, nonostante il loro potenziale teorico, le frequenze \gls{mmwave} e \gls{thz} presentano condizioni di propagazione estremamente sfavorevoli, che rendono difficile fornire una connessione wireless veloce ed onnipresente. 
A questo fine, questa tesi studia infrastrutture innovative per ovviare all'intrinseca copertura limitata delle comunicazioni mmWave e THz, aprendo la strada ad un loro utilizzo diffuso nel contesto delle reti cellulari 6G.

% In particular, this thesis
% \begin{enumerate*}[label=(\roman*)]
%     \item presents novel simulation tools, which model innovative coverage enhancement technologies such as \glspl{irs}, \gls{af} relays, and \glspl{ntn};
%     \item presents novel simulation models which improve the computational complexity of \gls{mimo} simulations;
%     \item introduces schemes for optimizing \gls{iab} networks;
%     \item analyzes the potential of mixed \gls{mmwave} and \gls{thz} links for wireless backhauling; and
%     \item analyzes the impact of non-ideal control channels in \gls{irs}-aided deployments, and introduces algorithms for mitigating the corresponding perofrmance degradation
% \end{enumerate*}.
In particolare, questa tesi
\begin{enumerate*}[label=(\roman*)]
\item presenta nuovi strumenti di simulazione, che modellano tecnologie di miglioramento della copertura innovative come \glspl{irs}, ripetitori \gls{af} e \glspl{ntn};
\item presenta nuovi modelli di simulazione che migliorano la complessità computazionale delle simulazioni \gls{mimo};
\item introduce schemi di ottimizzazione per reti \gls{iab};
\item analizza il potenziale di collegamenti misti \gls{mmwave} e \gls{thz} per wireless backhauling;
\item analizza l'impatto di canali di controllo non ideali in reti supportate da \glspl{irs} e introduce algoritmi per mitigare la conseguente perdita di prestazioni
\end{enumerate*}.

% This thesis adopts a system-level approach, thus characterizing the network behavior in an end-to-end fashion, and capturing the interplay between the physical signal propagation and the different layers of the communications protocol stack. Results demonstrate the effectiveness of the proposed solutions, which pave the way towards ubiquitous high-performance mobile networks.
Questa tesi adotta un approccio di studio a livello di sistema, quindi caratterizzando il comportamento della rete nel suo complesso e catturando l'in\-te\-ra\-zio\-ne tra la propagazione del segnale fisico ed i diversi livelli dello stack protocollare. I risultati presentati dimostrano l'efficacia delle soluzioni proposte, che aprono quindi la strada verso reti cellulari ubique con alte prestazioni.