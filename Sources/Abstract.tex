{\noindent\huge\itshape Abstract}\\
%

Cellular networks are constantly evolving in order to support the ever-increasing number of mobile users, and the corresponding growth in wireless data traffic, coupled with the emergence of new applications. 
Specifically, the last generation of mobile networks, i.e., 5G, brought high peak performance and extreme flexibility, making it possible to support a diverse set of applications with heterogeneous yet stringent requirements. 
One of 5G's main novelties is represented by the support for \gls{mmwave} frequencies, which unlocks an unprecedented amount of previously unused radio resources. In turn, the latter enables extremely high data rates and low latencies. Moreover, it is envisioned that the upcoming generation, i.e., 6G, will unleash additional bandwidth, by further expanding the supported spectrum bands to include \gls{thz} frequencies as well. 
However, despite their theoretical potential, \gls{mmwave} and \gls{thz} frequencies exhibit harsh propagation conditions which make it challenging to provide ubiquitous high speed wireless connectivity. 
To fill this gap, this thesis studies innovative deployment solutions to overcome the unfavorable propagation characteristics of \gls{mmwave} and \gls{thz} communications, paving the way for their widespread use in the context of 6G cellular networks. 

In particular, this thesis
\begin{enumerate*}[label=(\roman*)]
    \item presents novel simulation tools, which model innovative coverage enhancement technologies such as \glspl{irs}, \gls{af} relays, and \glspl{ntn};
    \item presents novel simulation models which improve the computational complexity of \gls{mimo} simulations;
    \item introduces schemes for optimizing \gls{iab} networks;
    \item analyzes the potential of mixed \gls{mmwave} and \gls{thz} links for wireless backhauling; and
    \item analyzes the impact of non-ideal control channels in \gls{irs}-aided deployments, and introduces algorithms for mitigating the corresponding performance degradation.
\end{enumerate*}

This thesis adopts a system-level approach, thus characterizing the network behavior in an end-to-end fashion, and capturing the interplay between the physical signal propagation and the different layers of the communications protocol stack. Results demonstrate the effectiveness of the proposed solutions, which pave the way towards ubiquitous high-performance mobile networks.