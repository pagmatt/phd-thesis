%%%%%%%%%%%%%%%%%%%%%%%%%%%%%%%%%%%%%%%%%%%%%%%%%%%%%%%%%%%%%%%%%%%%%%%%
\chapter{Introduction}
%%%%%%%%%%%%%%%%%%%%%%%%%%%%%%%%%%%%%%%%%%%%%%%%%%%%%%%%%%%%%%%%%%%%%%%%

The significance of mobile networks in modern society is underscored by their paramount role in facilitating social, professional, and educational interactions. The recent COVID-19 pandemic has served as a bitter reminder of this critical importance, highlighting how the absence of Internet connectivity can represents significant hindrance to our daily lives.
However, the current generation of cellular networks, i.e., \gls{5g}, are found wanting in providing adequate broadband coverage to rural regions~\cite{yaacoub2020key}. Furthermore, even in technologically advanced nations, cellular infrastructures often fall short of meeting the stringent reliability, availability, and responsiveness requirements of emerging wireless applications. The vulnerability of mobile networks to natural disasters and cyberattacks underscores the need for solutions which offer reliability from both a technological and a sociopolitical standpoint.
Indeed, recent geopolitical turmoils have expanded the scope of conflict to include the cyberspace, emphasizing the imperative of maintaining uninterrupted connectivity in emergency situations. In these contexts, connectivity outages can compromise the delivery of critical services, inflict significant economic damage, and even result in loss of lives~\cite{internet_ukr_afg}.

In response to the pressing need for reliable wireless connectivity, the \gls{itu} foresees a future where ubiquitous broadband coverage will be achieved by 2030. This vision is driven by the imperative to provide seamless connectivity to both humans and an increasingly vast array of intelligent devices, including wearables, autonomous vehicles, \glspl{uas}, and robots~\cite{mozaffari2018beyond}.
The advent of novel use cases such as holographic communications, \gls{xr}, and tactile applications will further exacerbate the requirements for peak throughput and latency identified for the 5G's \gls{embb} and \gls{urllc} use cases. 
To achieve these ambitions goals, future cellular systems will continue to evolve and enhance the 5G network paradigm, which has revolutionized the wireless landscape by introducing flexible virtualized architectures, the support for \gls{mmwave} communications, and the adoption of \gls{mimo} technologies~\cite{ghosh20195g}. 
Notably, both academic and industry researchers are exploring a more central role for \gls{mmwave} technology, with plans to allocate additional spectrum towards the \gls{thz} band, and the design of \gls{ai}-native networks.
In fact, as mobile networks become increasingly complex and heterogeneous, the push for spectrum expansion will be coupled with an \gls{ai}-native design which, thanks to the ongoing virtualization, will not be limited to the radio link level, but will encompass the orchestration of large scale deployments as well~\cite{polese2023understanding}.
Nevertheless, how to design, test and eventually deploy management and orchestration algorithms is an open research challenge~\cite{polese2022colo}.
First, the training data must accurately capture the interplay of the whole protocol stack with the wireless channel. Furthermore, optimization frameworks such as \gls{drl} also call for preliminary testing in isolated yet realistic environments, with the goal of minimizing the performance degradation to actual network deployments~\cite{lacava2022programmable, amir2023safehaul}.

The THz and mmWave frequency bands hold significant promise as a means to achieve peak data rates exceeding Tb/s, as envisioned by the ITU~\cite{imt2030}. However, this portion of the spectrum is hampered by unfavorable propagation characteristics that make it challenging to realize its full potential.
Specifically, the THz and mmWave bands are plagued by a pronounced free-space propagation loss, and a marked susceptibility to blockages, for instance due to buildings, foliage, and other obstacles, which can cause significant attenuation or even complete loss of signal~\cite{han2018propagation, jornet2011channel}.
With 5G, preliminary support for \gls{mmwave} bands has been introduced thanks to a major redesign not only of the physical layer, but of the whole cellular protocol stack~\cite{shafi2018microwave}. For instance, the intrinsic directionality of the communication requires ad hoc control procedures~\cite{heng2021six}, while the frequent transitions between \gls{los} and \gls{nlos} conditions call for an ad hoc transport layer design, such as novel \gls{tcp} algorithms~\cite{zhang2019will}. 
Albeit the harsh propagation environment can be partially mitigated by using directional links and densifying network deployments~\cite{polese2020toward}, these unfavorable propagation characteristics pose a significant challenge to the successful deployment of THz- and mmWave-based 6G wireless systems. Traditional network densification is costly for network operators,
especially in terms of sites acquisition campaigns, rental fees, and fiber optic layout to provide wired backhauling~\cite{lopez2015towards}.

To solve this issue, the 3GPP approved, as part of its 5G NR specifications for Rel-16~\cite{3gpp_38_874}, \gls{iab} as a new paradigm to replace fiber-like infrastructures with self-configuring relays operating through wireless backhaul links.
Previous research has demonstrated that \gls{iab} systems represent a cost-performance trade-off~\cite{polese2020integrated}, since base stations need to multiplex access and backhaul resources, and wireless backhaul at mmWave frequencies are less reliable compared to their fiber counterpart.
Furthermore, if the partitioning of these spectrum resources is not optimally configured, IAB networks can experience excessive buffering, which in turn leads to increased latency and reduced throughput~\cite{8514996}. Finally, while lower than that of wired backhaul deployments, the \gls{capex} costs of \gls{iab} installation may still prove prohibitive for \glspl{mno}~\cite{chaoub20216g}.
In light of these limitations, new technologies based on \glspl{irs} and \gls{af} relays are also emerging as promising alternatives to overcome the coverage issues of mmWave networks with energy and cost efficiency in mind.
An \gls{irs} is a meta-surface whose elements can be programmed to manipulate electromagnetic fields in favor of a specific destination. By passively beamforming incoming signals without amplification, \glspl{irs} can meet the minimum capacity requirements in dead spots with reduced power consumption compared to approaches such as IAB~\cite{bjornson2019intelligent}. In contrast, \gls{af} relays are designed to capture incident electromagnetic waves from a base station, amplify the received signal, and re-radiate it towards a targeted area to be served.
While IRSs offer advantages in terms of lower power consumption, AF relays have the potential to achieve higher capacity by actively amplifying signals. However, similarly to IAB, this comes at the cost of increased complexity, higher system costs, and potential amplification noise issues~\cite{huang2019reconfigurable}.
Nevertheless, to truly achieve ubiquitous connectivity \gls{ntn} integration within the depicted 6G wireless ecosystem will be essential, owing to its capability to deliver services anywhere and anytime. In fact, 3GPP foresee the support for the transparent integration of satellite gateways in the \gls{ran}, thus providing coverage to handheld devices in areas that are unreachable by conventional terrestrial deployment.

Whether these technologies will be able to fulfill future mobile service requirements and, if so, how these platforms can be optimally integrated in 5G and 6G systems, are still crucial issues that remain unsolved. 
Indeed, for the next generation of cellular networks to fully achieve the ubiquitous goal, the above innovative deployment solutions will need to be properly studied and optimized, characterizing their impact on the end-to-end systems, i.e., from the signal propagation to the \gls{qoe} perceived by the end users.
However, performing accurate performance evaluations of these new solutions on real-word testbeds is typically impractical, especially at scale, due to unavailability of suitable platforms and/or excessive deployment costs.
While these limitations are partly being addressed with initiatives such as the US-based \gls{pawr}, and the European IMAGINE-B5G and 6G-SANDBOX, promoting the development of \gls{sdr}-based platforms to enable the emulation-based experimentation of cellular technologies, system-level simulators can still play a key role in wireless research.
In particular, system-level simulator can provide a playground for preliminary experimentation, at a scale which makes it unfeasible to perform network emulation, and or encompassing technologies for which hardware prototypes are not yet available. 

Notably, the ns3-mmwave~\cite{mezzavilla2018end} 5G-NR-compliant open-source simulator represents a de-facto standard for the end-to-end simulation of 5G and beyond cellular networks.

% To sum up, in this dissertation, besides giving a detailed overview of the aforementioned
% problems, we contribute to the state of the art with improvements that span from the creation
% of novel ns-3 modules, to the design of artificial intelligence algorithms to proactively assist
% teleoperated driving operations, to the setup of a reproducible workflow for carrying out end-
% to-end measurements on a public platform for wireless research.
% Where possible, after assessing their
% peculiarities and the problems that might arise for specific use cases, we tried to propose new
% ad hoc solutions, or optimization to existing ones. It is also important to mention that the
% presented work is the fruit of the collaboration with several researchers from world-leading uni-
% versities, industries and research institutions, without whom reaching such relevant outcomes
% would have not been possible. To properly acknowledge this collaborative effort from all the
% authors, besides including at the end of this thesis a list of all the publications from these
% 3-years of Ph.D. (accepted or in the process of review), at the beginning of each chapter we also
% list the papers to refer to for the original work.
% The topics of this thesis are organized as follows