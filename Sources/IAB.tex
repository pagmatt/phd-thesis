\chapter{Towards wireless-backhauled next-generation cellular networks}
\label{ch:iab}

%%%%%%%%%%%%%%%%%%%%%
%%% RES MAN INTRO/SoA %%%
%%%%%%%%%%%%%%%%%%%%%

Future wireless networks will accommodate data-rate intensive use cases which include untethered \gls{vr} and mobile metaverse applications. This will further exacerbate the congestion of mobile access networks and backhaul systems~\cite{holo1}.
To accommodate this traffic increase, the \gls{3gpp} has introduced various technological advancements with the specifications of the \gls{5g} \gls{ran} and \gls{cn}, namely \gls{nr} and \gls{5gc}~\cite{3gpp_38_300}. In particular, \gls{nr} features a user and control plane split, a flexible \gls{ofdm} frame structure, and the support for \gls{mmwave} communications, while the \gls{cn} introduces virtualization and slicing~\cite{yousaf2017nfv}. 

Notably, the use of the \gls{mmwave} band, with typical deployments in the spectrum around 28 GHz and 39 GHz~\cite{shafi2017deployment}, possibly coupled with sub-terahertz mobile links~\cite{polese2020toward, 8869705}, represents the major technological enabler toward the Gbit/s capacity target. Indeed, these frequencies are characterized by the availability of vast chunks of contiguous and currently unused spectrum, in stark contrast with the crowded sub-6 GHz bands. 
However, \glspl{mmwave} and terahertz frequencies exhibit unfavorable propagation characteristics, such as high isotropic losses and a marked susceptibility to blockages and signal attenuation~\cite{khan2011mmwave, rangan2014millimeter}.
These issues can be partially mitigated using beamforming through large antenna arrays, thanks to the small wavelengths and advances in low-power \gls{cmos} RF circuits~\cite{hemadeh2017millimeter}; nevertheless, their introduction alone is not enough for meeting the high service availability requirement. 
In fact, wireless networks operating at such high frequencies will be deployed with extremely high density, to improve the probability of \gls{los} coverage and mitigate the impact of the harsh propagation environment. 
Nonetheless, while the theoretical effectiveness of this technique is well understood~\cite{gomez2017capacity}, achieving dense cellular deployments is extremely challenging from a practical point of view. Specifically, providing a fiber backhaul among base stations and the \gls{cn} is deemed economically impractical, even more so in the initial \gls{5g} deployments~\cite{polese2020integrated}. 

To make ultra-dense deployments viable, the \gls{3gpp} has standardized an extension of \gls{5g} NR, i.e., \gls{iab}, which exploits the same waveform and protocol stack to provide access to mobile users and wireless backhaul for \glspl{gnb} (i.e., the \gls{iab} nodes) thus limiting the need for fiber drops. The wireless backhaul topology terminates at a \gls{gnb} with fiber connectivity to the data core, the \gls{iab} donor~\cite{9187867,stoch_geom2,3gpp_38_174}. \gls{iab} also simplifies the deployment of cellular networks in on-demand or ad hoc contexts, as it removes the need for part of the wired backhaul.
Prior research has highlighted that \gls{iab} represents a cost-performance trade-off~\cite{stoch_geom2, polese2020integrated}, as base stations need to multiplex access and backhaul resources, and as the wireless backhaul at \glspl{mmwave} is less reliable than a fiber connection. In particular, \gls{iab} networks may suffer from excessive buffering (and, consequently, high latency and low throughput) when a suboptimal partition of access and backhaul resources is selected, thus hampering the benefits that the high bandwidth \gls{mmwave} links introduce~\cite{polese2020integrated,polese2018end}. Therefore, it is fundamental to solve these non-trivial challenges to enable a smooth integration of \gls{iab} in \gls{5g} and beyond deployments.

In this chapter, we introduce several solutions for optimizing routing and backhaul/access resource partitioning in \gls{iab} networks.
In particular, Section~\ref{sec:iab-res-man} describes a semi-centralized resource allocation scheme for IAB networks, designed to be flexible, with low complexity, and compliant with the 3GPP IAB specifications. The proposed solution, which is based on the \gls{mwm} problem, is compared with state-of-the-art distributed approaches through end-to-end, full-stack system-level simulations with a 3GPP-compliant channel model, protocol stack, and a diverse set of user applications. Results show that this scheme can increase the throughput of cell-edge users up to $5$ times, while decreasing the overall network congestion with an end-to-end delay reduction of up to $25$ times.
Section~\ref{sec:iab-safehaul} describes Safehaul, a risk-averse learning-based solution for IAB mmWave networks. Instead of optimizing the average latency performance, Safehaul ensures reliability by minimizing the losses in the tail of the performance distribution.
We show via extensive simulations that Safehaul not only reduces the latency by up to $43.2\%$ compared to the benchmarks, but also exhibits significantly more reliable performance, e.g., $71.4\%$ less variance in latency.
Finally, in Section~\ref{sec:iab-wons} we consider the deployment of mixed \gls{mmwave} and sub-terahertz links to increase the capacity of the backhaul network, and provide the first performance evaluation of the potential of sub-terahertz frequencies for 6G IAB. 
To do so, we develop a greedy algorithm that allocates frequency bands to the backhaul links (considering constraints on spectrum licenses, sharing, and congestion) and generates the wireless network mesh. Then, we profile the performance through a custom extension of the open-source
system-level simulator Sionna that supports Release 17 IAB specifications and channel models up to 140 GHz. Results
show that IAB with sub-terahertz links can outperform a mmWave-only deployment with improvements of $4\times$ for average
user throughput and a reduction of up to $50\%$ for median latency.
%%%%%%%%%%%%%%%%%%%%%
%%% ThZ INTRO/SoA %%%
%%%%%%%%%%%%%%%%%%%%%

% Wireless networks operating at such high frequencies will be deployed with extremely high density, to improve the probability of \gls{los} coverage and mitigate the impact of the harsh propagation environment. To make ultra-dense deployments viable, the \gls{3gpp} has standardized an extension of \gls{5g} NR, i.e., \gls{iab}, which exploits the same waveform and protocol stack to provide access to mobile users and wireless backhaul for \glspl{gnb} (i.e., the \gls{iab} nodes) thus limiting the need for fiber drops. The wireless backhaul topology terminates at a \gls{gnb} with fiber connectivity to the data core, the \gls{iab} donor~\cite{9187867,stoch_geom2,3gpp_38_174}. \gls{iab} also simplifies the deployment of cellular networks in on-demand or ad hoc contexts, as it removes the need for part of the wired backhaul.

% \gls{iab} networks in 5G systems are a natural application for \gls{mmwave} deployments, as telecom operators can easily fit carriers with 400 MHz of bandwidth in this spectrum. In addition, the directionality that \gls{mmwave} arrays introduce helps reduce the interference. 
% Nonetheless, studies have shown that bottlenecks can emerge at \gls{iab} donors, creating congestion, high latency, and degraded \gls{qos} for the end users, especially when backhaul links are constrained to re-using the same spectrum of the access (as in in-band \gls{iab})~\cite{polese2020integrated}.