\chapter{Towards Wireless-Backhauled Next-Generation Cellular Networks}
\label{ch:iab}

%%%%%%%%%%%%%%%%%%%%%
%%% RES MAN INTRO/SoA %%%
%%%%%%%%%%%%%%%%%%%%%

Future wireless networks will accommodate data-rate intensive use cases which include untethered \gls{vr} and mobile metaverse applications. This will further exacerbate the congestion on mobile access networks and backhaul systems~\cite{holo1}.
To this end, the \gls{3gpp} has introduced various technological advancements with the specifications of the \gls{5g} \gls{ran} and \gls{cn}, namely \gls{nr} and \gls{5gc}~\cite{3gpp_38_300}. In particular, \gls{nr} features a user and control plane split, a flexible \gls{ofdm} frame structure, and the support for \gls{mmwave} communications, while the \gls{cn} introduces virtualization and slicing~\cite{yousaf2017nfv}. 

Specifically, the use of the \gls{mmwave} band, with typical deployments in the spectrum around 28 GHz and 39 GHz~\cite{shafi2017deployment}, represents the major technological enabler toward the Gbit/s capacity target. Moreover, sub-terahertz mobile links are being considered for \gls{6g} applications~\cite{polese2020toward,8869705}. 
These frequencies are characterized by the availability of vast chunks of contiguous and currently unused spectrum, in stark contrast with the crowded sub-6 GHz frequencies. However, \glspl{mmwave} and terahertz bands exhibit unfavorable propagation characteristics such as high isotropic losses and a marked susceptibility to blockages and signal attenuation~\cite{khan2011mmwave, rangan2014millimeter}.
These issues can be partially mitigated using beamforming through large antenna arrays, thanks to the small wavelengths and advances in low-power \gls{cmos} RF circuits~\cite{hemadeh2017millimeter}; nevertheless, their introduction alone is not enough for meeting the high service availability requirement. 
Therefore, \glspl{mmwave} and terahertz networks also need densification, to decrease the average distance between mobile terminals and base stations and improve the average \gls{sinr}. The theoretical effectiveness of this technique is well understood~\cite{gomez2017capacity}; however, achieving dense \gls{5g} deployments is extremely challenging from a practical point of view. Specifically, providing a fiber backhaul among base stations and toward the \gls{cn} is deemed economically impractical, even more so in the initial \gls{5g} deployments~\cite{polese2020integrated}.

Recently, wireless backhaul solutions for \gls{5g} networks have emerged as a viable strategy toward cost effective, dense \gls{mmwave} deployments. Notably, the \gls{3gpp} has promoted \gls{iab}~\cite{3gpp_38_874}, i.e., a wireless backhaul architecture which dynamically splits the overall system bandwidth for backhaul and access purposes. \gls{iab} has been integrated in the latest release of the \gls{3gpp} \gls{nr} specifications. Prior research has highlighted that \gls{iab} represents a cost-performance trade-off~\cite{stoch_geom2, polese2020integrated}, as base stations need to multiplex access and backhaul resources, and as the wireless backhaul at \glspl{mmwave} is less reliable than a fiber connection. In particular, \gls{iab} networks may suffer from excessive buffering (and, consequently, high latency and low throughput) when a suboptimal partition of access and backhaul resources is selected, thus hampering the benefits that the high bandwidth \gls{mmwave} links introduce~\cite{polese2020integrated,polese2018end}. Therefore, it is fundamental to solve these non-trivial challenges to enable a smooth integration of \gls{iab} in \gls{5g} and beyond deployments.


%%%%%%%%%%%%%%%%%%%%%
%%% ThZ INTRO/SoA %%%
%%%%%%%%%%%%%%%%%%%%%

Future wireless networks will accommodate data-rate intensive use cases which include untethered \gls{vr} and mobile metaverse applications. This will further exacerbate the congestion on mobile access networks and backhaul systems~\cite{holo1}. For this reason, \gls{5g} cellular systems have pushed into the \gls{mmwave} band, with typical deployments in the spectrum around 28 GHz and 39 GHz~\cite{shafi2017deployment}, and sub-terahertz mobile links are being considered for \gls{6g} applications~\cite{polese2020toward,8869705}.

Wireless networks operating at such high frequencies will be deployed with extremely high density, to improve the probability of \gls{los} coverage and mitigate the impact of the harsh propagation environment. To make ultra-dense deployments viable, the \gls{3gpp} has standardized an extension of \gls{5g} NR, i.e., \gls{iab}, which exploits the same waveform and protocol stack to provide access to mobile users and wireless backhaul for \glspl{gnb} (i.e., the \gls{iab} nodes) thus limiting the need for fiber drops. The wireless backhaul topology terminates at a \gls{gnb} with fiber connectivity to the data core, the \gls{iab} donor~\cite{9187867,stoch_geom2,3gpp_38_174}. \gls{iab} also simplifies the deployment of cellular networks in on-demand or ad hoc contexts, as it removes the need for part of the wired backhaul.

\gls{iab} networks in 5G systems are a natural application for \gls{mmwave} deployments, as telecom operators can easily fit carriers with 400 MHz of bandwidth in this spectrum. In addition, the directionality that \gls{mmwave} arrays introduce helps reduce the interference. Nonetheless, studies have shown that bottlenecks can emerge at \gls{iab} donors, creating congestion, high latency, and degraded \gls{qos} for the end users, especially when backhaul links are constrained to re-using the same spectrum of the access (as in in-band \gls{iab})~\cite{polese2020integrated}.

In this context, out-of-band \gls{iab} with sub-terahertz links is seen as a solution
%
to support immersive multimedia data-hungry streams. 
%
Specifically, the spectrum above 100 GHz has several sub-bands that could provide bandwidths wider than 10 GHz, thus potentially data rates in the excess of tens of Gbps~\cite{akyildiz2014terahertz}. Backhaul---a static deployment---is a promising use case for sub-terahertz links, which need pencil-sharp beams to close the link budget and are thus less resilient to mobility compared to traditional sub-6 GHz or \gls{mmwave} frequencies. 

In recent years, the literature has closed several gaps in terms of circuit, antenna design~\cite{singh2020design} and physical and \gls{mac} layer solutions for sub-terahertz systems~\cite{ghafoor2020mac}.
%
When it comes to \gls{iab} with mixed sub-terahertz and \gls{mmwave} links,\footnote{In this paper, we consider the FR2 range of 3GPP NR (24.25 GHz to 71 GHz) as \glspl{mmwave}.} however, there are still several open questions in terms of network design and path selection. In this paper, we consider the problem of identifying a viable topology between \gls{iab} nodes and the \gls{iab} donors, including the carrier frequency of the backhaul links, and profile the performance that network planners can expect when mixing sub-terahertz and \gls{mmwave} \gls{iab} links.