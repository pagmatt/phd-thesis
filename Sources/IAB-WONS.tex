%%%%%%%%%%%%
%%% WONS %%%
%%%%%%%%%%%%


% \documentclass[a4paper, conference]{IEEEtran}
% \IEEEoverridecommandlockouts

% \pagestyle{empty}
% \usepackage{url}
% \usepackage[utf8]{inputenc}
% \usepackage{xcolor}
% \usepackage{amsmath}
% \usepackage{amssymb}
%\usepackage{algorithm}
%\usepackage{algorithmic}
\newcommand{\red}[1]{\begin{color}{red}#1\end{color}}

%\usepackage{geometry}
%\geometry{letterpaper, margin=1.3in}

% \usepackage[bottom=4.0cm, left=1.57cm, right=1.57cm, top=1.75cm]{geometry}

% \usepackage[acronyms,nonumberlist,nopostdot,nomain,nogroupskip]{glossaries}
% \usepackage{tablefootnote}
% \usepackage{booktabs}
% \usepackage{tabularx}
% \usepackage{tikz}
% \usepackage{pgfplots}
% \pgfplotsset{compat=newest}
% \pgfplotsset{plot coordinates/math parser=false}
% \usepackage{soul}
% \newlength\fheight
% \newlength\fwidth
% \usetikzlibrary{plotmarks,patterns, patterns.meta,decorations.pathreplacing,backgrounds,calc,arrows,arrows.meta,spy,matrix,backgrounds}
% \usepgfplotslibrary{patchplots,groupplots}
% \usepackage{tikzscale}
% \usepackage{hyperref}
% \usepackage{algorithm} 
% \usepackage{algpseudocode} 
% \usepackage{comment}
% \usepackage{bm}
% \usepackage{multirow}
% \usepackage{bbm}

% \renewcommand{\figurename}{Fig.}
% %\renewcommand{\tablename}{Tab.}
% \usepackage[font=small]{subcaption}
% \usepackage[font=small]{caption}

%\usepackage{mathtools}

% \newcommand{\MP}[1]{\color{red}{\textbf{TODO: #1 }}\color{black}}

%\usepackage{dblfloatfix}    % To enable Figures/IabThzLinks at the bottom of page
% \usepackage{colortbl}

%\usepackage{cite}

%\newacronym{3gpp}{3GPP}{3rd Generation Partnership Project}
\newacronym{adc}{ADC}{Analog to Digital Converter}
\newacronym{afbw}{AFBW}{Average Fading Bandwidth}
\newacronym{5g}{5G}{5th generation}
\newacronym{4g}{4G}{4th generation}
\newacronym{aimd}{AIMD}{Additive Increase Multiplicative Decrease}
\newacronym{am}{AM}{Acknowledged Mode}
\newacronym{amf}{AMF}{Access and Mobility Management Function}
\newacronym{an}{AN}{Access Network}
\newacronym{amc}{AMC}{Adaptive Modulation and Coding}
\newacronym{aqm}{AQM}{Active Queue Management}
\newacronym{awgn}{AGWN}{Additive White Gaussian Noise}
\newacronym{balia}{BALIA}{Balanced Link Adaptation}
\newacronym{bsr}{BSR}{Buffer Status Report}
\newacronym{msr}{MSR}{Max Sum-Rate}
\newacronym{ba}{BA}{Backlog Avoidance}
\newacronym{mrba}{MRBA}{Max-Rate Backlog Avoidance}
\newacronym{bdp}{BDP}{Bandwidth-Delay Product}
\newacronym{bf}{BF}{Beamforming}
\newacronym{v2x}{V2X}{Vehicle-to-Everything}
\newacronym{vr}{VR}{Virtual Reality}
\newacronym{mdp}{MDP}{Markov Decision Process}
\newacronym{mwm}{MWM}{Maximum Weighted Matching}
\newacronym{tdm}{TDM}{Time Division Multiplexing}
\newacronym{fdm}{FDM}{Frequency Division Multiplexing}
\newacronym{sdm}{SDM}{Space Division Multiplexing}
\newacronym{st}{ST}{Spanning Tree}
\newacronym{rl}{RL}{Reinforcement Learning}
\newacronym{lp}{LP}{Linear Programming}
\newacronym[firstplural=Deep Neural Networks (DNNs)]{dnn}{DNN}{Deep Neural Network}
\newacronym{dag}{DAG}{Directed Acyclic Graph}
\newacronym{gtp}{GTP}{GPRS Tunneling Protocol}
\newacronym{nas}{NAS}{Non-Access Stratum}
\newacronym{isp}{ISP}{Internet Service Provider}
\newacronym{ngnm}{NGNM}{Next Generation Mobile Networks Alliance}
\newacronym{cc}{CC}{Component Carrier}
\newacronym{drb}{DRB}{Data Radio Bearer}
\newacronym{qos}{QoS}{Quality of Service}
\newacronym{ca}{CA}{Carrier Aggregation}
\newacronym{sdap}{SDAP}{Service Data Adaptation Protocol}
\newacronym{lc}{LC}{Logical Channel}
\newacronym{rnti}{RNTI}{Radio Network Temporary Identifier}
\newacronym{qci}{QCI}{Quality Class Identifier}
\newacronym{cdf}{CDF}{Cumulative Distribution Function}
\newacronym{cmos}{CMOS}{Complementary Metal-Oxide Semiconductor}
\newacronym{cn}{CN}{Core Network}
\newacronym{cqi}{CQI}{Channel Quality Information}
\newacronym{cir}{CIR}{Channel Impulse Response}
\newacronym{cp}{CP}{Control Plane}
\newacronym{cu}{CU}{Central Unit}
\newacronym{du}{DU}{Distributed Unit}
\newacronym{csirs}{CSI-RS}{Channel State Information - Reference Signal}
\newacronym{dc}{DC}{Dual Connectivity}
\newacronym{imsi}{IMSI}{International Mobile Subscriber Identity}
\newacronym{dce}{DCE}{Direct Code Execution}
\newacronym{dci}{DCI}{Downlink Control Information}
\newacronym{uci}{UCI}{Uplink Control Information}
\newacronym{dl}{DL}{Downlink}
\newacronym{dmr}{DMR}{Deadline Miss Ratio}
\newacronym{dmrs}{DMRS}{DeModulation Reference Signal}
\newacronym{e2e}{E2E}{End-to-End}
\newacronym{ecn}{ECN}{Explicit Congestion Notification}
\newacronym{edf}{EDF}{Earliest Deadline First}
\newacronym{enb}{eNB}{evolved Node Base}
\newacronym{embb}{eMBB}{enhanced Mobile Broadband}
\newacronym{epc}{EPC}{Evolved Packet Core}
\newacronym{es}{ES}{Edge Server}
\newacronym{fdma}{FDMA}{Frequency Division Multiple Access}
\newacronym{fdd}{FDD}{Frequency Division Duplexing}
\newacronym[firstplural=Radio Access Technologies (RATs)]{rat}{RAT}{Radio Access Technology}
\newacronym[firstplural=Markov Chains (MCs)]{mc}{MC}{Markov Chain}
\newacronym{milp}{MILP}{Mixed-Integer Linear Programming}
\newacronym{fs}{FS}{Fast Switching}
\newacronym{ip}{IP}{Internet Protocol}
\newacronym{fr}{FR}{Frequency Range}
\newacronym{ftp}{FTP}{File Transfer Protocol}
\newacronym{gnb}{gNB}{Next Generation Node Base}
\newacronym{arq}{ARQ}{Automatic Repeat reQuest}
\newacronym{harq}{HARQ}{Hybrid Automatic Repeat reQuest}
\newacronym{hetnet}{HetNet}{Heterogeneous Network}
\newacronym{hh}{HH}{Hard Handover}
\newacronym{hol}{HOL}{Head-of-Line}
\newacronym{ia}{IA}{Initial Access}
\newacronym{imt}{IMT}{International Mobile Telecommunication}
\newacronym{iot}{IoT}{Internet of Things}
\newacronym{lcr}{LCR}{Level Crossing Rate}
\newacronym{lcf}{LCF}{Level Crossing Frequency}
\newacronym{los}{LoS}{Line-of-Sight}
\newacronym{lte}{LTE}{Long Term Evolution}
\newacronym{m2m}{M2M}{Machine to Machine}
\newacronym{mac}{MAC}{Medium Access Control}
\newacronym{num}{NUM}{Network Utility Maximization}
\newacronym{ri}{RI}{Rank Index}
\newacronym{pmi}{PMI}{Precoding Matrix Index}
\newacronym{mcs}{MCS}{Modulation and Coding Scheme}
\newacronym{mec}{MEC}{Mobile Edge Cloud}
\newacronym{mi}{MI}{Mutual Information}
\newacronym{mimo}{MIMO}{Multiple Input, Multiple Output}
\newacronym{mmwave}{mmWave}{millimeter wave}
\newacronym{mptcp}{MPTCP}{Multipath TCP}
\newacronym{mr}{MR}{Maximum Rate}
\newacronym{mt}{MT}{Mobile Termination}
\newacronym{mss}{MSS}{Maximum Segment Size}
\newacronym{mtd}{MTD}{Machine-Type Device}
\newacronym{mtu}{MTU}{Maximum Transmission Unit}
\newacronym{nfv}{NFV}{Network Function Virtualization}
\newacronym{nf}{NF}{Network Function}
\newacronym{nlos}{NLoS}{Non-Line-of-Sight}
\newacronym{nr}{NR}{New Radio}
\newacronym{csi}{CSI}{Channel State Information}
\newacronym{o2i}{O2I}{Outdoor-to-Indoor}
\newacronym{ofdm}{OFDM}{Orthogonal Frequency Division Multiplexing}
\newacronym{pdcch}{PDCCH}{Physical Downlink Control Channel}
\newacronym{pdcp}{PDCP}{Packet-Data Convergence Protocol}
\newacronym{pdsch}{PDSCH}{Physical Downlink Shared Channel}
\newacronym{pdu}{PDU}{Packet Data Unit}
\newacronym{sdu}{SDU}{Service Data Unit}
\newacronym{pf}{PF}{Proportional Fair}
\newacronym{pgw}{PGW}{Packet Gateway}
\newacronym{phy}{PHY}{Physical}
\newacronym{pbch}{PBCH}{Physical Broadcast Channel}
\newacronym[plural=\gls{mme}s,firstplural=Mobility Management Entities (MMEs)]{mme}{MME}{Mobility Management Entity}
\newacronym{prb}{PRB}{Physical Resource Block}
\newacronym{pss}{PSS}{Primary Synchronization Signal}
\newacronym{pucch}{PUCCH}{Physical Uplink Control Channel}
\newacronym{pusch}{PUSCH}{Physical Uplink Shared Channel}
\newacronym{rach}{RACH}{Random Access Channel}
\newacronym{ran}{RAN}{Radio Access Network}
\newacronym{ngran}{NG-RAN}{Next Generation RAN}
\newacronym{red}{RED}{Random Early Detection}
\newacronym{rf}{RF}{Radio Frequency}
\newacronym{rlc}{RLC}{Radio Link Control}
\newacronym{rlf}{RLF}{Radio Link Failure}
\newacronym{rrc}{RRC}{Radio Resource Control}
\newacronym{rrm}{RRM}{Radio Resource Management}
\newacronym{rr}{RR}{Round Robin}
\newacronym{rs}{RS}{Remote Server}
\newacronym{rsrp}{RSRP}{Reference Signal Received Power}
\newacronym{rss}{RSS}{Received Signal Strength}
\newacronym{rtt}{RTT}{Round Trip Time}
\newacronym{rw}{RW}{Receive Window}
\newacronym{rx}{RX}{Receiver}
\newacronym{sa}{SA}{standalone}
\newacronym{sack}{SACK}{Selective Acknowledgment}
\newacronym{sap}{SAP}{Service Access Point}
\newacronym{sch}{SCH}{Secondary Cell Handover}
\newacronym{scoot}{SCOOT}{Split Cycle Offset Optimization Technique}
\newacronym{sdma}{SDMA}{Spatial Division Multiple Access}
\newacronym{sdn}{SDN}{Software Defined Networking}
\newacronym{sinr}{SINR}{Signal-to-Interference-plus-Noise Ratio}
\newacronym{sir}{SIR}{Signal-to-Interference Ratio}
\newacronym{sm}{SM}{Saturation Mode}
\newacronym{snr}{SNR}{Signal-to-Noise Ratio}
\newacronym{son}{SON}{Self-Organizing Network}
\newacronym{ss}{SS}{Synchronization Signal}
\newacronym{srs}{SRS}{Sounding Reference Signal}
\newacronym{sss}{SSS}{Secondary Synchronization Signal}
\newacronym{tb}{TB}{Transport Block}
\newacronym{tcp}{TCP}{Transmission Control Protocol}
\newacronym{tdd}{TDD}{Time Division Duplexing}
\newacronym{tdma}{TDMA}{Time Division Multiple Access}
\newacronym{tfl}{TfL}{Transport for London}
\newacronym{tm}{TM}{Transparent Mode}
\newacronym{trp}{TRP}{Transmitter Receiver Pair}
\newacronym{tti}{TTI}{Transmission Time Interval}
\newacronym{ttt}{TTT}{Time-to-Trigger}
\newacronym{tx}{TX}{Transmitter}
\newacronym{qam}{QAM}{Quadrature Amplitude Modulation}
\newacronym{ue}{UE}{User Equipment}
\newacronym{ul}{UL}{Uplink}
\newacronym{uml}{UML}{Unified Modeling Language}
\newacronym{um}{UM}{Unacknowledged Mode}
\newacronym{uma}{UMa}{Urban Macro}
\newacronym{utc}{UTC}{Urban Traffic Control}
\newacronym{vm}{VM}{Virtual Machine}
\newacronym{rsrq}{RSRQ}{Reference Signal Received Quality}
\newacronym{rssi}{RSSI}{Received Signal Strength Indicator}
\newacronym{crs}{CRS}{Cell Reference Signal}
\newacronym{nsa}{NSA}{Non Stand Alone}
\newacronym{mrdc}{MR-DC}{Multi \gls{rat} \gls{dc}}
\newacronym{eutra}{E-UTRA}{Evolved Universal Terrestrial Radio Access}
\newacronym{endc}{EN-DC}{E-UTRAN-\gls{nr} \gls{dc}}
\newacronym{5gc}{5GC}{5G Core}
\newacronym{si}{SI}{Study Item}
\newacronym{iab}{IAB}{Integrated Access and Backhaul}
\newacronym{wf}{WF}{Wired-first}
\newacronym{hqf}{HQF}{Highest-quality-first}
\newacronym{pa}{PA}{Position-aware}
\newacronym{mlr}{MLR}{Maximum-local-rate}
\newacronym{wbf}{WBF}{Wired Bias Function}
\newacronym{mib}{MIB}{Master Information Block}
\newacronym{sib}{SIB}{Secondary Information Block}
\newacronym{kpi}{KPI}{Key Performance Indicator}
\newacronym{ppp}{PPP}{Poisson Point Process}
\newacronym{mpc}{MPC}{Multi Path Component}
\newacronym{rt}{RT}{Ray Tracer}
\newacronym{aoa}{AoA}{Angle of Arrival}
\newacronym{aod}{AoD}{Angle of Departure}
\newacronym{scm}{SCM}{Spatial Channel Model}
\newacronym{inr}{INR}{Interference to Noise Ratio}
\newacronym{qd}{QD}{Quasi Deterministic}
\newacronym{wlan}{WLAN}{Wireless Local Area Network}
\newacronym{cad}{CAD}{Computer-aided Design}
\newacronym{ap}{AP}{Access Point}
\newacronym{sta}{STA}{Station}
\newacronym{urllc}{URLLC}{Ultra-Reliable Low-Latency Communication}
\newacronym{udp}{UDP}{User Datagram Protocol}
\newacronym{upf}{UPF}{User Plane Function}
\newacronym{umi}{UMi}{Urban Microcell}
\newacronym{uma2}{UMa}{Urban Macrocell}
\newacronym{rma}{RMa}{Rural Macrocell}
\newacronym{in}{In}{Indoor Office}
\newacronym{wpans}{WPANs}{Wireless Personal Area Networks}
\newacronym{wpan}{WPAN}{Wireless Personal Area Network}
\newacronym{fd}{FD}{Full Duplex}
\newacronym{crc}{CRC}{Cyclic Redundancy Check}
\newacronym{wb}{WB}{Wideband}
\newacronym{sb}{SB}{Subband}
\newacronym{bap}{BAP}{Backhaul Adaptation Protocol}
\newacronym{lcg}{LCG}{Logical Channel Group}
\newacronym{ecdf}{ECDF}{Empirical Cumulative Distribution Function}
\newacronym{mu}{MU}{Multi-User}
\newacronym{ce}{CE}{Control Element}
\newacronym{ric}{RIC}{RAN Intelligent Controller}
\newacronym{bler}{BLER}{Block Error Rate}
\newacronym{ieee}{IEEE}{Institute of Electrical and Electronics Engineers}
\newacronym{ilp}{ILP}{Integer Linear Program}
\newacronym{ldpc}{LDPC}{Low-Density Parity Check}
\newacronym{nlosv}{NLOSv}{Vehicle Non-Line-of-Sight}
\newacronym{pscch}{PSCCH}{Physical Sidelink Control Channel}
\newacronym{sc}{SC}{Single Carrier}
\newacronym{sl}{SL}{Sidelink}
\newacronym{dft}{DFT}{Discrete Fourier Transform}
\newacronym{v2v}{V2V}{Vehicle-to-Vehicle}
\newacronym{wave}{WAVE}{Wireless Access in Vehicular Environments}
\newacronym{upa}{UPA}{Uniform Planar Array}
\newacronym{fec}{FEC}{Forward Error Correction}
\newacronym{psfch}{PSFCH}{Physical Sidelink Feedback Channel}
\newacronym{pssch}{PSSCH}{Physical Sidelink Shared Channel}
\newacronym{csma}{CSMA}{Carrier Sense Multiple Access}
\newacronym{v2n}{V2N}{Vehicle-to-Network}
\newacronym{cav}{CAV}{Connected and Autonomous Vehicle}
\newacronym{v2i}{V2I}{Vehicle-to-Infrastructure}
\newacronym{d2d}{D2D}{Device-to-Device}
\newacronym{c-its}{C-ITS}{Connected Intelligent Transportation System}
\newacronym{fr2}{FR2}{Frequency Range 2}
\newacronym{bs}{BS}{Base Station}
\newacronym{scs}{SCS}{Subcarrier Spacing}
\newacronym{sumo}{SUMO}{Simulation of Urban MObility}
\newacronym{prr}{PRR}{Packet Reception Ratio}
\newacronym{edca}{EDCA}{Enhanced Distribution Channel Access}
\newacronym{thz}{THz}{terahertz}
\newacronym{6g}{6G}{6th generation}
\newacronym{uav}{UAV}{unmanned aerial vehicles}
%\input{myTikz.tex}


% \begin{document}

% \title{6G Integrated Access and Backhaul Networks\\with Sub-Terahertz Links} % temp

% \author{\IEEEauthorblockN{Amir Ashtari Gargari$^*$, Matteo Pagin$^*$, Michele Polese$^{\circ}$, Michele Zorzi$^*$}\\
%   \IEEEauthorblockA{$^*$Department of Information Engineering, University of Padova, Italy\\email: \texttt{\{amirashtari, paginmatte, zorzi\}@dei.unipd.it}\\
%   $^{\circ}$Institute for the Wireless Internet of Things, Northeastern University, Boston, MA\\email: \texttt{m.polese@northeastern.edu}}
%   \thanks{This work was partially supported by the U.S. National Science Foundation under Grant CNS-2225590 and in part by the EU MSCA ITN project MINTS “MIllimeter-wave NeTworking and Sensing for Beyond 5G” (grant no. 861222).}
% }

% \flushbottom
% \setlength{\parskip}{0ex plus0.1ex}

% \maketitle
% \thispagestyle{empty}

% % 3GPP does not want us to use New Radio, so we disable the expansion of the acronym
% \glsunset{nr}

\section{High-capacity Integrated Access and Backhaul Networks using sub-Terahertz Links}
\label{sec:iab-wons}

The spectrum above 100 GHz has several sub-bands that could provide bandwidths wider than 10 GHz, thus potentially data rates in the excess of tens of Gbps~\cite{akyildiz2014terahertz}. Backhaul---a static deployment---is a promising use case for sub-terahertz links, which need pencil-sharp beams to close the link budget and are thus less resilient to mobility compared to traditional sub-6 GHz or \gls{mmwave} frequencies. 

In recent years, the literature has closed several gaps in terms of circuit, antenna design~\cite{singh2020design} and physical and \gls{mac} layer solutions for sub-terahertz systems~\cite{ghafoor2020mac}.
%
When it comes to \gls{iab} with mixed sub-terahertz and \gls{mmwave} links,\footnote{In this section, we consider the FR2 range of 3GPP NR (24.25 GHz to 71 GHz) as \glspl{mmwave}.} however, there are still several open questions in terms of network design and path selection. In this work, we consider the problem of identifying a viable topology between \gls{iab} nodes and the \gls{iab} donors, including the carrier frequency of the backhaul links, and profile the performance that network planners can expect when mixing sub-terahertz and \gls{mmwave} \gls{iab} links.

To this end, 
%
we develop a greedy path generation algorithm that automatically selects the frequency band of an \gls{iab} link (between 28 GHz and 140 GHz) and assigns routes so that each \gls{iab} node can reach the \gls{iab} donor. The frequency selection aims at avoiding bottlenecks, i.e., the algorithm selects the band that provides the highest capacity when accounting for the congestion that may arise in the proximity of the IAB donor. In addition, we consider and compare different ratios of sub-terahertz and mmWave links, which can be mapped to licensing constraints for out-of-band backhaul, and two different bandwidths for the sub-terahertz links (10 GHz and 32 GHz), which consider exclusive licensing or sharing with other services, respectively~\cite{polese2022dynamic}.

We model the \gls{iab} network in a custom-developed \gls{3gpp} Release 17 simulator based on the open-source tool Sionna~\cite{hoydis2022sionna}, with \gls{3gpp} and state-of-the-art \gls{mmwave} and sub-terahertz channel models, and realistic and detailed \gls{3gpp}-based physical and \gls{mac} layers. Our results quantify for the first time the performance improvement that sub-terahertz links can introduce in \gls{iab} networks, which can push beyond the limits of the in-band \gls{mmwave} backhaul and support more than 50 users with 120 Mbps streams and a single donor without congestion (compared to about 33 Mbps for in-band \glspl{mmwave}). 

This is the first work that provides a numerical evaluation of the potential associated with sub-terahertz links for \gls{iab}. Notably, \cite{9135258} evaluates the sub-\gls{thz} potential in backhaul networks from a physical layer perspective. This research demonstrates that sub-THz spectrum links can achieve multi-Gbps ratios in outdoor backhaul scenarios. \cite{9163026} proposed \gls{uav}-assisted backhaul solution to improve network coverage and data rate in heterogeneous networks with multiple tiers composed of sub-6 GHz, THz and \gls{uav} layers. In addition, the authors of \cite{9136652} successfully adopted concurrent scheduling to increase system throughput in dense THz backhaul scheduling. Finally,~\cite{saha2018integrated} considers a multi-band \gls{iab} deployment, but with a bandwidth that is more limited than those considered in future 6G scenarios. 

The rest of the section is organized as follows. Sec.~\ref{sec:system} introduces the system model. Sec.~\ref{sub:THzLinkSel} describes the algorithm for frequency and path selection, which is then numerically evaluated in Sec.~\ref{sec:PerfEval}.


% \hl{Matteo: Rename this ?}
\subsection{System Model}
\label{sec:system}

We consider a \gls{tdma} system in which a single \gls{iab} donor, featuring a fiber connectivity towards the \gls{cn} and the Internet, exchanges data with $N_{\mathrm{U}}$ \glspl{ue}. Without loss of generality, we consider uplink traffic only.
To achieve uniform coverage, the donor is aided by $N_{\mathrm{I}}$ \gls{iab} nodes, which can be connected either to the former or to neighboring base stations, thus possibly realizing a multi-hop wireless backhaul. 

We partition the time resources in $T$ radio subframes of duration $T_{sub} = 1$~ms, and we equip all nodes with buffers. Accordingly, the data that node $i$ transmits to \gls{gnb} $k$ during subframe $t$ is stored in its buffer $B_k (t)$, and represents either successfully received packets, in the case of the donor, or data to be relayed to the next hop along the path during subframe $t + 1$, in the case of \gls{iab} nodes.

We assume that the backhaul links operate \textit{either in the \gls{mmwave} or in the \gls{thz} band} and that each \gls{iab} node features two \gls{rf} chains, which are used for the backhaul and the fronthaul communications, respectively. In both cases, \glspl{gnb} are equipped with directional antennas.
%In the former case, the multiple antennas are used to steer the beam towards the intended destination using analog beamforming.

When \gls{gnb} $k = 0, \ldots, N_{\mathrm{I}}$, with index $0$ denoting the \gls{iab} donor, receives data from node $j$, packets experience a \gls{sinr} $\gamma_{s, d}$ which can be expressed as
\begin{equation}
\label{eq:sinr_common}
    \gamma_{s, d} = \frac{ \vert h_{s, d}^{l} \vert^2 \sigma_{x}^2 }{\sigma_{n}^2 + \sum_{i \in \mathcal{I}} \sigma_{i}^ 2}, 
\end{equation}
where $h_{s, d}^{l}, \, l \in \{mW, sT\} $ represents the equivalent channel response between the communication endpoints when using \gls{mmwave} or sub-\gls{thz} links, respectively. $\mathcal{I}$ denotes the set of interferers, $\sigma_{x}^2$, $\sigma_{i}^2$ and $\sigma_{n}^2$ are the powers of the transmitted signal, the $i$-th received interfering signal, and the thermal noise at the receiver, respectively. 
% Furthermore, $\bm{w}_{s}$ and $\bm{w}_{d}$ denote the beamforming vector used at S and D, respectively.

The corresponding access (backhaul) throughput $R_{j, k}^{\mathrm{A}} (t)$ ($R_{j, k}^{\mathrm{B}} (t)$) reads
%Assuming that the \gls{snr} at \gls{gnb} $k$, when receiving data from node $j$ is $\gamma_{j, k}$, the corresponding access (backhaul) throughput $R_{j, k}^{\mathrm{A}} (t)$ ($R_{j, k}^{\mathrm{B}} (t)$) reads:
\begin{equation}
  R_{j, k}^{\mathrm{A}} (t) = \frac{1}{T_{sub}} \sum_{l=1}^{B_{j}^t} \mathbbm{1} \left\{ \hat{b}_{l} (\gamma_{j, k}) = b_l \right\} ,
\end{equation}
where $B_{j}^t$ denotes the number of bits transmitted from user (\gls{iab} node) $j$ to \gls{gnb} $k$ during subframe $t$ and %and $b_l$ one of such bits. 
$\hat{b}_{l} (\gamma_{j, k})$ is the $l$-th decoded bit at the receiver, as a function of $\gamma_{j, k}$.

Our goal is to maximize the average system sum-rate, defined as
%\frac{1}{T} \sum_{k = 1}^{N_{\mathrm{U}}} \sum_{t = 1}^{T} R_{k, 0}^{\mathrm{A}} (t) +
\begin{equation}
 \bar{R} \doteq \frac{1}{T} \sum_{j = 1}^{N_{\mathrm{I}}} \sum_{t = 1}^{T} R_{j, 0}^{\mathrm{B}} (t),
\end{equation}
%
by tuning the carrier frequency (either \gls{mmwave} or \gls{thz}) of each backhaul link. We remark that in this metric we take into account only the packets which are received at their final destination, i.e., the \gls{iab} donor.

\subsubsection{Channel Models}
\label{sub:channelmodel}
\paragraph{\gls{mmwave} channel model}

%\hl{TODO, Matteo: Trim further?}
%\hl{TODO, Matteo: Consider adding a couiple of sentences on mmWave's propagation characteristics}\\
For the \gls{mmwave} links, we consider the 3GPP~38.901 \gls{scm}~\cite{3gpp.38.901}, which models \gls{mimo} wireless channels for frequencies between $0.5$ and $100$~GHz. %The model relies on deterministic formulas for the path-loss, and on random distributions for phenomena such as multipath fading and shadowing. 

In particular,~\cite{3gpp.38.901} outlines the procedures for generating a channel matrix $\bm{H}_{s, d}$ whose entries $h_{s, d}^{j, k}$ correspond to the impulse response of the channel between the $j$-th element of the antenna array of the transmitter (S), and the $k$-th radiating element of the antenna array of the receiver (D). %at time $t$ and with propagation delay $\tau$. 
%To model small-scale fading, each of the entries of entries $\bm{H}_{p, q}$ 
% Each of these entries is computed as the superposition of $N$ different clusters, each of which consists of $M$ rays. %that arrive (depart) to (from) the transmitter (receiver) antenna array, 
% with specific powers and angles of departure and arrival. 
Then, the channel matrix entries are combined with a frequency-flat path loss term $PL$. 

When considering analog beamforming at both the transmitter and the receiver, the equivalent channel response $h_{s, d}^{mW}$ can be evaluated as
\begin{equation}
\label{eq:sinr}
    h_{s, d}^{mW} = \sqrt{10^{PL/10}} \cdot \bm{w}_{d} \bm{H}_{s, d} \bm{w}_{s},
\end{equation}
%
with $\bm{w}_{s}$ and $\bm{w}_{d}$ the beamforming vectors used at S and D, respectively.

\begin{figure*}[t!]
% \begin{strip}
\begin{subequations}
    \begin{equation}
        \argmax_{\substack{\bm{P}, \{\bm{S}(t) \}_t, \bm{T}}} \bar{R}, \label{eq:opt_overall}
    \end{equation}
    \vspace{-10pt}
    \begin{alignat}{1}
     \text{s.t.}\; \text{C1:} \;& R_{j, k}^{\mathrm{B}} (t) T_{sub} \leq B_j (t) \;\; \forall \, j, \,  \forall\, t \label{eq:con_rate_vs_buff} \\ 
     \text{C2:} \;& B_j (t + 1) = \, B_j (t) + T_{sub} \left( \sum_{k=1}^{N_{\mathrm{U}}} R_{k, j}^{\mathrm{A}} (t) + \sum_{k=1}^{N_{\mathrm{I}}} R_{k, j}^{\mathrm{B}} (t) 
     - \sum_{k=0}^{N_{\mathrm{I}}} R_{j, k}^{\mathrm{B}} (t) \right) \;\; \forall \, j, \, \forall t \label{eq:con_buff_time} \\
     \text{C3:} \;& \sum_{k=0}^{N_{\mathrm{I}}} \bm{S} \left[ j, k \right] (t)  +
     \sum_{k=1}^{N_{\mathrm{I}}} \bm{S} \left[ k, j \right] (t) \leq 1 \;\; \forall \, j, \, \forall t \label{eq:con_tdd}\\
     \text{C4:} \;& R_{j, k}^{\mathrm{B}} (t) \bm{S} \left[ j, k \right] (t) = R_{j, k}^{\mathrm{B}} (t) \;\;
     \forall \, j, \, \forall k, \, \forall t \label{eq:con_rate_if_active} \\
     \text{C5:} \;& \sum_{j, k=0}^{N_{\mathrm{I}}} \bm{T} [j, k] \leq \rho_{max} \sum_{j, k=0}^{N_{\mathrm{I}}} \bm{P} [j, k]  \label{eq:con_num_thz}
    \end{alignat}
\end{subequations}
% \end{strip}
\end{figure*}

\paragraph{THz channel model}
\label{sub:thzchannel}
% This section describes the \gls{thz} channel model we utilize in the system model. 
%
% MP so far you have discussed generic modeling, now you mention simulation for the first time without providing details on what this simulation is. Be consistent and either introduce this as the channel for the system model here, or move the simulation part first and talk about the channel for simulation later. [done]
%
% Several measurement campaigns and channel modeling methodologies have been done to characterize the properties of the \gls{thz} spectrum in various scenarios and environments. 
For sub-THz, we use the physics-based channel modeling approach from~\cite{5995306}, which includes molecular absorption and path loss.
%, a channel model based on electromagnetic transmission between devices in the \gls{thz} range, which is general and easy to employ. This~\cite{5995306} research uses 
%to provide a channel model for EM nano communication devices. 
At THz-band frequencies, molecular absorption, which causes both molecular absorption loss and molecular absorption noise, is the principal factor affecting electromagnetic wave propagation. $h_{s, d}^{tH}$ is the THz-band channel model introduced in~\cite{5995306}, with additional transmit and receive antenna gains $G_S$ and $G_D$, and is given by
%
% MP why this one and not one of the SCM models which come with fading and are more similar to the mmWave channel structure? [done]

\begin{equation}
% \setlength{\abovedisplayskip}{0pt}
% \setlength{\belowdisplayskip}{2pt}
h_{s, d}^{tH}(f,d)=  \frac{c}{4 \pi f d} \exp \left( -\frac{k_{abs}(f)d}{2} \right)  G_{S}  G_{D} ,
\label{EQ_PHY}
\end{equation}
%
where $c$ stands for the speed of light and $k_{abs}$ for the medium's molecular absorption coefficient, based on the type and composition of molecules~\cite{hossain2018terasim}.
% , and $ G_{TX}$ and $ G_{RX}$ determines antenna gain for transmitter and receiver, respectively.
%the \gls{sinr} for \gls{thz} is formulated in(\ref{eq:thzSINR}) which is highly aligned with \gls{mmwave} \gls{sinr} formula (\ref{eq:sinr}) with the exception that beamforming is not used in \gls{thz}
% MP why not? if no antenna gain is modeled in the simulations, it is a problem. 
% MP also checking spacing, capitalization

%\begin{equation}
%\label{eq:thzSINR}
%    \gamma_{s, d} = \frac{ \vert \bm{h}_{{THz}}\vert^2 \sigma_{x}^2 }{\sigma_{n}^2 + \sum_{i \in \mathcal{I}} \sigma_{i}^ 2},
%\end{equation}

\subsection{Sum-rate optimization via THz Link Selection}
\label{sub:THzLinkSel}

We define $\bm{P} \in \{0, 1\}^{N_{\mathrm{I}} + 1 \times N_{\mathrm{I}} + 1}$ as the matrix which represents the possible active links among \glspl{gnb}, i.e., $\bm{P} [i, j] = 1$ if and only if the wireless backhaul link between \glspl{gnb} $i$ and $j$ is a feasible link; index $0$ refers to the donor. Similarly, $\bm{S} (t) \in \{0, 1\}^{N_{\mathrm{I}} + 1 \times N_{\mathrm{I}} + 1}$ and $\bm{T} \in \{0, 1\}^{N_{\mathrm{I}} + 1 \times N_{\mathrm{I}} + 1}$ represent the links which are active during subframe $t$, and whether they use \gls{thz} spectrum or not, respectively. 
Our objective is to maximize the average system sum-rate, by choosing whether each link is operating in the \gls{thz} or the \gls{mmwave} band and the active links in each subframe. We perform the choice of $\bm{T}$ and $\bm{P}$ only once, with the goal of reducing the computational complexity of the algorithm.

% \hl{TODO, Matteo: Move equation on top of the page (not float: needs to be done manually, once the rest is finished).}
%\begin{equation}
%    \argmax_{\substack{P, M}} \hat{R}     
%\end{equation}


The optimization problem is thus formulated as \eqref{eq:opt_overall}. Constraint C1 ensures that nodes do not transmit more data than available in their buffer. C2 enforces the proper evolution over time of the buffers occupancy, i.e., the buffer occupancy at time $t$ must be equal to the one in subframe $t - 1$, minus (plus) the outgoing (incoming) traffic from other nodes. Constraint C3 relates to the \gls{tdma} mode of operation, and ensures that each backhaul \gls{rf} chain is used at most for one transmission/reception in any given subframe, while C4 imposes that only active links can exhibit a positive rate. 
Finally, with C5 we set an upper bound $\rho_{max}$ on the maximum percentage of \gls{thz} links. 
% MP in general I prefer to introduce constraints one by one with comments and then mention the optimization subject to (1) (2) etc. It's ok to keep it as it is, but need to provide more details on the constraints (why are they needed, etc) [done]

\subsubsection{Backhaul Scheduler}
\label{sub:BackSche}

We remark that due to the binary nature of the $\bm{P}, \bm{S}(t)$ and $\bm{T}$ optimization variables, \eqref{eq:opt_overall} is an \gls{ilp}, thus NP-hard and not solvable in polynomial time. Therefore, in this section, we present a set of algorithms that solve the path selection and configuration problem heuristically and with low complexity. 

Specifically, we first describe the pre-processing steps, referred to as \textit{distance-aware path generation} (Alg.~\ref{algo:distanceAware}) and \textit{THz-link selection} (Alg.~\ref{algo:thzLink}), which prune the set of possible links established among \glspl{gnb} and decide which of them are to operate in the THz bands, respectively. Then, we describe the \textit{\gls{sinr}-based scheduler} (Alg.~\ref{algo:sinr}), which differs from the former procedures as it is executed in each subframe to track the dynamic nature of the backhaul network.
% , links must be scheduled for each time step.
% MP explain why [done]

\begin{algorithm}[t]
\small
	\caption{Distance Aware Path Generation} 
	\begin{algorithmic}
        %  \hspace*{\algorithmicindent} \textbf{Input}
        % \hspace*{\algorithmicindent} \textbf{Output}
        % \Output{11 round keys each of 4 words as $w[0], \dots, w[43]$}
        %\State $N_{\mathrm{I}} =$ Number of IAB nodes
        %\State $d_{max} =$ Max distance for which a link is activated
    \State $d_{max} \gets$ Max distance between IAB nodes of the same tier
	\State $\bm{P} = [0]_{N_{\mathrm{I}} + 1 \times N_{\mathrm{I}} + 1}$
		\For {$n_i=1,2,\ldots, N_{\mathrm{I}}$}
                \State $d_i \gets $ 3D distance between $n_i$ and IAB donor
                \If {$d_i~<~d_{max}$}
                		\State $ \bm{P} [n_i, 0]~=~1$
                \EndIf
                %\If{$n_1 \neq N_{\mathrm{I}}-1$}
			\For {$n_j=n_1+1,\ldots, N_{\mathrm{I}}$}  
				\State $d_{i, j} \gets$ 3D distance between $n_i$ and $n_j$ 
				\If{$d_{i, j} < d_{max}$}
    				\State $d_j \gets$ 3D distance between $n_j$ and IAB donor
    				\If {$d_i~<~d_j$}
        				\State $ \bm{P} [n_j, n_i]~=~1$
    				\Else 
    				    \State $ \bm{P} [n_i, n_j]~=~1$
				    \EndIf
				\EndIf
			\EndFor
            %\EndIf
		\EndFor
	\end{algorithmic} 
\label{algo:distanceAware}
\end{algorithm}

The distance-aware path generation algorithm 
% is in charge of computing 
% MP you can be more direct when writing, I changed to "computes" [done]
computes the $\bm{P}$ matrix, which encodes the potential connections between IAB nodes. $\bm{P}$ reduces the system complexity by restricting possible paths from each IAB node and by avoiding loops.
Specifically, Alg.~\ref{algo:distanceAware} iterates over each IAB node $n_j$, establishing a connection towards the donor whenever the distance between them is smaller than $d_{max}$, i.e., 
% $d_{max}$ specifies 
a scenario- and frequency-dependent distance that guarantees a link performance above a certain threshold.
% and defined by user in simulations, which varies based on scenario and frequency. 
In our case, the considered scenario involves a small and dense deployment of IAB nodes, so the path loss distance can be compensated by the antenna gain, and $d_{max}$ for THz and mmWave are assumed to have the same value.
% MP what is d_max? Who chooses this? And why? [done]
% MP also, in terms of distance, are THz and mmWave link equivalent (e.g., the scenario is small and the pathloss distance can be compensated through beamforming?) or not? If not, it may make sense to do step 1 and 2 jointly to exploit the fact that THz link will be worse than mmWaves in some conditions (NLOS, long distance, etc) [done]
Moreover, the proposed pre-processing step performs additional attachments between neighboring nodes, as long as the resulting link exhibits a lower length than $d_{max}$. The direction of such link is determined in such a way that the destination node is the closer to the donor.
Even though this link may be topologically redundant, it can provide an alternative route for load balancing purposes, while still avoiding the creation of cycles. 

%In the opposite case, i.e., whenever a node $n_k$, neighbor of $n_j$, is closer to the donor than to the former, an attachment from $n_j$ towards $n_k$ is performed. 
% MP do you still have a DAG or does this create cycles? 3GPP IAB need to be a DAG (or a spanning tree which is a special case)
% MP why is NLOS not modeled? due to high path loss, we don't go through nlos in outdoor. [done]

The THz link selection policy 
% aims at identifying  
% MP aims at identifying -> identifies 
identifies
bottleneck links based on two heuristics:
\begin{enumerate*}[label=\arabic*)]
    \item links involving IAB nodes which are closer to the donor are more likely to be congested since they are usually used also for relaying traffic of subtending nodes; and
    \item the average buffer occupancy provides an estimate of the loads incurred on each link.
% MP average buffer occupancy is something that changes very dynamically, need some good motivation to consider this as a preprocessing step done once in a while vs. something done in each slot.
   % \item marking a link as THz can help prevent congestion, thanks to the higher bandwidth available at these frequencies.
% MP this is not a criterion, it's a consequence of the selection
\end{enumerate*}
Accordingly, Alg.~\ref{algo:thzLink} partitions the \gls{iab} nodes into disjoint sets, referred to as \textit{tiers}. Nodes are assigned to tiers based on their distance with respect to the donor, with tier $0$ indicating the closest level to the donor. 
Then, the various backhaul links are marked as \gls{thz} in descending order with respect to the tier of the corresponding transmitting node, until the maximum ratio of non-mmWave links $\rho_{max}$ is reached.
% MP the algorithm algo:thzLink is not very clear, and this description does not provide enough clarity
%\hl{todo, AMIR: add details here}
Note that the algorithm may eventually reach a tier whose IAB nodes are not all set as \gls{thz}. In this case, ties within the same tier are broken by sorting its nodes with respect to their average traffic load, which we estimate by measuring the respective buffers. That is to say, nodes with higher buffer occupancy are given priority and thus are set as \gls{thz} before nodes exhibiting a lower traffic load. Note that this procedure can be based on long-term statistics, thus averaging the load of the nodes over multiple frames.
%Nsort is the average load per IAB node that we calculate in advance for various configurations. Therefore, the tier and Nsort parameters indicate which links must be converted from mmwave to THz. remainL is initiated by L, which specifies the number of links that must be converted to THz and the total number of THz links in the network, respectively. To activate links, we perform the following steps: if remainL is greater than the number of links in tier i all links in tier i are selected as Sub-THz; otherwise, we sort links in tier I based on Nsort and select remainL links with the highest load ratio. We traverse all tiers until remainL becomes zero.

Finally, the \gls{sinr}-based scheduler dynamically allocates resources, with the objective of maximizing the average sum rate %by minimizing Interference with respect to network load.  
by choosing a list of paths to be activated in each subframe. The rationale behind the proposed scheme is to schedule links based on their load. Specifically, in Alg.~\ref{algo:sinr} we assign a transmission resource allocation priority which is directly proportional to the buffer occupancy of the transmitting node. Once the first endpoint is chosen, we determine the outgoing link by selecting the one with the highest \gls{sinr} among those calculated in Alg.~\ref{algo:distanceAware}. Then, we set all links involving the corresponding transmitting and receiving nodes as infeasible (assigning zero to the corresponding transmitting ($n$) and receiving node ($p_n^{*}$) indices in $\bm{P}_{temp}$), and repeat the procedure by considering the remaining nodes and links only, thus ensuring that the \gls{tdma} constraint is satisfied. 
% older version

\begin{algorithm}[t]
\small
	\caption{\gls{thz} Link Selection}
	\begin{algorithmic}
        %  \hspace*{\algorithmicindent} \textbf{Input}
        % \hspace*{\algorithmicindent} \textbf{Output}
        % \Output{11 round keys each of 4 words as $w[0], \dots, w[43]$}
        %\State $N_{\mathrm{I}} =$ Number of IAB nodes
        %\State $T =$ Number of IAB node tiers 
        %\State $\rho_{max} = $ Max fraction of THz links  
        \State $\bm{N}_{\mathrm{T}} =$ Vector of IAB nodes tier index
        \State $\bm{N}_{sort} =$ Vector of IAB node indices, sorted with respect to their load 
        \State $\bm{T} \gets [0]_{N_{\mathrm{I}} + 1 \times N_{\mathrm{I}} + 1} $
        %\State $\bar{B}_{j} =$ Average buffer occupancy of the $j$-th \gls{gnb}
         %\State $\bm{N}_{\mathrm{T}} \gets [0]_{N_{\mathrm{I}}}$
        \State $d_{max} \gets$ Max distance between IAB nodes of the same tier
        
	\For {$n=1, 2,\ldots, N_{\mathrm{I}}$}
            \State $d \gets $ 3D distance between $n$ and IAB donor
            \State $\bm{N}_{\mathrm{T}}[n] \gets \left\lfloor d~/~d_{max} \right\rfloor $
        \EndFor
            %\State $remain_L = L$ \hl{Not clear, ask Amir}
            \For {$i=1,\ldots, \operatorname{max}(\bm{N}_{\mathrm{T}})$}
                \State $\bm{N}_{T}^{i} \gets \left\{ j  \; | \; \bm{N}_{T}[j] == i \right\}$ 
                \State $\bm{L}_{i} \gets$ links in $\bm{P}$ where nodes of $\bm{N}_{T}^{i}$ are the transmitting node 
                \If {$ \textstyle\sum_{j, k} \bm{T} [j, k] + \operatorname{dim} (\bm{L}_{i}) < \rho_{max} \sum_{j, k} \bm{P} [j, k]$}
                    \State $\bm{T}[j, k] \gets 1 \; \forall \; (j, k) \in \bm{L}_i$ 
                \Else
                    \While {$ \textstyle\sum_{j, k} \bm{T} [j, k] < \rho_{max} \sum_{j, k} \bm{P} [j, k]$} 
                    \State $n^{*} \gets \operatorname{min}_n \, | \, \bm{N}_{sort} [n] \cap \bm{N}_{T}^{i} \neq \emptyset $ 
                    \State $(n^{*}, k) \gets \text{ link } \in \bm{L}_{i} \, | \, n^{*}$ is the transmitting node
                    \State $\bm{T} [n^{*}, k] \gets 1$; $\bm{L}_{i} \gets \bm{L}_{i} \setminus (n^{*}, k) $ 
                    \EndWhile
                \EndIf                
	\EndFor
	\end{algorithmic} 
\label{algo:thzLink}
\end{algorithm}


\begin{algorithm}[t]
\small
	\caption{\gls{sinr}-based Scheduler} 
	\begin{algorithmic}
        %  \hspace*{\algorithmicindent} \textbf{Input}
        % \hspace*{\algorithmicindent} \textbf{Output}
        % \Output{11 round keys each of 4 words as $w[0], \dots, w[43]$}
        %\State $N_{\mathrm{I}} =$ Number of IAB nodes
        %\State $K =$ Number of paths 
        \State $\bm{N}_{sort} =$ Vector of IAB nodes, sorted with respect to their load 
        \State $\bm{P}_{temp} =  \bm{P} $
        \State $\bm{S} (t) \gets [0]_{N_{\mathrm{I}} + 1 \times N_{\mathrm{I}} + 1} $
        
	\For {$n$ in $\bm{N}_{sort}$}
            \State $\gamma_{max} \gets - \infty$
            %\State $temp_{list} \gets \text{where~} P_{temp}[n]==1$
            %\For {$i$ in $temp_{list}$}
            \For {$i$ in $0, \ldots, N_{\mathrm{I}}$}
                \If {$\gamma_{n, i} > \gamma_{max}$}
                    \State $\gamma_{max} \gets \gamma_{n, i}$
                    \State $p_n^{*} \gets i$
                \EndIf
		    \EndFor
		\State $\bm{S} (t) [n, p_n^{*}] \gets 1$
        \State $\bm{P}_{temp} [:, n], [n, :] \gets [0] $; $\bm{P}_{temp} [:, p_n^{*}], [p_n^{*}, :] \gets [0] $
	\EndFor
	\end{algorithmic} 
\label{algo:sinr}
\end{algorithm}




\subsection{Performance Evaluation}
\label{sec:PerfEval}
This section introduces a performance evaluation based on a novel simulation setup (Sec.~\ref{sec:SimSetup}) in a dense cellular network (Sec.~\ref{sub:SimScenario}), with a comparison between different results of THz and mmWave networks (Sec.~\ref{sub:results-iab-thz}).
% MP add one sentence here on what comes next [done]

\subsubsection{Simulation Setup}
\label{sec:SimSetup}

We have developed a system-level simulator that runs on top of Sionna~\cite{hoydis2022sionna}, an open-source TensorFlow-based GPU-accelerated toolbox, and that includes 
% MP what is Sionna? need to clarify here and not later [done]
the \gls{iab} networks described in Rel. 17. 
% MP what does it mean "to come closer"? [done]
The proposed simulator, which is written in Python, is a system-level simulator which features 3GPP-compliant channel modeling and lower layers of the protocol stack. 
% and scenarios that
% % MP "and is 3GPP-compliant for what concerns XYZ" [?]
% % is not a full-stack simulator.
% is a link-level system simulator.
% MP do not characterize what you do through lack of features or negative terms (I don't do X, I don't do Y -> just say I do Z). See also below (that lacks...)
%
% Due to the absence of an updated 5g-NR and sub-THz simulator for the IAB in other open source network simulators such as ns-3, we have chosen Sionna as the appropriate baseline simulator.
% MP why did you choose this? [done]
% The tensor-based approach supports the natural integration of neural networks and the prototyping of intricate communication systems.
% MP we don't care about this as there is no neural network in play. [done] 
% Sionna is a physical layer-focused simulator that 
However, it
lacks the implementation of 5G NR higher layers. 
% , protocol stack since it does not explicitly mimic \gls{5g} networks. 
% MP here you need to spin it positevly, as a contribution: we added XYZ as K does not do it
Therefore, we added system-level functions like MAC-level scheduling and RLC-level buffering~\cite{INFOCOMSIM}. In addition, in this research we use the Terasim channel simulator~\cite{hossain2018terasim}
% MP what does this mean? Do you have calls to MATLAB terasim implementation? Do you pre-generate traces? If so, how, and do they change, remain static, etc? Or do you take the channel model you discuss in Sec 2 and implement it here? [done]
to generate channel responses and integrate them into Sionna. To accomplish this, we generate traces for each IAB node's channel response and load them into Sionna. Terasim channel model integration allows us to generate channels up to 10 \gls{thz};
% MP avoid "a few", "some", "a lot" in scientific writing - be precise [done] 
in this simulation campaign, the sub-THz carrier frequency is 140 GHz. Several system-level \glspl{kpi}, including latency, throughput, and packet loss rate, are produced by our simulator.


\subsubsection{Simulation Scenario}
\label{sub:SimScenario}
We take a dense cellular base station deployment into account in our models. As shown in Fig.~\ref{fig:SimulationScenario}, we place IAB nodes at a density of $150$ \gls{gnb}/$\mathrm{km}^2$, thus with an average intersite distance of 40 m. 
% MP mention the density and area, no need to say how many nodes you have [done]
In Table~\ref{Tab:parameters-iab-thz}, the specific simulation settings are displayed. For \gls{mmwave}, we used the channel model outlined by \gls{3gpp} in TR 38.901~\cite{3gpp.38.901}, a statistical 3GPP channel model for 0.5-100 GHz, while for sub-\gls{thz} we used the \gls{thz}-band channel model introduced in~\cite{5995306} and detailed in Sec.~\ref{sub:thzchannel}. The range of the user rate is 20 Mbps to 500 Mbps. 
% MP put it as user rate [done]
We used a phased array antenna for \gls{mmwave} and a horn antenna for THz, respectively. In mmWave we do beamforming based on a pre-generated codebook, in order to find the best beam pair for connection. For the purposes of SINR calculation, we assume that each interfering device utilizes the beamforming vector with the greatest SINR towards its intended target. In a similar fashion, both the transmitter and the receiver utilize the beamforming configuration calculated by the hierarchical search technique.
We consider a scenario with a single donor to focus on the issues related to the bottleneck in the air interface of the donor itself, while extension to multiple donors is left for future work. We also set $d_{max} = 70$ m, as it has been experimentally shown that sub-THz links can operate in this range also in adverse weather conditions~\cite{sen2022terahertz}. 

% MP so there is beamforming :)  [done]
% MP why do you use the bandwidth you use? [done]
% MP do you model misalignment? How do you model interference when using beamforming?  [done]


\begin{figure}
    \centering
    \setlength\fwidth{0.5\columnwidth}
    \setlength\fheight{0.45\columnwidth}
    % This file was created with tikzplotlib v0.10.1.
\begin{tikzpicture}

\definecolor{darkgray150}{RGB}{150,150,150}
\definecolor{green01270}{HTML}{F09C35}
\definecolor{lightgray204}{RGB}{204,204,204}
\definecolor{blue}{HTML}{5BE9F0}
\definecolor{orange}{RGB}{0,128,0}

%\definecolor{darkgray176}{RGB}{150,150,150}
%\definecolor{gray}{RGB}{128,128,128}
%\definecolor{green}{RGB}{0,128,0}
%\definecolor{lightgray204}{RGB}{204,204,204}
%\definecolor{orange}{RGB}{255,165,0}
%\definecolor{blue}{HTML}{5bc4eb}

\begin{axis}[
    width=\fwidth,
    height=\fheight,
    at={(0\fwidth,0\fheight)},
    scale only axis,
    legend cell align={left},
    legend style={
        legend columns=3,
        draw opacity=1,
        text opacity=1,
        at={(0.5, 1.05)},
        anchor=south,
        draw=black,
        font=\footnotesize
    },
    tick align=outside,
    tick pos=left,
    x grid style={darkgray150},
    xmajorgrids,
    xmin=0, xmax=400,
    xtick style={color=black},
    y grid style={darkgray150},
    ymajorgrids,
    ymin=0, ymax=390,
    ytick style={color=black},
    xlabel={X [m]},
    ylabel={Y [m]}
    ]
    
\path [draw=white!10!black, fill=darkgray150, thick]
(axis cs:200,200)
--(axis cs:202.5,195)
--(axis cs:200.0005,195)
--(axis cs:200.0005,160)
--(axis cs:199.9995,160)
--(axis cs:199.9995,195)
--(axis cs:197.5,195)
--cycle;

\addplot [very thick, white!10!black]
table {%
-10 -10
-5 -5
};
%\addlegendimage{area legend, draw=darkgray150, fill=darkgray150, thick}
\addlegendentry{Path link}

\path [draw=white!10!black, fill=white!10!black, thick]
(axis cs:200,200)
--(axis cs:196.222584491411,204.120816918461)
--(axis cs:198.625913598539,204.80748237764)
--(axis cs:179.999519238026,269.999862639436)
--(axis cs:180.000480761974,270.000137360564)
--(axis cs:198.626875122487,204.807757098768)
--(axis cs:201.030204229615,205.494422557948)
--cycle;
\path [draw=white!10!black, fill=white!10!black, thick]
(axis cs:200,200)
--(axis cs:205.584403908235,200.253836541283)
--(axis cs:204.569260812334,197.969764575507)
--(axis cs:245.000203069233,180.000456905774)
--(axis cs:244.999796930767,179.999543094226)
--(axis cs:204.568854673868,197.968850763958)
--(axis cs:203.553711577968,195.684778798182)
--cycle;
\path [draw=white!10!black, fill=white!10!black, thick]
(axis cs:200,200)
--(axis cs:196.64589803375,195.527864045)
--(axis cs:195.528087651798,197.763484808905)
--(axis cs:140.000223606798,169.999552786405)
--(axis cs:139.999776393202,170.000447213595)
--(axis cs:195.527640438203,197.764379236096)
--(axis cs:194.409830056251,200)
--cycle;
\path [draw=white!10!black, fill=white!10!black, thick]
(axis cs:270,239)
--(axis cs:264.947166780705,241.391417248827)
--(axis cs:266.913906897386,242.933958516812)
--(axis cs:229.999606573291,289.999691430032)
--(axis cs:230.000393426709,290.000308569968)
--(axis cs:266.914693750803,242.934575656747)
--(axis cs:268.881433867484,244.477116924732)
--cycle;
\path [draw=white!10!black, fill=white!10!black, thick]
(axis cs:245,180)
--(axis cs:244.648865869234,185.579131188833)
--(axis cs:246.95028479506,184.60395367789)
--(axis cs:269.99953962414,239.000195074517)
--(axis cs:270.00046037586,238.999804925483)
--(axis cs:246.951205546781,184.603563528856)
--(axis cs:249.252624472607,183.628386017913)
--cycle;
\path [draw=white!10!black, fill=white!10!black, thick]
(axis cs:270,239)
--(axis cs:274.926560997556,241.641779085646)
--(axis cs:274.997946152417,239.143298665538)
--(axis cs:304.999985720113,240.000499796043)
--(axis cs:305.000014279887,239.999500203957)
--(axis cs:274.99797471219,239.142299073451)
--(axis cs:275.069359867051,236.643818653343)
--cycle;
\path [draw=white!10!black, fill=white!10!black, thick]
(axis cs:200,160)
--(axis cs:204.853626716971,157.226499018874)
--(axis cs:202.773917006273,155.840025878409)
--(axis cs:240.000416025147,100.000277350098)
--(axis cs:239.999583974853,99.9997226499019)
--(axis cs:202.773084955979,155.839471178213)
--(axis cs:200.693375245282,154.452998037748)
--cycle;
\path [draw=white!10!black, fill=white!10!black, thick]
(axis cs:200,160)
--(axis cs:203.553711577968,164.315221201818)
--(axis cs:204.568854673868,162.031149236042)
--(axis cs:244.999796930767,180.000456905774)
--(axis cs:245.000203069233,179.999543094226)
--(axis cs:204.569260812334,162.030235424493)
--(axis cs:205.584403908235,159.746163458717)
--cycle;
\path [draw=white!10!black, fill=white!10!black, thick]
(axis cs:200,160)
--(axis cs:194.657032912576,158.356010126946)
--(axis cs:195.067948181346,160.821501739565)
--(axis cs:139.999917800506,169.999506803038)
--(axis cs:140.000082199494,170.000493196962)
--(axis cs:195.068112580333,160.822488133489)
--(axis cs:195.479027849103,163.287979746107)
--cycle;
\path [draw=white!10!black, fill=white!10!black, thick]
(axis cs:200,160)
--(axis cs:202.807022876983,155.165682822974)
--(axis cs:200.312390457065,155.009768296729)
--(axis cs:204.000499026289,96.0000311891431)
--(axis cs:203.999500973711,95.9999688108569)
--(axis cs:200.311392404487,155.009705918442)
--(axis cs:197.816759984569,154.853791392198)
--cycle;
\path [draw=white!10!black, fill=white!10!black, thick]
(axis cs:180,270)
--(axis cs:183.713906763541,274.178145108984)
--(axis cs:184.642197759088,271.857417620116)
--(axis cs:229.999814304662,290.000464238345)
--(axis cs:230.000185695338,289.999535761655)
--(axis cs:184.642569149764,271.856489143425)
--(axis cs:185.570860145312,269.535761654557)
--cycle;
\path [draw=white!10!black, fill=white!10!black, thick]
(axis cs:180,270)
--(axis cs:175.321197620133,273.059216940682)
--(axis cs:177.480212982929,274.318642568979)
--(axis cs:144.99956811055,329.999748064487)
--(axis cs:145.00043188945,330.000251935513)
--(axis cs:177.48107676183,274.319146440005)
--(axis cs:179.640092124626,275.578572068302)
--cycle;
\path [draw=white!10!black, fill=white!10!black, thick]
(axis cs:180,270)
--(axis cs:175,267.5)
--(axis cs:175,269.9995)
--(axis cs:104,269.9995)
--(axis cs:104,270.0005)
--(axis cs:175,270.0005)
--(axis cs:175,272.5)
--cycle;
\path [draw=white!10!black, fill=white!10!black, thick]
(axis cs:240,100)
--(axis cs:245.576923076923,100.384615384615)
--(axis cs:244.615576923077,98.0773846153846)
--(axis cs:300.000192307692,75.0004615384615)
--(axis cs:299.999807692308,74.9995384615385)
--(axis cs:244.615192307692,98.0764615384615)
--(axis cs:243.653846153846,95.7692307692308)
--cycle;
\path [draw=white!10!black, fill=white!10!black, thick]
(axis cs:204,96)
--(axis cs:208.693339858181,99.0368669670583)
--(axis cs:208.969363457605,96.5526545722416)
--(axis cs:239.999944784237,100.000496941867)
--(axis cs:240.000055215763,99.9995030581327)
--(axis cs:208.969473889131,96.5516606885069)
--(axis cs:209.245497488555,94.0674482936902)
--cycle;
\path [draw=white!10!black, fill=white!10!black, thick]
(axis cs:245,180)
--(axis cs:250.155223315592,182.161867842023)
--(axis cs:249.988959048788,179.66790383996)
--(axis cs:320.000033259505,175.000498892579)
--(axis cs:319.999966740495,174.999501107421)
--(axis cs:249.988892529778,179.666906054802)
--(axis cs:249.822628262974,177.17294205274)
--cycle;
\path [draw=white!10!black, fill=white!10!black, thick]
(axis cs:320,175)
--(axis cs:324.244373438136,178.638034375545)
--(axis cs:324.850591232914,176.213163196432)
--(axis cs:379.999878732187,190.00048507125)
--(axis cs:380.000121267813,189.99951492875)
--(axis cs:324.850833768539,176.212193053932)
--(axis cs:325.457051563317,173.787321874818)
--cycle;
\path [draw=white!10!black, fill=white!10!black, thick]
(axis cs:305,240)
--(axis cs:308.56027552273,235.690192788274)
--(axis cs:306.124784729092,235.12815645128)
--(axis cs:320.000487195598,175.000112429753)
--(axis cs:319.999512804402,174.999887570247)
--(axis cs:306.123810337896,235.127931591774)
--(axis cs:303.688319544257,234.56589525478)
--cycle;
\path [draw=white!10!black, fill=white!10!black, thick]
(axis cs:140,170)
--(axis cs:134.696699141101,171.767766952966)
--(axis cs:136.464112540677,173.535180352542)
--(axis cs:99.9996464466094,209.999646446609)
--(axis cs:100.000353553391,210.000353553391)
--(axis cs:136.464819647458,173.535887459323)
--(axis cs:138.232233047034,175.303300858899)
--cycle;
\path [draw=white!10!black, fill=white!10!black, thick]
(axis cs:100,210)
--(axis cs:101.693503643551,204.672519787997)
--(axis cs:99.224304768602,205.060536754061)
--(axis cs:89.0004939385627,139.999922381083)
--(axis cs:88.9995060614373,140.000077618917)
--(axis cs:99.2233168914766,205.060691991895)
--(axis cs:96.7541180165281,205.448708957958)
--cycle;
\path [draw=white!10!black, fill=white!10!black, thick]
(axis cs:100,210)
--(axis cs:97.8381321579773,215.155223315592)
--(axis cs:100.33209616004,214.988959048788)
--(axis cs:103.999501107421,270.000033259505)
--(axis cs:104.000498892579,269.999966740495)
--(axis cs:100.333093945198,214.988892529778)
--(axis cs:102.82705794726,214.822628262974)
--cycle;
\path [draw=white!10!black, fill=white!10!black, thick]
(axis cs:100,210)
--(axis cs:94.6383855262484,208.417884253647)
--(axis cs:95.0777742271383,210.878460978631)
--(axis cs:43.9999121046808,219.999507786212)
--(axis cs:44.0000878953192,220.000492213788)
--(axis cs:95.0779500177768,210.879445406206)
--(axis cs:95.5173387186667,213.340022131189)
--cycle;
\path [draw=white!10!black, fill=white!10!black, thick]
(axis cs:165,55)
--(axis cs:159.514152183162,53.9250703602141)
--(axis cs:160.181216451074,56.3339135498957)
--(axis cs:99.9998665604585,72.9995181349891)
--(axis cs:100.000133439541,73.0004818650109)
--(axis cs:160.181483330156,56.3348772799176)
--(axis cs:160.848547598068,58.7437204695991)
--cycle;
\path [draw=white!10!black, fill=white!10!black, thick]
(axis cs:204,96)
--(axis cs:202.365326324237,90.6541752765578)
--(axis cs:200.554293449043,92.3768650846685)
--(axis cs:165.000362279031,54.999655393117)
--(axis cs:164.999637720969,55.000344606883)
--(axis cs:200.553568890982,92.3775542984345)
--(axis cs:198.742536015788,94.1002441065453)
--cycle;
\path [draw=white!10!black, fill=white!10!black, thick]
(axis cs:104,270)
--(axis cs:104.756840921376,275.538699470068)
--(axis cs:106.820539702807,274.128505302757)
--(axis cs:144.999587177679,330.000282095252)
--(axis cs:145.000412822321,329.999717904748)
--(axis cs:106.821365347449,274.127941112252)
--(axis cs:108.88506412888,272.71774694494)
--cycle;
\path [draw=white!10!black, fill=white!10!black, thick]
(axis cs:85,25)
--(axis cs:80.306660141819,21.9631330329417)
--(axis cs:80.0306365423949,24.4473454277584)
--(axis cs:40.000055215763,19.9995030581327)
--(axis cs:39.999944784237,20.0004969418673)
--(axis cs:80.0305261108689,24.4483393114931)
--(axis cs:79.7545025114448,26.9325517063098)
--cycle;
\path [draw=white!10!black, fill=white!10!black, thick]
(axis cs:100,73)
--(axis cs:100.894824979408,67.481912626985)
--(axis cs:98.5091022743093,68.2274509723283)
--(axis cs:85.000477239989,24.9998508625034)
--(axis cs:84.999522760011,25.0001491374966)
--(axis cs:98.5081477943313,68.2277492473214)
--(axis cs:96.1224250892327,68.9732875926647)
--cycle;
\path [draw=white!10!black, fill=white!10!black, thick]
(axis cs:140,170)
--(axis cs:136.95787924062,165.310063829289)
--(axis cs:135.690582434274,167.464468400075)
--(axis cs:89.0002535100633,139.999569032892)
--(axis cs:88.9997464899367,140.000430967108)
--(axis cs:135.690075414148,167.465330334291)
--(axis cs:134.422778607803,169.619734905077)
--cycle;
\path [draw=white!10!black, fill=white!10!black, thick]
(axis cs:89,140)
--(axis cs:92.2770234761656,135.471079914962)
--(axis cs:89.8105441414632,135.066135546578)
--(axis cs:100.000493394546,73.0000810050747)
--(axis cs:99.9995066054541,72.9999189949253)
--(axis cs:89.8095573523715,135.065973536429)
--(axis cs:87.3430780176691,134.661029168045)
--cycle;
\path [draw=white!10!black, fill=white!10!black, thick]
(axis cs:30,290)
--(axis cs:26.0471529247895,293.95284707521)
--(axis cs:28.4183868282668,294.74325837637)
--(axis cs:9.99952565835098,349.999841886117)
--(axis cs:10.000474341649,350.000158113883)
--(axis cs:28.4193355115648,294.743574604136)
--(axis cs:30.7905694150421,295.533985905295)
--cycle;
\path [draw=white!10!black, fill=white!10!black, thick]
(axis cs:104,270)
--(axis cs:98.5209104720017,268.891136643143)
--(axis cs:99.1730525803461,271.304062444017)
--(axis cs:29.9998695454874,289.999517318304)
--(axis cs:30.0001304545126,290.000482681696)
--(axis cs:99.1733134893712,271.30502780741)
--(axis cs:99.8254555977156,273.717953608285)
--cycle;
\path [draw=white!10!black, fill=white!10!black, thick]
(axis cs:44,220)
--(axis cs:40.5679676350818,224.412613040609)
--(axis cs:43.0189290339712,224.902805320387)
--(axis cs:29.9995097096622,289.999901941932)
--(axis cs:30.0004902903378,290.000098058068)
--(axis cs:43.0199096146469,224.903001436522)
--(axis cs:45.4708710135364,225.3931937163)
--cycle;
\path [draw=white!10!black, fill=white!10!black, thick]
(axis cs:104,270)
--(axis cs:101.759354601174,264.878524802684)
--(axis cs:100.159213694213,266.798693891038)
--(axis cs:44.0003200921998,219.99961588936)
--(axis cs:43.9996799078002,220.00038411064)
--(axis cs:100.158573509813,266.799462112317)
--(axis cs:98.5584326028519,268.719631200671)
--cycle;

\addplot [semithick, orange, mark=*, mark size=2.5, mark options={solid}, forget plot]
table {%
270 239
};

\addplot [semithick, orange, mark=*, mark size=2.5, mark options={solid}, only marks]
table {%
-10 -10
};
\addlegendentry{IAB node}

\addplot [semithick, orange, mark=*, mark size=2.5, mark options={solid}, forget plot]
table {%
200 160
};
\addplot [semithick, orange, mark=*, mark size=2.5, mark options={solid}, forget plot]
table {%
180 270
};
\addplot [semithick, orange, mark=*, mark size=2.5, mark options={solid}, forget plot]
table {%
230 290
};
\addplot [semithick, orange, mark=*, mark size=2.5, mark options={solid}, forget plot]
table {%
240 100
};
\addplot [semithick, orange, mark=*, mark size=2.5, mark options={solid}, forget plot]
table {%
320 175
};
\addplot [semithick, orange, mark=*, mark size=2.5, mark options={solid}, forget plot]
table {%
100 210
};
\addplot [semithick, orange, mark=*, mark size=2.5, mark options={solid}, forget plot]
table {%
165 55
};
\addplot [semithick, orange, mark=*, mark size=2.5, mark options={solid}, forget plot]
table {%
145 330
};
\addplot [semithick, orange, mark=*, mark size=2.5, mark options={solid}, forget plot]
table {%
85 25
};
\addplot [semithick, orange, mark=*, mark size=2.5, mark options={solid}, forget plot]
table {%
245 180
};
\addplot [semithick, orange, mark=*, mark size=2.5, mark options={solid}, forget plot]
table {%
380 190
};
\addplot [semithick, orange, mark=*, mark size=2.5, mark options={solid}, forget plot]
table {%
305 240
};
\addplot [semithick, orange, mark=*, mark size=2.5, mark options={solid}, forget plot]
table {%
40 20
};
\addplot [semithick, orange, mark=*, mark size=2.5, mark options={solid}, forget plot]
table {%
140 170
};
\addplot [semithick, orange, mark=*, mark size=2.5, mark options={solid}, forget plot]
table {%
300 75
};
\addplot [semithick, orange, mark=*, mark size=2.5, mark options={solid}, forget plot]
table {%
100 73
};
\addplot [semithick, orange, mark=*, mark size=2.5, mark options={solid}, forget plot]
table {%
204 96
};
\addplot [semithick, orange, mark=*, mark size=2.5, mark options={solid}, forget plot]
table {%
89 140
};
\addplot [semithick, orange, mark=*, mark size=2.5, mark options={solid}, forget plot]
table {%
10 350
};
\addplot [semithick, orange, mark=*, mark size=2.5, mark options={solid}, forget plot]
table {%
30 290
};
\addplot [semithick, orange, mark=*, mark size=2.5, mark options={solid}, forget plot]
table {%
104 270
};
\addplot [semithick, orange, mark=*, mark size=2.5, mark options={solid}, forget plot]
table {%
44 220
};
\addplot [semithick, red, mark=square*, mark size=3.0, mark options={solid}, forget plot]
table {%
200 200
};

\addplot [semithick, red, mark=square*, mark size=2.5, mark options={solid}, only marks]
table {%
200 200
};
\addlegendentry{IAB donor}

\end{axis}

\end{tikzpicture}

    %     \setlength\abovecaptionskip{0cm}
    % \setlength\belowcaptionskip{-.3cm}
    \caption{Simulation Scenario}
    \label{fig:SimulationScenario}
\end{figure}

\begin{table}[]
\caption{Simulation parameters.}
\label{Tab:parameters-iab-thz}
\centering
\footnotesize
\begin{tabular}{l|l}
    \toprule
    Parameter & Value \\ \midrule

    Carrier frequency (\gls{mmwave}) & $28$~GHz \\
    Bandwidth (\gls{mmwave}) & $400$~MHz \\
    Carrier frequency (THz) & 140~GHz  \\
    Bandwidth (THz) & $\{ 10,32 \}$~GHz \\
    % Total bandwidth & 400 MHz \\
    % \hline
    IAB RF Chains & 2 (1 access + 1 backhaul) \\
    % \hline
    % \hline 
    Pathloss model (\gls{mmwave}) & UMi-Street Canyon~\cite{3gpp.38.901} \\
    Pathloss model (THz) &  Physics-based~\cite{5995306} \\
    % \hline
    Number of IAB nodes $N_{\mathrm{I}}$ & $23$  \\
    Number of users $N_{\mathrm{U}}$ & $50$   \\
    Per-UE source rate & \{$40, 80, 100, 200$\} Mbps \\
    $\rho_{max}$ & \{$0, 0.1, 0.3, 0.5, 0.7, 1$\} \\
    \gls{gnb} antenna array & $8$H $\times$ $8$V\\
    % \hline
    % IAB Access antenna array & 4H×4V\\
    UE antenna array & $4$H $\times$ $4$V\\
    % \hline 
    \gls{gnb} and UE height & $15$~m and $1.5$~m \\
    % UE height & 1.5~m \\
    \gls{gnb} antenna gain (\gls{mmwave}) & $30$~dB \\
    % MP where does this 30 come from? 
    \gls{gnb} antenna gain (THz) & $38$~dB \\
    Noise power & $10$~dBm \\
\bottomrule
\end{tabular}
\end{table}



\subsubsection{Results}
\label{sub:results-iab-thz}

In this section we report the outcomes of our numerical evaluation, focusing on end-to-end metrics measured at the IAB donor. We compare the performance achieved by different backhaul configurations, i.e., different maximum ratios of THz links and bandwidth, in terms of throughput, latency and packet drop ratio. We consider two baselines: \textit{Random Scheduler (RS)} and \textit{Random Links (RL)}. The former uses Alg.~\ref{algo:thzLink} and chooses at random a feasible set of active links during each subframe. On the contrary, RL randomly picks which links to set as \gls{thz}, and uses Alg.~\ref{algo:sinr} for scheduling. 10 simulations per configuration are executed, to obtain estimates which are averaged over the realizations of the wireless channels.
% MP how many simulations? Where do the randomness come from (channel instances, traffic generation, noise)? [done]

% \hl{New comments on the random policies vs. our algs.}
Fig.~\ref{fig:ThroughputPerDifferAlg} reports the UE throughput achieved by the proposed solution, versus that achieved by RS and RL. Focusing on the former, it can be seen that Alg.~\ref{algo:sinr} leads to a throughput increase of up to 40\% compared to a random scheduling policy, thanks to the prioritization of the backhaul links incurring a higher load and exhibiting a higher number of subtending IAB nodes. 
Moreover, Alg.~\ref{algo:thzLink} introduces an additional throughput increase of up to 15\% compared to RL.
%MPag: Figures/IabThzLinks are kind of hard to read, probably adding a RORS baseline would make sense as well..
\begin{figure}
    \centering
    \setlength\fwidth{0.55\columnwidth}
    \setlength\fheight{0.35\columnwidth}
    % This file was created with tikzplotlib v0.10.1.
\begin{tikzpicture}

\definecolor{darkgray176}{RGB}{176,176,176}
\definecolor{gray}{RGB}{128,128,128}
\definecolor{green}{RGB}{0,128,0}
\definecolor{lightgray204}{RGB}{204,204,204}
\definecolor{orange}{RGB}{255,165,0}
\definecolor{blue}{HTML}{5bc4eb}

\definecolor{color1}{RGB}{255,20,189}
\definecolor{color2}{RGB}{40, 215, 235}
\definecolor{color3}{RGB}{240, 131, 22}
\definecolor{color4}{RGB}{73, 214, 34}
\definecolor{color5}{RGB}{255,225,82}

\begin{axis}[
    width=\fwidth,
    height=\fheight,
    at={(0\fwidth,0\fheight)},
    scale only axis,
    legend cell align={left},
    legend style={
        legend columns=2,
        draw opacity=1,
        text opacity=1,
        at={(0.39, 0.96)},
        anchor=north,
        draw=black,
        font=\footnotesize
    },
    tick align=outside,
    tick pos=left,
    x grid style={darkgray176},
    xlabel={UE Source rate [Mbps]},
    xmajorgrids,
    xmin=2, xmax=26.69,
    xtick style={color=black},
    xtick={4,11,18,25},
    xticklabels={20,40,80,200},
    y grid style={darkgray176},
    ylabel={Throughput [Mbps]},
    ymajorgrids,
    ymin=0, ymax=139.49591583826,
    ytick style={color=black}
]

\addlegendimage{area legend, draw=black, fill=color2}
\addlegendentry{RS}
\addlegendimage{area legend, draw=black, fill=color3, postaction={pattern=north east lines}}
\addlegendentry{RL}
\addlegendimage{area legend, draw=black, fill=color4, postaction={pattern={Dots[radius=0.3mm]}}}
\addlegendentry{Alg.~\ref{algo:sinr} and Alg.~\ref{algo:thzLink}}


\draw[draw=black,fill=color2] (axis cs:2.6,0) rectangle (axis cs:3.4,15.2);
\draw[draw=black,fill=color2] (axis cs:9.6,0) rectangle (axis cs:10.4,27.25);
\draw[draw=black,fill=color2] (axis cs:16.6,0) rectangle (axis cs:17.4,39.72);
\draw[draw=black,fill=color2] (axis cs:23.6,0) rectangle (axis cs:24.4,75.03);
\draw[draw=black,fill=color3, postaction={pattern=north east lines}] (axis cs:3.6,0) rectangle (axis cs:4.4,19.7);
\draw[draw=black,fill=color3, postaction={pattern=north east lines}] (axis cs:10.6,0) rectangle (axis cs:11.4,33.74);
\draw[draw=black,fill=color3, postaction={pattern=north east lines}] (axis cs:17.6,0) rectangle (axis cs:18.4,54);
\draw[draw=black,fill=color3, postaction={pattern=north east lines}] (axis cs:24.6,0) rectangle (axis cs:25.4,101.1);
\draw[draw=black,fill=color4, postaction={pattern={Dots[radius=0.3mm]}}] (axis cs:4.6,0) rectangle (axis cs:5.4,19.408);
\draw[draw=black,fill=color4, postaction={pattern={Dots[radius=0.3mm]}}] (axis cs:11.6,0) rectangle (axis cs:12.4,36.952);
\draw[draw=black,fill=color4, postaction={pattern={Dots[radius=0.3mm]}}] (axis cs:18.6,0) rectangle (axis cs:19.4,67.694);
\draw[draw=black,fill=color4, postaction={pattern={Dots[radius=0.3mm]}}] (axis cs:25.6,0) rectangle (axis cs:26.4,124.292);


\path [draw=black, semithick]
(axis cs:3,12.94956038652)
--(axis cs:3,19.45043961348);

\addplot [semithick, black, mark=-, mark size=3, mark options={solid}, only marks, forget plot]
table {%
3 12.94956038652
};
\addplot [semithick, black, mark=-, mark size=3, mark options={solid}, only marks, forget plot]
table {%
3 19.45043961348
};
\path [draw=black, semithick]
(axis cs:10,21.5481379562758)
--(axis cs:10,34.3018620437242);

\addplot [semithick, black, mark=-, mark size=3, mark options={solid}, only marks, forget plot]
table {%
10 21.5481379562758
};
\addplot [semithick, black, mark=-, mark size=3, mark options={solid}, only marks, forget plot]
table {%
10 34.3018620437242
};
\path [draw=black, semithick]
(axis cs:17,31.1272449380274)
--(axis cs:17,49.2267550619726);

\addplot [semithick, black, mark=-, mark size=3, mark options={solid}, only marks, forget plot]
table {%
17 31.1272449380274
};
\addplot [semithick, black, mark=-, mark size=3, mark options={solid}, only marks, forget plot]
table {%
17 49.2267550619726
};
\path [draw=black, semithick]
(axis cs:24,95.1794034733883)
--(axis cs:24,58.9565965266117);

\addplot [semithick, black, mark=-, mark size=3, mark options={solid}, only marks, forget plot]
table {%
24 95.1794034733883
};
\addplot [semithick, black, mark=-, mark size=3, mark options={solid}, only marks, forget plot]
table {%
24 58.9565965266117
};
\path [draw=black, semithick]
(axis cs:4,18.6492520270418)
--(axis cs:4,18.9267479729582);

\addplot [semithick, black, mark=-, mark size=3, mark options={solid}, only marks, forget plot]
table {%
4 18.6492520270418
};
\addplot [semithick, black, mark=-, mark size=3, mark options={solid}, only marks, forget plot]
table {%
4 18.9267479729582
};
\path [draw=black, semithick]
(axis cs:11,33.6369402565865)
--(axis cs:11,35.9760597434135);

\addplot [semithick, black, mark=-, mark size=3, mark options={solid}, only marks, forget plot]
table {%
11 33.6369402565865
};
\addplot [semithick, black, mark=-, mark size=3, mark options={solid}, only marks, forget plot]
table {%
11 35.9760597434135
};
\path [draw=black, semithick]
(axis cs:18,56.5834678670515)
--(axis cs:18,49.6525321329485);

\addplot [semithick, black, mark=-, mark size=3, mark options={solid}, only marks, forget plot]
table {%
18 56.5834678670515
};
\addplot [semithick, black, mark=-, mark size=3, mark options={solid}, only marks, forget plot]
table {%
18 49.6525321329485
};
\path [draw=black, semithick]
(axis cs:25,111.445065217728)
--(axis cs:25,91.094934782272);

\addplot [semithick, black, mark=-, mark size=3, mark options={solid}, only marks, forget plot]
table {%
25 111.445065217728
};
\addplot [semithick, black, mark=-, mark size=3, mark options={solid}, only marks, forget plot]
table {%
25 91.094934782272
};
\path [draw=black, semithick]
(axis cs:5,19.3021888474687)
--(axis cs:5,19.5138111525313);

\addplot [semithick, black, mark=-, mark size=3, mark options={solid}, only marks, forget plot]
table {%
5 19.3021888474687
};
\addplot [semithick, black, mark=-, mark size=3, mark options={solid}, only marks, forget plot]
table {%
5 19.5138111525313
};
\path [draw=black, semithick]
(axis cs:12,36.6352414168487)
--(axis cs:12,37.2687585831513);

\addplot [semithick, black, mark=-, mark size=3, mark options={solid}, only marks, forget plot]
table {%
12 36.6352414168487
};
\addplot [semithick, black, mark=-, mark size=3, mark options={solid}, only marks, forget plot]
table {%
12 37.2687585831513
};
\path [draw=black, semithick]
(axis cs:19,66.296529427859)
--(axis cs:19,69.091470572141);

\addplot [semithick, black, mark=-, mark size=3, mark options={solid}, only marks, forget plot]
table {%
19 66.296529427859
};
\addplot [semithick, black, mark=-, mark size=3, mark options={solid}, only marks, forget plot]
table {%
19 69.091470572141
};
\path [draw=black, semithick]
(axis cs:26,115.730746820704)
--(axis cs:26,132.853253179296);

\addplot [semithick, black, mark=-, mark size=3, mark options={solid}, only marks, forget plot]
table {%
26 115.730746820704
};
\addplot [semithick, black, mark=-, mark size=3, mark options={solid}, only marks, forget plot]
table {%
26 132.853253179296
};
\end{axis}

\end{tikzpicture}

    %     \setlength\abovecaptionskip{0cm}
    % \setlength\belowcaptionskip{-.3cm}
    \caption{Throughput per \gls{ue} for different schedulers and THz link selection policies, for \gls{thz} bandwidth $32$~GHz and $\rho_{max} = 0.3$.} 
    \label{fig:ThroughputPerDifferAlg}
\end{figure}

Fig.~\ref{fig:Throughput} illustrates the \gls{ue} throughput for various configurations of sub-\gls{thz} backhauling links and different UE source rates. 
The performance always improves by adding more bandwidth to the system through sub-THz links, despite the harsher propagation environment at higher frequencies.

The performance gap increases with the user source rate. Indeed, 
\glspl{mmwave} successfully sustain a source system rate of 1 Gbps (20 Mbps for 50 UE), 
but cannot match higher source rates, as the capacity saturates.
The configuration with sub-THz links achieves a higher throughput in all scenarios and in particular for $\rho_{max} = 0.3$, 32 GHz achieves the highest throughput for all source rates. It is obvious that increasing the bandwidth improves the performance; nevertheless, increasing the percentage of the \gls{thz} links from $\rho_{max} = 0.1$ to $\rho_{max} = 0.3$ has a more significant impact on throughput. 
% MP well this is also increasing the system bandwidth! [not]
This may be explained by considering the effects of replacing bottleneck backhauling mmWave links with \gls{thz} links with higher bandwidth.

% MP what does this mean (second part of the sentence on max source rate)? [done]

\begin{figure}
    \centering
    \setlength\fwidth{0.55\columnwidth}
    \setlength\fheight{0.35\columnwidth}
    % This file was created with tikzplotlib v0.10.1.
\begin{tikzpicture}

\definecolor{darkgray176}{RGB}{176,176,176}
\definecolor{gray}{RGB}{128,128,128}
\definecolor{green}{RGB}{0,128,0}
\definecolor{lightgray204}{RGB}{204,204,204}
\definecolor{orange}{RGB}{255,165,0}
\definecolor{blue}{HTML}{5bc4eb}

\definecolor{color1}{RGB}{255,20,189}
\definecolor{color2}{RGB}{40, 215, 235}
\definecolor{color3}{RGB}{240, 131, 22}
\definecolor{color4}{RGB}{73, 214, 34}
\definecolor{color5}{RGB}{255,225,82}

\begin{axis}[
    width=\fwidth,
    height=\fheight,
    at={(0\fwidth,0\fheight)},
    scale only axis,
    legend cell align={left},
    legend style={
        legend columns=2,
        draw opacity=1,
        text opacity=1,
        at={(0.5, 1.35)},
        anchor=north,
        draw=black,
        font=\footnotesize
    },
    tick align=outside,
    tick pos=left,
    x grid style={darkgray176},
    xlabel={UE Source rate [Mbps]},
    xmajorgrids,
    xmin=-0.69, xmax=27.69,
    xtick style={color=black},
    xtick={3,10,17,24},
    xticklabels={20,40,80,200},
    y grid style={darkgray176},
    ylabel={Throughput [Mbps]},
    ymajorgrids,
    ymin=0, ymax=139.49591583826,
    ytick style={color=black}
]

\addlegendimage{area legend, draw=black, fill=color1, postaction={pattern=north east lines}}
\addlegendentry{$\rho_{max} = 0.1$, 10 GHz}

\addlegendimage{area legend, draw=black, fill=color2, postaction={pattern={Dots[radius=0.3mm]}}}
\addlegendentry{$\rho_{max} = 0.1$, 32 GHz}

\addlegendimage{area legend, draw=black, fill=color3, postaction={pattern=horizontal lines}}
\addlegendentry{$\rho_{max} = 0.3$, 10 GHz}

\addlegendimage{area legend, draw=black, fill=color4,}
\addlegendentry{$\rho_{max} = 0.3$, 32 GHz}

\addlegendimage{area legend, draw=black, fill=white}
\addlegendentry{$\rho_{max} = 0$}

\draw[draw=black,fill=white] (axis cs:0.6,0) rectangle (axis cs:1.4,15.844);
\draw[draw=black,fill=white] (axis cs:7.6,0) rectangle (axis cs:8.4,23.135);
\draw[draw=black,fill=white] (axis cs:14.6,0) rectangle (axis cs:15.4,30.488);
\draw[draw=black,fill=white] (axis cs:21.6,0) rectangle (axis cs:22.4,33.018);

\draw[draw=black,fill=color1, postaction={pattern=north east lines}] (axis cs:1.6,0) rectangle (axis cs:2.4,17.051);
\draw[draw=black,fill=color1, postaction={pattern=north east lines}] (axis cs:8.6,0) rectangle (axis cs:9.4,29.6715);
\draw[draw=black,fill=color1, postaction={pattern=north east lines}] (axis cs:15.6,0) rectangle (axis cs:16.4,42.635);
\draw[draw=black,fill=color1, postaction={pattern=north east lines}] (axis cs:22.6,0) rectangle (axis cs:23.4,49.777);

\draw[draw=black,fill=color2, postaction={pattern={Dots[radius=0.3mm]}}] (axis cs:2.6,0) rectangle (axis cs:3.4,18.2);
\draw[draw=black,fill=color2, postaction={pattern={Dots[radius=0.3mm]}}] (axis cs:9.6,0) rectangle (axis cs:10.4,32.925);
\draw[draw=black,fill=color2, postaction={pattern={Dots[radius=0.3mm]}}] (axis cs:16.6,0) rectangle (axis cs:17.4,56.677);
\draw[draw=black,fill=color2, postaction={pattern={Dots[radius=0.3mm]}}] (axis cs:23.6,0) rectangle (axis cs:24.4,85.068);

\draw[draw=black,fill=color3, postaction={pattern=horizontal lines}] (axis cs:3.6,0) rectangle (axis cs:4.4,18.788);
\draw[draw=black,fill=color3, postaction={pattern=horizontal lines}] (axis cs:10.6,0) rectangle (axis cs:11.4,35.3065);
\draw[draw=black,fill=color3, postaction={pattern=horizontal lines}] (axis cs:17.6,0) rectangle (axis cs:18.4,58.618);
\draw[draw=black,fill=color3, postaction={pattern=horizontal lines}] (axis cs:24.6,0) rectangle (axis cs:25.4,117.77);

\draw[draw=black,fill=color4] (axis cs:4.6,0) rectangle (axis cs:5.4,19.408);
\draw[draw=black,fill=color4] (axis cs:11.6,0) rectangle (axis cs:12.4,36.952);
\draw[draw=black,fill=color4] (axis cs:18.6,0) rectangle (axis cs:19.4,67.694);
\draw[draw=black,fill=color4] (axis cs:25.6,0) rectangle (axis cs:26.4,124.292);
\path [draw=black, semithick]
(axis cs:1,15.6557873543037)
--(axis cs:1,16.0322126456963);

\addplot [semithick, black, mark=-, mark size=3, mark options={solid}, only marks, forget plot]
table {%
1 15.6557873543037
};
\addplot [semithick, black, mark=-, mark size=3, mark options={solid}, only marks, forget plot]
table {%
1 16.0322126456963
};
\path [draw=black, semithick]
(axis cs:8,21.9589026400846)
--(axis cs:8,24.3110973599154);

\addplot [semithick, black, mark=-, mark size=3, mark options={solid}, only marks, forget plot]
table {%
8 21.9589026400846
};
\addplot [semithick, black, mark=-, mark size=3, mark options={solid}, only marks, forget plot]
table {%
8 24.3110973599154
};
\path [draw=black, semithick]
(axis cs:15,28.5107453375956)
--(axis cs:15,32.4652546624044);

\addplot [semithick, black, mark=-, mark size=3, mark options={solid}, only marks, forget plot]
table {%
15 28.5107453375956
};
\addplot [semithick, black, mark=-, mark size=3, mark options={solid}, only marks, forget plot]
table {%
15 32.4652546624044
};
\path [draw=black, semithick]
(axis cs:22,32.4217986917156)
--(axis cs:22,33.6142013082844);

\addplot [semithick, black, mark=-, mark size=3, mark options={solid}, only marks, forget plot]
table {%
22 32.4217986917156
};
\addplot [semithick, black, mark=-, mark size=3, mark options={solid}, only marks, forget plot]
table {%
22 33.6142013082844
};
\path [draw=black, semithick]
(axis cs:2,16.9148970977532)
--(axis cs:2,17.1871029022468);

\addplot [semithick, black, mark=-, mark size=3, mark options={solid}, only marks, forget plot]
table {%
2 16.9148970977532
};
\addplot [semithick, black, mark=-, mark size=3, mark options={solid}, only marks, forget plot]
table {%
2 17.1871029022468
};
\path [draw=black, semithick]
(axis cs:9,28.3491402720893)
--(axis cs:9,30.9938597279107);

\addplot [semithick, black, mark=-, mark size=3, mark options={solid}, only marks, forget plot]
table {%
9 28.3491402720893
};
\addplot [semithick, black, mark=-, mark size=3, mark options={solid}, only marks, forget plot]
table {%
9 30.9938597279107
};
\path [draw=black, semithick]
(axis cs:16,41.302958333985)
--(axis cs:16,43.967041666015);

\addplot [semithick, black, mark=-, mark size=3, mark options={solid}, only marks, forget plot]
table {%
16 41.302958333985
};
\addplot [semithick, black, mark=-, mark size=3, mark options={solid}, only marks, forget plot]
table {%
16 43.967041666015
};
\path [draw=black, semithick]
(axis cs:23,47.5924918173648)
--(axis cs:23,51.9615081826352);

\addplot [semithick, black, mark=-, mark size=3, mark options={solid}, only marks, forget plot]
table {%
23 47.5924918173648
};
\addplot [semithick, black, mark=-, mark size=3, mark options={solid}, only marks, forget plot]
table {%
23 51.9615081826352
};
\path [draw=black, semithick]
(axis cs:3,17.94956038652)
--(axis cs:3,18.45043961348);

\addplot [semithick, black, mark=-, mark size=3, mark options={solid}, only marks, forget plot]
table {%
3 17.94956038652
};
\addplot [semithick, black, mark=-, mark size=3, mark options={solid}, only marks, forget plot]
table {%
3 18.45043961348
};
\path [draw=black, semithick]
(axis cs:10,32.5481379562758)
--(axis cs:10,33.3018620437242);

\addplot [semithick, black, mark=-, mark size=3, mark options={solid}, only marks, forget plot]
table {%
10 32.5481379562758
};
\addplot [semithick, black, mark=-, mark size=3, mark options={solid}, only marks, forget plot]
table {%
10 33.3018620437242
};
\path [draw=black, semithick]
(axis cs:17,53.1272449380274)
--(axis cs:17,60.2267550619726);

\addplot [semithick, black, mark=-, mark size=3, mark options={solid}, only marks, forget plot]
table {%
17 53.1272449380274
};
\addplot [semithick, black, mark=-, mark size=3, mark options={solid}, only marks, forget plot]
table {%
17 60.2267550619726
};
\path [draw=black, semithick]
(axis cs:24,80.1794034733883)
--(axis cs:24,89.9565965266117);

\addplot [semithick, black, mark=-, mark size=3, mark options={solid}, only marks, forget plot]
table {%
24 80.1794034733883
};
\addplot [semithick, black, mark=-, mark size=3, mark options={solid}, only marks, forget plot]
table {%
24 89.9565965266117
};
\path [draw=black, semithick]
(axis cs:4,18.6492520270418)
--(axis cs:4,18.9267479729582);

\addplot [semithick, black, mark=-, mark size=3, mark options={solid}, only marks, forget plot]
table {%
4 18.6492520270418
};
\addplot [semithick, black, mark=-, mark size=3, mark options={solid}, only marks, forget plot]
table {%
4 18.9267479729582
};
\path [draw=black, semithick]
(axis cs:11,34.6369402565865)
--(axis cs:11,35.9760597434135);

\addplot [semithick, black, mark=-, mark size=3, mark options={solid}, only marks, forget plot]
table {%
11 34.6369402565865
};
\addplot [semithick, black, mark=-, mark size=3, mark options={solid}, only marks, forget plot]
table {%
11 35.9760597434135
};
\path [draw=black, semithick]
(axis cs:18,56.5834678670515)
--(axis cs:18,60.6525321329485);

\addplot [semithick, black, mark=-, mark size=3, mark options={solid}, only marks, forget plot]
table {%
18 56.5834678670515
};
\addplot [semithick, black, mark=-, mark size=3, mark options={solid}, only marks, forget plot]
table {%
18 60.6525321329485
};
\path [draw=black, semithick]
(axis cs:25,111.445065217728)
--(axis cs:25,124.094934782272);

\addplot [semithick, black, mark=-, mark size=3, mark options={solid}, only marks, forget plot]
table {%
25 111.445065217728
};
\addplot [semithick, black, mark=-, mark size=3, mark options={solid}, only marks, forget plot]
table {%
25 124.094934782272
};
\path [draw=black, semithick]
(axis cs:5,19.3021888474687)
--(axis cs:5,19.5138111525313);

\addplot [semithick, black, mark=-, mark size=3, mark options={solid}, only marks, forget plot]
table {%
5 19.3021888474687
};
\addplot [semithick, black, mark=-, mark size=3, mark options={solid}, only marks, forget plot]
table {%
5 19.5138111525313
};
\path [draw=black, semithick]
(axis cs:12,36.6352414168487)
--(axis cs:12,37.2687585831513);

\addplot [semithick, black, mark=-, mark size=3, mark options={solid}, only marks, forget plot]
table {%
12 36.6352414168487
};
\addplot [semithick, black, mark=-, mark size=3, mark options={solid}, only marks, forget plot]
table {%
12 37.2687585831513
};
\path [draw=black, semithick]
(axis cs:19,66.296529427859)
--(axis cs:19,69.091470572141);

\addplot [semithick, black, mark=-, mark size=3, mark options={solid}, only marks, forget plot]
table {%
19 66.296529427859
};
\addplot [semithick, black, mark=-, mark size=3, mark options={solid}, only marks, forget plot]
table {%
19 69.091470572141
};
\path [draw=black, semithick]
(axis cs:26,115.730746820704)
--(axis cs:26,132.853253179296);

\addplot [semithick, black, mark=-, mark size=3, mark options={solid}, only marks, forget plot]
table {%
26 115.730746820704
};
\addplot [semithick, black, mark=-, mark size=3, mark options={solid}, only marks, forget plot]
table {%
26 132.853253179296
};
\end{axis}

\end{tikzpicture}

    %     \setlength\abovecaptionskip{0cm}
    % \setlength\belowcaptionskip{-.3cm}
    \caption{Throughput per UE for different configurations.}
    % MP the color palette can be improved, the gray and green are very similar
    \label{fig:Throughput}
\end{figure}


Similar considerations can be drawn from the results shown in Fig.~\ref{fig:packetDrop}, which reports the packet drop percentages for various backhaul configurations. %This performance is consistent with the behaviors seen in Figure~\ref{fig:Throughput}'s behavior.
The highest and lowest packet drop percentages across all \gls{ue} source rates are achieved when using the \gls{mmwave} and $\rho_{max} = 0.3$, 32 GHz configurations, respectively. Packet drop percentages at 20 Mbps source rates are close to zero for all configurations. The highest packet drop percentages among configurations including \gls{thz} is $\rho_{max} = 0.1$, 10 GHz. It is noteworthy that the system performance is influenced directly by both the \gls{thz} bandwidth and the link ratio, as seen in Fig~\ref{fig:Throughput}.

\begin{figure}
    \centering
    \setlength\fwidth{0.55\columnwidth}
    \setlength\fheight{0.35\columnwidth}
    % This file was created with tikzplotlib v0.10.1.
\begin{tikzpicture}

\definecolor{darkgray176}{RGB}{176,176,176}
\definecolor{gray}{RGB}{128,128,128}
\definecolor{green}{RGB}{0,128,0}
\definecolor{lightgray204}{RGB}{204,204,204}
\definecolor{color2}{RGB}{255,165,0}
\definecolor{blue}{HTML}{5bc4eb}

\definecolor{color1}{RGB}{255,20,189}
\definecolor{color2}{RGB}{40, 215, 235}
\definecolor{color3}{RGB}{240, 131, 22}
\definecolor{color4}{RGB}{73, 214, 34}
\definecolor{color5}{RGB}{255,225,82}


\begin{axis}[
    width=\fwidth,
    height=\fheight,
    at={(0\fwidth,0\fheight)},
    scale only axis,
    legend cell align={left},
    legend style={
        legend columns=2,
        draw opacity=1,
        text opacity=1,
        at={(0.5, 1.2)},
        anchor=north,
        draw=black,
        font=\footnotesize
    },
    tick align=outside,
    tick pos=left,
    x grid style={darkgray176},
    xlabel={UE Source rate [Mbps]},
    xmajorgrids,
    xmin=-0.69, xmax=27.69,
    xtick style={color=black},
    xtick={3,10,17,24},
    xticklabels={20,40,80,200},
    y grid style={darkgray176},
    ylabel={Packet drop percentage [\%]},
    ymajorgrids,
    ymin=-0, ymax=10.6924489573116,
    ytick style={color=black}
]

\addlegendimage{area legend, draw=black, fill=color1, postaction={pattern=north east lines}}
\addlegendentry{$\rho_{max} = 0.1$, 10 GHz}

\addlegendimage{area legend, draw=black, fill=color2, postaction={pattern={Dots[radius=0.3mm]}}}
\addlegendentry{$\rho_{max} = 0.1$, 32 GHz}

\addlegendimage{area legend, draw=black, fill=color3, postaction={pattern=horizontal lines}}
\addlegendentry{$\rho_{max} = 0.3$, 10 GHz}

\addlegendimage{area legend, draw=black, fill=color4,}
\addlegendentry{$\rho_{max} = 0.3$, 32 GHz}

\addlegendimage{area legend, draw=black, fill=white}
\addlegendentry{$\rho_{max} = 0$}

\draw[draw=black,fill=white] (axis cs:0.6,0) rectangle (axis cs:1.4,3.67518695221753);
\draw[draw=black,fill=white] (axis cs:7.6,0) rectangle (axis cs:8.4,4.845278309726);
\draw[draw=black,fill=white] (axis cs:14.6,0) rectangle (axis cs:15.4,8.89268590509084);
\draw[draw=black,fill=white] (axis cs:21.6,0) rectangle (axis cs:22.4,8.11544584296259);

\draw[draw=black,fill=color1, postaction={pattern=north east lines}] (axis cs:1.6,0) rectangle (axis cs:2.4,0.581668654041138);
\draw[draw=black,fill=color1, postaction={pattern=north east lines}] (axis cs:8.6,0) rectangle (axis cs:9.4,4.57986560435684);
\draw[draw=black,fill=color1, postaction={pattern=north east lines}] (axis cs:15.6,0) rectangle (axis cs:16.4,5.35337607181025);
\draw[draw=black,fill=color1, postaction={pattern=north east lines}] (axis cs:22.6,0) rectangle (axis cs:23.4,5.7788314351186);

\draw[draw=black,fill=color2, postaction={pattern={Dots[radius=0.3mm]}}] (axis cs:2.6,0) rectangle (axis cs:3.4,0.0273015296974269);
\draw[draw=black,fill=color2, postaction={pattern={Dots[radius=0.3mm]}}] (axis cs:9.6,0) rectangle (axis cs:10.4,0.686435208728916);
\draw[draw=black,fill=color2, postaction={pattern={Dots[radius=0.3mm]}}] (axis cs:16.6,0) rectangle (axis cs:17.4,3.84389963584301);
\draw[draw=black,fill=color2, postaction={pattern={Dots[radius=0.3mm]}}] (axis cs:23.6,0) rectangle (axis cs:24.4,5.02363721123376);

\draw[draw=black,fill=color3, postaction={pattern=horizontal lines}] (axis cs:3.6,0) rectangle (axis cs:4.4,0.00521920668058455);
\draw[draw=black,fill=color3, postaction={pattern=horizontal lines}] (axis cs:10.6,0) rectangle (axis cs:11.4,0.784183441815596);
\draw[draw=black,fill=color3, postaction={pattern=horizontal lines}] (axis cs:17.6,0) rectangle (axis cs:18.4,3.03795413291424);
\draw[draw=black,fill=color3, postaction={pattern=horizontal lines}] (axis cs:24.6,0) rectangle (axis cs:25.4,4.58954607419068);

\draw[draw=black,fill=color4] (axis cs:4.6,0) rectangle (axis cs:5.4,0.063979816817083);
\draw[draw=black,fill=color4] (axis cs:11.6,0) rectangle (axis cs:12.4,0.0486662469364497);
\draw[draw=black,fill=color4] (axis cs:18.6,0) rectangle (axis cs:19.4,1.25546066423633);
\draw[draw=black,fill=color4] (axis cs:25.6,0) rectangle (axis cs:26.4,1.02178627398367);

\path [draw=black, semithick]
(axis cs:1,2.33783016488447)
--(axis cs:1,5.01254373955058);

\addplot [semithick, black, mark=-, mark size=3, mark options={solid}, only marks, forget plot]
table {%
1 2.33783016488447
};
\addplot [semithick, black, mark=-, mark size=3, mark options={solid}, only marks, forget plot]
table {%
1 5.01254373955058
};
\path [draw=black, semithick]
(axis cs:8,2.71283312611119)
--(axis cs:8,5.57773343408332);

\addplot [semithick, black, mark=-, mark size=3, mark options={solid}, only marks, forget plot]
table {%
8 2.71283312611119
};
\addplot [semithick, black, mark=-, mark size=3, mark options={solid}, only marks, forget plot]
table {%
8 5.57773343408332
};
\path [draw=black, semithick]
(axis cs:15,7.60391715661735)
--(axis cs:15,10.1814546535643);

\addplot [semithick, black, mark=-, mark size=3, mark options={solid}, only marks, forget plot]
table {%
15 7.60391715661735
};
\addplot [semithick, black, mark=-, mark size=3, mark options={solid}, only marks, forget plot]
table {%
15 10.1814546535643
};
\path [draw=black, semithick]
(axis cs:22,7.29084697599827)
--(axis cs:22,8.94004470992691);

\addplot [semithick, black, mark=-, mark size=3, mark options={solid}, only marks, forget plot]
table {%
22 7.29084697599827
};
\addplot [semithick, black, mark=-, mark size=3, mark options={solid}, only marks, forget plot]
table {%
22 8.94004470992691
};
\path [draw=black, semithick]
(axis cs:2,0.26631959316497)
--(axis cs:2,0.897017714917306);

\addplot [semithick, black, mark=-, mark size=3, mark options={solid}, only marks, forget plot]
table {%
2 0.26631959316497
};
\addplot [semithick, black, mark=-, mark size=3, mark options={solid}, only marks, forget plot]
table {%
2 0.897017714917306
};
\path [draw=black, semithick]
(axis cs:9,3.08920658143186)
--(axis cs:9,6.07052462728182);

\addplot [semithick, black, mark=-, mark size=3, mark options={solid}, only marks, forget plot]
table {%
9 3.08920658143186
};
\addplot [semithick, black, mark=-, mark size=3, mark options={solid}, only marks, forget plot]
table {%
9 6.07052462728182
};
\path [draw=black, semithick]
(axis cs:16,4.55072276597036)
--(axis cs:16,6.15602937765014);

\addplot [semithick, black, mark=-, mark size=3, mark options={solid}, only marks, forget plot]
table {%
16 4.55072276597036
};
\addplot [semithick, black, mark=-, mark size=3, mark options={solid}, only marks, forget plot]
table {%
16 6.15602937765014
};
\path [draw=black, semithick]
(axis cs:23,4.57189564106147)
--(axis cs:23,6.98576722917574);

\addplot [semithick, black, mark=-, mark size=3, mark options={solid}, only marks, forget plot]
table {%
23 4.57189564106147
};
\addplot [semithick, black, mark=-, mark size=3, mark options={solid}, only marks, forget plot]
table {%
23 6.98576722917574
};
\path [draw=black, semithick]
(axis cs:3,-0.0384314213817879)
--(axis cs:3,0.0930344807766418);

\addplot [semithick, black, mark=-, mark size=3, mark options={solid}, only marks, forget plot]
table {%
3 -0.0384314213817879
};
\addplot [semithick, black, mark=-, mark size=3, mark options={solid}, only marks, forget plot]
table {%
3 0.0930344807766418
};
\path [draw=black, semithick]
(axis cs:10,0.277146956148178)
--(axis cs:10,1.09572346130965);

\addplot [semithick, black, mark=-, mark size=3, mark options={solid}, only marks, forget plot]
table {%
10 0.277146956148178
};
\addplot [semithick, black, mark=-, mark size=3, mark options={solid}, only marks, forget plot]
table {%
10 1.09572346130965
};
\path [draw=black, semithick]
(axis cs:17,2.94584928920027)
--(axis cs:17,4.74194998248574);

\addplot [semithick, black, mark=-, mark size=3, mark options={solid}, only marks, forget plot]
table {%
17 2.94584928920027
};
\addplot [semithick, black, mark=-, mark size=3, mark options={solid}, only marks, forget plot]
table {%
17 4.74194998248574
};
\path [draw=black, semithick]
(axis cs:24,4.28831300304668)
--(axis cs:24,5.75896141942083);

\addplot [semithick, black, mark=-, mark size=3, mark options={solid}, only marks, forget plot]
table {%
24 4.28831300304668
};
\addplot [semithick, black, mark=-, mark size=3, mark options={solid}, only marks, forget plot]
table {%
24 5.75896141942083
};
\path [draw=black, semithick]
(axis cs:4,-0.0104384133611691)
--(axis cs:4,0.0208768267223382);

\addplot [semithick, black, mark=-, mark size=3, mark options={solid}, only marks, forget plot]
table {%
4 -0.0104384133611691
};
\addplot [semithick, black, mark=-, mark size=3, mark options={solid}, only marks, forget plot]
table {%
4 0.0208768267223382
};
\path [draw=black, semithick]
(axis cs:11,0.358684360442467)
--(axis cs:11,1.20968252318872);

\addplot [semithick, black, mark=-, mark size=3, mark options={solid}, only marks, forget plot]
table {%
11 0.358684360442467
};
\addplot [semithick, black, mark=-, mark size=3, mark options={solid}, only marks, forget plot]
table {%
11 1.20968252318872
};
\path [draw=black, semithick]
(axis cs:18,2.42458429084184)
--(axis cs:18,3.65132397498664);

\addplot [semithick, black, mark=-, mark size=3, mark options={solid}, only marks, forget plot]
table {%
18 2.42458429084184
};
\addplot [semithick, black, mark=-, mark size=3, mark options={solid}, only marks, forget plot]
table {%
18 3.65132397498664
};
\path [draw=black, semithick]
(axis cs:25,3.73665064193202)
--(axis cs:25,5.44244150644933);

\addplot [semithick, black, mark=-, mark size=3, mark options={solid}, only marks, forget plot]
table {%
25 3.73665064193202
};
\addplot [semithick, black, mark=-, mark size=3, mark options={solid}, only marks, forget plot]
table {%
25 5.44244150644933
};
\path [draw=black, semithick]
(axis cs:5,-0.00266078212037386)
--(axis cs:5,0.13062041575454);

\addplot [semithick, black, mark=-, mark size=3, mark options={solid}, only marks, forget plot]
table {%
5 -0.00266078212037386
};
\addplot [semithick, black, mark=-, mark size=3, mark options={solid}, only marks, forget plot]
table {%
5 0.13062041575454
};
\path [draw=black, semithick]
(axis cs:12,-0.0382525113056243)
--(axis cs:12,0.135585005178524);

\addplot [semithick, black, mark=-, mark size=3, mark options={solid}, only marks, forget plot]
table {%
12 -0.0382525113056243
};
\addplot [semithick, black, mark=-, mark size=3, mark options={solid}, only marks, forget plot]
table {%
12 0.135585005178524
};
\path [draw=black, semithick]
(axis cs:19,0.476781037644174)
--(axis cs:19,2.03414029082848);

\addplot [semithick, black, mark=-, mark size=3, mark options={solid}, only marks, forget plot]
table {%
19 0.476781037644174
};
\addplot [semithick, black, mark=-, mark size=3, mark options={solid}, only marks, forget plot]
table {%
19 2.03414029082848
};
\path [draw=black, semithick]
(axis cs:26,0.765662642533051)
--(axis cs:26,1.2779099054343);

\addplot [semithick, black, mark=-, mark size=3, mark options={solid}, only marks, forget plot]
table {%
26 0.765662642533051
};
\addplot [semithick, black, mark=-, mark size=3, mark options={solid}, only marks, forget plot]
table {%
26 1.2779099054343
};
\end{axis}

\end{tikzpicture}

    %     \setlength\abovecaptionskip{0cm}
    % \setlength\belowcaptionskip{-.3cm}
    \caption{Backhaul packet drop percentage for different configurations.}
    \label{fig:packetDrop}
\end{figure}

Fig.~\ref{fig:Latency} depicts the \gls{ecdf} of the \gls{e2e} latency experienced by packets which reach the donor, for different bachkaul configurations. Accordingly, both latencies accumulated over the fronthaul and backhaul links are taken into account, from the time packets are generated at the \gls{ue} until they eventually reach the \gls{iab} donor. The plot shows that packet latency decreases as more sub-THz links are added to the network. In accordance with the aforementioned observations (Fig.~\ref{fig:Throughput} and Fig.~\ref{fig:packetDrop}), $\rho_{max} = 0.3$, 32 GHz has the lowest latency, whereas \gls{mmwave} has the highest latency. The average latency for $\rho_{max} = 0.3$, 32 GHz, $\rho_{max} = 0.3$, 10 GHz, $\rho_{max} = 0.1$, 32 GHz, and $\rho_{max} = 0.1$, 10 GHz is approximately 51\%, 24\%, 24\%, and 18\% less than in \gls{mmwave}.

\begin{figure}
    \centering
    \setlength\fwidth{0.55\columnwidth}
    \setlength\fheight{0.35\columnwidth}
    % This file was created with tikzplotlib v0.10.1.
\begin{tikzpicture}

\definecolor{darkgray176}{RGB}{150,150,150}
\definecolor{gray}{RGB}{128,128,128}
\definecolor{green}{RGB}{0,128,0}
\definecolor{lightgray204}{RGB}{204,204,204}
\definecolor{orange}{RGB}{255,165,0}
\definecolor{blue}{HTML}{5bc4eb}

\definecolor{color1}{RGB}{255,20,189}
\definecolor{color2}{RGB}{40, 215, 235}
\definecolor{color3}{RGB}{240, 131, 22}
\definecolor{color4}{RGB}{73, 214, 34}
\definecolor{color5}{RGB}{255,225,82}

\begin{axis}[
    width=\fwidth,
    height=\fheight,
    at={(0\fwidth,0\fheight)},
    scale only axis,
    legend cell align={left},
    legend style={
        legend columns=1,
        draw opacity=1,
        text opacity=1,
        at={(0.745, 0.02)},
        anchor=south,
        draw=black,
        font=\footnotesize
    },
    tick align=outside,
    tick pos=left,
    x grid style={darkgray176},
    xlabel={Latency [ms]},
    xmajorgrids,
    xmin=2.66546001599534, xmax=31.4967128271509,
    xtick style={color=black},
    y grid style={darkgray176},
    ylabel={ECDF},
    ymajorgrids,
    ymin=0.0, ymax=1.0
]
\addplot [color1, very thick, mark=triangle*, mark size=3, mark repeat=50, mark options={solid,rotate=90}]
table {%
7.26957559585571 0
8.30400657653809 0.00200402736663818
8.68300628662109 0.00400805473327637
8.72667694091797 0.00601208209991455
9.00158023834229 0.00801599025726318
9.20892810821533 0.0100200176239014
9.25952816009521 0.0120240449905396
9.43859195709229 0.0140280723571777
9.48785972595215 0.0160320997238159
9.51481151580811 0.0180361270904541
9.59027862548828 0.0200400352478027
9.79505634307861 0.0220440626144409
9.81936454772949 0.0240480899810791
9.84952449798584 0.0260521173477173
9.86929512023926 0.0280561447143555
9.87447357177734 0.0300601720809937
9.87716293334961 0.0320640802383423
9.97320938110352 0.0340681076049805
10.0722312927246 0.0360721349716187
10.1157627105713 0.0380761623382568
10.1440114974976 0.040080189704895
10.1474313735962 0.0420842170715332
10.1782493591309 0.0440881252288818
10.1924743652344 0.04609215259552
10.2432508468628 0.0480961799621582
10.2983360290527 0.0501002073287964
10.3524713516235 0.0521042346954346
10.3864812850952 0.0541082620620728
10.4384689331055 0.0561121702194214
10.4411153793335 0.0581161975860596
10.4414882659912 0.0601202249526978
10.4619016647339 0.0621242523193359
10.4726591110229 0.0641282796859741
10.5135107040405 0.0661323070526123
10.5472364425659 0.0681362152099609
10.5595769882202 0.0701402425765991
10.5614604949951 0.0721442699432373
10.5714950561523 0.0741482973098755
10.6297082901001 0.0761523246765137
10.6329383850098 0.0781563520431519
10.6401805877686 0.08016037940979
10.6418952941895 0.0821642875671387
10.6515302658081 0.0841683149337769
10.6549167633057 0.086172342300415
10.657374382019 0.0881763696670532
10.6640625 0.0901803970336914
10.6978006362915 0.0921844244003296
10.7068948745728 0.0941883325576782
10.7546586990356 0.0961923599243164
10.8114013671875 0.0981963872909546
10.8276948928833 0.100200414657593
10.8532295227051 0.102204442024231
10.8557653427124 0.104208469390869
10.8587942123413 0.106212377548218
10.8861999511719 0.108216404914856
10.8984031677246 0.110220432281494
10.9201745986938 0.112224459648132
10.9449968338013 0.114228487014771
10.9493522644043 0.116232514381409
10.9647045135498 0.118236422538757
10.9715461730957 0.120240449905396
10.9752750396729 0.122244477272034
10.9878597259521 0.124248504638672
10.9907722473145 0.12625253200531
10.9914474487305 0.128256559371948
10.9922695159912 0.130260467529297
10.9943838119507 0.132264494895935
11.0041980743408 0.134268522262573
11.0061197280884 0.136272549629211
11.0137195587158 0.13827657699585
11.0660982131958 0.140280604362488
11.0773973464966 0.142284631729126
11.0850095748901 0.144288539886475
11.1086120605469 0.146292567253113
11.1401796340942 0.148296594619751
11.141565322876 0.150300621986389
11.1614475250244 0.152304649353027
11.1667737960815 0.154308557510376
11.1811351776123 0.156312584877014
11.1838808059692 0.158316612243652
11.2331409454346 0.160320639610291
11.2599143981934 0.162324666976929
11.2635402679443 0.164328694343567
11.2732448577881 0.166332721710205
11.3018674850464 0.168336629867554
11.3057165145874 0.170340657234192
11.316520690918 0.17234468460083
11.3266887664795 0.174348711967468
11.3276338577271 0.176352739334106
11.3572492599487 0.178356647491455
11.3640069961548 0.180360674858093
11.3725538253784 0.182364702224731
11.3742160797119 0.18436872959137
11.3769416809082 0.186372756958008
11.3776941299438 0.188376784324646
11.3902759552002 0.190380811691284
11.3928337097168 0.192384719848633
11.4140577316284 0.194388747215271
11.4199447631836 0.196392774581909
11.4460096359253 0.198396801948547
11.449652671814 0.200400829315186
11.4743127822876 0.202404856681824
11.4767007827759 0.204408884048462
11.4809637069702 0.206412792205811
11.4867534637451 0.208416819572449
11.4975280761719 0.210420846939087
11.5055418014526 0.212424874305725
11.5113182067871 0.214428901672363
11.5149412155151 0.216432809829712
11.526346206665 0.21843683719635
11.5271663665771 0.220440864562988
11.5313129425049 0.222444891929626
11.5395946502686 0.224448919296265
11.5469169616699 0.226452946662903
11.5478267669678 0.228456974029541
11.5579309463501 0.23046088218689
11.5611152648926 0.232464909553528
11.5611696243286 0.234468936920166
11.574089050293 0.236472964286804
11.5817012786865 0.238476991653442
11.5871906280518 0.240480899810791
11.5915861129761 0.242484927177429
11.5992937088013 0.244488954544067
11.6175622940063 0.246492981910706
11.6233329772949 0.248497009277344
11.6235122680664 0.250501036643982
11.6254329681396 0.25250506401062
11.6374006271362 0.254508972167969
11.6392698287964 0.256512999534607
11.6681299209595 0.258517026901245
11.6724090576172 0.260521054267883
11.67418384552 0.262525081634521
11.6743440628052 0.26452898979187
11.7026863098145 0.266533136367798
11.7095394134521 0.268537044525146
11.715576171875 0.270541071891785
11.7230072021484 0.272545099258423
11.7446556091309 0.274549126625061
11.7489128112793 0.276553153991699
11.7531881332397 0.278557062149048
11.7812309265137 0.280561089515686
11.787971496582 0.282565116882324
11.7903108596802 0.284569144248962
11.7908887863159 0.286573171615601
11.7919368743896 0.288577198982239
11.7937669754028 0.290581226348877
11.7991905212402 0.292585134506226
11.8081607818604 0.294589161872864
11.8225612640381 0.296593189239502
11.8256597518921 0.29859721660614
11.8318090438843 0.300601243972778
11.8373374938965 0.302605152130127
11.8389873504639 0.304609179496765
11.8451595306396 0.306613206863403
11.8561773300171 0.308617234230042
11.8601512908936 0.31062126159668
11.8677740097046 0.312625288963318
11.8882713317871 0.314629316329956
11.8921031951904 0.316633224487305
11.9010143280029 0.318637251853943
11.903904914856 0.320641279220581
11.9048738479614 0.322645306587219
11.9129657745361 0.324649333953857
11.943902015686 0.326653242111206
11.9528789520264 0.328657388687134
11.9565782546997 0.330661296844482
11.9625549316406 0.332665324211121
11.9661474227905 0.334669351577759
11.9741849899292 0.336673378944397
11.9743843078613 0.338677406311035
11.9976043701172 0.340681314468384
12.0010118484497 0.342685341835022
12.0559663772583 0.34468936920166
12.0576086044312 0.346693396568298
12.0610437393188 0.348697423934937
12.0747003555298 0.350701332092285
12.0759611129761 0.352705478668213
12.0823822021484 0.354709386825562
12.1073150634766 0.3567134141922
12.1097364425659 0.358717441558838
12.1170778274536 0.360721468925476
12.1242656707764 0.362725496292114
12.1407794952393 0.364729404449463
12.1414566040039 0.366733431816101
12.1416883468628 0.368737459182739
12.1565780639648 0.370741486549377
12.1648817062378 0.372745513916016
12.1759901046753 0.374749541282654
12.1817789077759 0.376753568649292
12.1898469924927 0.378757476806641
12.1987533569336 0.380761504173279
12.2155179977417 0.382765531539917
12.2280235290527 0.384769558906555
12.2718896865845 0.386773586273193
12.2901782989502 0.388777494430542
12.298056602478 0.39078152179718
12.3028345108032 0.392785549163818
12.3051834106445 0.394789576530457
12.3093004226685 0.396793603897095
12.3184099197388 0.398797631263733
12.3303422927856 0.400801658630371
12.3320970535278 0.40280556678772
12.3422088623047 0.404809594154358
12.3468742370605 0.406813621520996
12.3476257324219 0.408817648887634
12.3482046127319 0.410821676254272
12.352840423584 0.412825584411621
12.3733530044556 0.414829730987549
12.3789863586426 0.416833639144897
12.3805236816406 0.418837666511536
12.3947677612305 0.420841693878174
12.4004850387573 0.422845721244812
12.4508590698242 0.42484974861145
12.4579000473022 0.426853656768799
12.4606580734253 0.428857684135437
12.4769325256348 0.430861711502075
12.4816846847534 0.432865738868713
12.5094890594482 0.434869766235352
12.5107231140137 0.4368736743927
12.5245561599731 0.438877820968628
12.5370216369629 0.440881729125977
12.5527505874634 0.442885756492615
12.5552139282227 0.444889783859253
12.5569372177124 0.446893811225891
12.5602798461914 0.448897838592529
12.5666723251343 0.450901746749878
12.5731801986694 0.452905774116516
12.5812492370605 0.454909801483154
12.5867404937744 0.456913828849792
12.602593421936 0.458917856216431
12.605260848999 0.460921883583069
12.6152992248535 0.462925910949707
12.6192150115967 0.464929819107056
12.6216011047363 0.466933846473694
12.6602954864502 0.468937873840332
12.6707382202148 0.47094190120697
12.6732730865479 0.472945928573608
12.6751747131348 0.474949836730957
12.682879447937 0.476953864097595
12.6894254684448 0.478957891464233
12.6911935806274 0.480961918830872
12.7021961212158 0.48296594619751
12.7249135971069 0.484969973564148
12.7362070083618 0.486974000930786
12.7468929290771 0.488977909088135
12.7512273788452 0.490981936454773
12.7807493209839 0.492985963821411
12.7954664230347 0.494989991188049
12.80393409729 0.496994018554688
12.8068580627441 0.498997926712036
12.8154335021973 0.501002073287964
12.8162670135498 0.503005981445312
12.833155632019 0.505010008811951
12.8370409011841 0.507014036178589
12.8372840881348 0.509018063545227
12.8374052047729 0.511022090911865
12.8405933380127 0.513025999069214
12.8445196151733 0.515030145645142
12.8454427719116 0.51703405380249
12.8480262756348 0.519038081169128
12.8523807525635 0.521042108535767
12.8592796325684 0.523046016693115
12.8673877716064 0.525050163269043
12.8786239624023 0.527054071426392
12.9080801010132 0.52905809879303
12.9487829208374 0.531062126159668
12.9519691467285 0.533066153526306
12.9599008560181 0.535070180892944
12.961012840271 0.537074089050293
12.9646453857422 0.539078235626221
12.9650135040283 0.541082143783569
12.9694404602051 0.543086171150208
12.9750318527222 0.545090198516846
12.9773721694946 0.547094106674194
12.9902763366699 0.549098253250122
13.0140705108643 0.551102161407471
13.0175466537476 0.553106188774109
13.0248708724976 0.555110216140747
13.0279159545898 0.557114243507385
13.0387115478516 0.559118270874023
13.0500640869141 0.561122179031372
13.0539255142212 0.5631263256073
13.0575838088989 0.565130233764648
13.0581045150757 0.567134261131287
13.0623550415039 0.569138288497925
13.0730562210083 0.571142196655273
13.0992498397827 0.573146343231201
13.10671043396 0.57515025138855
13.1134090423584 0.577154397964478
13.130069732666 0.579158306121826
13.1313161849976 0.581162333488464
13.1418342590332 0.583166360855103
13.1582813262939 0.585170269012451
13.165717124939 0.587174415588379
13.1755228042603 0.589178323745728
13.1876907348633 0.591182351112366
13.190128326416 0.593186378479004
13.1940336227417 0.595190405845642
13.2087821960449 0.59719443321228
13.213921546936 0.599198341369629
13.2162303924561 0.601202487945557
13.2251996994019 0.603206396102905
13.2259311676025 0.605210423469543
13.250470161438 0.607214450836182
13.2540817260742 0.60921835899353
13.2658700942993 0.611222505569458
13.2757358551025 0.613226413726807
13.279314994812 0.615230441093445
13.3023204803467 0.617234468460083
13.3039236068726 0.619238495826721
13.3174562454224 0.621242523193359
13.3451299667358 0.623246431350708
13.3527812957764 0.625250577926636
13.3529272079468 0.627254486083984
13.3576431274414 0.629258513450623
13.3586654663086 0.631262540817261
13.3696346282959 0.633266448974609
13.4131689071655 0.635270595550537
13.4148626327515 0.637274503707886
13.4233407974243 0.639278531074524
13.4286985397339 0.641282558441162
13.428750038147 0.6432865858078
13.4430103302002 0.645290613174438
13.4490785598755 0.647294521331787
13.4568166732788 0.649298667907715
13.4582319259644 0.651302576065063
13.4762554168701 0.653306603431702
13.4821043014526 0.65531063079834
13.4885597229004 0.657314658164978
13.488823890686 0.659318685531616
13.4951515197754 0.661322593688965
13.5066022872925 0.663326740264893
13.5075073242188 0.665330648422241
13.5217123031616 0.667334675788879
13.5232820510864 0.669338703155518
13.5302000045776 0.671342611312866
13.5304861068726 0.673346757888794
13.5654134750366 0.675350666046143
13.5738763809204 0.677354693412781
13.5755596160889 0.679358720779419
13.5777311325073 0.681362748146057
13.5796718597412 0.683366775512695
13.6226100921631 0.685370683670044
13.6227512359619 0.687374830245972
13.6266803741455 0.68937873840332
13.6503400802612 0.691382765769958
13.6516580581665 0.693386793136597
13.6809177398682 0.695390701293945
13.6959571838379 0.697394847869873
13.6969404220581 0.699398756027222
13.7062520980835 0.70140278339386
13.7103776931763 0.703406810760498
13.7279844284058 0.705410838127136
13.7392740249634 0.707414865493774
13.7429027557373 0.709418773651123
13.7458219528198 0.711422920227051
13.776665687561 0.713426828384399
13.7865467071533 0.715430855751038
13.8237276077271 0.717434883117676
13.8320360183716 0.719438791275024
13.8717699050903 0.721442937850952
13.8770122528076 0.723446846008301
13.879714012146 0.725450873374939
13.8867702484131 0.727454900741577
13.887186050415 0.729458928108215
13.8896522521973 0.731462955474854
13.9079074859619 0.733466863632202
13.9276723861694 0.73547101020813
13.9523296356201 0.737474918365479
13.961597442627 0.739478945732117
13.9710721969604 0.741482973098755
14.0380334854126 0.743487000465393
14.0425081253052 0.745491027832031
14.0555334091187 0.74749493598938
14.1088218688965 0.749499082565308
14.1413631439209 0.751502990722656
14.1724262237549 0.753507018089294
14.1950645446777 0.755511045455933
14.1996221542358 0.757514953613281
14.2387380599976 0.759519100189209
14.2806606292725 0.761523008346558
14.2990531921387 0.763527035713196
14.3111658096313 0.765531063079834
14.3201885223389 0.767535090446472
14.3214797973633 0.76953911781311
14.3278646469116 0.771543025970459
14.3285713195801 0.773547172546387
14.3429098129272 0.775551080703735
14.3528165817261 0.777555108070374
14.3737840652466 0.779559135437012
14.4163942337036 0.78156304359436
14.4392061233521 0.783567190170288
14.4467458724976 0.785571098327637
14.4551210403442 0.787575125694275
14.473669052124 0.789579153060913
14.4832811355591 0.791583180427551
14.4864120483398 0.793587207794189
14.5096044540405 0.795591115951538
14.5232801437378 0.797595262527466
14.5245361328125 0.799599170684814
14.5335636138916 0.801603198051453
14.544849395752 0.803607225418091
14.5486249923706 0.805611133575439
14.5629281997681 0.807615280151367
14.6305246353149 0.809619188308716
14.6434955596924 0.811623334884644
14.6524381637573 0.813627243041992
14.6791963577271 0.81563127040863
14.6811351776123 0.817635297775269
14.6954011917114 0.819639205932617
14.6982088088989 0.821643352508545
14.7090044021606 0.823647260665894
14.713041305542 0.825651288032532
14.751953125 0.82765531539917
14.7744808197021 0.829659342765808
14.7996444702148 0.831663370132446
14.8443384170532 0.833667278289795
14.8519897460938 0.835671424865723
14.9180116653442 0.837675333023071
14.9863700866699 0.839679360389709
15.0091972351074 0.841683387756348
15.017936706543 0.843687295913696
15.0245494842529 0.845691442489624
15.0557403564453 0.847695350646973
15.0936164855957 0.849699378013611
15.1217317581177 0.851703405380249
15.1651563644409 0.853707432746887
15.1729431152344 0.855711460113525
15.187328338623 0.857715368270874
15.2348642349243 0.859719514846802
15.2827796936035 0.86172342300415
15.3374948501587 0.863727450370789
15.3530559539795 0.865731477737427
15.3578500747681 0.867735385894775
15.3927822113037 0.869739532470703
15.3956489562988 0.871743440628052
15.4001808166504 0.87374746799469
15.4255418777466 0.875751495361328
15.4719581604004 0.877755522727966
15.5011167526245 0.879759550094604
15.5339393615723 0.881763458251953
15.5588607788086 0.883767604827881
15.5749979019165 0.885771512985229
15.5948095321655 0.887775540351868
15.6050519943237 0.889779567718506
15.6087760925293 0.891783595085144
15.6099376678467 0.893787622451782
15.6222534179688 0.895791530609131
15.6459188461304 0.897795677185059
15.6561059951782 0.899799585342407
15.6653594970703 0.901803612709045
15.6839084625244 0.903807640075684
15.8613567352295 0.905811548233032
15.9259614944458 0.90781569480896
16.0217189788818 0.909819602966309
16.026388168335 0.911823630332947
16.0853290557861 0.913827657699585
16.3136119842529 0.915831685066223
16.3537788391113 0.917835712432861
16.3624210357666 0.91983962059021
16.493408203125 0.921843767166138
16.5190448760986 0.923847675323486
16.540283203125 0.925851702690125
16.5654926300049 0.927855730056763
16.6783466339111 0.929859638214111
16.7182197570801 0.931863784790039
16.8720531463623 0.933867692947388
16.9313945770264 0.935871720314026
16.9573230743408 0.937875747680664
17.0378952026367 0.939879775047302
17.0395183563232 0.94188380241394
17.2514305114746 0.943887710571289
17.2606678009033 0.945891857147217
17.3162708282471 0.947895765304565
17.3250312805176 0.949899792671204
17.3634643554688 0.951903820037842
17.4218978881836 0.95390772819519
17.4437198638916 0.955911874771118
17.4920978546143 0.957915782928467
17.722038269043 0.959919810295105
17.8022327423096 0.961923837661743
17.9452476501465 0.963927865028381
18.1799259185791 0.96593189239502
18.5119781494141 0.967935800552368
18.5834045410156 0.969939947128296
18.9076118469238 0.971943855285645
19.0850620269775 0.973947882652283
19.6867523193359 0.975951910018921
20.4365692138672 0.977955937385559
21.5915699005127 0.979959964752197
21.9610843658447 0.981963872909546
22.0942344665527 0.983968019485474
22.2704544067383 0.985971927642822
22.7198925018311 0.98797595500946
22.7546806335449 0.989979982376099
22.9234771728516 0.991983890533447
25.0151176452637 0.993988037109375
25.2643814086914 0.995991945266724
25.6212978363037 0.997995972633362
25.9865493774414 1
};
\addlegendentry{$\rho_{max} = 0.1$, 10 GHz}
\addplot [color2, very thick, mark=o, mark size=3, mark repeat=50, mark options={solid}]
table {%
6.68273067474365 0
6.94367265701294 0.00200402736663818
7.21634197235107 0.00400805473327637
7.29909181594849 0.00601208209991455
7.76562452316284 0.00801599025726318
7.78824806213379 0.0100200176239014
7.86021137237549 0.0120240449905396
7.94470930099487 0.0140280723571777
7.98532390594482 0.0160320997238159
7.98587989807129 0.0180361270904541
8.04297161102295 0.0200400352478027
8.0598783493042 0.0220440626144409
8.13270378112793 0.0240480899810791
8.21036624908447 0.0260521173477173
8.21049213409424 0.0280561447143555
8.25088214874268 0.0300601720809937
8.3647289276123 0.0320640802383423
8.3744068145752 0.0340681076049805
8.39549446105957 0.0360721349716187
8.40424060821533 0.0380761623382568
8.48134517669678 0.040080189704895
8.49883651733398 0.0420842170715332
8.50697898864746 0.0440881252288818
8.50709915161133 0.04609215259552
8.56597900390625 0.0480961799621582
8.60895442962646 0.0501002073287964
8.61983203887939 0.0521042346954346
8.64633464813232 0.0541082620620728
8.67569255828857 0.0561121702194214
8.74547004699707 0.0581161975860596
8.75257873535156 0.0601202249526978
8.77276706695557 0.0621242523193359
8.78633117675781 0.0641282796859741
8.80135536193848 0.0661323070526123
8.84790515899658 0.0681362152099609
8.85816287994385 0.0701402425765991
8.86294841766357 0.0721442699432373
8.89396572113037 0.0741482973098755
8.98305511474609 0.0761523246765137
8.98806953430176 0.0781563520431519
9.02350807189941 0.08016037940979
9.06827926635742 0.0821642875671387
9.10812091827393 0.0841683149337769
9.1668815612793 0.086172342300415
9.1697359085083 0.0881763696670532
9.19346046447754 0.0901803970336914
9.22722244262695 0.0921844244003296
9.25268173217773 0.0941883325576782
9.32009410858154 0.0961923599243164
9.32338809967041 0.0981963872909546
9.39274120330811 0.100200414657593
9.39995288848877 0.102204442024231
9.41912651062012 0.104208469390869
9.46143245697021 0.106212377548218
9.46944427490234 0.108216404914856
9.51492023468018 0.110220432281494
9.52069091796875 0.112224459648132
9.54543304443359 0.114228487014771
9.56208610534668 0.116232514381409
9.56819438934326 0.118236422538757
9.58984851837158 0.120240449905396
9.59662532806396 0.122244477272034
9.60497379302979 0.124248504638672
9.60696125030518 0.12625253200531
9.62654781341553 0.128256559371948
9.62831497192383 0.130260467529297
9.64695167541504 0.132264494895935
9.69279670715332 0.134268522262573
9.70371532440186 0.136272549629211
9.74161911010742 0.13827657699585
9.74912357330322 0.140280604362488
9.75155735015869 0.142284631729126
9.75424671173096 0.144288539886475
9.78405857086182 0.146292567253113
9.78907108306885 0.148296594619751
9.80731773376465 0.150300621986389
9.80825424194336 0.152304649353027
9.81164741516113 0.154308557510376
9.82319927215576 0.156312584877014
9.82712650299072 0.158316612243652
9.84104061126709 0.160320639610291
9.84790802001953 0.162324666976929
9.8638391494751 0.164328694343567
9.86722755432129 0.166332721710205
9.87102317810059 0.168336629867554
9.89676475524902 0.170340657234192
9.90913581848145 0.17234468460083
9.92875862121582 0.174348711967468
9.942626953125 0.176352739334106
9.96391582489014 0.178356647491455
9.98983287811279 0.180360674858093
9.9985523223877 0.182364702224731
10.0003004074097 0.18436872959137
10.0044507980347 0.186372756958008
10.0067806243896 0.188376784324646
10.0168828964233 0.190380811691284
10.0272674560547 0.192384719848633
10.0278654098511 0.194388747215271
10.0344295501709 0.196392774581909
10.0483074188232 0.198396801948547
10.0488576889038 0.200400829315186
10.0511617660522 0.202404856681824
10.0641565322876 0.204408884048462
10.0708961486816 0.206412792205811
10.0723466873169 0.208416819572449
10.0832529067993 0.210420846939087
10.0961856842041 0.212424874305725
10.0997200012207 0.214428901672363
10.1042900085449 0.216432809829712
10.1143941879272 0.21843683719635
10.1703567504883 0.220440864562988
10.1791887283325 0.222444891929626
10.1873273849487 0.224448919296265
10.1904191970825 0.226452946662903
10.2043981552124 0.228456974029541
10.2213296890259 0.23046088218689
10.2214689254761 0.232464909553528
10.2226724624634 0.234468936920166
10.2554998397827 0.236472964286804
10.2849197387695 0.238476991653442
10.3047227859497 0.240480899810791
10.3075218200684 0.242484927177429
10.3344917297363 0.244488954544067
10.3560056686401 0.246492981910706
10.3607082366943 0.248497009277344
10.3731145858765 0.250501036643982
10.3842430114746 0.25250506401062
10.3920383453369 0.254508972167969
10.4110307693481 0.256512999534607
10.4115934371948 0.258517026901245
10.4121265411377 0.260521054267883
10.4244871139526 0.262525081634521
10.4251079559326 0.26452898979187
10.4269495010376 0.266533136367798
10.4687118530273 0.268537044525146
10.4690237045288 0.270541071891785
10.4696092605591 0.272545099258423
10.482307434082 0.274549126625061
10.4915466308594 0.276553153991699
10.4989738464355 0.278557062149048
10.5013666152954 0.280561089515686
10.544059753418 0.282565116882324
10.5468235015869 0.284569144248962
10.5495138168335 0.286573171615601
10.5901355743408 0.288577198982239
10.6304235458374 0.290581226348877
10.6450328826904 0.292585134506226
10.6638078689575 0.294589161872864
10.6762046813965 0.296593189239502
10.6873188018799 0.29859721660614
10.6883878707886 0.300601243972778
10.733470916748 0.302605152130127
10.7378549575806 0.304609179496765
10.7587108612061 0.306613206863403
10.7631206512451 0.308617234230042
10.7858104705811 0.31062126159668
10.7891979217529 0.312625288963318
10.806923866272 0.314629316329956
10.8161630630493 0.316633224487305
10.8171148300171 0.318637251853943
10.8395318984985 0.320641279220581
10.8464660644531 0.322645306587219
10.8586349487305 0.324649333953857
10.8629369735718 0.326653242111206
10.8910474777222 0.328657388687134
10.914605140686 0.330661296844482
10.9205894470215 0.332665324211121
10.9476099014282 0.334669351577759
10.9543676376343 0.336673378944397
10.9555892944336 0.338677406311035
10.9564380645752 0.340681314468384
10.960334777832 0.342685341835022
10.9673328399658 0.34468936920166
10.980354309082 0.346693396568298
10.9811630249023 0.348697423934937
11.0015077590942 0.350701332092285
11.005313873291 0.352705478668213
11.0067930221558 0.354709386825562
11.0085506439209 0.3567134141922
11.0145969390869 0.358717441558838
11.0286693572998 0.360721468925476
11.0343475341797 0.362725496292114
11.0413589477539 0.364729404449463
11.0480737686157 0.366733431816101
11.0502710342407 0.368737459182739
11.061731338501 0.370741486549377
11.0652284622192 0.372745513916016
11.0839643478394 0.374749541282654
11.1057109832764 0.376753568649292
11.1146402359009 0.378757476806641
11.1209907531738 0.380761504173279
11.1300430297852 0.382765531539917
11.1440572738647 0.384769558906555
11.1535558700562 0.386773586273193
11.156364440918 0.388777494430542
11.1573390960693 0.39078152179718
11.1629257202148 0.392785549163818
11.1808547973633 0.394789576530457
11.2081708908081 0.396793603897095
11.2216806411743 0.398797631263733
11.2292881011963 0.400801658630371
11.2410135269165 0.40280556678772
11.2464561462402 0.404809594154358
11.2467460632324 0.406813621520996
11.253306388855 0.408817648887634
11.2554321289062 0.410821676254272
11.2711601257324 0.412825584411621
11.3121919631958 0.414829730987549
11.312219619751 0.416833639144897
11.3128776550293 0.418837666511536
11.3669672012329 0.420841693878174
11.3761901855469 0.422845721244812
11.3847665786743 0.42484974861145
11.4146480560303 0.426853656768799
11.4177465438843 0.428857684135437
11.4200286865234 0.430861711502075
11.4302930831909 0.432865738868713
11.4353542327881 0.434869766235352
11.4604263305664 0.4368736743927
11.4627838134766 0.438877820968628
11.4630613327026 0.440881729125977
11.4633884429932 0.442885756492615
11.4702234268188 0.444889783859253
11.4761600494385 0.446893811225891
11.4771919250488 0.448897838592529
11.4817523956299 0.450901746749878
11.4831647872925 0.452905774116516
11.4836826324463 0.454909801483154
11.4929046630859 0.456913828849792
11.5167427062988 0.458917856216431
11.5445861816406 0.460921883583069
11.5712938308716 0.462925910949707
11.5894403457642 0.464929819107056
11.5945091247559 0.466933846473694
11.5955238342285 0.468937873840332
11.5957765579224 0.47094190120697
11.6026773452759 0.472945928573608
11.6069192886353 0.474949836730957
11.6137933731079 0.476953864097595
11.6143436431885 0.478957891464233
11.624475479126 0.480961918830872
11.6720304489136 0.48296594619751
11.6728525161743 0.484969973564148
11.6776447296143 0.486974000930786
11.7025489807129 0.488977909088135
11.7113265991211 0.490981936454773
11.7243986129761 0.492985963821411
11.7281007766724 0.494989991188049
11.7501029968262 0.496994018554688
11.7535257339478 0.498997926712036
11.756965637207 0.501002073287964
11.8163299560547 0.503005981445312
11.8366613388062 0.505010008811951
11.8523845672607 0.507014036178589
11.8538389205933 0.509018063545227
11.855899810791 0.511022090911865
11.8610782623291 0.513025999069214
11.8617630004883 0.515030145645142
11.8774709701538 0.51703405380249
11.8824672698975 0.519038081169128
11.9062356948853 0.521042108535767
11.9184131622314 0.523046016693115
11.9220066070557 0.525050163269043
11.9388599395752 0.527054071426392
11.9489145278931 0.52905809879303
11.9554815292358 0.531062126159668
11.975435256958 0.533066153526306
11.9980497360229 0.535070180892944
12.0000162124634 0.537074089050293
12.043607711792 0.539078235626221
12.0554733276367 0.541082143783569
12.0748701095581 0.543086171150208
12.0909233093262 0.545090198516846
12.0921983718872 0.547094106674194
12.0934190750122 0.549098253250122
12.1324005126953 0.551102161407471
12.1344404220581 0.553106188774109
12.1421527862549 0.555110216140747
12.1488971710205 0.557114243507385
12.1619882583618 0.559118270874023
12.1808481216431 0.561122179031372
12.1826839447021 0.5631263256073
12.190767288208 0.565130233764648
12.1939525604248 0.567134261131287
12.2042474746704 0.569138288497925
12.2220830917358 0.571142196655273
12.2271146774292 0.573146343231201
12.2288246154785 0.57515025138855
12.229642868042 0.577154397964478
12.2368650436401 0.579158306121826
12.2456579208374 0.581162333488464
12.2525053024292 0.583166360855103
12.2589292526245 0.585170269012451
12.266149520874 0.587174415588379
12.2753114700317 0.589178323745728
12.2859401702881 0.591182351112366
12.2878446578979 0.593186378479004
12.2911939620972 0.595190405845642
12.2959537506104 0.59719443321228
12.3003950119019 0.599198341369629
12.3022489547729 0.601202487945557
12.3042554855347 0.603206396102905
12.3096599578857 0.605210423469543
12.3214998245239 0.607214450836182
12.3298673629761 0.60921835899353
12.3580980300903 0.611222505569458
12.3825044631958 0.613226413726807
12.3866968154907 0.615230441093445
12.4131727218628 0.617234468460083
12.4147901535034 0.619238495826721
12.4160089492798 0.621242523193359
12.4184761047363 0.623246431350708
12.4210710525513 0.625250577926636
12.4213409423828 0.627254486083984
12.4404363632202 0.629258513450623
12.4440479278564 0.631262540817261
12.4553813934326 0.633266448974609
12.4632158279419 0.635270595550537
12.4640426635742 0.637274503707886
12.4910106658936 0.639278531074524
12.4975671768188 0.641282558441162
12.4983730316162 0.6432865858078
12.509425163269 0.645290613174438
12.556134223938 0.647294521331787
12.5569829940796 0.649298667907715
12.5585279464722 0.651302576065063
12.5627412796021 0.653306603431702
12.5881261825562 0.65531063079834
12.5899152755737 0.657314658164978
12.5975742340088 0.659318685531616
12.611289024353 0.661322593688965
12.6219892501831 0.663326740264893
12.6481666564941 0.665330648422241
12.6552076339722 0.667334675788879
12.6693077087402 0.669338703155518
12.7130346298218 0.671342611312866
12.7281808853149 0.673346757888794
12.7284526824951 0.675350666046143
12.7349700927734 0.677354693412781
12.7498235702515 0.679358720779419
12.7691793441772 0.681362748146057
12.7707500457764 0.683366775512695
12.781566619873 0.685370683670044
12.7912645339966 0.687374830245972
12.8000640869141 0.68937873840332
12.8014631271362 0.691382765769958
12.8284883499146 0.693386793136597
12.8371610641479 0.695390701293945
12.85364818573 0.697394847869873
12.8638687133789 0.699398756027222
12.8650102615356 0.70140278339386
12.8746404647827 0.703406810760498
12.8923721313477 0.705410838127136
12.8950872421265 0.707414865493774
12.8987321853638 0.709418773651123
12.9054079055786 0.711422920227051
12.9329795837402 0.713426828384399
12.942798614502 0.715430855751038
12.9627819061279 0.717434883117676
12.9904861450195 0.719438791275024
12.9914741516113 0.721442937850952
12.9943504333496 0.723446846008301
13.0028419494629 0.725450873374939
13.015362739563 0.727454900741577
13.027684211731 0.729458928108215
13.0361328125 0.731462955474854
13.0506038665771 0.733466863632202
13.0628528594971 0.73547101020813
13.0767917633057 0.737474918365479
13.0924425125122 0.739478945732117
13.0931491851807 0.741482973098755
13.1060638427734 0.743487000465393
13.1157808303833 0.745491027832031
13.1266670227051 0.74749493598938
13.1549606323242 0.749499082565308
13.1664762496948 0.751502990722656
13.1777601242065 0.753507018089294
13.1846170425415 0.755511045455933
13.2001504898071 0.757514953613281
13.2091236114502 0.759519100189209
13.2286834716797 0.761523008346558
13.242488861084 0.763527035713196
13.286735534668 0.765531063079834
13.2908658981323 0.767535090446472
13.3140296936035 0.76953911781311
13.3411064147949 0.771543025970459
13.3734550476074 0.773547172546387
13.3767032623291 0.775551080703735
13.3784780502319 0.777555108070374
13.3803339004517 0.779559135437012
13.3822164535522 0.78156304359436
13.3838119506836 0.783567190170288
13.3933277130127 0.785571098327637
13.398832321167 0.787575125694275
13.4112854003906 0.789579153060913
13.436728477478 0.791583180427551
13.440052986145 0.793587207794189
13.4406337738037 0.795591115951538
13.4941596984863 0.797595262527466
13.5103349685669 0.799599170684814
13.5126924514771 0.801603198051453
13.5194034576416 0.803607225418091
13.5357294082642 0.805611133575439
13.6134014129639 0.807615280151367
13.6321249008179 0.809619188308716
13.6647434234619 0.811623334884644
13.6759338378906 0.813627243041992
13.6830425262451 0.81563127040863
13.7077913284302 0.817635297775269
13.7135581970215 0.819639205932617
13.7210083007812 0.821643352508545
13.7450132369995 0.823647260665894
13.7480897903442 0.825651288032532
13.786301612854 0.82765531539917
13.8048477172852 0.829659342765808
13.8908414840698 0.831663370132446
13.95432472229 0.833667278289795
13.9679002761841 0.835671424865723
14.0161046981812 0.837675333023071
14.0623216629028 0.839679360389709
14.0804996490479 0.841683387756348
14.0907764434814 0.843687295913696
14.1720304489136 0.845691442489624
14.175705909729 0.847695350646973
14.2104787826538 0.849699378013611
14.245099067688 0.851703405380249
14.2453927993774 0.853707432746887
14.2564888000488 0.855711460113525
14.2690849304199 0.857715368270874
14.274169921875 0.859719514846802
14.4487285614014 0.86172342300415
14.4641370773315 0.863727450370789
14.4753751754761 0.865731477737427
14.5054121017456 0.867735385894775
14.509069442749 0.869739532470703
14.5405616760254 0.871743440628052
14.5826654434204 0.87374746799469
14.5924253463745 0.875751495361328
14.7773323059082 0.877755522727966
14.8057928085327 0.879759550094604
14.8221807479858 0.881763458251953
14.8233470916748 0.883767604827881
14.839545249939 0.885771512985229
14.9551906585693 0.887775540351868
14.9644451141357 0.889779567718506
14.9688205718994 0.891783595085144
14.98548412323 0.893787622451782
14.9904832839966 0.895791530609131
15.0147018432617 0.897795677185059
15.0956039428711 0.899799585342407
15.096173286438 0.901803612709045
15.1539745330811 0.903807640075684
15.217059135437 0.905811548233032
15.2172431945801 0.90781569480896
15.3365335464478 0.909819602966309
15.4488306045532 0.911823630332947
15.4644079208374 0.913827657699585
15.5381374359131 0.915831685066223
15.5932312011719 0.917835712432861
15.6689071655273 0.91983962059021
15.6825160980225 0.921843767166138
15.8406620025635 0.923847675323486
15.8664054870605 0.925851702690125
15.8717212677002 0.927855730056763
15.9007358551025 0.929859638214111
15.9531774520874 0.931863784790039
15.9681243896484 0.933867692947388
16.1325931549072 0.935871720314026
16.1396827697754 0.937875747680664
16.1720142364502 0.939879775047302
16.1888523101807 0.94188380241394
16.2404136657715 0.943887710571289
16.3004455566406 0.945891857147217
16.4060516357422 0.947895765304565
16.4616050720215 0.949899792671204
16.716703414917 0.951903820037842
16.7260627746582 0.95390772819519
16.7397899627686 0.955911874771118
16.7628765106201 0.957915782928467
16.8432579040527 0.959919810295105
17.0240383148193 0.961923837661743
17.0414085388184 0.963927865028381
17.0630722045898 0.96593189239502
17.0959529876709 0.967935800552368
17.1194534301758 0.969939947128296
17.1992206573486 0.971943855285645
17.2089900970459 0.973947882652283
17.4308090209961 0.975951910018921
17.946949005127 0.977955937385559
17.9654769897461 0.979959964752197
18.2437515258789 0.981963872909546
18.8081836700439 0.983968019485474
18.8422908782959 0.985971927642822
19.0760746002197 0.98797595500946
19.2514762878418 0.989979982376099
19.7179622650146 0.991983890533447
20.2868824005127 0.993988037109375
23.077709197998 0.995991945266724
23.5995655059814 0.997995972633362
24.9995517730713 1
};
\addlegendentry{$\rho_{max} = 0.1$, 32 GHz}
\addplot [color3, very thick, mark=asterisk, mark size=3, mark repeat=50, mark options={solid}]
table {%
7.47811317443848 0
7.50093841552734 0.00200402736663818
7.77655363082886 0.00400805473327637
7.99266004562378 0.00601208209991455
8.10205364227295 0.00801599025726318
8.24157428741455 0.0100200176239014
8.32883167266846 0.0120240449905396
8.37247085571289 0.0140280723571777
8.43817138671875 0.0160320997238159
8.6501989364624 0.0180361270904541
8.65605735778809 0.0200400352478027
8.71331882476807 0.0220440626144409
8.7425012588501 0.0240480899810791
8.75556182861328 0.0260521173477173
8.76078510284424 0.0280561447143555
8.78781414031982 0.0300601720809937
8.833176612854 0.0320640802383423
8.89107227325439 0.0340681076049805
8.89297866821289 0.0360721349716187
8.92455768585205 0.0380761623382568
8.94040298461914 0.040080189704895
8.94541835784912 0.0420842170715332
8.9455451965332 0.0440881252288818
8.97389888763428 0.04609215259552
8.98245716094971 0.0480961799621582
8.99011325836182 0.0501002073287964
9.01055145263672 0.0521042346954346
9.03248500823975 0.0541082620620728
9.06686305999756 0.0561121702194214
9.07719039916992 0.0581161975860596
9.08371353149414 0.0601202249526978
9.11580181121826 0.0621242523193359
9.12099647521973 0.0641282796859741
9.15651798248291 0.0661323070526123
9.15995788574219 0.0681362152099609
9.16719722747803 0.0701402425765991
9.17057991027832 0.0721442699432373
9.23663234710693 0.0741482973098755
9.24318408966064 0.0761523246765137
9.27216529846191 0.0781563520431519
9.27767276763916 0.08016037940979
9.28299331665039 0.0821642875671387
9.28938007354736 0.0841683149337769
9.32191562652588 0.086172342300415
9.35446357727051 0.0881763696670532
9.40345859527588 0.0901803970336914
9.40823936462402 0.0921844244003296
9.42057514190674 0.0941883325576782
9.4491548538208 0.0961923599243164
9.46571350097656 0.0981963872909546
9.46750354766846 0.100200414657593
9.4700231552124 0.102204442024231
9.47603893280029 0.104208469390869
9.4883918762207 0.106212377548218
9.58513641357422 0.108216404914856
9.58723258972168 0.110220432281494
9.59123802185059 0.112224459648132
9.5970344543457 0.114228487014771
9.61660194396973 0.116232514381409
9.6169261932373 0.118236422538757
9.70308399200439 0.120240449905396
9.71422004699707 0.122244477272034
9.72094249725342 0.124248504638672
9.76147747039795 0.12625253200531
9.76881694793701 0.128256559371948
9.78404808044434 0.130260467529297
9.81488227844238 0.132264494895935
9.83660411834717 0.134268522262573
9.84489440917969 0.136272549629211
9.8486499786377 0.13827657699585
9.85194110870361 0.140280604362488
9.90711975097656 0.142284631729126
9.93228721618652 0.144288539886475
9.93749523162842 0.146292567253113
9.95103073120117 0.148296594619751
9.95162677764893 0.150300621986389
9.95812034606934 0.152304649353027
10.0080614089966 0.154308557510376
10.0100183486938 0.156312584877014
10.0147724151611 0.158316612243652
10.0563640594482 0.160320639610291
10.0632181167603 0.162324666976929
10.0751209259033 0.164328694343567
10.0917139053345 0.166332721710205
10.106707572937 0.168336629867554
10.1088876724243 0.170340657234192
10.1221990585327 0.17234468460083
10.1262702941895 0.174348711967468
10.1445941925049 0.176352739334106
10.1459178924561 0.178356647491455
10.1580181121826 0.180360674858093
10.1801843643188 0.182364702224731
10.1820363998413 0.18436872959137
10.1906719207764 0.186372756958008
10.1911058425903 0.188376784324646
10.1934089660645 0.190380811691284
10.197850227356 0.192384719848633
10.2184381484985 0.194388747215271
10.2213220596313 0.196392774581909
10.2232732772827 0.198396801948547
10.2249193191528 0.200400829315186
10.2262487411499 0.202404856681824
10.2272386550903 0.204408884048462
10.2613735198975 0.206412792205811
10.2740087509155 0.208416819572449
10.2823667526245 0.210420846939087
10.2840270996094 0.212424874305725
10.2844047546387 0.214428901672363
10.2887010574341 0.216432809829712
10.2946462631226 0.21843683719635
10.2979640960693 0.220440864562988
10.3045310974121 0.222444891929626
10.3204584121704 0.224448919296265
10.3230171203613 0.226452946662903
10.3266229629517 0.228456974029541
10.3270673751831 0.23046088218689
10.3283348083496 0.232464909553528
10.3308868408203 0.234468936920166
10.3466081619263 0.236472964286804
10.3540515899658 0.238476991653442
10.3665895462036 0.240480899810791
10.394248008728 0.242484927177429
10.4227857589722 0.244488954544067
10.4265995025635 0.246492981910706
10.4379768371582 0.248497009277344
10.4788093566895 0.250501036643982
10.4869632720947 0.25250506401062
10.4881458282471 0.254508972167969
10.4886407852173 0.256512999534607
10.4959926605225 0.258517026901245
10.5070161819458 0.260521054267883
10.5291271209717 0.262525081634521
10.5334720611572 0.26452898979187
10.5486793518066 0.266533136367798
10.5563507080078 0.268537044525146
10.5712862014771 0.270541071891785
10.5723829269409 0.272545099258423
10.5798320770264 0.274549126625061
10.585765838623 0.276553153991699
10.5975255966187 0.278557062149048
10.6140775680542 0.280561089515686
10.6176519393921 0.282565116882324
10.6267204284668 0.284569144248962
10.6389093399048 0.286573171615601
10.645899772644 0.288577198982239
10.6556997299194 0.290581226348877
10.6584157943726 0.292585134506226
10.686502456665 0.294589161872864
10.6902523040771 0.296593189239502
10.6982336044312 0.29859721660614
10.7059392929077 0.300601243972778
10.7370233535767 0.302605152130127
10.7379198074341 0.304609179496765
10.7417106628418 0.306613206863403
10.7427062988281 0.308617234230042
10.7508001327515 0.31062126159668
10.7526397705078 0.312625288963318
10.7569007873535 0.314629316329956
10.7581214904785 0.316633224487305
10.7673387527466 0.318637251853943
10.7932920455933 0.320641279220581
10.8040685653687 0.322645306587219
10.8059062957764 0.324649333953857
10.8064727783203 0.326653242111206
10.8163938522339 0.328657388687134
10.8273801803589 0.330661296844482
10.8509578704834 0.332665324211121
10.8571081161499 0.334669351577759
10.8695163726807 0.336673378944397
10.8701248168945 0.338677406311035
10.880012512207 0.340681314468384
10.8928098678589 0.342685341835022
10.9074668884277 0.34468936920166
10.9090347290039 0.346693396568298
10.9205207824707 0.348697423934937
10.9283800125122 0.350701332092285
10.935097694397 0.352705478668213
10.9463272094727 0.354709386825562
10.9547538757324 0.3567134141922
10.9641342163086 0.358717441558838
10.9856910705566 0.360721468925476
10.9861278533936 0.362725496292114
11.0049772262573 0.364729404449463
11.0268993377686 0.366733431816101
11.0282001495361 0.368737459182739
11.0294046401978 0.370741486549377
11.031774520874 0.372745513916016
11.0383462905884 0.374749541282654
11.0475244522095 0.376753568649292
11.0510492324829 0.378757476806641
11.0643539428711 0.380761504173279
11.0706443786621 0.382765531539917
11.0788459777832 0.384769558906555
11.0896644592285 0.386773586273193
11.0914144515991 0.388777494430542
11.0924377441406 0.39078152179718
11.101508140564 0.392785549163818
11.1180753707886 0.394789576530457
11.1488485336304 0.396793603897095
11.1555194854736 0.398797631263733
11.1564159393311 0.400801658630371
11.1733484268188 0.40280556678772
11.1914491653442 0.404809594154358
11.2040100097656 0.406813621520996
11.2048559188843 0.408817648887634
11.2140092849731 0.410821676254272
11.2168455123901 0.412825584411621
11.2183027267456 0.414829730987549
11.224515914917 0.416833639144897
11.2396755218506 0.418837666511536
11.2601718902588 0.420841693878174
11.2622318267822 0.422845721244812
11.2849168777466 0.42484974861145
11.287052154541 0.426853656768799
11.3045616149902 0.428857684135437
11.3074970245361 0.430861711502075
11.3187561035156 0.432865738868713
11.3207368850708 0.434869766235352
11.3226566314697 0.4368736743927
11.3233442306519 0.438877820968628
11.3311681747437 0.440881729125977
11.3425235748291 0.442885756492615
11.3569974899292 0.444889783859253
11.3613729476929 0.446893811225891
11.36350440979 0.448897838592529
11.3743858337402 0.450901746749878
11.3856964111328 0.452905774116516
11.3871307373047 0.454909801483154
11.3904256820679 0.456913828849792
11.3923482894897 0.458917856216431
11.3950357437134 0.460921883583069
11.3993501663208 0.462925910949707
11.4042634963989 0.464929819107056
11.4260511398315 0.466933846473694
11.4300765991211 0.468937873840332
11.4397897720337 0.47094190120697
11.4463815689087 0.472945928573608
11.459789276123 0.474949836730957
11.4672374725342 0.476953864097595
11.4699850082397 0.478957891464233
11.47034740448 0.480961918830872
11.4915103912354 0.48296594619751
11.5103616714478 0.484969973564148
11.5285062789917 0.486974000930786
11.5289402008057 0.488977909088135
11.5541219711304 0.490981936454773
11.5685615539551 0.492985963821411
11.587043762207 0.494989991188049
11.6010093688965 0.496994018554688
11.6017274856567 0.498997926712036
11.6070261001587 0.501002073287964
11.6171932220459 0.503005981445312
11.6247177124023 0.505010008811951
11.6254758834839 0.507014036178589
11.6261072158813 0.509018063545227
11.6326675415039 0.511022090911865
11.6451864242554 0.513025999069214
11.6560640335083 0.515030145645142
11.6801490783691 0.51703405380249
11.695484161377 0.519038081169128
11.7136573791504 0.521042108535767
11.7148704528809 0.523046016693115
11.7210626602173 0.525050163269043
11.7297325134277 0.527054071426392
11.7374601364136 0.52905809879303
11.7397985458374 0.531062126159668
11.7405519485474 0.533066153526306
11.7516860961914 0.535070180892944
11.7623157501221 0.537074089050293
11.7726631164551 0.539078235626221
11.7853193283081 0.541082143783569
11.7876758575439 0.543086171150208
11.790807723999 0.545090198516846
11.7908821105957 0.547094106674194
11.7951354980469 0.549098253250122
11.7954750061035 0.551102161407471
11.837474822998 0.553106188774109
11.8419494628906 0.555110216140747
11.8481254577637 0.557114243507385
11.8567485809326 0.559118270874023
11.8707418441772 0.561122179031372
11.8855352401733 0.5631263256073
11.8992872238159 0.565130233764648
11.9105224609375 0.567134261131287
11.9127635955811 0.569138288497925
11.9139795303345 0.571142196655273
11.9139976501465 0.573146343231201
11.9268884658813 0.57515025138855
11.9438810348511 0.577154397964478
11.9462299346924 0.579158306121826
11.9516410827637 0.581162333488464
11.9781837463379 0.583166360855103
11.9825305938721 0.585170269012451
11.9855241775513 0.587174415588379
11.9888010025024 0.589178323745728
11.9960813522339 0.591182351112366
12.007661819458 0.593186378479004
12.0393943786621 0.595190405845642
12.0430812835693 0.59719443321228
12.0473890304565 0.599198341369629
12.0961742401123 0.601202487945557
12.0972604751587 0.603206396102905
12.1246643066406 0.605210423469543
12.1277904510498 0.607214450836182
12.129075050354 0.60921835899353
12.1618938446045 0.611222505569458
12.163158416748 0.613226413726807
12.1693840026855 0.615230441093445
12.1822624206543 0.617234468460083
12.1843557357788 0.619238495826721
12.1957330703735 0.621242523193359
12.2032775878906 0.623246431350708
12.2122364044189 0.625250577926636
12.2131195068359 0.627254486083984
12.2210779190063 0.629258513450623
12.2433176040649 0.631262540817261
12.2511053085327 0.633266448974609
12.2568340301514 0.635270595550537
12.2590284347534 0.637274503707886
12.2684488296509 0.639278531074524
12.29150390625 0.641282558441162
12.3110609054565 0.6432865858078
12.3121480941772 0.645290613174438
12.3173484802246 0.647294521331787
12.3192558288574 0.649298667907715
12.3351573944092 0.651302576065063
12.3447570800781 0.653306603431702
12.3478851318359 0.65531063079834
12.3526020050049 0.657314658164978
12.3536653518677 0.659318685531616
12.3575458526611 0.661322593688965
12.3605461120605 0.663326740264893
12.3632049560547 0.665330648422241
12.3818731307983 0.667334675788879
12.3875513076782 0.669338703155518
12.3888864517212 0.671342611312866
12.4091033935547 0.673346757888794
12.4211645126343 0.675350666046143
12.4284801483154 0.677354693412781
12.4406213760376 0.679358720779419
12.458836555481 0.681362748146057
12.4607849121094 0.683366775512695
12.4760866165161 0.685370683670044
12.4801635742188 0.687374830245972
12.4848680496216 0.68937873840332
12.4894113540649 0.691382765769958
12.507643699646 0.693386793136597
12.5124864578247 0.695390701293945
12.5130081176758 0.697394847869873
12.5167303085327 0.699398756027222
12.5251379013062 0.70140278339386
12.5472345352173 0.703406810760498
12.5559597015381 0.705410838127136
12.5805063247681 0.707414865493774
12.5816078186035 0.709418773651123
12.584623336792 0.711422920227051
12.5849962234497 0.713426828384399
12.5995693206787 0.715430855751038
12.6038780212402 0.717434883117676
12.6137771606445 0.719438791275024
12.6149291992188 0.721442937850952
12.6201496124268 0.723446846008301
12.6349582672119 0.725450873374939
12.658953666687 0.727454900741577
12.6750822067261 0.729458928108215
12.7067937850952 0.731462955474854
12.7091817855835 0.733466863632202
12.7236175537109 0.73547101020813
12.7256765365601 0.737474918365479
12.7382574081421 0.739478945732117
12.7389678955078 0.741482973098755
12.7399377822876 0.743487000465393
12.749903678894 0.745491027832031
12.7549180984497 0.74749493598938
12.7810287475586 0.749499082565308
12.7820625305176 0.751502990722656
12.8009815216064 0.753507018089294
12.8032083511353 0.755511045455933
12.8077154159546 0.757514953613281
12.811619758606 0.759519100189209
12.8131885528564 0.761523008346558
12.8146295547485 0.763527035713196
12.8258790969849 0.765531063079834
12.8275985717773 0.767535090446472
12.8462505340576 0.76953911781311
12.8534755706787 0.771543025970459
12.8714876174927 0.773547172546387
12.8877429962158 0.775551080703735
12.8886566162109 0.777555108070374
12.8981485366821 0.779559135437012
12.9099655151367 0.78156304359436
12.9283800125122 0.783567190170288
12.9321250915527 0.785571098327637
12.9550294876099 0.787575125694275
12.9724073410034 0.789579153060913
12.9975757598877 0.791583180427551
12.9988994598389 0.793587207794189
13.0006475448608 0.795591115951538
13.0070848464966 0.797595262527466
13.0359020233154 0.799599170684814
13.0412321090698 0.801603198051453
13.0429944992065 0.803607225418091
13.0449266433716 0.805611133575439
13.0644330978394 0.807615280151367
13.0895538330078 0.809619188308716
13.1001453399658 0.811623334884644
13.1025476455688 0.813627243041992
13.1264734268188 0.81563127040863
13.1559190750122 0.817635297775269
13.1749696731567 0.819639205932617
13.2182140350342 0.821643352508545
13.2216148376465 0.823647260665894
13.2391271591187 0.825651288032532
13.2499656677246 0.82765531539917
13.2565212249756 0.829659342765808
13.2775707244873 0.831663370132446
13.2880115509033 0.833667278289795
13.2930107116699 0.835671424865723
13.2994604110718 0.837675333023071
13.3502101898193 0.839679360389709
13.3845510482788 0.841683387756348
13.3953943252563 0.843687295913696
13.4130611419678 0.845691442489624
13.4367666244507 0.847695350646973
13.4738340377808 0.849699378013611
13.4842967987061 0.851703405380249
13.4944105148315 0.853707432746887
13.5140695571899 0.855711460113525
13.525218963623 0.857715368270874
13.5263366699219 0.859719514846802
13.5436000823975 0.86172342300415
13.5463953018188 0.863727450370789
13.5724086761475 0.865731477737427
13.5951194763184 0.867735385894775
13.6154975891113 0.869739532470703
13.6628217697144 0.871743440628052
13.6976327896118 0.87374746799469
13.7043685913086 0.875751495361328
13.7048540115356 0.877755522727966
13.7208843231201 0.879759550094604
13.7255821228027 0.881763458251953
13.7321920394897 0.883767604827881
13.7542505264282 0.885771512985229
13.7642345428467 0.887775540351868
13.7710123062134 0.889779567718506
13.7889766693115 0.891783595085144
13.8502836227417 0.893787622451782
13.9076194763184 0.895791530609131
13.9403533935547 0.897795677185059
13.9655656814575 0.899799585342407
13.9889335632324 0.901803612709045
14.0978851318359 0.903807640075684
14.1103763580322 0.905811548233032
14.1367378234863 0.90781569480896
14.1373720169067 0.909819602966309
14.2710914611816 0.911823630332947
14.3168563842773 0.913827657699585
14.4230823516846 0.915831685066223
14.4653272628784 0.917835712432861
14.4705648422241 0.91983962059021
14.4916620254517 0.921843767166138
14.504487991333 0.923847675323486
14.5207214355469 0.925851702690125
14.5791425704956 0.927855730056763
14.6148881912231 0.929859638214111
14.7584133148193 0.931863784790039
14.7877006530762 0.933867692947388
14.8166952133179 0.935871720314026
14.8544149398804 0.937875747680664
14.9077758789062 0.939879775047302
14.9621524810791 0.94188380241394
15.0087041854858 0.943887710571289
15.2158679962158 0.945891857147217
15.2349557876587 0.947895765304565
15.2383060455322 0.949899792671204
15.2384262084961 0.951903820037842
15.2486457824707 0.95390772819519
15.5810470581055 0.955911874771118
15.8180961608887 0.957915782928467
16.0676536560059 0.959919810295105
16.0714282989502 0.961923837661743
16.0720443725586 0.963927865028381
16.1165180206299 0.96593189239502
16.3342227935791 0.967935800552368
16.3533763885498 0.969939947128296
16.4737911224365 0.971943855285645
16.5634651184082 0.973947882652283
16.7094097137451 0.975951910018921
16.8807506561279 0.977955937385559
16.880895614624 0.979959964752197
16.9174995422363 0.981963872909546
17.0128707885742 0.983968019485474
17.1079654693604 0.985971927642822
17.6791763305664 0.98797595500946
17.8170204162598 0.989979982376099
17.8368301391602 0.991983890533447
18.1501846313477 0.993988037109375
19.2914485931396 0.995991945266724
19.5246543884277 0.997995972633362
23.8200874328613 1
};
\addlegendentry{$\rho_{max} = 0.3$, 10 GHz}
\addplot [color4, very thick, mark=square*, mark size=3, mark repeat=50, mark options={solid}]
table {%
3.97597146034241 0
4.35706520080566 0.00200402736663818
5.50065851211548 0.00400805473327637
5.58865642547607 0.00601208209991455
5.61657953262329 0.00801599025726318
5.65336990356445 0.0100200176239014
5.67034149169922 0.0120240449905396
5.68871068954468 0.0140280723571777
5.70548629760742 0.0160320997238159
5.75426578521729 0.0180361270904541
5.80736207962036 0.0200400352478027
5.81433773040771 0.0220440626144409
5.95028162002563 0.0240480899810791
5.95076560974121 0.0260521173477173
5.96782636642456 0.0280561447143555
5.97684955596924 0.0300601720809937
6.03634595870972 0.0320640802383423
6.08578300476074 0.0340681076049805
6.12075471878052 0.0360721349716187
6.22419500350952 0.0380761623382568
6.23951864242554 0.040080189704895
6.2588210105896 0.0420842170715332
6.30331230163574 0.0440881252288818
6.4094409942627 0.04609215259552
6.41543436050415 0.0480961799621582
6.48015642166138 0.0501002073287964
6.49058198928833 0.0521042346954346
6.49321126937866 0.0541082620620728
6.49747276306152 0.0561121702194214
6.50270843505859 0.0581161975860596
6.52225732803345 0.0601202249526978
6.55393838882446 0.0621242523193359
6.61643362045288 0.0641282796859741
6.62155103683472 0.0661323070526123
6.63166666030884 0.0681362152099609
6.64800786972046 0.0701402425765991
6.65104866027832 0.0721442699432373
6.66537475585938 0.0741482973098755
6.66796970367432 0.0761523246765137
6.68013334274292 0.0781563520431519
6.69700813293457 0.08016037940979
6.7029857635498 0.0821642875671387
6.77530097961426 0.0841683149337769
6.78284358978271 0.086172342300415
6.79089212417603 0.0881763696670532
6.87838172912598 0.0901803970336914
6.88020372390747 0.0921844244003296
6.88147354125977 0.0941883325576782
6.89366340637207 0.0961923599243164
6.8979172706604 0.0981963872909546
6.90713405609131 0.100200414657593
7.08438205718994 0.102204442024231
7.08701276779175 0.104208469390869
7.15314722061157 0.106212377548218
7.18606185913086 0.108216404914856
7.23866891860962 0.110220432281494
7.25467157363892 0.112224459648132
7.2823691368103 0.114228487014771
7.28614568710327 0.116232514381409
7.30743360519409 0.118236422538757
7.32017803192139 0.120240449905396
7.32019090652466 0.122244477272034
7.3374342918396 0.124248504638672
7.33775043487549 0.12625253200531
7.36244773864746 0.128256559371948
7.36692571640015 0.130260467529297
7.36822462081909 0.132264494895935
7.37232828140259 0.134268522262573
7.38481378555298 0.136272549629211
7.39967632293701 0.13827657699585
7.41697216033936 0.140280604362488
7.46001577377319 0.142284631729126
7.47671985626221 0.144288539886475
7.50991201400757 0.146292567253113
7.51067590713501 0.148296594619751
7.55495452880859 0.150300621986389
7.55966901779175 0.152304649353027
7.59611225128174 0.154308557510376
7.59762668609619 0.156312584877014
7.60360717773438 0.158316612243652
7.60502672195435 0.160320639610291
7.61962938308716 0.162324666976929
7.6216025352478 0.164328694343567
7.62735652923584 0.166332721710205
7.65650224685669 0.168336629867554
7.6677393913269 0.170340657234192
7.66810369491577 0.17234468460083
7.68853807449341 0.174348711967468
7.71294355392456 0.176352739334106
7.71596479415894 0.178356647491455
7.72802543640137 0.180360674858093
7.73293876647949 0.182364702224731
7.73440742492676 0.18436872959137
7.73822546005249 0.186372756958008
7.74983072280884 0.188376784324646
7.76162099838257 0.190380811691284
7.76406812667847 0.192384719848633
7.7906551361084 0.194388747215271
7.83662557601929 0.196392774581909
7.83807706832886 0.198396801948547
7.88156509399414 0.200400829315186
7.88386917114258 0.202404856681824
7.9012279510498 0.204408884048462
7.90914154052734 0.206412792205811
7.91477012634277 0.208416819572449
7.92072534561157 0.210420846939087
7.93452310562134 0.212424874305725
7.93683195114136 0.214428901672363
7.93860244750977 0.216432809829712
7.9424991607666 0.21843683719635
7.94435453414917 0.220440864562988
7.94792318344116 0.222444891929626
7.96693086624146 0.224448919296265
7.97128438949585 0.226452946662903
7.99412298202515 0.228456974029541
7.99633121490479 0.23046088218689
8.02351570129395 0.232464909553528
8.02579402923584 0.234468936920166
8.05665016174316 0.236472964286804
8.07192993164062 0.238476991653442
8.0873851776123 0.240480899810791
8.10541820526123 0.242484927177429
8.10765743255615 0.244488954544067
8.11143589019775 0.246492981910706
8.11791133880615 0.248497009277344
8.13333797454834 0.250501036643982
8.13580131530762 0.25250506401062
8.15318870544434 0.254508972167969
8.15801620483398 0.256512999534607
8.17502021789551 0.258517026901245
8.18872356414795 0.260521054267883
8.19509887695312 0.262525081634521
8.2068338394165 0.26452898979187
8.21023273468018 0.266533136367798
8.21131706237793 0.268537044525146
8.22483062744141 0.270541071891785
8.23640537261963 0.272545099258423
8.24865818023682 0.274549126625061
8.25882053375244 0.276553153991699
8.26182460784912 0.278557062149048
8.2625789642334 0.280561089515686
8.27424907684326 0.282565116882324
8.2825813293457 0.284569144248962
8.28443908691406 0.286573171615601
8.2863302230835 0.288577198982239
8.30081653594971 0.290581226348877
8.31116485595703 0.292585134506226
8.31187915802002 0.294589161872864
8.33293437957764 0.296593189239502
8.34233283996582 0.29859721660614
8.34793090820312 0.300601243972778
8.37421798706055 0.302605152130127
8.41434001922607 0.304609179496765
8.43378639221191 0.306613206863403
8.43718719482422 0.308617234230042
8.45590877532959 0.31062126159668
8.46490478515625 0.312625288963318
8.4962100982666 0.314629316329956
8.4965648651123 0.316633224487305
8.50219440460205 0.318637251853943
8.50870609283447 0.320641279220581
8.51428127288818 0.322645306587219
8.52950859069824 0.324649333953857
8.53402614593506 0.326653242111206
8.58524513244629 0.328657388687134
8.59153175354004 0.330661296844482
8.62325763702393 0.332665324211121
8.63903045654297 0.334669351577759
8.65587329864502 0.336673378944397
8.65625667572021 0.338677406311035
8.66550350189209 0.340681314468384
8.67814540863037 0.342685341835022
8.67855262756348 0.34468936920166
8.69894027709961 0.346693396568298
8.70107460021973 0.348697423934937
8.70313739776611 0.350701332092285
8.70939064025879 0.352705478668213
8.7146692276001 0.354709386825562
8.72077369689941 0.3567134141922
8.73792839050293 0.358717441558838
8.76175212860107 0.360721468925476
8.76873207092285 0.362725496292114
8.76993751525879 0.364729404449463
8.77688026428223 0.366733431816101
8.77913093566895 0.368737459182739
8.79397678375244 0.370741486549377
8.79679298400879 0.372745513916016
8.80179405212402 0.374749541282654
8.80185890197754 0.376753568649292
8.80625247955322 0.378757476806641
8.81217098236084 0.380761504173279
8.81443786621094 0.382765531539917
8.81461811065674 0.384769558906555
8.81845855712891 0.386773586273193
8.82893848419189 0.388777494430542
8.83817386627197 0.39078152179718
8.84775257110596 0.392785549163818
8.85000896453857 0.394789576530457
8.85313701629639 0.396793603897095
8.85625457763672 0.398797631263733
8.89295673370361 0.400801658630371
8.89497661590576 0.40280556678772
8.90146255493164 0.404809594154358
8.91585350036621 0.406813621520996
8.92396831512451 0.408817648887634
8.9394063949585 0.410821676254272
8.94058132171631 0.412825584411621
8.96748065948486 0.414829730987549
8.97428131103516 0.416833639144897
8.98636722564697 0.418837666511536
8.98964977264404 0.420841693878174
9.00645542144775 0.422845721244812
9.01698589324951 0.42484974861145
9.02509021759033 0.426853656768799
9.05855846405029 0.428857684135437
9.06698322296143 0.430861711502075
9.06820583343506 0.432865738868713
9.07570648193359 0.434869766235352
9.07648277282715 0.4368736743927
9.08152389526367 0.438877820968628
9.09021472930908 0.440881729125977
9.09342002868652 0.442885756492615
9.09641742706299 0.444889783859253
9.10157299041748 0.446893811225891
9.10859298706055 0.448897838592529
9.11513805389404 0.450901746749878
9.11636352539062 0.452905774116516
9.12379169464111 0.454909801483154
9.1459379196167 0.456913828849792
9.15312385559082 0.458917856216431
9.15750122070312 0.460921883583069
9.16543674468994 0.462925910949707
9.17067337036133 0.464929819107056
9.1801061630249 0.466933846473694
9.20773029327393 0.468937873840332
9.21407604217529 0.47094190120697
9.23328399658203 0.472945928573608
9.23664474487305 0.474949836730957
9.25447463989258 0.476953864097595
9.25922393798828 0.478957891464233
9.27557373046875 0.480961918830872
9.30230140686035 0.48296594619751
9.30632877349854 0.484969973564148
9.32210922241211 0.486974000930786
9.32954216003418 0.488977909088135
9.34469509124756 0.490981936454773
9.34972476959229 0.492985963821411
9.37253189086914 0.494989991188049
9.37371826171875 0.496994018554688
9.3786792755127 0.498997926712036
9.41618156433105 0.501002073287964
9.42679882049561 0.503005981445312
9.46420669555664 0.505010008811951
9.46441459655762 0.507014036178589
9.46828556060791 0.509018063545227
9.47974491119385 0.511022090911865
9.49527740478516 0.513025999069214
9.50290298461914 0.515030145645142
9.51093864440918 0.51703405380249
9.51749515533447 0.519038081169128
9.52451038360596 0.521042108535767
9.56099510192871 0.523046016693115
9.56569671630859 0.525050163269043
9.56596183776855 0.527054071426392
9.57254219055176 0.52905809879303
9.57265281677246 0.531062126159668
9.58080291748047 0.533066153526306
9.5882682800293 0.535070180892944
9.59027481079102 0.537074089050293
9.61399269104004 0.539078235626221
9.65236759185791 0.541082143783569
9.65286827087402 0.543086171150208
9.67540454864502 0.545090198516846
9.6813497543335 0.547094106674194
9.69523525238037 0.549098253250122
9.70618915557861 0.551102161407471
9.71511840820312 0.553106188774109
9.74129962921143 0.555110216140747
9.7755708694458 0.557114243507385
9.78338623046875 0.559118270874023
9.78411960601807 0.561122179031372
9.79224491119385 0.5631263256073
9.79251098632812 0.565130233764648
9.80429458618164 0.567134261131287
9.80478000640869 0.569138288497925
9.81476402282715 0.571142196655273
9.81836795806885 0.573146343231201
9.81858825683594 0.57515025138855
9.82454776763916 0.577154397964478
9.83193397521973 0.579158306121826
9.84251689910889 0.581162333488464
9.84642505645752 0.583166360855103
9.86779308319092 0.585170269012451
9.87319183349609 0.587174415588379
9.8764705657959 0.589178323745728
9.88482189178467 0.591182351112366
9.88555812835693 0.593186378479004
9.89927291870117 0.595190405845642
9.90559577941895 0.59719443321228
9.90927791595459 0.599198341369629
9.93606948852539 0.601202487945557
9.94846343994141 0.603206396102905
9.95506954193115 0.605210423469543
9.9587869644165 0.607214450836182
9.97750854492188 0.60921835899353
9.98145389556885 0.611222505569458
10.0014390945435 0.613226413726807
10.0122566223145 0.615230441093445
10.0264501571655 0.617234468460083
10.0276727676392 0.619238495826721
10.0334577560425 0.621242523193359
10.0455465316772 0.623246431350708
10.0472230911255 0.625250577926636
10.051441192627 0.627254486083984
10.0826921463013 0.629258513450623
10.1076278686523 0.631262540817261
10.1156044006348 0.633266448974609
10.1183204650879 0.635270595550537
10.1339139938354 0.637274503707886
10.1569337844849 0.639278531074524
10.1591234207153 0.641282558441162
10.168249130249 0.6432865858078
10.1823892593384 0.645290613174438
10.2057247161865 0.647294521331787
10.2057399749756 0.649298667907715
10.2240467071533 0.651302576065063
10.296293258667 0.653306603431702
10.3341112136841 0.65531063079834
10.3394727706909 0.657314658164978
10.3470640182495 0.659318685531616
10.3471002578735 0.661322593688965
10.3604297637939 0.663326740264893
10.3779582977295 0.665330648422241
10.3831596374512 0.667334675788879
10.3854665756226 0.669338703155518
10.3979425430298 0.671342611312866
10.4091310501099 0.673346757888794
10.412127494812 0.675350666046143
10.4164772033691 0.677354693412781
10.4633769989014 0.679358720779419
10.466402053833 0.681362748146057
10.4888687133789 0.683366775512695
10.4936933517456 0.685370683670044
10.4965505599976 0.687374830245972
10.5421085357666 0.68937873840332
10.5459890365601 0.691382765769958
10.5627098083496 0.693386793136597
10.587230682373 0.695390701293945
10.5888357162476 0.697394847869873
10.5900983810425 0.699398756027222
10.594801902771 0.70140278339386
10.596884727478 0.703406810760498
10.6059732437134 0.705410838127136
10.6405401229858 0.707414865493774
10.6892414093018 0.709418773651123
10.7040481567383 0.711422920227051
10.7169094085693 0.713426828384399
10.7269592285156 0.715430855751038
10.7424449920654 0.717434883117676
10.7628479003906 0.719438791275024
10.7630004882812 0.721442937850952
10.7711868286133 0.723446846008301
10.7821235656738 0.725450873374939
10.790355682373 0.727454900741577
10.796238899231 0.729458928108215
10.8023147583008 0.731462955474854
10.8077440261841 0.733466863632202
10.8440532684326 0.73547101020813
10.8571290969849 0.737474918365479
10.8596153259277 0.739478945732117
10.8724269866943 0.741482973098755
10.8751096725464 0.743487000465393
10.8773918151855 0.745491027832031
10.8991556167603 0.74749493598938
10.9005641937256 0.749499082565308
10.9147109985352 0.751502990722656
10.9177265167236 0.753507018089294
10.9457855224609 0.755511045455933
10.9571170806885 0.757514953613281
10.9636754989624 0.759519100189209
10.9929332733154 0.761523008346558
11.0112400054932 0.763527035713196
11.0202598571777 0.765531063079834
11.0280199050903 0.767535090446472
11.0454139709473 0.76953911781311
11.0660543441772 0.771543025970459
11.0693025588989 0.773547172546387
11.0776968002319 0.775551080703735
11.0789518356323 0.777555108070374
11.0856742858887 0.779559135437012
11.0896091461182 0.78156304359436
11.0900020599365 0.783567190170288
11.1231851577759 0.785571098327637
11.1710166931152 0.787575125694275
11.1909856796265 0.789579153060913
11.2020702362061 0.791583180427551
11.2086219787598 0.793587207794189
11.2205667495728 0.795591115951538
11.2609958648682 0.797595262527466
11.2630109786987 0.799599170684814
11.2769641876221 0.801603198051453
11.3290271759033 0.803607225418091
11.3691740036011 0.805611133575439
11.4186773300171 0.807615280151367
11.463303565979 0.809619188308716
11.4670991897583 0.811623334884644
11.4753513336182 0.813627243041992
11.4984169006348 0.81563127040863
11.4994373321533 0.817635297775269
11.5396575927734 0.819639205932617
11.5520973205566 0.821643352508545
11.5535230636597 0.823647260665894
11.5694942474365 0.825651288032532
11.5753440856934 0.82765531539917
11.6201047897339 0.829659342765808
11.6405162811279 0.831663370132446
11.6555013656616 0.833667278289795
11.6567859649658 0.835671424865723
11.7158393859863 0.837675333023071
11.7561178207397 0.839679360389709
11.762734413147 0.841683387756348
11.7853469848633 0.843687295913696
11.8335981369019 0.845691442489624
11.8519153594971 0.847695350646973
11.8529710769653 0.849699378013611
11.8688440322876 0.851703405380249
11.8906888961792 0.853707432746887
11.9054574966431 0.855711460113525
11.9098777770996 0.857715368270874
11.9257879257202 0.859719514846802
11.9302062988281 0.86172342300415
11.9484987258911 0.863727450370789
11.9800662994385 0.865731477737427
12.0117807388306 0.867735385894775
12.085807800293 0.869739532470703
12.1121549606323 0.871743440628052
12.1128330230713 0.87374746799469
12.1468801498413 0.875751495361328
12.1504430770874 0.877755522727966
12.2204427719116 0.879759550094604
12.3268241882324 0.881763458251953
12.4464302062988 0.883767604827881
12.4677972793579 0.885771512985229
12.5020017623901 0.887775540351868
12.5357093811035 0.889779567718506
12.535964012146 0.891783595085144
12.5770978927612 0.893787622451782
12.5853366851807 0.895791530609131
12.6071033477783 0.897795677185059
12.6138648986816 0.899799585342407
12.6245851516724 0.901803612709045
12.6862077713013 0.903807640075684
12.7737283706665 0.905811548233032
12.7988424301147 0.90781569480896
12.8189287185669 0.909819602966309
12.9064388275146 0.911823630332947
12.929181098938 0.913827657699585
13.0180435180664 0.915831685066223
13.0464468002319 0.917835712432861
13.0930423736572 0.91983962059021
13.2086839675903 0.921843767166138
13.2762002944946 0.923847675323486
13.3169984817505 0.925851702690125
13.3184633255005 0.927855730056763
13.3938493728638 0.929859638214111
13.4360513687134 0.931863784790039
13.4427461624146 0.933867692947388
13.4881687164307 0.935871720314026
13.5786457061768 0.937875747680664
13.5935983657837 0.939879775047302
13.6784868240356 0.94188380241394
13.7024831771851 0.943887710571289
13.7367105484009 0.945891857147217
13.8603401184082 0.947895765304565
13.9678688049316 0.949899792671204
13.9729633331299 0.951903820037842
14.035493850708 0.95390772819519
14.0668182373047 0.955911874771118
14.186261177063 0.957915782928467
14.5148773193359 0.959919810295105
14.5513610839844 0.961923837661743
14.5554103851318 0.963927865028381
14.5573797225952 0.96593189239502
14.5674228668213 0.967935800552368
14.5881862640381 0.969939947128296
14.6922159194946 0.971943855285645
14.702223777771 0.973947882652283
14.7315187454224 0.975951910018921
14.7322788238525 0.977955937385559
15.2409505844116 0.979959964752197
15.3915901184082 0.981963872909546
15.9272470474243 0.983968019485474
16.3101539611816 0.985971927642822
16.4903450012207 0.98797595500946
17.1189880371094 0.989979982376099
17.1404476165771 0.991983890533447
17.3780078887939 0.993988037109375
17.741512298584 0.995991945266724
18.3086338043213 0.997995972633362
19.9264793395996 1
};
\addlegendentry{$\rho_{max} = 0.3$, 32 GHz}
\addplot [color5, very thick, mark=*, mark size=3, mark repeat=50, mark options={solid}]
table {%
11.3012056350708 0
11.6080961227417 0.00200402736663818
11.9382762908936 0.00400805473327637
12.0418653488159 0.00601208209991455
12.1170721054077 0.00801599025726318
12.1793222427368 0.0100200176239014
12.2360305786133 0.0120240449905396
12.5000295639038 0.0140280723571777
12.5636301040649 0.0160320997238159
12.7068128585815 0.0180361270904541
12.9631633758545 0.0200400352478027
12.9662895202637 0.0220440626144409
12.9979791641235 0.0240480899810791
13.1474304199219 0.0260521173477173
13.1836977005005 0.0280561447143555
13.1896295547485 0.0300601720809937
13.2609748840332 0.0320640802383423
13.3040189743042 0.0340681076049805
13.306360244751 0.0360721349716187
13.3864011764526 0.0380761623382568
13.4054269790649 0.040080189704895
13.4364356994629 0.0420842170715332
13.4588689804077 0.0440881252288818
13.4807596206665 0.04609215259552
13.4989795684814 0.0480961799621582
13.558253288269 0.0501002073287964
13.5807723999023 0.0521042346954346
13.5907201766968 0.0541082620620728
13.6031866073608 0.0561121702194214
13.6051406860352 0.0581161975860596
13.6169557571411 0.0601202249526978
13.6256923675537 0.0621242523193359
13.6885271072388 0.0641282796859741
13.736930847168 0.0661323070526123
13.7420234680176 0.0681362152099609
13.7563714981079 0.0701402425765991
13.7723093032837 0.0721442699432373
13.7726411819458 0.0741482973098755
13.8390998840332 0.0761523246765137
13.8573007583618 0.0781563520431519
13.9137697219849 0.08016037940979
13.9167451858521 0.0821642875671387
13.918116569519 0.0841683149337769
13.9300842285156 0.086172342300415
13.9771184921265 0.0881763696670532
14.0122394561768 0.0901803970336914
14.0562582015991 0.0921844244003296
14.1143674850464 0.0941883325576782
14.114598274231 0.0961923599243164
14.117018699646 0.0981963872909546
14.1192970275879 0.100200414657593
14.1233739852905 0.102204442024231
14.1245622634888 0.104208469390869
14.1293306350708 0.106212377548218
14.1365966796875 0.108216404914856
14.1413879394531 0.110220432281494
14.1592903137207 0.112224459648132
14.1599407196045 0.114228487014771
14.1651515960693 0.116232514381409
14.1920900344849 0.118236422538757
14.1999378204346 0.120240449905396
14.2145118713379 0.122244477272034
14.2227516174316 0.124248504638672
14.2228393554688 0.12625253200531
14.2253618240356 0.128256559371948
14.2268943786621 0.130260467529297
14.2292718887329 0.132264494895935
14.2376441955566 0.134268522262573
14.2671489715576 0.136272549629211
14.269907951355 0.13827657699585
14.2727165222168 0.140280604362488
14.2982807159424 0.142284631729126
14.310661315918 0.144288539886475
14.3260736465454 0.146292567253113
14.3280572891235 0.148296594619751
14.3389120101929 0.150300621986389
14.3491849899292 0.152304649353027
14.3765678405762 0.154308557510376
14.380350112915 0.156312584877014
14.3835906982422 0.158316612243652
14.399001121521 0.160320639610291
14.4260330200195 0.162324666976929
14.4337377548218 0.164328694343567
14.4353837966919 0.166332721710205
14.4435539245605 0.168336629867554
14.4732627868652 0.170340657234192
14.4742937088013 0.17234468460083
14.510703086853 0.174348711967468
14.5263547897339 0.176352739334106
14.531476020813 0.178356647491455
14.5331945419312 0.180360674858093
14.5449132919312 0.182364702224731
14.5677909851074 0.18436872959137
14.5751476287842 0.186372756958008
14.5797300338745 0.188376784324646
14.5815477371216 0.190380811691284
14.5865354537964 0.192384719848633
14.5869626998901 0.194388747215271
14.5904712677002 0.196392774581909
14.6037158966064 0.198396801948547
14.6121454238892 0.200400829315186
14.614372253418 0.202404856681824
14.6157960891724 0.204408884048462
14.6307220458984 0.206412792205811
14.6350469589233 0.208416819572449
14.6375169754028 0.210420846939087
14.6389293670654 0.212424874305725
14.6412506103516 0.214428901672363
14.6473932266235 0.216432809829712
14.6642904281616 0.21843683719635
14.6713314056396 0.220440864562988
14.6784534454346 0.222444891929626
14.6906099319458 0.224448919296265
14.7054653167725 0.226452946662903
14.7134275436401 0.228456974029541
14.7256364822388 0.23046088218689
14.7311782836914 0.232464909553528
14.7424688339233 0.234468936920166
14.7503890991211 0.236472964286804
14.7539443969727 0.238476991653442
14.7696514129639 0.240480899810791
14.7702312469482 0.242484927177429
14.7746267318726 0.244488954544067
14.7767343521118 0.246492981910706
14.7984142303467 0.248497009277344
14.8021335601807 0.250501036643982
14.8133239746094 0.25250506401062
14.8298788070679 0.254508972167969
14.8299016952515 0.256512999534607
14.830171585083 0.258517026901245
14.8392276763916 0.260521054267883
14.8406801223755 0.262525081634521
14.8645868301392 0.26452898979187
14.8658285140991 0.266533136367798
14.8684377670288 0.268537044525146
14.8720865249634 0.270541071891785
14.8969125747681 0.272545099258423
14.9038543701172 0.274549126625061
14.9268636703491 0.276553153991699
14.9321355819702 0.278557062149048
14.9359292984009 0.280561089515686
14.9493246078491 0.282565116882324
14.9638290405273 0.284569144248962
14.9739189147949 0.286573171615601
14.9760599136353 0.288577198982239
14.9791593551636 0.290581226348877
14.9887361526489 0.292585134506226
15.0262498855591 0.294589161872864
15.0446844100952 0.296593189239502
15.0492925643921 0.29859721660614
15.0623931884766 0.300601243972778
15.0677623748779 0.302605152130127
15.0683965682983 0.304609179496765
15.0714750289917 0.306613206863403
15.080846786499 0.308617234230042
15.0919981002808 0.31062126159668
15.1052436828613 0.312625288963318
15.1063041687012 0.314629316329956
15.1101655960083 0.316633224487305
15.1261281967163 0.318637251853943
15.1267490386963 0.320641279220581
15.1297178268433 0.322645306587219
15.1341876983643 0.324649333953857
15.1374244689941 0.326653242111206
15.1530485153198 0.328657388687134
15.188627243042 0.330661296844482
15.2044172286987 0.332665324211121
15.2066287994385 0.334669351577759
15.2181377410889 0.336673378944397
15.2406949996948 0.338677406311035
15.2515544891357 0.340681314468384
15.2799005508423 0.342685341835022
15.2862110137939 0.34468936920166
15.2942762374878 0.346693396568298
15.3115329742432 0.348697423934937
15.3140316009521 0.350701332092285
15.3243789672852 0.352705478668213
15.3264818191528 0.354709386825562
15.3329420089722 0.3567134141922
15.3604755401611 0.358717441558838
15.3671808242798 0.360721468925476
15.3744163513184 0.362725496292114
15.3774785995483 0.364729404449463
15.3781080245972 0.366733431816101
15.3822727203369 0.368737459182739
15.390754699707 0.370741486549377
15.3987722396851 0.372745513916016
15.4110860824585 0.374749541282654
15.4180364608765 0.376753568649292
15.4227323532104 0.378757476806641
15.4308109283447 0.380761504173279
15.4448757171631 0.382765531539917
15.4472017288208 0.384769558906555
15.4685344696045 0.386773586273193
15.4741277694702 0.388777494430542
15.4772024154663 0.39078152179718
15.4794645309448 0.392785549163818
15.4838495254517 0.394789576530457
15.4978485107422 0.396793603897095
15.5034284591675 0.398797631263733
15.509069442749 0.400801658630371
15.5219116210938 0.40280556678772
15.5302314758301 0.404809594154358
15.5356245040894 0.406813621520996
15.5454530715942 0.408817648887634
15.5486898422241 0.410821676254272
15.5721235275269 0.412825584411621
15.5813236236572 0.414829730987549
15.5817308425903 0.416833639144897
15.5843667984009 0.418837666511536
15.5891847610474 0.420841693878174
15.5899686813354 0.422845721244812
15.6038513183594 0.42484974861145
15.6119737625122 0.426853656768799
15.6139822006226 0.428857684135437
15.6340198516846 0.430861711502075
15.6379537582397 0.432865738868713
15.6595573425293 0.434869766235352
15.677864074707 0.4368736743927
15.6870040893555 0.438877820968628
15.6949548721313 0.440881729125977
15.6960477828979 0.442885756492615
15.7207164764404 0.444889783859253
15.7296257019043 0.446893811225891
15.7352361679077 0.448897838592529
15.7367420196533 0.450901746749878
15.7432985305786 0.452905774116516
15.7505474090576 0.454909801483154
15.7534656524658 0.456913828849792
15.7725629806519 0.458917856216431
15.7788228988647 0.460921883583069
15.7833013534546 0.462925910949707
15.8121910095215 0.464929819107056
15.8176879882812 0.466933846473694
15.8406934738159 0.468937873840332
15.8485670089722 0.47094190120697
15.8642721176147 0.472945928573608
15.8665208816528 0.474949836730957
15.8929586410522 0.476953864097595
15.8931427001953 0.478957891464233
15.906813621521 0.480961918830872
15.9243259429932 0.48296594619751
15.9455127716064 0.484969973564148
15.9523620605469 0.486974000930786
15.9530534744263 0.488977909088135
15.9613971710205 0.490981936454773
15.970682144165 0.492985963821411
15.981746673584 0.494989991188049
15.993860244751 0.496994018554688
16.0050640106201 0.498997926712036
16.0072116851807 0.501002073287964
16.0099239349365 0.503005981445312
16.0251655578613 0.505010008811951
16.0545635223389 0.507014036178589
16.0554027557373 0.509018063545227
16.0581550598145 0.511022090911865
16.060245513916 0.513025999069214
16.0646553039551 0.515030145645142
16.0681114196777 0.51703405380249
16.0804309844971 0.519038081169128
16.0829925537109 0.521042108535767
16.0977592468262 0.523046016693115
16.0986595153809 0.525050163269043
16.1024837493896 0.527054071426392
16.1120338439941 0.52905809879303
16.1130847930908 0.531062126159668
16.1143131256104 0.533066153526306
16.1267471313477 0.535070180892944
16.1282768249512 0.537074089050293
16.1452407836914 0.539078235626221
16.1508712768555 0.541082143783569
16.1515674591064 0.543086171150208
16.165994644165 0.545090198516846
16.1770820617676 0.547094106674194
16.2005443572998 0.549098253250122
16.2153301239014 0.551102161407471
16.2155799865723 0.553106188774109
16.2174663543701 0.555110216140747
16.2176303863525 0.557114243507385
16.2588520050049 0.559118270874023
16.2854614257812 0.561122179031372
16.2956809997559 0.5631263256073
16.2958946228027 0.565130233764648
16.2989749908447 0.567134261131287
16.3030452728271 0.569138288497925
16.3276481628418 0.571142196655273
16.3283100128174 0.573146343231201
16.3337993621826 0.57515025138855
16.3352546691895 0.577154397964478
16.3354110717773 0.579158306121826
16.3522396087646 0.581162333488464
16.3524646759033 0.583166360855103
16.3640556335449 0.585170269012451
16.3648834228516 0.587174415588379
16.3756008148193 0.589178323745728
16.3860721588135 0.591182351112366
16.3976249694824 0.593186378479004
16.4081859588623 0.595190405845642
16.4186229705811 0.59719443321228
16.4405479431152 0.599198341369629
16.4406337738037 0.601202487945557
16.4764938354492 0.603206396102905
16.4789085388184 0.605210423469543
16.506010055542 0.607214450836182
16.5072250366211 0.60921835899353
16.515811920166 0.611222505569458
16.5276374816895 0.613226413726807
16.5334796905518 0.615230441093445
16.5379180908203 0.617234468460083
16.5444774627686 0.619238495826721
16.5450172424316 0.621242523193359
16.583101272583 0.623246431350708
16.6129970550537 0.625250577926636
16.630744934082 0.627254486083984
16.6482524871826 0.629258513450623
16.6580791473389 0.631262540817261
16.6634635925293 0.633266448974609
16.6649322509766 0.635270595550537
16.6763134002686 0.637274503707886
16.688138961792 0.639278531074524
16.6897277832031 0.641282558441162
16.7408065795898 0.6432865858078
16.7591876983643 0.645290613174438
16.7795276641846 0.647294521331787
16.7945766448975 0.649298667907715
16.795955657959 0.651302576065063
16.817964553833 0.653306603431702
16.8226985931396 0.65531063079834
16.8488082885742 0.657314658164978
16.8600540161133 0.659318685531616
16.8617916107178 0.661322593688965
16.8709011077881 0.663326740264893
16.889820098877 0.665330648422241
16.8948707580566 0.667334675788879
16.9102802276611 0.669338703155518
16.913595199585 0.671342611312866
16.9137954711914 0.673346757888794
16.9193058013916 0.675350666046143
16.9418144226074 0.677354693412781
16.9453411102295 0.679358720779419
16.9723491668701 0.681362748146057
16.976432800293 0.683366775512695
16.9890098571777 0.685370683670044
16.9903774261475 0.687374830245972
16.9987926483154 0.68937873840332
17.0151977539062 0.691382765769958
17.0222129821777 0.693386793136597
17.0505657196045 0.695390701293945
17.0592746734619 0.697394847869873
17.0806579589844 0.699398756027222
17.0818347930908 0.70140278339386
17.1014842987061 0.703406810760498
17.1314697265625 0.705410838127136
17.140495300293 0.707414865493774
17.1540012359619 0.709418773651123
17.1883506774902 0.711422920227051
17.1888427734375 0.713426828384399
17.2326736450195 0.715430855751038
17.2349720001221 0.717434883117676
17.2382888793945 0.719438791275024
17.242130279541 0.721442937850952
17.2458896636963 0.723446846008301
17.2471904754639 0.725450873374939
17.2524471282959 0.727454900741577
17.2835254669189 0.729458928108215
17.3007049560547 0.731462955474854
17.3072776794434 0.733466863632202
17.3194694519043 0.73547101020813
17.3246536254883 0.737474918365479
17.3371829986572 0.739478945732117
17.3599796295166 0.741482973098755
17.3725910186768 0.743487000465393
17.3946151733398 0.745491027832031
17.4292507171631 0.74749493598938
17.4312210083008 0.749499082565308
17.4446926116943 0.751502990722656
17.4565315246582 0.753507018089294
17.4739437103271 0.755511045455933
17.4927921295166 0.757514953613281
17.5028209686279 0.759519100189209
17.5233974456787 0.761523008346558
17.5364589691162 0.763527035713196
17.5425701141357 0.765531063079834
17.5449600219727 0.767535090446472
17.5885219573975 0.76953911781311
17.5909996032715 0.771543025970459
17.5981063842773 0.773547172546387
17.6313133239746 0.775551080703735
17.6377601623535 0.777555108070374
17.6388397216797 0.779559135437012
17.640323638916 0.78156304359436
17.6707668304443 0.783567190170288
17.6830310821533 0.785571098327637
17.7082977294922 0.787575125694275
17.7117900848389 0.789579153060913
17.7173805236816 0.791583180427551
17.7213439941406 0.793587207794189
17.7298717498779 0.795591115951538
17.7357215881348 0.797595262527466
17.7674217224121 0.799599170684814
17.7724590301514 0.801603198051453
17.7761974334717 0.803607225418091
17.8277435302734 0.805611133575439
17.8413505554199 0.807615280151367
17.8604297637939 0.809619188308716
17.862361907959 0.811623334884644
17.8877830505371 0.813627243041992
17.9559516906738 0.81563127040863
17.9631080627441 0.817635297775269
17.9658603668213 0.819639205932617
17.9665641784668 0.821643352508545
17.9716720581055 0.823647260665894
17.9998416900635 0.825651288032532
18.0310726165771 0.82765531539917
18.0321197509766 0.829659342765808
18.0385761260986 0.831663370132446
18.0486621856689 0.833667278289795
18.0729694366455 0.835671424865723
18.1245346069336 0.837675333023071
18.1346836090088 0.839679360389709
18.176061630249 0.841683387756348
18.200963973999 0.843687295913696
18.2083835601807 0.845691442489624
18.2214889526367 0.847695350646973
18.2676086425781 0.849699378013611
18.30859375 0.851703405380249
18.357234954834 0.853707432746887
18.3727474212646 0.855711460113525
18.4554862976074 0.857715368270874
18.458366394043 0.859719514846802
18.5128040313721 0.86172342300415
18.6047286987305 0.863727450370789
18.6138820648193 0.865731477737427
18.6433792114258 0.867735385894775
18.6541900634766 0.869739532470703
18.785572052002 0.871743440628052
18.8110446929932 0.87374746799469
18.8357772827148 0.875751495361328
18.8417339324951 0.877755522727966
18.8837375640869 0.879759550094604
18.9793243408203 0.881763458251953
18.999927520752 0.883767604827881
19.0043106079102 0.885771512985229
19.0073699951172 0.887775540351868
19.0229663848877 0.889779567718506
19.0230331420898 0.891783595085144
19.0580654144287 0.893787622451782
19.0877876281738 0.895791530609131
19.1487464904785 0.897795677185059
19.1615447998047 0.899799585342407
19.171350479126 0.901803612709045
19.1784820556641 0.903807640075684
19.2631187438965 0.905811548233032
19.3209056854248 0.90781569480896
19.3358211517334 0.909819602966309
19.370288848877 0.911823630332947
19.3791275024414 0.913827657699585
19.3845443725586 0.915831685066223
19.4318256378174 0.917835712432861
19.4808311462402 0.91983962059021
19.4957447052002 0.921843767166138
19.5361919403076 0.923847675323486
19.5860900878906 0.925851702690125
19.6215534210205 0.927855730056763
19.6916103363037 0.929859638214111
19.8893089294434 0.931863784790039
20.0295066833496 0.933867692947388
20.0735149383545 0.935871720314026
20.0899848937988 0.937875747680664
20.0986747741699 0.939879775047302
20.1820411682129 0.94188380241394
20.2969837188721 0.943887710571289
20.3130588531494 0.945891857147217
20.334602355957 0.947895765304565
20.4585304260254 0.949899792671204
20.6030540466309 0.951903820037842
20.8942623138428 0.95390772819519
21.1887989044189 0.955911874771118
21.2335720062256 0.957915782928467
21.2600498199463 0.959919810295105
21.3390674591064 0.961923837661743
21.4812049865723 0.963927865028381
21.6298007965088 0.96593189239502
21.7051639556885 0.967935800552368
21.947395324707 0.969939947128296
21.962423324585 0.971943855285645
22.0701770782471 0.973947882652283
22.5856266021729 0.975951910018921
22.769926071167 0.977955937385559
22.9973487854004 0.979959964752197
23.4712047576904 0.981963872909546
23.7861728668213 0.983968019485474
25.1045761108398 0.985971927642822
25.2777786254883 0.98797595500946
25.3154945373535 0.989979982376099
25.9642639160156 0.991983890533447
27.2881660461426 0.993988037109375
27.8921756744385 0.995991945266724
29.8678398132324 0.997995972633362
30.1862010955811 1
};
\addlegendentry{$\rho_{max} = 0$}
\end{axis}

\end{tikzpicture}

    %     \setlength\abovecaptionskip{0cm}
    % \setlength\belowcaptionskip{-.3cm}
    \caption{\Gls{e2e} latency \gls{ecdf} for different configurations for 80 Mbps user rate.}
    \label{fig:Latency}
\end{figure}

Finally, the average system throughput for different ratios $\rho_{max}$ of \gls{thz} link is shown in Fig.~\ref{fig:THZtotal}.
The system throughput increases with the inclusion of additional \gls{thz} links. 
The figure also shows that system source rates of 2 Gbps, 4 Gbps, and 10 Gbps can be satisfied by a single donor when $\rho_{max}$ is properly set.
%
% converge to maximum source rates in the entire presence of \gls{thz} links, 
However, the larger demand of the 25 Gbps system source rate still cannot be satisfied, as the system becomes saturated.
% , and prevents convergence to required source rates. 
$\rho_{max} = 0.1$ and $1$ can increase the system throughput by up to four times and twelve times, respectively.

\begin{figure}
    \centering
    \setlength\fwidth{0.55\columnwidth}
    \setlength\fheight{0.35\columnwidth}
    % This file was created with tikzplotlib v0.10.1.
\begin{tikzpicture}

\definecolor{darkgray176}{RGB}{150,150,150}
\definecolor{gray}{RGB}{128,128,128}
\definecolor{green}{RGB}{0,128,0}
\definecolor{lightgray204}{RGB}{204,204,204}
\definecolor{orange}{RGB}{255,165,0}
\definecolor{blue}{HTML}{5bc4eb}

\definecolor{color1}{RGB}{255,20,189}
\definecolor{color2}{RGB}{40, 215, 235}
\definecolor{color3}{RGB}{240, 131, 22}
\definecolor{color4}{RGB}{73, 214, 34}
\definecolor{color5}{RGB}{255,225,82}

\begin{axis}[
    width=\fwidth,
    height=\fheight,
    at={(0\fwidth,0\fheight)},
    scale only axis,
    legend cell align={left},
    legend style={
        legend columns=4,
        draw opacity=1,
        text opacity=1,
        at={(0.5, 1.05)},
        anchor=north,
        draw=black,
        font=\footnotesize
    },
    tick align=outside,
    tick pos=left,
    x grid style={darkgray176},
    xlabel={Maximum ratio of \gls{thz} links $\rho_{max}$},
    xmajorgrids,
    xmin=0, xmax=1,
    xtick style={color=black},
    y grid style={darkgray176},
    ylabel={Throughput [Gbps]},
    ymajorgrids,
    ymin=0.9, ymax=24.0983225,
    ytick style={color=black}
]
\addplot [very thick, color1, mark size = 4, mark=triangle*]
table {%
0 0.8568
0.10 1.327
0.30 1.686
0.50 1.7442
0.70 1.8454
1 1.9762
};
\addlegendentry{2 Gbps}
\addplot [very thick, mark size = 4,color2, mark=*]
table {%
0 1.198
0.10 2.2434
0.30 3.0524
0.50 3.3556
0.70 3.5194
1 3.923
};
\addlegendentry{4 Gbps}
\addplot [very thick, mark size = 4, color3, mark=asterisk]
table {%
0 1.5812
0.10 4.1734
0.30 6.2714
0.50 7.2284
0.70 8.2628
1 9.4924
};
\addlegendentry{10 Gbps}
\addplot [very thick, mark size = 4, color4, mark=triangle*]
table {%
0 1.671
0.10 6.9562
0.30 11.6684
0.50 13.88
0.70 17.144
1 20.635
};
\addlegendentry{25 Gbps}
\end{axis}

\end{tikzpicture}

    %     \setlength\abovecaptionskip{0cm}
    % \setlength\belowcaptionskip{-.3cm}
    \caption{System throughput for different source rate and ratio of \gls{thz} links.}
    \label{fig:THZtotal}
\end{figure}


\section{Conclusions and Future Work}
\label{sec:concl-iab}

%%%%%%%%%
%% RES MAN
%%%%%%%%

In this paper we proposed a semi-centralized resource partitioning scheme for \gls{5g} and beyond \gls{iab} networks, coupled with a set of allocation policies. We showed that the introduction of this light resource allocation cooperation dramatically improves the end-to-end throughput and delay achieved by the system already, preventing (or at the very least limiting) the insurgence of network congestion on the backhaul links. Specifically, the \gls{mrba} policy exhibits the most promising results, offering up to a 3-fold increase in the worst-case throughput and approximately a 30\% smaller worst-case latency, compared to the distributed scheduler. On the other hand, the effectiveness of the \gls{ba} and \gls{msr} policies varies quite significantly across the specific system configuration and inspected metric. %Nevertheless, better than SoA

We provided considerations on the implementation of a semi-centralized resource allocation controller in real world deployments. In particular, we acknowledged that the proposed scheme relies on the assumption of \gls{iab}-nodes being capable of exchanging timely feedback information with the \gls{iab}-donor. Even though the amount of signaling data which the proposed solution requires is quite low, and its performance is quite robust with respect to an increase of the central allocation period, we argue that this remains a significant constraint. Moreover, such drawback is exacerbated by the unfavorable \glspl{mmwave} propagation characteristics.
As a consequence, we deem that solutions involving a central controller, which rely on timely exchange of control information with the \gls{iab}-donor, are likely to require dedicated control channels, possibly at sub-6~{GHz}, in order to grant the utmost priority and reliability to the feedback information. Therefore, we can conclude that the aforementioned framework can bring dramatic performance benefits to \gls{iab} networks, although its introduction in \gls{5g} and beyond deployments requires additional research efforts.

For this reason, as part of our future work we plan to design machine-learning algorithms which predict the network evolution at the \gls{iab}-donor. This improvement will allow us to relax the timely feedback assumption, by increasing the minimum semi-centralized allocation period which leads to performance benefits over distributed strategies. 
Moreover, we foresee to implement mechanisms which adapt the parameters of the \gls{mrba} policy to the system load and configuration, and additional resource partitioning strategies. 
Finally, the generalization of the proposed framework to \gls{sdma} systems will be studied. 
The use of such multiple access scheme should significantly improve the performance of \gls{mmwave} wireless backhauling by introducing the possibility of concurrently serving multiple terminals, provided that they exhibit a sufficient distance among them.

%%%%%%%%%%
%% SAFEHAUL
%%%%%%%%%

In this work, we proposed the first reliability-focused scheduling and path selection algorithm for \gls{iab} mmWave networks. We illustrated that our \gls{rl}-based solution can cope with the network dynamics including channel, interference, and load.  Furthermore, we demonstrated that \name{} not only  exhibits highly reliable performance in the presence of the above-mentioned network dynamics, but also outperforms the benchmark schemes in terms of throughput, latency and packet-drop rate. The reliability of \name{} stems from the joint minimization of the average latency, and the expected value of its tail losses, by leveraging \gls{cvar} as a risk metric.

Reliability is a highly under-explored topic that definitely deserves more investigation. Some interesting research directions are the maximization of reliability under the assumption of statistical system knowledge, or the evaluation of the network's reliability when the functionality of the \gls{bap} layer is compromised.
Furthermore, our system-level extension to Sionna can be further developed to support an arbitrary number of RF chains and in-band backhauling, allowing more extensive investigation of IAB protocols and architectures.

%%%%%%%%
%%% WONS
%%%%%%%%%


This paper provides the first performance evaluation of the possibilities of sub-terahertz frequencies for 6G IAB using a customized extension of the open-source Sionna simulator. This permits the use of greedy algorithms to evaluate the deployment of mixed mmWave and sub-terahertz links to boost the backhaul network's capacity.
We will broaden the analysis of the network performance to cover a broader range of source traffic patterns, scenarios (including multi-donor instances, deployments with lower node density, or more realistic map-based scenarios as in~\cite{gemmi2023on,gemmi2022on}), and protocol stack implementations as future work.