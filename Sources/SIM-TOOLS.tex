\chapter{Simulation tools for future cellular networks}
\label{ch:sim-tools}

The preliminary, yet reliable performance evaluation of disruptive research ideas is a vital step in identifying the most promising key technology enablers for future cellular networks, before further research and development, and eventual deployment in real-world networks.
To this end, while providing reliable performance estimates, experiments with real testbeds are often impractical due to limitations in the scalability and flexibility of the involved platforms, as well as the high cost of hardware components. At the same time, analytical models usually introduce strong assumptions for the sake of tractability. End-to-end simulations fill these gaps, by coupling (possibly simplified) analytical models with the protocol stack accuracy exhibited by testbeds and/or digital twins.

Additionally, the ever increasing complexity and heterogeneous nature of wireless networks is poised to be coupled with an \gls{ai}-native design which, thanks to the ongoing virtualization, will not be limited to the radio link level, but will encompass the orchestration of large scale deployments as well~\cite{polese2023understanding}.
Nevertheless, how to design, test and eventually deploy management and orchestration algorithms is an open research challenge~\cite{polese2022colo}.
First, the training data must accurately capture the interplay of the whole protocol stack with the wireless channel. Secondly, optimization frameworks such as \gls{drl} also call for preliminary testing in isolated yet realistic environments, with the goal of minimizing the performance degradation to actual network deployments~\cite{lacava2023programmable, amir2023safehaul}.
Both these requirements are met by end-to-end simulations, which will thus play a fundamental role in designing and evaluating the performance of the next generation of cellular networks.

%In all these cases, the importance of testing network configurations via simulations on a sandbox environment becomes even more paramount.
%Additionally, simulations can adapt better than testbeds to the large number of evolving use cases and deployment scenarios that such networks will serve. ns-3 is well positioned to be an important simulation tool for future wireless networks, thanks to the already available modules for mmWaves and NR [46, 47], IEEE 802.11ad/ay [48, 49], and to the activity to extend the wifi module to also support IEEE 802.11ax [50].

Recently, both academic and industry researchers have been strongly favoring open-source simulators~\cite{10465179}, i.e., simulators which are made freely available for both use, modification and redistribution. In general, these characteristics of these softwares foster decentralized development and open collaboration.
In this class of simulators, ns-3 is the facto standard in the wireless research space, thanks to the already available modules for 5G NR~\cite{mezzavilla2018end, patriciello2019e2e}, IEEE 802.11ad/ay/ax~\cite{magrinValid2021, 10.1145/3460797.3460799, 7461452} and its implementation of the 3GPP TR 38.901 statistical channel model~\cite{zugno2020implementation}.
Nevertheless, ns-3 currently lacks physical propagation models for most disruptive 6G deployment solutions. Most notably, ns-3 lacks channel models for \gls{af} and \glspl{irs} relays (also referred to as ``smart relays"), and/or \gls{ntn}. Both these technologies are expected to play a key role in achieving the ubiquitous connectivity target set for 6G networks.
Moreover, the ns-3 TR 38.901~\cite{3gpp.38.901} channel modeling framework exhibits limitation in its scalability, thus rendering infeasible the simulation of large-scale deployments and/or terminals featuring massive-\gls{mimo} arrays.

Despite their widespread use, the short- and medium-term suitability of end-to-end network simulators as performance evaluation tools will largely depend on their scalability for realistically-sized deployments, and on the accuracy of their channel model~\cite{testolina2020scalable}.
In fact, system-level simulators generally abstract the actual link-level transmission via an error model, which maps the \gls{sinr} of the wireless link to a packet error probability~\cite{lagen2020new}. Eventually, the latter is used to determine whether the packet has been successfully decoded by the receiver. As a consequence, the accuracy of network simulators heavily depends on the reliability of the \gls{sinr} estimation, especially when considering the \gls{mmwave} and \gls{thz} bands. 
Indeed, these portions of the spectrum entail a major redesign not only of the physical layer, but also of the whole cellular protocol stack~\cite{shafi2018microwave}, which makes it paramount to accurately model the peculiar propagation characteristics which they exhibit. For instance, the intrinsic directionality of the communication requires ad hoc control procedures~\cite{heng2021six}, while the frequent transitions between \gls{los} and \gls{nlos} conditions call for an ad hoc transport layer design, such as novel \gls{tcp} algorithms~\cite{zhang2019will}.

To fill these gaps, in the first part of this chapter we introduce the design and implementation of channel models for innovative deployment solutions in ns-3. 
In particular, Section~\ref{sec:ch_model_ext} describes a new channel model for ns-3 for IRS/AF-aided communications which is based on the current 3GPP channel model for 5G networks standardized in~\cite{3gpp.38.901}. 
Then, we present how the former can be used in conjunction with the \texttt{ns3-mmwave} module~\cite{mezzavilla2018end}, which models the \gls{phy} and \gls{mac} layers of the \gls{5g} NR protocol stack to achieve an end-to-end simulation framework for smart relays. The latter incorporates the interplay with the \gls{5g} NR protocol stack and relative control tasks, as well as the impact of the upper (including transport and application) layers. Then, we leverage this novel framework to conduct an extensive simulation campaign to study the performance of IRS/AF nodes for relaying connectivity requests from end users, compared to a baseline solution in which relays are not deployed. We demonstrate that IRSs and AF relays are valid solutions, especially in small networks, even though high-EIRP AF relays are required to support more aggressive traffic applications. Based on our simulations, we provide guidelines towards the optimal dimensioning of IRS and AF configurations, in terms of number of antenna elements and amplification power.

Section~\ref{sec:channel-ntn} presents a new open-source module for ns-3 that implements the \gls{ntn} channel model based on the 3GPP specifications described in TR 38.811~\cite{38811}.
While the described implementation is mainly related to the channel and the physical layer, a deep understanding of the propagation model is the first step towards proper protocol design~\cite{lecci2021accuracy}, which makes our module a valuable and accurate tool in the study of NTNs.
Specifically, the module introduces: (i) new simulation scenarios for NTN; (ii) a new \gls{pl} model for the air/space channel in a wide range of frequencies (from 0.5 GHz to 100 GHz); (iii) the characterization of atmospheric absorptions; (iv) a new fast fading model for the space environment; (v) an antenna model for both terrestrial and non-terrestrial nodes; and (vi) a new coordinate system to account for the Geocentric Cartesian coordinate system of satellites. 

The second part of this chapter focuses on novel solutions to improve the scalability of the ns-3 5G NR simulation framework. Specifically, Section~\ref{sec:ch-perf-improv} presents optimizations to the ns-3 implementation of the TR 38.901 channel model of~\cite{tommaso:20}, both at the codebase and at the design level, which aim to provide wireless researchers with the tools for simulating future dense wireless scenarios in a computationally efficient manner. Specifically, we significantly improve the runtime of simulations involving the 3GPP TR 38.901 channel model~\cite{TR38901} by porting the intensive linear algebra operations to the open-source library \texttt{Eigen}~\cite{eigenweb}. To this end, we also design and implement a set of common linear algebra APIs, which increase the modularity of the \texttt{spectrum} module with respect to the underlying data structures and algorithms.
Additionally, we propose a simplified channel model, based on~\cite{TR38901}, which aims to provide an additional order of magnitude of runtime reduction, at the cost of a slight accuracy penalty. Profiling results show that the support for \texttt{Eigen}, coupled with further TR 38.901 optimizations, leads to a decrease of up to 5 times in the simulation time of typical \gls{mimo} scenarios. Furthermore, the proposed performance-oriented channel model further improved the runtime of simulations, which now take as low as $6$~\% with respect to the full TR 38.901 channel model, with a negligible loss in accuracy.
