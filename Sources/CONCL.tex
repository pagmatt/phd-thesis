\chapter{Conclusions and future work}

% The advent of 5G mobile networks marked a significant milestone in wireless communications, offering unprecedented peak performance and flexibility, which enable the support of diverse applications with heterogeneous yet stringent requirements. 
One of the key innovations of 5G is its ability to harness \gls{mmwave} frequencies, thereby unlocking substantial unused radio resources that were previously inaccessible. The next iteration of mobile networks, i.e., 6G is predicted to extend the bandwidth capabilities even further by expanding the supported spectrum bands to encompass \gls{thz} frequencies. 
While these advancements enable unprecedented data rates and ultra-low latency, \gls{mmwave} and \gls{thz} frequencies are hindered by challenging propagation conditions that impede the provision of ubiquitous high-speed wireless connectivity.
% These limitations underscore the need for continued research and development aimed at mitigating the unfavorable propagation characteristics exhibited by these frequency regimes, enabling ubiquitous high-performing wireless connectivity.
This thesis tackles the these limitations by studying innovative coverage enhancement solutions which hold promise for the widespread adoption of \gls{mmwave} and \gls{thz} deployments in 6G cellular networks, overcoming their unfavorable propagation characteristics.

Specifically, In Chapter 2, TODO

In Chapter 3, TODO

In Chapter 4, we analyzed the performance of \gls{irs}-aided cellular deployments with practical constraints on the \gls{irs} reconfiguration period. In this context, we maximized the average sum capacity, subject to a fixed number of \gls{irs} reconfigurations per time frame. We proposed a clustering-based scheduling algorithms, which groups users with similar (optimal) \gls{irs} configurations based on either a distance metric or the achievable capacity, thereby mitigating the impact of the reconfiguration limit.
The efficacy of the proposed algorithms was assessed in terms of computational complexity, sum capacity, and fairness across diverse channel conditions, as a function of the \gls{irs} configuration and the number of users.
% Additionally, the influence of phase shift quantization on algorithm performance was examined.
Our results indicate that capacity-based clustering outperforms distance-based approaches, yielding up to $85$\% of the sum capacity achievable in an ideal scenario (i.e., with no reconfiguration constraints), while reducing the number of IRS reconfigurations by up to $50$\%.
% Different clustering algorithms were numerically evaluated in terms of computational complexity, sum capacity, and fairness under different channel conditions, as a function of the size of the \gls{irs} size and the number of users, and with or without quantization of phase shifts.
% The results showed that capacity-based clustering outperforms distance-based clustering, and can achieve up to $85$\% of the sum capacity obtained in an ideal deployment (with no reconfiguration constraints), reducing by $50$\% the number of \gls{irs} reconfigurations.