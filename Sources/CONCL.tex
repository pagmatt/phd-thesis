\chapter{Conclusions}
\label{ch:concl}

% The advent of 5G mobile networks marked a significant milestone in wireless communications, offering unprecedented peak performance and flexibility, which enable the support of diverse applications with heterogeneous yet stringent requirements. 
One of the key innovations of 5G is its ability to harness \gls{mmwave} frequencies, thereby unlocking substantial unused radio resources that were previously inaccessible. The next iteration of mobile networks, i.e., 6G is predicted to extend the bandwidth capabilities even further by expanding the supported spectrum bands to encompass \gls{thz} frequencies. 
While these advancements enable unprecedented data rates and ultra-low latency, \gls{mmwave} and \gls{thz} frequencies are hindered by challenging propagation conditions that impede the provision of ubiquitous high-speed wireless connectivity.
% These limitations underscore the need for continued research and development aimed at mitigating the unfavorable propagation characteristics exhibited by these frequency regimes, enabling ubiquitous high-performing wireless connectivity.
This thesis tackles these limitations by studying innovative coverage enhancement solutions which hold promise for the widespread adoption of \gls{mmwave} and \gls{thz} deployments in 6G cellular networks, overcoming their unfavorable propagation characteristics.

In Chapter~\ref{ch:sim-tools}, we proposed a signal model for 5G NR deployments featuring \glspl{irs} and \gls{af} relays based on the 3GPP TR 38.901 channel model. Our simulation framework provides numerical guidelines for dimensioning IRS/AF-assisted networks. We also developed an ns-3 implementation of the 3GPP channel model for \gls{ntn} scenarios, which we open-sourced and validated against 3GPP calibration results. Additionally, we presented optimizations for simulating \gls{mimo} wireless channels in ns-3, including improved linear algebra routines and a performance-oriented statistical channel model that significantly reduces simulation time.

Specifically, in Chapter~\ref{ch:iab} we proposed a semi-centralized resource partitioning scheme for 5G and beyond \gls{iab} networks, coupled with a set of allocation policies. Our analysis demonstrated that the integration of this lightweight resource allocation cooperation significantly enhances the system end-to-end throughput and delay, while preventing or mitigating network congestion in the backhaul links. We also provided considerations for implementing a semi-centralized resource allocation controller in real-world deployments. 
% Although the proposed solution requires relatively low signaling data and exhibits robust performance despite increased central allocation periods, this remains a significant constraint.
Moreover, we introduced the first reliability-focused scheduling and path selection algorithm specifically designed for \gls{iab} networks. Our \gls{rl}-based solution was shown to effectively account for the complexities of the network, including channel variations, interference, and load dynamics. Results demonstrated that our proposed algorithm not only achieves highly reliable performance in the presence of these challenges, but also outperforms benchmark schemes in terms of throughput, latency, and packet-drop rate. The reliability of our approach is rooted in its ability to jointly minimize the average latency and the expected value of tail losses through the use of \gls{cvar} as a risk metric.
Additionally, we conducted the first comprehensive evaluation of the potential benefits of sub-terahertz frequencies for 6G \gls{iab} networks, utilizing a customized extension of the open-source Sionna simulator. This simulation framework enabled us to assess the feasibility of employing mixed mmWave and sub-terahertz links in conjunction with greedy algorithms to optimize the deployment of backhaul networks.

In Chapter~\ref{ch:irs-block}, we analyzed the performance of \gls{irs}-aided cellular deployments with practical constraints on the \gls{irs} reconfiguration period. In this context, we maximized the average sum capacity, subject to a fixed number of \gls{irs} reconfigurations per time frame. We proposed a clustering-based scheduling algorithms, which groups users with similar (optimal) \gls{irs} configurations based on either a distance metric or the achievable capacity, thereby mitigating the impact of the reconfiguration limit.
The efficacy of the proposed algorithms was assessed in terms of computational complexity, sum capacity, and fairness across diverse channel conditions, as a function of the \gls{irs} configuration and the number of users.
% Additionally, the influence of phase shift quantization on algorithm performance was examined.
Our results indicate that capacity-based clustering outperforms distance-based approaches, yielding up to $85$\% of the sum capacity achievable in an ideal scenario (i.e., with no reconfiguration constraints), while reducing the number of IRS reconfigurations by up to $50$\%.
% Different clustering algorithms were numerically evaluated in terms of computational complexity, sum capacity, and fairness under different channel conditions, as a function of the size of the \gls{irs} size and the number of users, and with or without quantization of phase shifts.
% The results showed that capacity-based clustering outperforms distance-based clustering, and can achieve up to $85$\% of the sum capacity obtained in an ideal deployment (with no reconfiguration constraints), reducing by $50$\% the number of \gls{irs} reconfigurations.